\documentclass[output=paper
,modfonts
,nonflat]{langsci/langscibook} 

\title{Opacity in agreement}
\author{Laura Kalin\affiliation{Princeton University}}
\ChapterDOI{10.5281/zenodo.3541751}

\abstract{In this paper, I use a complex pattern of agreement in progressives in the Neo-Aramaic language Senaya as a window into the timing of and relationship among the different components of $\varphi$-agreement. In particular, I argue that we need to recognize three distinct, ordered operations underlying what we see on the surface as $\varphi$-agreement -- Match, Value, and Vocabulary Insertion -- based on data that reveal that opacity can arise both in the relation between Match and Value, as well as in the relation between Value and Vocabulary Insertion. This work builds on and extends earlier research that recognizes the need for (at least a subset of) these distinct components, including (among many others) \citet{HalleMarantz93,HalleMarantz94}, \citet{Bejar03}, \citet{ArregiNevins12}, \citet{BhattWalkow13}, \citet{Bonet13}, and \citet{Marusicetal15}.}

\begin{document}

%the forest-set definition needs to be here so it does not apply to all papers as would happen by placing it in the localcommands or local packages	
\forestset{
		nice empty nodes/.style={ %style to get the branching without node
			for tree={calign=fixed edge angles, calign primary angle=-70,calign secondary angle=70}, %added calign for space between sisters
			delay={where content={}{shape=coordinate,for siblings={anchor=north}}{}}
		},
	}

\maketitle
\section{Introduction} \label{sec-kalin:1}

Every aspect of the mechanism(s) behind $\varphi$-agreement is highly debated, even within (broadly speaking) Minimalist approaches. This paper engages with two central questions about $\varphi$-agreement --  (i) the number of steps and (ii) the timing of these steps -- bringing to bear complex data from Neo-Aramaic progressives.

First, how many steps are involved in deriving what we see on the surface as $\varphi$-agreement? The traditional view is that there is one unified operation of Agree, which consists of matching (a probe finds a goal) and valuation (the value of a feature on the goal is transferred/copied to the probe) (\citealt{Chomsky00,Chomsky01,Bejar03,Preminger11,Preminger14}, i.a.). A competing view is that these two subcomponents of Agree are (at least potentially) separate operations (\citealt{vanKoppen07,BBP09,ArregiNevins12,BhattWalkow13,Bonet13,Marusicetal15, Smith17, AtlamazBakerTA}; \citetv{chapters/06-marusic-nevins}, i.a.). Further, once matching and valuation have taken place, it might be that there is a trivial and transparent relation between these newly valued $\varphi$-features and the phonological form that realizes them, or there may be yet further operations or steps required to understand the surface form of $\varphi$-agreement (see, e.g., \citealt{ArregiNevins12}).

The second question is -- given the operation(s) involved in $\varphi$-agreement (however many there are) -- when does each of these steps take place derivationally? Agree might initiate in the syntax, immediately upon merge of a probe ({\citealt{Bejar03,Preminger11,Preminger14}, i.a.}), it might be triggered upon completion of the phase \citep{Chomsky08}, or it might be entirely post-syntactic \citep{Bobaljik08}. Valuation might occur concurrently with matching (the first step of Agree), or could be a separate operation, even potentially taking place in a different component of the grammar (\citealt{vanKoppen07,BBP09,ArregiNevins12,BhattWalkow13,Bonet13,Marusicetal15,Smith17,AtlamazBakerTA}; \citetv{chapters/06-marusic-nevins}, i.a.). And again, beyond matching and valuation, the phonological realization of $\varphi$-features  might be straightforward, potentially even present from the start of the syntactic derivation, as in some lexicalist approaches to morphology. Or, the choice of an exponent might be an operation in its own right,  taking place after all (or nearly all) other operations  (\citealt{HalleMarantz93,HalleMarantz94,EmbickNoyer07}, i.a.). 

I will argue that we need to recognize (at least) three separate operations implicated in deriving surface $\varphi$-agreement, given in (\ref{OPS}), as well as the Activity Condition in (\ref{kalin:AC}). Though all three of these operations, as well as the Activity Condition, have been independently proposed elsewhere (see citations above), I will offer new evidence from Senaya regarding their precise formulation, their relative timing, and their interaction(s) with each other.

\eal \label{OPS}
\ex \textit{Match} {(takes place in the syntax)} \- \hfill (``{$\alpha$} Matches with {$\beta$}'')\\
An unvalued feature F (a probe, {$\alpha$}) Matches with the closest active valued instance of F (on a goal, {$\beta$}) in its c-command domain.
\ex \textit{Value} (takes place early in the post-syntax) \- \hfill (``{$\beta$} Values {$\alpha$}'')\\
The probe {$\alpha$} copies the value of F from an active goal {$\beta$} that {$\alpha$} has Matched with.
\ex \textit{Vocabulary Insertion} (VI) (takes place late in the post-syntax)\\
Phonetic exponents are matched to morphemes, obeying the subset principle \citep{Halle97}.
\zl

\ea \textit{Activity Condition} (constrains relations in the syntax and post-syntax)\\A feature F is \textit{active} (visible to Match and Value) iff it has not yet been copied (has not Valued a probe).\label{kalin:AC}
\z

\noindent Empirical evidence for the need for \xxref{OPS}{kalin:AC}, in the precise ways they are stated above, comes from agreement in progressives in Senaya, which furnish us with a testing ground involving (up to) three agreement ``slots'', for which there is variation -- within strict limits -- as to what sort of agreement can appear in these different positions.

The paper is laid out as follows. \sectref{sec-kalin:2} introduces the core data to be dealt with in the paper. \sectref{sec-kalin:3} introduces some basic assumptions about syntax generally and the syntax of Senaya in particular, putting this together in \sectref{sec-kalin:4} with the three steps of agreement and the Activity Condition above to account for the complex Senaya data. Finally, \sectref{sec-kalin:5} shows that an account of agreement as a single operation with a transparent spell-out is not well-suited to account for the data at hand.

\section{Agreement (in)variability in Senaya} \label{sec-kalin:2}

Neo-Aramaic languages (Semitic) are rich in agreement. Subjects, direct objects, and indirect objects can all trigger agreement under the right conditions (see, e.g., \citealt{DoronKhan12,KalinvanUrk15}). Most commonly, there are one or two agreement slots, each filled predictably by agreement with a certain argument. The language of interest here is the Neo-Aramaic language Senaya, originally spoken in the town of Sanandaj, Iran \citep{Panoussi90}; all data in this paper come from original fieldwork conducted by the author with \ia{McPherson, Laura@McPherson, Laura}Laura McPherson and \ia{Ryan, Kevin@Ryan, Kevin}Kevin Ryan in Los Angeles, between 2012 and 2014.

We'll start in Senaya with the invariable agreement that is found in transitive clauses in imperfective aspect, (\ref{hit}).

\eal \label{hit}
\ex[]{
\gll (\=Oya) ({axn\=i}) maxy-\=a-lan.\\
     she us hit.\textsc{impf-S.3f.sg-L.1pl}\\
\glt `She hits/will hit us.'\footnotemark
}
\ex[]{
\gll Axn\=i \=o ks\=uta kasw-{ox}-l\=a. \\
we that book.\textsc{f} write.\textsc{impf}-{\textsc{S.1pl}}-{\textsc{L.3f.sg}}\\
\glt `We write/will write that (specific) book.'
}
\zl\footnotetext{Pronominal arguments are always optional, though this is the only example that explicitly marks them as such. As a subject, a pronoun comes and goes freely, while as an object, the pronoun is overt only when under focus.}

\noindent The verb bases in (\ref{hit}) are in their imperfective root-and-template form, and are immediately followed by subject agreement. Subject agreement appears in the form of an ``S-suffix'', an agreement morpheme from the so-called ``S'' (for ``simple'') paradigm of agreement affixes (indicated by the capital S in the gloss for this agreement above). Object agreement appears next, in the form of an ``L-suffix'', from the ``L'' paradigm of agreement affixes (indicated by a capital L in glosses, all of which are underlyingly \textit{l}-initial). Full agreement paradigms are shown in \tabref{tab-kalin:1}. Note that in (\ref{hit}), both agreement morphemes are obligatory, and must appear in precisely this order and form. 


%\pagebreak

\begin{table}
\caption{Agreement morphemes in Senaya\label{tab-kalin:1}}
\begin{tabular}{lll}
\lsptoprule
  {}    & S-suffix  & L-suffix               \\ 
\midrule
\oldstylenums{1}\textsc{m.sg}   & -en  & \multirow{2}{*}{-l\=i}             \\ 
\oldstylenums{1}\textsc{f.sg}   & -an                              &              \\ 
\oldstylenums{1}\textsc{pl}   & -ox                              & -lan             \\ 
\oldstylenums{2}\textsc{m.sg} & -et & -lox \\ 
\oldstylenums{2}\textsc{f.sg} &       -at             & -lax             \\ 
\oldstylenums{2}\textsc{pl}   & -\=iton                             & {-l\=oxon}             \\ 
\oldstylenums{3}\textsc{m.sg} & -$\emptyset$   & -l\=e              \\ 
\oldstylenums{3}\textsc{f.sg} &          -a                        & -l\=a              \\ 
\oldstylenums{3}\textsc{pl}   &         -\=i                         & {-l\=u/-lun}             \\ \lspbottomrule
\end{tabular}
\end{table}
\noindent Intransitive clauses in imperfective aspect, (\ref{sleep}), are minimally different from transitive clauses, lacking just the L-suffix that encodes object agreement.

\eal \label{sleep}
\ex[]{
\gll \=Oya damx-\=a.\\
     she sleep.\textsc{impf-S.3f.sg}\\
\glt `She sleeps/will sleep.'
}
\ex[]{
 \gll Axn\=i palq-{ox}. \\
we leave.\textsc{impf}-{\textsc{S.1pl}} \\
\glt `We leave/will leave.'
}
\zl

\noindent Subject agreement takes the same shape as before, an S-suffix. And again, there is no other well-formed variant of this simple intransitive; agreement must appear in this fixed way. Note that nonspecific objects do not trigger agreement, and so the verb with such an object looks just like it would in an intransitive, bearing only subject agreement as an S-suffix, (\ref{ns}).

\eal \label{ns}
\ex[]{
\gll \=Oya xa ks\=uta kasw-{\=a}. \\
she one book.\textsc{f} write.\textsc{impf}-{\textsc{S.3f.sg}}\\
\glt `She writes/will write a (some) book.'
}
\ex[]{ 
\gll Axn\=i kod y\=oma xelya sh\=at-ox.\\
we each day milk drink.\textsc{impf-S.1{pl}}\\
\glt `We drink milk every day.'
}
\zl
The examples thus far show us that, in the imperfective, there are (up to) two agreement slots, filled predictably with subject agreement (the first slot; S-suffix) and object agreement (the second slot; L-suffix). Progressives in Senaya are of interest because they contain the imperfective verb stem and its two agreement slots, while adding an enclitic auxiliary (whose form is \textit{=y}, with surface variants \textit{=\=i} and \textit{=∅} depending on the adjacent sounds) which in turn bears its own agreement as well. This allows for (up to) three agreement slots within a single complex verb form.

Intransitive progressives, (\ref{prog1a}), and transitive progressives with a nonspecific object, (\ref{prog1b}), like the examples we have seen so far, have a fixed agreement pattern. Subject agreement, in these cases, appears twice: the subject agrees (i) in its usual spot on the imperfective stem, i.e., as an S-suffix adjacent to the verb (as in all the examples above), and (ii) on the auxiliary. There is no object agreement, consistent with (\ref{ns}).

\eal \label{prog1}
 \ex[]{ 
 \gll \=An\=i damx-{\=i}=$\emptyset$-{l\=u}.\\
they sleep.\textsc{impf}-{\textsc{S.3pl}}=\textsc{aux}-{\textsc{L.3pl}}\\
\glt `They are sleeping.'\label{prog1a}
}
\ex[]{
\gll Axn\=i xa ks\=uta kasw-{ox}=y-{ox}. \\
we one book.\textsc{f} write.\textsc{impf}-{\textsc{S.1pl}}=\textsc{aux}-{\textsc{S.1pl}}\\
\glt `We are writing some book.'\label{prog1b}
}
\zl

\noindent Agreement on the auxiliary takes the form of an L-suffix for third persons, (\ref{prog1a}), and an S-suffix for non-third persons, (\ref{prog1b}). One of the imperfective stem's agreement slots is filled, as is the auxiliary's agreement slot.

Where progressives with a \textit{single} agreeing argument utilize \textit{two} agreement positions with a \textit{fixed} pattern, (\ref{prog1}), progressives with \textit{two} agreeing arguments utilize \emph{three} agreement positions, and what fills them is \emph{variable}, depending in part on the person features of the object. We will walk through each possible configuration for a transitive progressive (with a specific object) in turn. 

First, to form a transitive progressive (with two agreeing arguments), it is possible to simply add the enclitic auxiliary to the typical imperfective verb form seen in  (\ref{hit}) (with its own subject and object agreement as usual) and then double the subject agreement on the auxiliary, just as in progressives with a single agreement argument like (\ref{prog1}); this is shown in (\ref{prog2}). I will refer to this as the \textsc{sbj-obj-sbj} variant, reflecting the three agreement morphemes in the order they appear.

\eal \label{prog2}
\ex[]{
\gll Axn\=i \=o ks\=uta kasw-{ox-l\=a}=y-{ox}. \\
we that book.\textsc{f} write.\textsc{impf}-{\textsc{S.1pl-L.3f.sg}}=\textsc{aux}-{\textsc{S.1pl}}\\ 
\glt `We are writing that book.' \label{prog2a}
}
\ex[]{
\gll \=Oya molp-\=a-l\=i=$\emptyset$-l\=a.\\
she teach.\textsc{impf-S.3f.sg-L.1sg=aux-L.3f.sg}\\
\glt `She is teaching me.'\label{prog2b}
}
\zl

\noindent Extrapolating from all the patterns we have seen so far, we might predict transitive progressives to always look like this, but this expectation is not borne out. There are in fact two other agreement configurations that are possible.

The second possible variant of a transitive progressive has default agreement (third singular masculine) on the auxiliary, rather than actual agreement, (\ref{prog3}), cf.~(\ref{prog2}). I will refer to this as the \textsc{sbj-obj-dflt} variant.

\eal \label{prog3}
\ex[]{
\gll Axn\=i \=o ks\=uta kasw-{ox-l\=a}=$\emptyset$-l\=e. \\
we that book.\textsc{f} write.\textsc{impf}-{\textsc{S.1pl-L.3f.sg}}=\textsc{aux}-{\textsc{L.dflt}}\\
\glt `We are writing that book.' \label{prog3a}
}
\ex[]{
\gll \=Oya molp-\=a-l\=i=$\emptyset$-l\=e.\\
she teach.\textsc{impf-S.3f.sg-L.1sg=aux-L.dflt}\\
\glt `She is teaching me.'\label{prog3b}
}
\zl

\noindent Note that default agreement is \textit{not} allowed to surface on the auxiliary when there is only one agreeing argument, (\ref{prog3.5}), cf. (\ref{prog1}).

\eal \label{prog3.5}
 \ex[*]{ 
 \gll \=An\=i damx-{\=i}=$\emptyset$-{l\=e}.\\
they sleep.\textsc{impf}-{\textsc{S.3pl}}=\textsc{aux}-{\textsc{L.dflt}}\\
\glt Intended: `They are sleeping.'\label{prog3.5a}
}
\ex[*]{ 
\gll Axn\=i xa ks\=uta kasw-{ox}=\=i-{l\=e}. \\
we one book.\textsc{f} write.\textsc{impf}-{\textsc{S.1pl}}=\textsc{aux}-{\textsc{L.dflt}}\\
\glt Intended: `We are writing some book.'\label{prog3.5b}
}
\zl

\noindent Default agreement on the auxiliary is limited to transitive progressives with an agreeing object.

The third and final possible configuration when there are two agreeing arguments in a transitive progressive is for the default agreement and object agreement of (\ref{prog3}) to swap positions. This means that object agreement appears on the auxiliary while default agreement appears in the usual object agreement slot on the verb stem, (\ref{prog4a}), cf.~(\ref{prog3a}). Note, however, that this \textsc{sbj-dflt-obj} variant is only possible when the object is third person, and so (\ref{prog4b}) (a swapped variant of (\ref{prog3b}), which has a first person object) is not possible.

\eal \label{prog4}
\ex[]{
\gll Axn\=i \=o ks\=uta kasw-{ox-l\=e}=$\emptyset$-l\=a. \\
we that book.\textsc{f} write.\textsc{impf}-{\textsc{S.1pl-L.dflt}}=\textsc{aux}-{\textsc{L.3f.sg}}\\
\glt `We are writing that book.' \label{prog4a}
}
\ex[*]{
\gll \=Oya molp-\=a-l\=e \{ =$\emptyset$-l\=i. / =y-an \}\\
she teach.\textsc{impf-S.3f.sg-L.dflt} \- \textsc{=aux-L.1sg} \- \textsc{=aux-S.1f.sg}\\
\glt Intended: `She is teaching me.'\label{prog4b}
}
\zl

\noindent In (\ref{prog4a}), the L-suffix on the verb base is default \textit{-l\=e}, while object agreement surfaces on the auxiliary. The two attempted auxiliary forms in (\ref{prog4b}) show that neither an L-suffix form nor S-suffix form for object agreement renders this configuration acceptable with a non-third person object. Instead, the argument configuration in (\ref{prog4b}) can only surface with the \textsc{sbj-obj-sbj} (\ref{prog2b}) or \textsc{sbj-obj-dflt} (\ref{prog3b}) agreement patterns. On the other hand, when the object is third person, all three agreement variants -- \textsc{sbj-obj-sbj} (\ref{prog2a}), \textsc{sbj-obj-dflt} (\ref{prog3a}), and \textsc{sbj-dflt-obj} (\ref{prog4a}) -- are possible.

All other logically possible agreement configurations for a progressive with two agreeing arguments are disallowed. For example, object agreement cannot be doubled, (\ref{prog5a}), subject agreement cannot swap with object agreement, (\ref{prog5b}), default agreement cannot be dropped on the verb base, (\ref{prog5c}), default agreement cannot go in the place of subject agreement when the subject agrees on the auxiliary, (\ref{prog5d}), etc.

\eal \label{prog5}
\ex[*]{
\gll Axn\=i \=o ks\=uta kasw-{ox-l\=a}=$\emptyset$-l\=a. \\
we that book.\textsc{f} write.\textsc{impf}-{\textsc{S.1pl-L.3f.sg}}=\textsc{aux}-{\textsc{L.3f.sg}}\\
\glt Intended: `We are writing that book.'\footnotemark \label{prog5a}
}
\ex[*]{
\gll Axn\=i \=o ks\=uta kasw-{\=a-lan}=\=i-l\=e. \\
we that book.\textsc{f} write.\textsc{impf}-{\textsc{S.3f.sg-L.1pl}}=\textsc{aux}-{\textsc{L.dflt}}\\
\glt Intended: `We are writing that book.'\label{prog5b}
}
\ex[*]{
\gll Axn\=i \=o ks\=uta kasw-{ox}=\=i-l\=a. \\
we that book.\textsc{f} write.\textsc{impf}-{\textsc{S.1pl}}=\textsc{aux}-{\textsc{L.3f.sg}}\\
\glt Intended: `We are writing that book.' \label{prog5c}
}
\ex[*]{
\gll Axn\=i \=o ks\=uta k\=as\=u-{$\emptyset$-l\=a}=y-ox. \\
we that book.\textsc{f} write.\textsc{impf}-{\textsc{S.dflt-L.1pl}}=\textsc{aux}-{\textsc{S.1pl}}\\
\glt Intended: `We are writing that book.' \label{prog5d}
}
\zl

\footnotetext{Note that this example is grammatical with the interpretation `We write/are writing/will write that book for her.' There are a number of puzzling properties of ditransitives in Senaya (one of which is their aspectual ambiguity), and so they are outside the scope of this paper. See \citet{Kalin14b} for the relevant data and an early theoretical account.}

\noindent The three available agreement slots do not result in a free-for-all, but rather, the observed variation is highly constrained.

\tabref{tab-kalin:2} summarizes all the grammatical agreement configurations introduced in this section.

\begin{table}
\caption{Imperfective and progressive agreement configurations}
\label{tab-kalin:2}
\begin{tabularx}{\textwidth}{XlXXXXX}
\lsptoprule Aspect & \# of agreeing & Verb & S slot & L slot & Aux & Aux agr  \\
& arguments & & & & & \\
\midrule
  \textsc{impf} & 1 & V.\textsc{impf} & \scshape sbj & -- & -- & -- \\
  \textsc{impf} & 2 & V.\textsc{impf} & \scshape sbj & \scshape obj & -- & -- \\
  \textsc{prog} & 1 & V.\textsc{impf} & \scshape sbj & -- & Aux           & \scshape sbj \\
 \textsc{prog}  & 2  & V.\textsc{impf}& \scshape sbj & \scshape obj & Aux & \scshape sbj\\
  \textsc{prog} & 2 & V.\textsc{impf} & \scshape sbj & \scshape obj & Aux & \scshape dflt \\
  \textsc{prog} & 2, 3rd p.\ \textsc{obj} & V.\textsc{impf}& \textsc{sbj} & \textsc{dflt} & Aux& \textsc{obj} \\
  	\lspbottomrule
 \end{tabularx}
\end{table}

\noindent Drawing on \tabref{tab-kalin:2} and the impossible forms in (\ref{prog5}), we can make the following generalizations about agreement in transitive progressives with an agreeing object: (i) subject agreement is always (at least) in its usual slot on the imperfective verb base; (ii) object agreement (with a specific object) must appear exactly once; (iii) Aux always hosts an agreement morpheme of some kind; (iv) Aux can agree with the subject, the object, or neither; and (v) when the object agrees on Aux (possible only for a third person object), default agreement must surface on the verb.

\section{Some preliminary notes on the syntax of Senaya} \label{sec-kalin:3}

In this section, I lay out my assumptions about syntax and agreement both generally and within Senaya, which I build on in \sectref{sec-kalin:4} to derive the agreement variation in progressives. First, I assume a phase-based theory of syntax, with phase boundaries at the clause level falling at $v$P and CP (\citealt{Chomsky01}, i.a.). Following \citet{Baker15}, I assume that the $v$P phase can be ``soft'', which means that the phase boundary can be transparent for establishing new case and agreement relations even after $v$P has been spelled out (see also \citetv{chapters/04-smith}). Finally, I assume that phases can be extended by head movement \citep{denDikken06,denDikken07,Gallego10}, and that it is the whole phase that is spelled out, not just the complement of the phase head \citep{FP05}.

Specific to Senaya, I adopt the general approach proposed by \citet{KalinvanUrk15}: (i) S-suffixes are the result of agreement with Asp; (ii) L-suffixes are the result of agreement with T; and (iii) the imperfective verb stem is formed by head movement of V/$v$ to Asp.\footnote{This agreement configuration is motivated by the imperfective-perfective agreement split, which is not discussed here.} In order to make  \citet{KalinvanUrk15} fully compatible with the theory of phases discussed above, I take $v$P to be a soft phase, extended to AspP in the context of imperfective Asp due to head movement (see also \citealt{Kalin15}). All of these components of Senaya syntax are shown in the representation of a simple imperfective clause in (\ref{tree1}), with a box around the extended $v$P phase; $\varphi$-probes are annotated with the morphological form they are spelled out with ($\varphi$\sub{S} for S-suffix, $\varphi$\sub{L} for L-suffix).

\begin{exe}
    \ex \label{tree1}
		\begin{forest} 
			[TP,nice empty nodes,for tree={l=3mm}, calign primary angle=-75,calign secondary angle=75 %extra command for separation
			[{T\\$ \phi $\sub{\textsc{l}}}]
			[AspP, tikz={\node [draw,black,inner sep=0,fit to=tree]{};} %this draws the box
			[{V+\textit{v}+Asp\sub{\textsc{impf}}\\$ \phi $\sub{\textsc{s}}},name=Asp]
			[\textit{v}P, 
			[Sbj]
			[{}, 
			[\sout{V+\textit{v}},name=v]
			[VP,
			[\sout{V},name=V] [Obj]]]]]]
			\draw[semithick,->,overlay] (V) to[out=south west,in=south,bend left=45] (v);
			\draw[semithick,->,overlay] (v) to[out=south west,in=south,bend left=45] (Asp);
		\end{forest}		
\end{exe}

\noindent A discussion of how (\ref{tree1}) can produce a fixed agreement pattern in the imperfective, deriving \xxref{hit}{ns}, is taken up in \sectref{sec-kalin:4}.

What is the structure of a progressive? A variety of evidence points to progressives having a biclausal structure, with the embedded clause truncated (``restructured''; \citealt{Wurmbrand98} et seq.). Evidence for biclausality comes from the fact that subject agreement can surface twice in progressives, \xxref{prog1}{prog2}, as can past tense marking, \textit{-w\=a}, (\ref{prog7}).

\ea[]{\label{prog7}
\gll Temal tamam y\=oma d\=amx-an\textbf{-w\=a}=y-an\textbf{-w\=a}.\\
yesterday all day sleep.\textsc{impf-S.1f.sg-pst=aux-S.1f.sg-pst}\\
\glt `Yesterday I was sleeping all day.'
}
\z

\noindent However, the embedded clause in a progressive is not fully independent from the matrix clause. The embedded clause \textit{must} be imperfective (no other verb base can appear), negation can only surface once, preceding the entire verbal complex, and, as shown in (\ref{prog7.5}), tense must match across the clauses.

\eal \label{prog7.5}
\ex[]{
\gll Temal tamam y\=oma d\=amx-an*(-w\=a)=y-an-w\=a.\\
yesterday all day sleep.\textsc{impf-S.1f.sg-pst=aux-S.1f.sg-pst}\\
\glt `Yesterday I was sleeping all day.'
}
\ex[]{
\gll Temal tamam y\=oma d\=amx-an-w\=a=y-an*(-w\=a).\\
yesterday all day sleep.\textsc{impf-S.1f.sg-pst=aux-S.1f.sg-pst}\\
\glt `Yesterday I was sleeping all day.'
}
\zl
For concreteness, I take progressive aspect to be expressed by a (silent) verb that selects for a clause truncated at TP, with control from matrix subject position into the embedded clause, similar to \citegen{Laka06} analysis of the Basque progressive.\footnote{I take there to be a control relation between the matrix and embedded subject, but nothing crucial hinges on this. If it is indeed control, then PRO must be able to have a full set of $\varphi$-features (potentially inherited from its controller; see, e.g., \citealt{Ussery08}). If this is instead a raising relation, then it must be that the embedded subject is able to agree both in the embedded clause and in the matrix clause after movement, i.e., the higher and lower copy of the subject must be independent in terms of agreement. I do not entertain a \textit{pro} analysis of the embedded subject because this position cannot be filled with an overt pronoun.} Within the matrix clause, the verb raises all the way to T, resulting in a complex head in T that contains both of the matrix clause's $\varphi$-probes.\footnote{It is unclear why it should be that head movement proceeds all the way up to T in the matrix clause of progressives, but only up to Asp in the embedded clause (and in imperfective clauses more generally). Empirically, some sort of unification of the S and L agreement loci is needed in the matrix clause of progressives because only one agreement suffix can be spelled out (unlike in the embedded clause), and the usual division of labor between S and L suffixes is leveled with respect to which agrees with the subject, and which the object. Independent of these facts, there does not seem to be a motivation for positing this additional head movement, and so it might be that some other mechanism is responsible for these effects. If head movement is the right model for these effects, then it also must be that $\varphi$-probes can c-command out of a complex head that contains them.} This structure is represented in (\ref{tree2}):

\begin{exe}
		\ex \label{tree2}
		\resizebox{\linewidth}{!}{\begin{forest} 
			[TP,nice empty nodes,for tree={l=3mm}, calign primary angle=-65,calign secondary angle=65
			[{V+\textit{v}+Asp+T\\$ \phi $\sub{\textsc{s}},$ \phi $\sub{\textsc{l}}},name=T-up]
			[AspP, calign primary angle=-65,calign secondary angle=65
			[\sout{{V+\textit{v}+Asp}},name=Asp-up]
			[\textit{v}P, calign primary angle=-65,calign secondary angle=65
			[Sbj\sub{i}]
			[{}, calign primary angle=-65,calign secondary angle=65
			[\sout{V+\textit{v}},name=v-up]
			[VP, calign primary angle=-65,calign secondary angle=65
			[{\sout{V}\\\textsc{prog}},name=prog-up]
			[TP, calign primary angle=-75,calign secondary angle=75
			[{T\\$ \phi $\sub{\textsc{l}}}]
			[AspP, tikz={\node [draw,black,inner sep=0,fit to=tree]{};}, calign primary angle=-65,calign secondary angle=65
			[{V+\textit{v}+Asp\sub{\textsc{impf}}\\$ \phi $\sub{\textsc{s}}},name=Asp]
			[vP
			[PRO\sub{i}]
			[{}
			[\sout{V+\textit{v}},name=v]
			[VP, calign primary angle=-65,calign secondary angle=65
			[\sout{V},name=V] [Obj]]]]]]]]]]]
			\draw[semithick,->,overlay] (V) to[out=south west,in=south,bend left=45] (v);
			\draw[semithick,->,overlay] (v) to[out=south west,in=south,bend left=45] (Asp);
			\draw[semithick,->,overlay] (prog-up) to[out=south,in=south,bend left=45] (v-up);
			\draw[semithick,->,overlay] (v-up) to[out=south west,in=south,bend left=60] (Asp-up);
			\draw[semithick,->,overlay] (Asp-up) to[out=south west,in=south,bend left=60] (T-up);
		\end{forest}}
\end{exe}

\noindent The final necessary step here is to adopt the last-resort analysis of auxiliaries proposed by \citet{Bjorkman11}, such that the enclitic auxiliary is inserted into the complex head in matrix T as a morphological host for the matrix clause's stranded $\varphi$-agreement.

Before moving on, it is worth considering how the surface morpheme order is derived from the above structures. While I have represented the structures as head-initial for ease of presentation (and will continue to do so below), Senaya is an SOV language. If heads follow their complements in the VP and the extended projection of the VP in Senaya, then the morpheme order in the verb falls out naturally:  the verb (stem) itself is initial, followed by the S-suffix on embedded Asp, followed by the L-suffix on embedded T. Taking the auxiliary to be a clitic inserted at the head of the (head-final) matrix TP, we can understand why it leans to its left (onto the complex verb already built in the embedded clause), with matrix agreement appearing last.

\section{The three steps of agreement} \label{sec-kalin:4}

In this section, I show how the ingredients from \sectref{sec-kalin:3}, coupled with  a three-step model of agreement, gives us the tools to account for the full range of complex agreement patterns in Senaya. In particular, the goal is to account for the following empirical facts: (i) imperfective clauses, as well as progressive clauses with one agreeing argument, have a fixed agreement pattern, (ii) progressive clauses with two agreeing arguments have a variable agreement pattern, and (iii) variation in such progressives is restricted to exactly three configurations, \textsc{sbj-obj-sbj}, \textsc{sbj-obj-dflt}, and \textsc{sbj-dflt-obj}. 

The three steps of agreement are repeated here from the introduction:

\eal \label{OPS2}
\ex \textit{Match} {(takes place in the syntax)} \- \hfill (``{$\alpha$} Matches with {$\beta$}'')\\
An unvalued feature F (a probe, {$\alpha$}) Matches with the closest active valued instance of F (on a goal, {$\beta$}) in its c-command domain.
\ex \textit{Value} (takes place early in the post-syntax) \- \hfill (``{$\beta$} Values {$\alpha$}'')\\
The probe {$\alpha$} copies the value of F from an active goal {$\beta$} that {$\alpha$} has Matched with.
\ex \textit{Vocabulary Insertion} (takes place late in the post-syntax)\\
Phonetic exponents are matched to morphemes, obeying the subset principle \citep{Halle97}.
\zl

\noindent Each phase triggers its own sequence of these operations, and when a phase is ``soft'', the arguments in the phase may remain accessible to later phases. In particular, I propose the following version of the Activity Condition \citep{Chomsky01}, where it is \textit{valuation} (Valuing a probe) that makes a goal ineligible for subsequent matching (being a target of Match for a different probe) or valuation (Valuing a different probe).

\ea \textit{Activity Condition}\\A feature F is \textit{active} (visible to Match and Value) iff it has not yet been copied (has not Valued a probe).\label{AC2}
\z

\noindent The operation that \textit{makes} a feature inactive -- Value -- is post-syntactic. However, a feature \textit{being} active is a precondition both for Match and for Value, (\ref{OPS2}), and so the Activity Condition constrains the application of both syntactic and post-syntactic operations. The intuition here is that a feature can only be copied once \citep{Bejar03} (after being copied, a feature cannot be copied again, i.e., cannot Value another probe) and is inert after being copied (it cannot be Matched with again). The crucial effects of this for the account of Senaya will be: (i) nominals that have not Valued a probe in a soft phase are accessible to probes in a later phase; (ii) multiple Matches with a single nominal are possible, so long as that nominal has not Valued a probe; and (iii) such multiply-Matched nominals can still only Value one probe.\largerpage

Matching and Valuation are related to VI at the probe site in the following way: 

\eal
\ex No Match (and so no Value): No features spelled out\footnotemark
\ex Match and Value: Valued features are eligible to be spelled out
\ex Match but no Value: Default features are eligible to be spelled out
\ex At a single insertion point, only one feature bundle can be spelled out
\zl 

\footnotetext{It might be that some languages have a dedicated exponent for such ``failed'' agreement or indeed use default agreement in these cases; see, e.g., \citet{Preminger11,Preminger14}, or \citet{Halpert12a}.}

\noindent In the remainder of this section, we will walk through the data in \sectref{sec-kalin:2} to see how all these pieces come together to derive the agreement facts in Senaya. 

We begin with a simple intransitive imperfective clause, (\ref{sleep2}), repeated from (\ref{sleep}), where there is a fixed agreement pattern: the subject triggers agreement in the form of an S-suffix.

\ea[]{ \label{sleep2}
\gll \=Oya damx-\=a.\\
     she sleep.\textsc{impf-S.3f.sg}\\
\glt `She sleeps/will sleep.'
}
\z

\noindent The derivation of such clauses proceeds as represented in (\ref{tree3}) and is discussed below.\largerpage

	\begin{exe}
		\ex \label{tree3}
		\begin{forest} 
			[TP,nice empty nodes,for tree={l=3mm}, calign primary angle=-75,calign secondary angle=75
			[{T\\$ \phi $\sub{\textsc{l}}}]
			[AspP, tikz={\node [draw,black,inner sep=0,fit to=tree]{};}
			[{V+\textit{v}+Asp\sub{\textsc{impf}}\\$ \phi $\sub{\textsc{s}}},name=Asp]
			[\textit{v}P, 
			[Sbj,name=Subj]
			[{}, %calign primary angle=-60,calign secondary angle=60
			[\sout{V+\textit{v}},name=v]
			[VP, %calign primary angle=-60,calign secondary angle=60
			[\sout{V},name=V] ]]]]]
			\node(Asp-mod) at (-0.6,-2.5){}; %two nodes by hand for the arrows closer to the head
			\node(Asp-mod2) at (0.1,-2.95){};
			\draw[overlay,semithick,->] (V) to[out=south west,in=south,bend left=45] (v);
			\draw[overlay,semithick,dashed,->] (Asp-mod2) to[out=south west,in=south,bend right=45] (Subj);
			\draw[overlay,semithick,->] (v) to[out=south west,in=south,bend left=70] (Asp-mod);
		\end{forest}
	\end{exe} 

\noindent In the first phase (boxed), the verb raises to Asp, and Asp's $\varphi$-probe Matches with the subject. After the AspP phase is spelled out, the subject Values the probe on Asp in the post-syntax, transferring its $\varphi$-features to Asp. Finally, VI applies, replacing these $\varphi$-features with the corresponding S-suffix. In the next phase, T's $\varphi$-probe searches its c-command domain, but finds no Match, as the subject is inactive (as per the Activity Condition, since it has already Valued a probe); $\varphi$ on T is therefore ineligible for VI. In an intransitive imperfective, then, the subject triggers S-suffix agreement, and this is invariable. This derivation is summarized in (\ref{layout1}).

\eal \label{layout1}
\ex Phase 1 (boxed)
\begin{xlist}
\ex Step 1 (syntax): $\varphi$ on Asp Matches with subject
\ex Step 2 (early post-syntax): Subject Values $\varphi$ on Asp
\ex Step 3 (late post-syntax): Subject features exponed on Asp
\end{xlist}
\ex Phase 2
\begin{xlist}
\ex Step 1 (syntax): No Match for $\varphi$ on T
\ex Step 2 (early post-syntax): No Value for $\varphi$ on T
\ex Step 3 (late post-syntax): No more features to be exponed
\end{xlist}
\zl 
\noindent Note that a clause with a nonspecific object is minimally different from (\ref{tree3}), as nonspecific objects are ineligible for Match \citep{KalinvanUrk15}. (Alternatively, it might be that T's $\varphi$-probe is present only when there is an object that needs licensing; see \citealt{Kalin18}.)

A transitive imperfective with an agreeing object, like that in (\ref{hit2}), repeated from (\ref{hit}), has a first phase identical to (\ref{tree3}), though of course with the addition of an object. This means that, again, the subject will agree in the form of an S-suffix. The second phase proceeds as shown in (\ref{tree4}), with the subject in angled brackets as it is inactive in the second phase.\largerpage[2]

\ea[]{\label{hit2}
\gll Axn\=i \=o ks\=uta kasw-{ox}-l\=a. \\
we that book.\textsc{f} write.\textsc{impf}-{\textsc{S.1pl}}-{\textsc{L.3f.sg}}\\
\glt `We write/will write that (specific) book.'
}
\z 

\begin{exe}
		\ex \label{tree4}
		\begin{forest} 
			[TP,nice empty nodes,for tree={l=3mm}, calign primary angle=-75,calign secondary angle=75
			[{T\\$ \phi $\sub{\textsc{l}}}]
			[AspP, %tikz={\node [draw,black,inner sep=0,fit to=tree]{};} %box by hand below
			[{V+\textit{v}+Asp\sub{\textsc{impf}}\\$ \phi $\sub{\textsc{s}}},name=Asp]
			[\textit{v}P, 
			[Sbj]
			[{}, %calign primary angle=-60,calign secondary angle=60
			[\sout{V+\textit{v}},name=v]
			[VP, %calign primary angle=-60,calign secondary angle=60
			[\sout{V},name=V] [Obj,name=Obj]]]]]]
			\node(T-mod) at (-1.8,-1.6){}; %node by hand at T so the arrow is closer
			\draw[overlay,semithick,dashed,->] (T-mod) to[out=south west,in=south,bend right=45] (Obj);
			\draw (-1,-0.75) rectangle (7.55,-6.2); %box by hand so the arrow is included
		\end{forest}
\end{exe}

\noindent Since the $v$P phase (extended to Asp) is ``soft'', T is still able to establish a Match relation with a nominal inside this phase; and since the subject is inactive, the closest nominal that is eligible for Match with T is the object. In the post-syntax, the object Values the $\varphi$-probe on T, and this is spelled out with an L-suffix that encodes object agreement. (\ref{layout2}) lays out the derivation of (\ref{hit2}/\ref{tree4}).

\eal \label{layout2}
\ex Phase 1 (boxed)
\begin{xlist}
\ex Step 1 (syntax): $\varphi$ on Asp Matches with subject
\ex Step 2 (early post-syntax): Subject Values $\varphi$ on Asp
\ex Step 3 (late post-syntax): Subject features exponed on Asp
\end{xlist}
\ex Phase 2
\begin{xlist}
\ex Step 1 (syntax): $\varphi$ on T Matches with object
\ex Step 2 (early post-syntax): Object Values $\varphi$ on T
\ex Step 3 (late post-syntax): Object features exponed on T
\end{xlist}
\zl

\noindent So far, then, we have derived the basic imperfective data, with up to two agreement slots and a fixed agreement pattern. 

Intransitive progressives, too, have a fixed agreement pattern, (\ref{prog1-2}), repeated from (\ref{prog1}): the subject agrees once on the imperfective base in its usual S-suffix form, and once on the auxiliary. 

 \ea[]{\label{prog1-2} 
 \gll \=An\=i damx-{\=i}=$\emptyset$-{l\=u}.\\
they sleep.\textsc{impf}-{\textsc{S.3pl}}=\textsc{aux}-{\textsc{L.3pl}}\\
\glt `They are sleeping.'
}
\z

\noindent The first phase, shown in the box in (\ref{tree5}), again proceeds just like in (\ref{tree3}): successful Match, Value, and VI resulting in subject agreement in the form of an S-suffix on (embedded) Asp. This component of the derivation is in fact invariant across all progressives, and is the reason that the inner S-suffix exponing subject agreement is the one thing that \textit{cannot} vary (cf. \tabref{tab-kalin:2}, page~\pageref{tab-kalin:2}). 

The second phase of (\ref{tree5}) is more complex, and will ultimately be the source of the three agreement variants of transitive progressives. The first thing to note here is that there is no phase boundary between the embedded clause and the matrix clause, precisely because this is a restructuring environment and so there is no embedded CP layer. Further, head movement of V to T in the matrix clause extends even the next-higher phase, $v$P, all the way to TP. This means that there are three $\varphi$-probes all at play in a single phase, unlike in all the previous derivations, where there was one $\varphi$-probe per phase. It is precisely this many-probes-to-one-phase relation that gives rise to the possibility of different surface agreement configurations in the progressive.

\begin{exe}
\ex\label{tree5}%
    \resizebox{\linewidth}{!}{\begin{forest} 
			[TP,nice empty nodes,for tree={l=3mm}, calign primary angle=-65,calign secondary angle=65
			[{V+\textit{v}+Asp+T\\$ \phi $\sub{\textsc{s}},$ \phi $\sub{\textsc{l}}},name=T-up]
			[AspP, calign primary angle=-65,calign secondary angle=65
			[\sout{{V+\textit{v}+Asp}},name=Asp-up]
			[\textit{v}P, calign primary angle=-65,calign secondary angle=65
			[Sbj\sub{i},name=Subj-up]
			[{}, calign primary angle=-65,calign secondary angle=65
			[\sout{V+\textit{v}},name=v-up]
			[VP, calign primary angle=-65,calign secondary angle=65
			[{\sout{V}\\\textsc{prog}},name=prog-up]
			[TP, calign primary angle=-75,calign secondary angle=75
			[{T\\$ \phi $\sub{\textsc{l}}}]
			[AspP, tikz={\node [draw,black,inner sep=0,fit to=tree]{};}, calign primary angle=-65,calign secondary angle=65
			[{V+\textit{v}+Asp\sub{\textsc{impf}}\\$ \phi $\sub{\textsc{s}}},name=Asp]
			[vP
			[PRO\sub{i},name=PRO]
			[{}
			[\sout{V+\textit{v}},name=v]
			[VP
			[\sout{V}]]]]]]]]]]]
			\node(Asp-mod) at (6.2,-9.1){}; %node by hand at low Asp so the arrow is closer
			\node(T-up-mod) at (-1.65,-2){}; %node by hand at high T so the arrow is closer
			\draw[semithick,dashed,->] (Asp-mod) to[out=south,in=south west,bend right=45] (PRO);
			\draw[semithick,dashed,->] (T-up-mod) to[out=south,in=south west,bend right=45] (Subj-up);
		\end{forest}}%
\end{exe}

With the embedded verb being intransitive, (\ref{prog1-2}), the derivation in the second phase of (\ref{tree5}) is still deterministic. The $\varphi$-probe on embedded T will fail to find a Match, as in intransitive imperfectives, (\ref{tree3}), and receive no exponent, hence there is no L-suffix on the verb base (before the auxiliary) in (\ref{prog1-2}). The complex head in matrix T will Match with the only active nominal left, the matrix subject, and the subject will Value T in the post-syntax, subsequently resulting in a second instance of subject agreement appearing in the matrix clause.\footnote{Since T bears $\varphi$-probes both from Asp and T, this helps account for why some of the agreement affixes on Aux are S-suffixes while others are L-suffixes, (\ref{prog1}). But, it is not clear from this analysis why it should be that first/second  person are expressed as S-suffixes while third person is expressed with an L-suffix. I do not attempt to derive this here.} This derivation is represented step-wise in (\ref{layout3}). (Note that transitive progressives with one agreeing argument are derived in essentially the same way, and so I do not provide a separate derivation of such clauses here.)\largerpage[-1]

\eal \label{layout3}
\ex Phase 1 (boxed)
\begin{xlist}
\ex Step 1 (syntax): $\varphi$ on embedded Asp Matches with embedded subj
\ex Step 2 (early post-syntax): Embedded subject Values $\varphi$ on embedded Asp
\ex Step 3 (late post-syntax): Embedded subject features exponed on embedded Asp
\end{xlist}
\ex Phase 2
\begin{xlist}
\ex Step 1 (syntax): \begin{itemize}
                    \item No Match for $\varphi$ on embedded T
                    \item $\varphi$ on matrix T Matches with matrix subject
                    \end{itemize}
\ex Step 2 (early post-syntax): \begin{itemize}
                    \item No Value for $\varphi$ on embedded T
                    \item Matrix subject Values $\varphi$ on matrix T
                    \end{itemize}
\ex Step 3 (late post-syntax): \begin{itemize}
                    \item No features to be exponed on embedded T
                    \item Matrix subject features exponed on matrix T
                    \end{itemize}
\end{xlist}
\zl

\noindent As discussed in \sectref{sec-kalin:4}, I take the auxiliary to surface simply as a morphological host for the stranded agreement in this matrix clause. 

Turning now to transitive progressives with two agreeing arguments, we will finally see where variation and the opacity in the agreement system comes in. The three agreement variants are shown in (\ref{3progs}), repeated from \sectref{sec-kalin:2}. 

\eal \label{3progs}
\ex[]{\label{sos}
\gll Axn\=i \=o ks\=uta kasw-{ox-l\=a}=y-{ox}. \\
we that book.\textsc{f} write.\textsc{impf}-{\textsc{S.1pl-L.3f.sg}}=\textsc{aux}-{\textsc{S.1pl}}\\ 
\glt `We are writing that book.' \hfill = \textsc{sbj-obj-sbj}
}
\ex[]{\label{sod}
\gll Axn\=i \=o ks\=uta kasw-{ox-l\=a}=$\emptyset$-l\=e. \\
we that book.\textsc{f} write.\textsc{impf}-{\textsc{S.1pl-L.3f.sg}}=\textsc{aux}-{\textsc{L.dflt}}\\
\glt `We are writing that book.' \hfill = \textsc{sbj-obj-dflt}
}
\ex[]{\label{sdo}
\gll Axn\=i \=o ks\=uta kasw-{ox-l\=e}=$\emptyset$-l\=a. \\
we that book.\textsc{f} write.\textsc{impf}-{\textsc{S.1pl-L.dflt}}=\textsc{aux}-{\textsc{L.3f.sg}}\\
\glt `We are writing that book.'  \hfill = \textsc{sbj-dflt-obj}
}
\zl
These three surface agreement configurations are all derived from a single syntactic structure, (\ref{tree6}), which shows the Match relations in the second phase. (The first phase, again, proceeds just as in (\ref{tree3}), and the embedded PRO subject is subsequently inactive, indicated via angled brackets.) 

	\begin{exe}
		\ex \label{tree6}
		\resizebox{\linewidth}{!}{%
		\begin{forest} 
			[TP,nice empty nodes,for tree={l=3mm}, calign primary angle=-65,calign secondary angle=65
			[{V+\textit{v}+Asp+T\\$ \phi $\sub{\textsc{s}},$ \phi $\sub{\textsc{l}}},name=T-up]
			[AspP, calign primary angle=-60,calign secondary angle=60
			[\sout{{V+\textit{v}+Asp}},name=Asp-up]
			[\textit{v}P, calign primary angle=-60,calign secondary angle=60
			[Sbj\sub{i},name=Subj-up]
			[{}, calign primary angle=-60,calign secondary angle=60
			[\sout{V+\textit{v}},name=v-up]
			[VP, calign primary angle=-60,calign secondary angle=60
			[{\sout{V}\\\textsc{prog}},name=prog-up]
			[TP, calign primary angle=-75,calign secondary angle=75
			[{T\\$ \phi $\sub{\textsc{l}}}]
			[AspP, calign primary angle=-60,calign secondary angle=60%, tikz={\node [draw,black,inner sep=0,fit to=tree]{};} 
			[{V+\textit{v}+Asp\sub{\textsc{impf}}\\$ \phi $\sub{\textsc{s}}},name=Asp]
			[vP, calign primary angle=-70,calign secondary angle=70%,
			[\trace{PRO\sub{i}},name=PRO]
			[{}, calign primary angle=-70,calign secondary angle=70%,
			[\sout{V+\textit{v}},name=v]
			[VP, calign primary angle=-60,calign secondary angle=60%,
			[\sout{V}] [Obj,name=Obj]]]]]]]]]]]
			\node(T-low-mod) at (3.75,-7.65){}; %node by hand at low T so the arrow is closer
			\node(T-up-mod) at (-1.52,-2){}; %node by hand at high T so the arrow is closer
			\draw[overlay,semithick,dashed,->] (T-up-mod) to[out=south,in=south west,bend right=45] (Subj-up);
			\draw[overlay,semithick,dashed,->] (T-up-mod) to[out=south,in=south west,bend right=50] (Obj);
			\draw[overlay,semithick,dashed,->] (T-low-mod) to[out=south,in=south west,bend right=40] (Obj);
			\draw (4.5,-6.8) rectangle (12.75,-12.3);
		\end{forest}}
	\end{exe}

\eal \label{layoutincomp}
\ex Phase 1 (boxed)
\begin{xlist}
\ex Step 1 (syntax): $\varphi$ on embedded Asp Matches with embedded subj
\ex Step 2 (early post-syntax): Embedded subject Values $\varphi$ on embedded Asp
\ex Step 3 (late post-syntax): Embedded subject features exponed on embedded Asp
\end{xlist}
\ex Phase 2 \label{phase2}
\begin{xlist}
\ex Step 1 (syntax):\begin{itemize}
                    \item $\varphi$ on embedded T Matches with object
                    \item $\varphi$ on matrix T Matches with matrix subject and with object
                    \end{itemize}
\ex Steps 2 \& 3 (post-syntax): several possible continuations
\end{xlist}
\zl\largerpage[-2]

\noindent Walking through the Match relations in the second phase, what we see first is that embedded T (unlike in an intransitive progressive, \ref{tree5}) does successfully find a Match, the object, just as in a transitive imperfective, (\ref{tree4}). Next, like in (\ref{tree5}), the complex head in matrix T Matches with the matrix subject. Empirically, we know that the matrix auxiliary can display agreement with not just the matrix subject, (\ref{sos}), but also with the object, (\ref{sdo}). There are several ways that we might understand this. What I will adopt for the remainder of the paper is to assume that after matrix T agrees with the matrix subject, the matrix subject raises to spec-TP (similar to \citeauthor{AnandNevins06}'s ``punting'', \citeyear{AnandNevins06}). This allows the second probe on matrix T to agree with a different active nominal, if there is one. In (\ref{tree6}), the object, not yet having Valued a probe (even though it is already in a Match relation), is still active, and so matrix T Matches with the object.\footnote{There are other alternatives for explaining why the second probe in matrix T does not simply Match with the matrix subject a second time. One possibility is that two probes on a single head are simply allowed to target different nominals (see, e.g., \citealt{Keine10}). For a more extensive discussion, see \citet[fn. 12]{Georgi12}.}\largerpage[-2]

The crucial pieces so far are that the object remain a potential target for Match at least until matrix T probes, and that matrix T be able to Match with the object. This provides us with the first crucial separation of and ordering among operations: Match and Value must not be one unified process; Value must follow Match, and must take place in a different component of the grammar. If Match and Value were one unified process, calculated cyclically in the syntax, then we would predict that embedded T should be the only probe that could successfully agree with the object, counter to fact.\footnote{Without an Activity Condition, so long as both embedded and matrix T are in the same phase, it would be possible for both embedded T and matrix T to Match/Value with the object. However, this would make the incorrect prediction that object agreement could be spelled out in both locations simultaneously, cf. (\ref{prog5a}).}  

After the basic Match relations are established in the syntax, there are several different possible continuations of the derivation of (\ref{tree6}/\ref{layoutincomp}) in the post-syntax. The choice among these continuations seems to be free, constrained only by the person restriction on object agreement on the auxiliary. 

The first possible continuation of (\ref{tree6}/\ref{layoutincomp}) is that the object Values embedded T, and becomes inactive. At matrix T, then, one of the $\varphi$-probes is Valued by the subject, while the other $\varphi$-probe has a Match (the object) but cannot get a Value because the object is no longer active (it has already Valued embedded T). This means that at VI, embedded T is spelled out with object agreement (as an L-suffix). At matrix T, VI can spell out (matrix) subject agreement, as one of the $\varphi$-probes on T Matched/Valued with the subject. This possibility is represented in (\ref{phase2-1}), the first possible continuation of (\ref{phase2}).

\begin{exe}
\ex Phase 2, Option A -- deriving \textsc{sbj-obj-sbj} \label{phase2-1}
\begin{xlist}
\ex {Step 1 (syntax)\begin{itemize}
                    \item $\varphi$ on embedded T Matches with object
                    \item $\varphi$ on matrix T Matches with matrix subject and with object
                    \end{itemize}}
\ex {Step 2 (early post-syntax)\begin{itemize}
                    \item Object Values $\varphi$ on embedded T
                    \item Matrix subject Values $\varphi$ on matrix T
                    \item No Value for second $\varphi$ on matrix T; gets 3\textsc{m.sg} default features
                    \end{itemize}
\label{phase2-1b}}
\ex {Step 3 (late post-syntax)\begin{itemize}
                    \item Object features exponed on embedded T
                    \item Subject features exponed on matrix T
                    \end{itemize}}
\end{xlist}
\end{exe}

\noindent Phase 2 of the derivation, as given above, produces the \textsc{sbj-obj-sbj} variant of the progressive, (\ref{sos}).

A minimally different derivation compared to (\ref{phase2-1}) involves just a difference at VI. Recall that matrix T has two $\varphi$-probes, one of which Matched with the matrix subject, and the other with the object; while the $\varphi$-probe that Matched with the subject was Valued by the subject (and so has the subject's $\varphi$-features), the other $\varphi$-probe Matched with the object but was not Valued by it (and so has a default features set), (\ref{phase2-1b}).  At matrix T, only one of these feature bundles can be spelled out, since they are at the same insertion site (by assumption). Spelling out the feature bundle that contains the subject's features derives \textsc{sbj-obj-sbj}, as above. Spelling out the other feature bundle instead is also a possibility.\footnote{It is important to recognize that it is not the two vocabulary items (the two agreement morphemes) that are competing with each other for insertion, in which case we'd always expect the more specific one -- the one whose features are the result of a successful Value relation -- to be inserted. Rather, it is that there are two bundles of $\varphi$-features at T, and either one (but only one) can be exponed (can be the target of VI).} (Note that this is not a possibility if the second $\varphi$-probe has not found a Match and so does not have at least default $\varphi$-features, cf.\ (\ref{tree5}).) This possibility leads to the \textsc{sbj-obj-dflt} variant, (\ref{sod}); the derivation is laid out in (\ref{phase2-2}).\largerpage[2]

\begin{exe}
\ex Phase 2, Option B -- deriving \textsc{sbj-obj-dflt} \label{phase2-2}
\begin{xlist}
\ex Step 1 (syntax) \begin{itemize}
            \item $\varphi$ on embedded T Matches with object
            \item $\varphi$ on matrix T Matches with matrix subject and with object
            \end{itemize}
\ex Step 2 (early post-syntax) \begin{itemize}
            \item Object Values $\varphi$ on embedded T
            \item Matrix subject Values $\varphi$ on matrix T
            \item No Value for second $\varphi$ on matrix T; gets 3\textsc{m.sg} default features
            \end{itemize}
\ex Step 3 (late post-syntax) \begin{itemize}
            \item Object features exponed on embedded T
            \item Default feature set exponed on matrix T
            \end{itemize}
\end{xlist}
\end{exe}

\noindent Whatever agreement is spelled out in the matrix clause, the auxiliary is inserted to host this stranded agreement. 

In (\ref{phase2-1}) and (\ref{phase2-2}), it is crucial in the early post-syntax that instances of Value be ordered with respect to each other: because the object Values embedded T first, it cannot subsequently Value matrix T. If operations in the post-syntax applied simultaneously (or if there were no Activity Condition), we could not explain why object agreement cannot be spelled out both on embedded T and matrix T, in the same derivation. On the other hand, if operations in the post-syntax were strictly cyclic, applying from the most embedded node up (rather than just applying sequentially), then we would expect (\ref{phase2-1}) and (\ref{phase2-2}) to be the only possible continuations of (\ref{layoutincomp}). This brings us to the third progressive variant.

The final possible continuation of the post-syntax of this second phase is that the object Values matrix T rather than embedded T. Matrix T thus has $\varphi$-features from the matrix subject (as above) as well as $\varphi$-features from the matrix object, and it is embedded T that has a Match but no Value, cf. (\ref{phase2-1}/\ref{phase2-2}). The first empirical piece that is derived here is that embedded T (the L-suffix position on the verb base) is necessarily spelled out with default features, corresponding to this Match without a Value. The next empirical piece is slightly trickier to account for. We might expect that either subject or object agreement could be exponed on matrix T, deriving the (attested) \textsc{sbj-dflt-obj} variant as well as a (unattested) \textsc{sbj-dflt-sbj} variant. VI should be able to produce either variant, and I assume it indeed does. However, the (unattested) \textsc{sbj-dflt-sbj} variant leaves no trace of the object's features; it is possible, then, that this alternative is ruled out by ``functional'' considerations of communicative effectiveness, or (similarly) a surface-level morphotactic constraint, which requires agreement with specific objects to be visible somewhere within the verbal complex. I do not attempt to formalize this constraint here.

The derivation of the (fully successful) \textsc{sbj-dflt-obj} variant is shown in (\ref{phase2-3}).\largerpage

\begin{exe}
\ex Phase 2, Option C -- deriving \textsc{sbj-dflt-obj} \label{phase2-3}
\begin{xlist}
\ex Step 1 (syntax) \begin{itemize}
            \item $\varphi$ on embedded T Matches with object
            \item $\varphi$ on matrix T Matches with matrix subject and with object
            \end{itemize}
\ex Step 2 (early post-syntax) \begin{itemize}
            \item Object Values $\varphi$ on matrix T
            \item Matrix subject Values $\varphi$ on matrix T
            \item No Value for $\varphi$ on embedded T; gets 3\textsc{m.sg} default features
            \end{itemize}
\ex Step 3 (late post-syntax) \begin{itemize}
            \item Default feature set exponed on embedded T
            \item Object features exponed on matrix T
            \end{itemize}
\end{xlist}
\end{exe}

\noindent That this final version of the post-syntax is only a possibility for third person objects seems to confirm that matrix agreement comes from one complex head, as multiple agreement relations from a single head are precisely where we expect to see a person case constraint effect (see, e.g., \citealt{Anagnostopoulou03,BejarRezac03,AdgerHarbour07,Rezac08,Rezac11}). Here, what this seems to mean is that the second $\varphi$-probe on matrix T is only able to be Valued by a third person object.\footnote{This may also reveal that nominal licensing has both a syntactic and a post-syntactic component -- licensing requires a successful Match and a successful (complete) Value.}

To understand all these variations of a successful post-syntax in the transitive progressive, we need to recognize the post-syntax as consisting of ordered operations that (within a phase) need not apply cyclically (from the most embedded node up), at least within post-syntactic operations of the same type.\footnote{A somewhat similar conclusion is reached by \citet{DealWolf17}, examining the relative order of insertion of vocabulary items in Nez Perce.} Each instance of Value in the post-syntax is crucially ordered with respect to other instances of Value, but this order is not fixed. In the derivations of the \textsc{sbj-obj-sbj} and \textsc{sbj-obj-dflt} variants, Value at embedded T bleeds Value at matrix T. In the derivation of \textsc{sbj-dflt-obj}, this relation is reversed -- Value at matrix T bleeds Value at embedded T. VI crucially takes place after all Value operations are completed, i.e., once every $\varphi$-probe has a real or default set of $\varphi$-features.

What these data are also revealing to us is that surface agreement can be opaque in several ways. In the \textsc{sbj-obj-dflt}  and \textsc{sbj-dflt-obj} variant, matrix T's Match/Value with the subject is what made Match/Value with the object possible, but no subject agreement is expressed on the auxiliary; this can be understood as opacity introduced by the relationship between Value and VI -- of the two sets of $\varphi$-features at matrix T, only one can be exponed. In these same variants (\textsc{sbj-obj-dflt}, \textsc{sbj-dflt-obj}), there is opacity in the relationship between Match and Value: default features arise when there is a successful Match relation but failed Value relation; these surface default features conceal which nominal the Match was with. While the three core operations involved in agreement (Match, Value, VI) feed each other, they do so imperfectly, and so are necessarily separate operations.

\section{Transparent agreement?} \label{sec-kalin:5}

The Senaya progressive data pose a number of problems for accounts of Agree as a unified operation with a transparent spell-out. In particular, the spell-out of $\varphi$-agreement in such a system should be entirely predictable from the lexical items present, coupled with the structure they are in: for every probe in a particular structure, there is one closest c-commanded nominal; agreement will always be with that nominal and the transferred $\varphi$-features will consistently be spelled out in a fixed, predictable way. In order to capture variation like that seen in Senaya, such a system would therefore need to appeal to different syntactic structures (either underlying or derived) and/or different enumerations for the progressive agreement variants. Whereas the account that I proposed in \sectref{sec-kalin:4} appeals to divergent derivations in the \textit{post-syntax}, an account that appeals to a unified Agree operation would necessitate divergent derivations in the \textit{syntax}.

Are there multiple progressive structures (or enumerations) in Senaya? As a first (potentially negative) indication, there seems to be no consistent semantic difference  across the variants of the progressive with differing agreement patterns, i.e., any grammatical variant can be used with the same meaning; grammatical variants seem to always be exchangeable for each other. Second, the basic word order of Senaya, SOVX, remains constant across the agreement variants, and c-command relations are similarly unaffected -- the subject remains the subject and c-commands the object across the variants.

Putting aside the lack of obvious independent evidence for different structures, it is still worth working out whether this sort of analysis has a chance of accounting for the agreement variants. The remainder of this section explores this possibility and concludes that positing different structures and/or enumerations does not help us understand the agreement variation in any meaningful way. I will take as a starting point all the same basic assumptions about Senaya syntax laid out in \sectref{sec-kalin:3}. 

One ingredient that can help us begin to account for the three agreement variants via the syntax is to posit that the matrix subject position of a progressive can either be filled by a lexical subject or a null expletive subject. When this position is filled by a lexical subject, the auxiliary agrees with it, (\ref{SOStrans}), yielding the \textsc{sbj-obj-sbj}, (\ref{prog2}), variant.
	
\begin{exe}
		\ex \label{SOStrans}\relax
		[ \ConnectDashTail{SBJ\sub{i}} \ \ConnectDashHead*{V+\textit{v}+Asp+T}\relax\  [ \sout{Asp} [ \sout{\textit{v}} [ \sout{V} [ \ConnectDashTail{T}[A]\relax\  [ \ConnectDashTail{V+\textit{v}+Asp}[B]\relax\  [ \ConnectDashHead{PRO\sub{i}}[B]\relax\  [ \sout{\textit{v}} [ \sout{V} \ConnectDashHead[2ex]{OBJ}[A]\relax ]]]]]]]]]
\end{exe}
	
\noindent When this position is filled by an expletive subject, the auxiliary agrees with the expletive, (\ref{SODtrans}),  yielding default agreement (or potentially true agreement with the expletive, assuming it is third person singular); this derives the \textsc{sbj-obj-dflt}, (\ref{prog3}), variant.

	\begin{exe}
		\ex \label{SODtrans}\relax
		[ \ConnectDashTail{EXPL}\ \ConnectDashHead*{V+\textit{v}+Asp+T}\relax\ [ \sout{Asp} [ \sout{\textit{v}} [ \sout{V} [ \ConnectDashTail{T}[A]\relax\ [ \ConnectDashTail{V+\textit{v}+Asp}[B]\relax\ [ \ConnectDashHead{SUBJ}[B]\relax\ [ \sout{\textit{v}} [ \sout{V} \ConnectDashHead[2ex]{OBJ}[A]\relax ]]]]]]]]]
	\end{exe}
	
\noindent So far, this seems like a sensible way to go, as the progressive matrix verb (arguably) need not assign its own theta role/introduce its own argument.

An account that posits an expletive matrix subject in the cases above makes the prediction that an expletive subject (and thus default agreement on the Aux) should be available in all progressives. However, for intransitive progressives as well as transitive progressives with a nonspecific (non-agreeing) object, this prediction is incorrect, (\ref{prog3.5}). For this account to survive, then, we need to supplement it by limiting the expletive to transitive progressives that have a specific/agreeing object, as this is the only time default agreement can surface. This would be a highly arbitrary and suspicious constraint on the expletive, since matrix T (in this analysis as compared to that in \sectref{sec-kalin:4}) has no relationship at all with the object; the availability of an expletive in the matrix clause's subject position should then not be able to be mediated by the nature/presence of an embedded object.

Turning now to the third agreement pattern, \textsc{sbj-dflt-obj} (\ref{prog4}), there are several components that are puzzling and need to be explained -- object agreement appears on the progressive auxiliary, this object agreement is limited to third person, and default agreement can appear in the slot usually reserved for object agreement on the imperfective verb stem.\footnote{An anonymous reviewer suggests that the \textsc{sbj-dflt-obj} variant could simply be an instance of post-syntactic local dislocation, à la \citet{EmbickNoyer01}, applying to the output of (\ref{SODtrans}). It is true that in the present tense, the only overt change needed to derive \textsc{sbj-dflt-obj} from \textsc{sbj-obj-dflt} is the swapping of the two final agreement morphemes, as the auxiliary (for predictable phonological reasons) will always be null in these cases. However, there is a special allomorph of the auxiliary in the past tense when it bears 3rd person agreement, in which case the enclitic auxiliary (including agreement) surfaces as \textit{ya-w\=a}. Deriving one progressive variant from the other via local dislocation is impossible in the past tense, and so I do not entertain it as a possible explanation for the \textsc{sbj-obj-dflt} variant more generally.} A first, seemingly  plausible account of this verb form is to take L-suffixes to be clitics -- there is indeed evidence for the clitichood of L-suffixes in some Neo-Aramaic languages \citep{DoronKhan12}, though there is no evidence of this Senaya-internally. If the object marking on the imperfective base is a clitic, then it is natural to think that this clitic could ``climb'' to the progressive auxiliary (especially since this is a restructuring environment), thereby explaining why object marking can appear so high. However, this account of the displacement of object marking raises more questions than it answers. Assuming the structure/Agree relations in (\ref{SOStrans}) underly the clitic-climbing structure, why does clitic-climbing then block subject agreement on the auxiliary? And why does default agreement show up where the clitic used to be? If instead the structure in (\ref{SODtrans}) is taken to feed clitic-climbing, why does the object clitic climbing ``up'' cause the default agreement on the auxiliary to climb ``down''? And regardless of what the underlying structure is, why should it be that only third person clitics climb? None of these are plausible side effects or properties of clitic climbing, to my knowledge.

An alternative analysis of the third agreement variant, \textsc{sbj-dflt-obj} (\ref{prog4}), is to posit that the object moves over the subject, and there is subsequent agreement with the object in its high position, (\ref{SDOtrans}).

	\begin{exe}
		\ex \label{SDOtrans}\relax
		[ \ConnectTail{OB}[A]\ConnectDashTail{J}[B]\relax\ \ConnectDashHead*{V+\textit{v}+Asp+T}[B]\relax\ [ \sout{Asp} [ \sout{\textit{v}} [ \sout{V} [ T [ \ConnectDashTail{V+\textit{v}+Asp}[C]\relax\ [ \ConnectDashHead{SUBJ}[C] [ \sout{\textit{v}} [ \sout{V} \ConnectHead*[2ex]{\sout{OBJ}}[A]\relax ]]]]]]]]]
	\end{exe}

\noindent Apart from the minimality violation that this would entail, this analysis again raises the question of why it should be that a third person object can raise, while a first/second person object cannot. This account also does not help us understand why there is default agreement on the imperfective verb base in these cases. Finally, as Senaya does not even have a productive passive construction, nor does the object c-command the subject in these progressives, positing that one of the progressive variants is derived via some sort of passivization (promotion of the object) is a non-starter.

An analysis of the progressive in Senaya that appeals to a system of transparent agreement fails, even allowing for different syntactic structures and enumerations for the different progressive agreement variants.

\section{Conclusion} \label{sec-kalin:6}

In this paper, I presented a complex case of variation in agreement marking in the Neo-Aramaic language Senaya. Progressives in Senaya furnish us with a clear case where a simple account of agreement, consisting of one step with a transparent spell-out, is insufficient to account for the empirical facts. Instead, I argued that these data show that there must be three separate (and ordered) core operations involved in deriving surface $\varphi$-agreement, Match, Value, and VI, building on much previous work (see, e.g., \citealt{vanKoppen07,BBP09,ArregiNevins12,BhattWalkow13,Bonet13,Marusicetal15, Smith17, AtlamazBakerTA}; \citetv{chapters/06-marusic-nevins}). While Match feeds Value and Value feeds VI, these feeding relations are imperfect and can obscure the application of prior operations -- not every Match is evident from transferred values (Value can fail after Match) and not every Value is evident in the exponed form (VI can fail to spell out every output of Value) -- leading to opacity in the surface form of agreement. I also proposed that operations in the post-syntax may apply counter-cyclically, contributing to the possibility of there being more than one grammatical surface form for the same underlying syntactic Match relations. Opacity and variation typically do not arise when there is one $\varphi$-probe per phase, and so it is only in environments that are more complex than this that we can tease apart hypotheses about the relationship(s) among the steps of $\varphi$-agreement.

\section*{Abbreviations}

\begin{multicols}{2}
	\begin{tabbing}
		\textsc{impf}\hspace{5mm} \= imperfective\kill
1, 2, 3 \> first, second, third person\\
\textsc{aux} \> auxiliary\\
\textsc{dflt} \> default\\
 \textsc{f} \> feminine\\
\textsc{impf} \> imperfective\\
 L \> L-suffix\\
 \textsc{m} \> masculine\\
 \textsc{pl} \> plural\\
 \textsc{prog} \> progressive\\
 \textsc{pst} \> past\\
 S \> S-suffix\\
 \textsc{sg} \> singular\\
	\end{tabbing} 
\end{multicols}

\section*{Acknowledgements}

Thank you to Nico Baier, Jonathan Bobaljik, Anke Himmelreich, Laura McPherson, Sandhya Sundaresan, Coppe van Urk, and audiences at University of Connecticut, Agreement Across Borders, Generative Syntax in the 21st Century, Multiple Agreement Across Domains, and the 90th meeting of the Linguistic Society of America for extremely helpful discussions of this work. Thank you also to two anonymous reviewers and the editors of this volume for invaluable feedback that greatly improved this paper. Finally, thank you to my Senaya consultant, \ia{Caldani, Paul@Caldani, Paul}Paul Caldani, for sharing his language with me.

{\sloppy\printbibliography[heading=subbibliography,notkeyword=this]}
\end{document}
