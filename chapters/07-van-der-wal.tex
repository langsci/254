\documentclass[output=paper
,modfonts
,nonflat]{langsci/langscibook} 

\title{The AWSOM correlation in comparative Bantu object marking}
\author{Jenneke {van der Wal}\affiliation{Leiden University Centre for Linguistics}}

\ChapterDOI{10.5281/zenodo.3541755}

\abstract{The Bantu languages show much variation in object marking, two parameters being (1) their behaviour in ditransitives (symmetric or asymmetric) and (2) the number of object markers allowed (single or multiple). This paper reveals that a combination of these parameter settings in a sample of 50+ Bantu languages results in an almost-gap, the AWSOM correlation: ``asymmetry wants single object marking''. A Minimalist featural analysis is presented of Bantu object marking as agreement with a defective goal \citep{Van_der_Wal2015} and parametric variation in the distribution of $\phi$ features on low functional heads (e.g. Appl) accounts for both the AWSOM and Sambaa as the one exception to the AWSOM.}

\begin{document}
	
\maketitle	
\section{Introduction: Bantu object marking} \label{sec-vdwal:1}


The Bantu languages are around 500 in number \citep[1]{Nurse_Philippson2003}, spread over most of sub-Saharan Africa. General typological properties include noun classes, agglutinative morphology and SVO basic word order. Finite verbs typically include derivational suffixes and inflectional prefixes. One of these prefixes can be the object marker, as shown in (\ref{ex-vdwal:1b}).

\protectedex{
\begin{exe}
\ex Lugwere \label{ex-vdwal:1}
	\xlist
	\ex \label{ex-vdwal:1a}
		\gll Swáya y-á-ßona óDéo.\footnotemark \\
		1.Swaya \textsc{1sm}-\textsc{fut}-see 1.Deo \\
		\footnotetext {Object markers referring to the Theme object are underlined, and object markers referring to the Recipient/Benefactive are in \textbf{boldface}. Where no source is mentioned, the data come from fieldwork.}	 
		\glt `Swaya will see Deo.'
	\ex  \label{ex-vdwal:1b}
		\gll Swáya y-á-\textbf{mu}-ßoná. \\
			 1.Swaya \textsc{1sm}-\textsc{fut}-\textsc{1om}-see \\ 
		\glt `Swaya will see him.'
	\endxlist
\end{exe}}
\footnotetext{Object markers referring to the Theme object are \uline{underlined}, and object markers referring to the Recipient/Benefactive are in \textbf{boldface}. Where no source is mentioned, the data come from fieldwork. }\noindent Within the Bantu languages that show object marking in a verbal prefix, there is much variation, which has been described along the following parameters (see \citealt{Hyman_Duranti1982,Polak1986,Morimoto2002,Beaudoin-Lietz_et_al2004,Marten_et_al2007,Riedel2009,Marten_Kula2012,Zeller2014,Marlo2015}, for typological overviews of Bantu object marking):

\begin{enumerate}
\item behaviour in ditransitives: only the highest object can be marked (asymmetric) or either object can be marked (symmetric); 
\item number of object markers allowed: one-two-multiple;
\item nature of the object marker: syntactic agreement (doubling) or pronominal clitic (non-doubling);
\item types of objects marked, specifically locative object markers, and animacy, definiteness, givenness (differential object marking);
\item position of object marker: pre-stem or enclitic.
\end{enumerate}
In the current paper I focus on parameters 1 and 2.\footnote{See \citet{Beaudoin-Lietz_et_al2004} and \citet{Marlo2015} for parameter 5, and see \citet{Van_der_Wal2017b} for the interaction between parameters 1 and 3/4, which shows a gap described as the RANDOM (the ``relation between asymmetry and non-doubling object marking'').} 
In \sectref{sec-vdwal:2}, I illustrate these two parameters and show their settings for 50+ Bantu languages. It is the first time that the parametric settings for such a large group of Bantu languages have been gathered, but this by itself is not the most interesting fact. What makes this overview of object marking typologically and theoretically fascinating is the interaction between the settings for both parameters. \citet[78]{Riedel2009} remarks that “Across the Bantu family, it has been observed that the languages which allow more than one object marker […] tend to be symmetric. […] these three properties [parameters 1-2-3, JvdW] do not correlate systematically with one another. For example, Sambaa is an asymmetric language with multiple object markers.” As will be shown in \sectref{sec-vdwal:3}, Sambaa turns out to be quite special in its combination of parameter settings, and all other languages in the current systematic comparative overview of object marking parameters provide evidence for the AWSOM correlation: ``asymmetry wants single object marking''. After providing a Minimalist featural analysis of object marking in \sectref{sec-vdwal:4}, I will use this ana-lysis to answer the following questions about the AWSOM in Sections~\ref{sec-vdwal:5} and~\ref{sec-vdwal:6}, proposing an explanation in the distribution of $\phi$ features on heads in the clausal spine:

\begin{enumerate}
\item What causes the correlation between symmetry and multiple object marking?
\item How can we account for object marking in Sambaa?
\item Why is this parameter setting for object marking so apparently rare?
\end{enumerate}
The paper is thus intended to contribute to the ongoing debate on the theory of Agree, as well as the upcoming field of Bantu typology, and formal approaches to language variation in general.

\section{Parameters of variation in number of object markers and symmetry} \label{sec-vdwal:2}

\citet{Bresnan_Moshi1990} divided Bantu languages into two classes -- symmetric and asymmetric -- based on the behaviour of objects in ditransitives. Languages are taken to be symmetric if both objects of a ditransitive verb behave alike with respect to object marking (see \citealt{Ngonyani1996} and \citealt{Buell2005}, for further tests). In Zulu, for example, either object can be object-marked on the verb (\ref{ex-vdwal:2}), making this a ``symmetric'' language.\footnote{One should, however, be careful in characterising a whole language as one type, since it has become more and more evident that languages can be partly symmetric (\citealt{Baker1988,Rugemalira1991,Alsina_Mchombo1993,Schadeberg1995,Simango1995,Ngonyani1996,Ngonyani_Githinji2006,Thwala2006,Riedel2009,Zeller_Ngoboka2006,Jerro2015,Jerro2016,Van_der_Wal2017a}, etc.).}\textsuperscript{,}\footnote{In this research I focus on recipient/benefactive/malefactive ditransitives, leaving aside instrumental/locative/reason applicatives, for which see \citet{Kimenyi1980,Baker1988,Alsina_Mchombo1993,Moshi1998,Ngonyani1998,Ngonyani_Githinji2006,Jerro2016}, among others. This is partly to keep ditransitives comparable across languages, and partly because it is debatable whether multitransitives with other thematic roles are underlyingly true double object constructions (the alternative being some sort of prepositional construction with a different hierarchical structure from that treated here).}


\begin{exe}	
\ex	Zulu (S42, \citealt[11]{Adams2010}) \label{ex-vdwal:2}
	\xlist
	\ex \label{ex-vdwal:2a}
		\gll U-mama u-nik-e aba-ntwana in-cwadi.\\  
	    1a-mama 1\textsc{sm}-give-\textsc{pfv} 2-children 9-book\\
		\glt `Mama gave the children a book.'
	\ex \label{ex-vdwal:2b}
		\gll U-mama u-\textbf{ba}-nik-e in-cwadi (aba-ntwana).\\
		1a-mama 1\textsc{sm}-\textbf{\textsc{2om}}-give-\textsc{pfv} 9-book 2-children\\
		\glt `Mama gave them a book (the children).'
	\ex \label{ex-vdwal:2c}
		\gll U-mama u-\uline{yi}-nik-e aba-ntwana (in-cwadi).\\ 
		1a-mama 1\textsc{sm}-\textsc{9om}-give-\textsc{pfv} 2-children 9-book\\
		\glt `Mama gave the children it (a book).'
	\endxlist
\end{exe}
Conversely, in asymmetric languages only the highest object (Benefactive, Recipient) can be object-marked; object-marking the lower object (Theme) is ungrammatical.


\begin{exe}
\ex Swahili (G42) \label{ex-vdwal:3}
	\xlist
	\ex[]{
		\gll A-li-\textbf{m}-pa kitabu.\\
		\textsc{1sm}{}-\textsc{pst}-\textsc{1om}{}-give 7.book\\
		\glt `She gave him a book.'}
	\ex[*]{
		\gll A-li-\uline{ki}-pa Juma.\\
	 	\textsc{1sm-pst-7om}-give 1.Juma\\
		\glt int. `She gave it to Juma.'}
	\endxlist
\end{exe}
This parameter splits the Bantu languages into two groups (where languages are classified as symmetric as soon as the Theme can be object-marked in any ditransitive construction, even if not all constructions are symmetric), as seen in \tabref{Table 1}.

\begin{table}[hb]
\caption{Parameterisation of Bantu languages according to the behaviour in ditransitives}
\label{Table 1}	
	\begin{tabularx}{\textwidth}{lX}
	\lsptoprule
	asymmetric & Bemba, Chichewa, Chimwiini, Chingoni, Chuwabo, Kagulu, Kiyaka,
	Lika, Lunda, Makhuwa, Matengo, Nsenga, Ruwund, Sambaa, Swahili, Tumbuka, Yao\\ 
	\midrule
	symmetric & Bembe, Chaga, Changana, Digo, Gitonga, Ha, Haya, Herero, Kimeru, Lugwere, Kikuyu, Kinande, Kinyarwanda, Kirundi, Kuria, Lozi, Lubukusu, Luganda, Luguru, Maragoli, Mongo, Ndebele, Nyaturu, Tshiluba, Totela, Setswana, Shona, Swati, Sotho, Tharaka, Xhosa, Zulu\\ 
	\midrule
	symm unknown & Ekoti, Fuliiru, Lucazi, Makwe, Rangi, Shimakonde\\
	\lspbottomrule
	\end{tabularx}
\end{table}

\hspace*{-0.7202pt}A second parameter distinguishes the number of object markers allowed. Many languages are restricted to only one object marker -- whether asymmetric as in (\ref{ex-vdwal:4}) or symmetric as in (\ref{ex-vdwal:5}). Other languages allow multiple markers to occur on the verb, the famous constructed example in (\ref{ex-vdwal:6}) illustrating the extreme of six object markers.

\begin{exe}
\ex Tumbuka (N20, Jean Chavula, personal communication) \label{ex-vdwal:4} 
	\xlist
	\ex[]{
		\gll  Wa-ka-cap-il-a mwaana vyakuvwara.\\
		\textsc{2sm-t}-wash-\textsc{appl}-\textsc{fv} 1.child 8.clothes \\
		\glt `They washed clothes for the child.'}
	\ex[]{
		\gll Wa-ka-\textbf{mu}-cap-il-a vyakuvwara.\\
		\textsc{2sm-t-1om}-wash-\textsc{appl}-\textsc{fv} 8.clothes \\
		\glt `They washed the clothes for him.'}
	\ex[*]{
		\gll Wa-ka-\uline{vi}-cap-il-a mwaana.\\
		\textsc{2sm-t-8om}-wash-\textsc{appl}-\textsc{fv} 1.child \\
		\glt int. `They washed them for the child.'}
	\ex[*]{
		\gll Wa-ka-\uline{vi}-\textbf{mu}-cap-il-a.\\
		\textsc{2sm-t-8om}-\textsc{1om}-wash-\textsc{appl-fv} \\
		\glt int. `They washed them for him.'}
	\endxlist
\end{exe}

\begin{exe}
\ex Zulu (S42, \citealt[220]{Zeller2012})\label{ex-vdwal:5}
	\xlist
	\ex[*]{
		\gll U-John u-\textbf{ba}-\uline{zi}-nik-ile.\\
		1a.John \textsc{1sm-2om-9om}-give-\textsc{pst}\\}
	\ex[*]{
		\gll U-John u-\uline{zi}-\textbf{ba}-nik-ile.\\
		1a.John \textsc{1sm-9om-2om}-give-\textsc{pst}\\ 
		\glt int. `John gave them to them.'}
	\endxlist
\end{exe}

\begin{exe}
\ex Kinyarwanda (JD62, \citealt[183]{Beaudoin-Lietz_et_al2004})\label{ex-vdwal:6}\\
	\gll Umugoré a-ra-na-ha-ki-zi-ba-ku-n-someesheesherereza.\\
	1.woman{} \textsc{1sm}-\textsc{dj}-also-\textsc{16om}-\textsc{7om}-\textsc{10om}-\textsc{2om}-\textsc{2sg}.\textsc{om}-\textsc{1sg}.\\ \textsc{om}-read.\textsc{caus}.\textsc{caus}.\textsc{appl}.\textsc{appl}\\
	\glt `The woman is also making us read it (book, cl. 7) with them (glasses, cl.10) to you for me there (at the house, cl.16).'
\end{exe}
There is a third type of languages where object marking is generally restricted to one marker, but under certain circumstances allows ``extra'' markers (1+). This is usually when the first marker is a reflexive, a 1\textsuperscript{st} person singular or sometimes also an animate object, as in (\ref{ex-vdwal:7b}). See \citet{Polak1986} and \citet{Marlo2014, Marlo2015} as well as \citet{Sikuku2012} for further discussion and illustration of this type of object marker.

\begin{exe}
\ex Bemba (M42, \citealt[245]{Marten_Kula2012})\label{ex-vdwal:7}
	\xlist
	\ex[*]{\label{ex-vdwal:7a}
		\gll N-àlíí-\uline{yà}-\textbf{mù}-péél-à.\\ 
		\textsc{1sg.sm-pst-6om-1om}-give-\textsc{fv}\\
		\glt Int: `I gave him it (e.g. water).'}
	\ex[]{\label{ex-vdwal:7b}
		\gll À-\uline{chí}-\textbf{m}-péél-é.\\ 
		1\textsc{sm-7om-1sg.om}-give-\textsc{opt}\\
		\glt `S/he should give it to me.'}
	\endxlist
\end{exe}
Classified according to the number of object markers, again the Bantu languages can be split as in \tabref{Table 2} (where languages are classified as ``multiple object markers'' as soon as they allow more than one object marker on the verb, even if the number is restricted, with the exception of the ``extra'' markers that are indicated as a separate group under “1+”):

\begin{table}
\caption{Parameterisation of Bantu languages according to the number of object markers}	
\label{Table 2}
\begin{tabularx}{\textwidth}{lX} 
	\lsptoprule
	single OM & Bembe, Changana, Chichewa, Chimwiini, Chingoni, Chuwabo, Digo, Ekoti, Gitonga, Herero, Kagulu, Kimeru, Kinande, Lika, Lozi, Luguru, Lunda, Makhuwa, Makwe, Maragoli, Matengo, Ndebele, Nsenga, Rangi, Swahili, Shimakonde, Shona, Sotho, Swati, Tumbuka, Xhosa, Yao, Zulu\\\midrule
	multiple OM & Chaga, Ha, Haya, Kinyarwanda, Kuria, Luganda, Lugwere, Kirundi, Sambaa, Setswana, Totela, Tshiluba\\\midrule
	1+ & Bemba, Fuliiru, Kikuyu, Kiyaka, Lubukusu, Mongo, Nyaturu, Ruwund, Tharaka\\ 
	\lspbottomrule
\end{tabularx}
\end{table}

\section{Interaction between multiple object markers and symmetry} \label{sec-vdwal:3}

Although the distribution of languages over parameter settings is quite even for the two parameters, the combination of parameters for behaviour in ditransitives and number of object markers is skewed, as already noted in the literature (\citealt[185]{Henderson2006}, \citealt[227]{Zeller_Ngoboka2015}). \citet{Riedel2009} describes the correlations as follows:

\begin{quote}
	“Across the Bantu family, it has been observed that the languages which allow more than one object marker, such as Haya and Rundi, tend to be symmetric. \citet{Baker2008a} suggests that this is a consequence of the properties of syntactic agreement as opposed to object clitics. \citet{Bentley1994} also lumps together agreement, animacy-sensitivity, having only one object marker and asymmetry as related properties. However, although this may well be a tendency across Bantu, these three properties do not correlate systematically with one another. For example, Sambaa is an asymmetric language with multiple object markers.” \citep[78]{Riedel2009}
\end{quote}
The question is thus what distribution a larger sample of languages will reveal, and the result of the current survey is summarised in \tabref{Table 3}.\largerpage[2]

\begin{table}[H]
\caption{Interaction between number of object markers and symmetry in Bantu languages} 
\label{Table 3}\label{tab-vdwal:3}
\begin{tabularx}{\textwidth}{lQQQ}
	\lsptoprule
             & \multicolumn{3}{c}{number of object markers}\\\cmidrule(lr){2-4}
	symmetry & multiple & single & 1+\\
	\midrule
	asymmetric & Sambaa & Chichewa, Chimwiini, Chingoni, Chuwabo, Kagulu, Lika, Lunda, Makhuwa, Nsenga, Swahili, Tumbuka, Matengo, Yao & Bemba, Kiyaka, Ruwund\\
	\midrule
	symmetric & Dzamba, Chaga, Ha, Haya, Kinyarwanda, Kirundi, Kuria, Luganda, Lugwere, Setswana & Bembe, Changana, Digo, Herero, Gitonga, Kikongo, Kimeru, Kinande, Lozi, Luguru, Maragoli, Ndebele, Shona, Sotho, Swati, Tshiluba, Totela, Zulu, Xhosa & Kikuyu, Lubukusu, Mongo, Nyaturu, Tharaka\\
	\midrule
	unknown &  & Ekoti, Makwe, Rangi, Shimakonde & Fuliiru\\
	\lspbottomrule
\end{tabularx}
\end{table} 


Perhaps surprisingly, in this combinations of parameters an almost-gap appears: there is a systematic correlation between multiple object marking and symmetry, which can be formulated as the AWSOM:

\begin{exe}
\ex \label{ex-vdwal:8} \textit{Asymmetry wants single object marking correlation} (AWSOM)\\ 
Asymmetric languages greatly prefer single object markers.\\
Languages with multiple object markers are overwhelmingly symmetric.
\end{exe}
\noindent Despite this strong correlation, \citet{Riedel2009} is correct to claim that Sambaa is an exception: Sambaa appears as the only language allowing multiple object markers but being asymmetric (and doubling). A first question to answer before any explanation is sought, then, is whether Sambaa is a true counterexample to the AWSOM. As can be seen in examples \xxref{ex-vdwal:9}{ex-vdwal:11} the answer is ``yes'': any kind of Theme in Sambaa can only be object-marked in a ditransitive if the Benefactive/Recipient is object-marked first (comparable to Greek clitic doubling where the Theme can only be reached once the Benefactive is clitic-doubled, see \citealt{Anagnostopoulou2003, Anagnostopoulou2014}).
It is grammatical to object-mark only the Recipient (\ref{ex-vdwal:9b}), or both the Recipient and the Theme (\ref{ex-vdwal:9c}), but object marking just the Theme is ungrammatical (\ref{ex-vdwal:9d}) and (\ref{ex-vdwal:9e}).

\begin{exe} \settowidth\jamwidth{(*OM only for Th)}
\ex Sambaa (G23, \citealt[106]{Riedel2009})\label{ex-vdwal:9} 
	\xlist
	\ex[]{\label{ex-vdwal:9a}
		\gll N-za-nka ng’wana kitabu. \\ 
		\textsc{1sg.sm-pfv.dj}-give 1.child 7.book\\
		\glt `I gave the child a book.' \jambox{(no OM)}}
	\ex[]{\label{ex-vdwal:9b}
		\gll N-za-\textbf{m}-nka ng’wana kitabu. \\
		\textsc{1sg.sm1-pfv.dj}-1\textsc{om}-give 1.child 7.book\\
		\glt `I gave the child a book.' \jambox{(OM only for R)}}
	\ex[]{\label{ex-vdwal:9c}
		\gll N-za-\uline{chi}-\textbf{m}-nka ng’wana kitabu. \\
		\textsc{1sg.sm-pfv.dj}-7\textsc{om-1om}-give 1.child 7.book\\
		\glt `I gave the child a book.' \jambox{(OM for both)}}
	\ex[*]{\label{ex-vdwal:9d}
		\gll N-za-\uline{chi}-nka ng’wana kitabu. \\
		\textsc{1sg.sm-pfv.dj}-7\textsc{om}-give 1.child 7.book\\
		\glt Int: `I gave the child a book.' \jambox{(*OM only for Th)}}
	\ex[*]{\label{ex-vdwal:9e}
		\gll N-za-\uline{chi}-nka ng’wana. \\
		\textsc{1sg.sm-pfv.dj}-7\textsc{om}-give 1.child\\
		\glt Int: `I gave it to the child.' \jambox{(*OM for null Th)}}
	\endxlist
\end{exe} 
\noindent Since Sambaa prefers object marking for arguments high on the hierarchies of animacy and definiteness, one might suspect that the reason for the ungrammaticality of (\ref{ex-vdwal:9d}) and (\ref{ex-vdwal:9e}) lies not in the marking of the Theme, but the non-marking of the Recipient, i.e. the examples are out because the animate \textit{ng’wana} ‘child’ is not object-marked. However, even with reversed animacy the same pattern holds: animate and even human Themes cannot be marked by themselves in the presence of an inanimate Benefactive (also indicated as ``R'' below) -- the result is a reversal of the roles, as indicated in the translations of (\ref{ex-vdwal:10}) and (\ref{ex-vdwal:11}).\footnote{There appears to be a restriction on the ordering of multiple markers in Sambaa as well, see also \sectref{sec-vdwal:5} on prefix ordering.
\begin{exe}
\ex \settowidth\jamwidth{(*OM order R-Th)}
\gll *Wa-za-\uline{wa}-\textbf{zi}-ghul-iya.\\
\textsc{2sm-pst.dj-2om-10om}-buy-\textsc{appl}	\\\jambox{(*OM order R-Th)}
\end{exe}
\begin{exe} 
\ex \settowidth\jamwidth{(*order R-Th)}
\gll Wa-za-\uline{zi}-\textbf{wa}-ghul-iya.\\
\textsc{2sm-pst.dj-10om-2om}-buy-\textsc{appl}\\
\glt `They bought them (10, farms) for them (2, slaves).'\jambox{(order Th-R)}
*`They bought them (2, slaves) for them (10, farms).'\jambox{(*order R-Th)}
\end{exe}}

\begin{exe} \settowidth\jamwidth{(*OM for Th)}
\ex Sambaa (own data) \label{ex-vdwal:10} \newline
	\gll N-za-\textbf{jí}-ghúl-íyá nyumbá.\\
		\textsc{1sg.sm-pst.dj-5om}-buy-\textsc{appl} 9.house\\
	\glt *`I bought it for the house (a/the dog, class 5).' \jambox{(*OM for Th)}
	instead: `I bought a house for it (the dog).' \jambox{(OM for R)}
\end{exe}\largerpage[-2]

\begin{exe} \settowidth\jamwidth{(*OM for inanimate Th)}
\ex \label{ex-vdwal:11}
	\xlist
	\ex
		\gll Wá-zá-\textbf{zi}-ghul-iya \textbf{khói} \textbf{z-áwe} \uline{wátuunghwa}.\\
		\textsc{2sm-pst.dj-10om}-buy-\textsc{appl} 10.farm 10-\textsc{poss}.2 2.slaves\\
		\glt `They bought slaves for their farms.' \jambox{(OM for inanimate R)}
	\ex 
		\gll Wá-zá-\uline{wa}-ghul-iya \textbf{khói} \textbf{z-áwe} \uline{watúúnghwa}.\\
		\textsc{2sm-pst.dj-2om}-buy-\textsc{appl} 10.farm 10-\textsc{poss}.2 2.slaves\\
		\glt `They bought farms for the slaves.'      \jambox{(OM for human R)}
		*`They bought slaves for their farms.'  \jambox{(*OM for inanimate Th)}
	\endxlist
\end{exe}
Having established that the AWSOM correlation in (\ref{ex-vdwal:8}) is real, and that Sambaa escapes it, the research questions are:

\begin{enumerate}\label{vdw:researchquestions}
\item What causes the correlation between symmetry and multiple object marking?
\item How can we account for object marking in Sambaa?
\item Why is this parameter setting for object marking so apparently rare?
\end{enumerate}
In order to address these questions, I first lay out my assumptions about object marking as involving Agree with a defective Goal (largely taken from \citealt{Van_der_Wal2015}), and about verbal head movement in the clause.

\section{Agree and head movement}\label{sec-vdwal:4}

There are two key ingredients for the analysis. The first is that object marking involves an Agree relation, and the second is that verb-movement takes place in the lower part of the clause but stops just above little v.

With respect to the first, it might seem straightforward that object marking is some sort of agreement, but a longstanding debate for Bantu object marking concerns the question whether object marking involves syntactic agreement or pronoun incorporation, and how this may differ crosslinguistically (see for recent discussion on the status of object markers in Bantu, among others, \citealt{Henderson2006}; \citealt{Riedel2009}; \citealt{Zeller2012}; \citealt{Iorio2014}; \citealt{Baker2016}; and object clitics in general \citealt{Preminger2009}; \citealt{Nevins2011}; \citealt{Anagnostopoulou2014, Anagnostopoulou2016}; \citealt{Kramer2014}; \citealt{Harizanov2014}; \citealt{Baker_Kramer2016}). As an alternative to this choice, \citet{Roberts2010} proposes a hybrid account of clitics that \textit{always} involves an Agree relation between a Probe and a Goal \citep{Chomsky2000, Chomsky2001}. The Probe with an uninterpretable feature (uF) searches its c-command domain for valuation by the closest Goal with a matching interpretable feature (iF). Upon Agree, the features on the Goal are shared with the Probe (unlike \citetv{chapters/05-kalin}, I assume Agree to consist of simultaneous match and value).

\citet{Roberts2010} proposes that Goals can be defective, in the sense of having a subset of the features that are present on the Probe. In an Agree relation with a defective Goal, the Probe will contain the features of the Goal, and potentially additional features that the Probe does not share with the Goal (such as D or Person, though it does not need to be a proper subset). This makes the relation indistinguishable from a copy/movement chain, where normally only the highest copy is spelled out. The lower copy is not spelled out, due to chain-reduction \citep{Nunes2004}. This gives the impression of “incorporation” of the Goal, because its features will be spelled out on the Probe.

Concretely for object marking, this can be seen as follows. Little v has uninterpretable $\phi $ features (u$\phi$), which probe down to find an internal argument (object) with interpretable $\phi$ features (i$\phi$). If the object Goal is a defective pronoun (a $\phi$P, following \citealt{Dechaine_Wiltschko2002}), the Goal’s nominal features are a subset of the Probe’s (\figref{fig-vdwal:12}). When Agree is established, the $\phi$ features are spelled out on v in the form of an object marker.

\begin{figure}
	\caption{Left: Agree with a defective $\phi$P-Goal. Right: Spell-out of $\phi$ on v: object marker.\label{fig-vdwal:13}\label{fig-vdwal:12}}
\begin{minipage}{.5\textwidth}\centering
\forestset{pretty nice empty nodes/.style={
	for tree={
		calign=fixed edge angles,
		parent anchor=children,
		delay={if content={}{
				inner sep=0pt,
				edge path={\noexpand\path [\forestoption{edge}] (!u.parent anchor) -- (.children)\forestoption{edge label};}
			}{}}
	},
},}
		\begin{forest} for tree={align=center}
			[, pretty nice empty nodes
			[v\\{[}u$\phi$: \_ {]}, name=objectmarker]
			[VP[ ]
			[
			[V]  
			[$\phi$P\\{[}i$\phi$: class 8 {]}, name=Goal, inner sep=0pt]
			] ] ]				
			\draw[->, thick,overlay] (objectmarker) to [out=south,in=195] node[near start,left]{\textit{Agree}} (Goal);			
			%%%NOTE: this arrow doesn't attach to $\phi$P, but the feature structure underneath
	\end{forest}
	\end{minipage}\begin{minipage}{.5\textwidth}\centering
\forestset{pretty nice empty nodes/.style={
	for tree={
		calign=fixed edge angles,
		parent anchor=children,
		delay={if content={}{
				inner sep=0pt,
				edge path={\noexpand\path [\forestoption{edge}] (!u.parent anchor) -- (.children)\forestoption{edge label};}
			}{}}
	},
},}
		\begin{forest} for tree={align=center}
			[, pretty nice empty nodes
			[v [{[}$\phi$: 8{]}\\-bi-] 
			[v] ]
			[VP[ ]
			[
			[V]  
			[\sout{$\phi$P}\\ {[}\sout{i$\phi$: class 8}{]}]
			] ] ]				
		\end{forest}
\end{minipage} 
\end{figure}

\noindent Assuming with \citet{Roberts2010} that this Agree relation only spells out on the Probe if the Goal has a subset of the features on the Probe, this also implies that if the Goal’s features are \textit{not} a subset, the features will not be spelled out on the Probe.\footnote{This is the strongest hypothesis. A weaker version would claim that if the Goal is defective, the Probe has to be spelled out, and if the Goal is not defective, the features can still be spelled out on the Probe (but do not need to be) -- see also the discussion below on doubling object marking.} If the Goal is a full DP, the Probe simply Agrees with it, valuing u$\phi$, but only the DP spells out. This is illustrated in \figref{fig-vdwal:14}.\largerpage

\begin{figure} 
	\caption{Left: Agree with DP object\label{fig-vdwal:14}. Right: No spell-out of $\phi$ on v, but spell-out of DP.\label{fig-vdwal:15}}
	\begin{minipage}{.5\textwidth}\centering
	\forestset{nice empty nodes/.style={for tree={calign=fixed edge angles}, delay={where content={}{shape=coordinate, for current and siblings={anchor=north}}{}}
		},
	} 
		\begin{forest} for tree={align=center}
			[, nice empty nodes
			[v {[}u$\phi$: \_ {]}, name=objectmarker]
			[VP [ ]
			[
			[V]  
			[DP [{[} i$\phi${]}, roof, name=Goal] ]
			] ] ]				
			\draw[->, thick,overlay] (objectmarker) to [out=south,in=west] node[near start,left]{\textit{Agree}} (Goal);			
	\end{forest}
	\end{minipage}\begin{minipage}{.5\textwidth}\centering
	\forestset{nice empty nodes/.style={for tree={calign=fixed edge angles}, delay={where content={}{shape=coordinate, for current and siblings={anchor=north}}{}}
		},
	} 
		\begin{forest} for tree={align=center}
			[, nice empty nodes
			[v {[}$\phi$: 8{]}] 
			[VP [ ]
			[
			[V]  
			[DP,name=DP [{[}i$\phi${]}, roof]]
			] ] ]				
	\end{forest}
	\end{minipage}
\end{figure}

Object marking thus always involves an Agree relation, with the spell-out of the object marker depending on the structure of the Goal, resulting in incorporation effects.

Although it is not immediately relevant to the present discussion, I briefly discuss the difference between so-called doubling and non-doubling object marking here (\figref{fig-vdwal:16}). The analysis presented thus far accounts for languages that have non-doubling object marking, that is, the pronominal or anaphoric kind of OM, with a complementary distribution between OM and DP. However, in other languages object marking ``doubles'' the object DP, and both the OM and the DP are overtly realised (i.e. ``agreement''). The DPs that are object-marked in such languages are typically high in animacy, definiteness and givenness. As explained in more detail in \citet{Van_der_Wal2015}, I assume that animate/definite/given DPs have a Person feature (following \citealt{Richards2008, Richards2015}), which in these languages projects a separate PersonP layer (following \citealt{Hoehn2017}).  
Where in a non-doubling language v agrees with the DP, in a doubling language v agrees with the features in the Person layer, if present. Since these form a subset of the Probe, this Agree relation spells out as an object marker, while the DP also spells out, leading to doubling. I refer to \citet{Van_der_Wal2015} for further details.

\begin{figure}
	\caption{Left: Spell-out of $\phi$ on v: doubling object marker\label{fig-vdwal:16}. Right: No spell-out of $\phi$ on v, but spell-out of DP\label{fig-vdwal:17}.}
	\begin{minipage}{.5\textwidth}\centering
	 \forestset{pretty nice empty nodes/.style={
			for tree={
				calign=fixed edge angles,
				parent anchor=children,
				delay={if content={}{
						inner sep=0pt,
						edge path={\noexpand\path [\forestoption{edge}] (!u.parent anchor) -- (.children)\forestoption{edge label};}
					}{}}
			},
		},}
\begin{forest} for tree={align=center}, for tree= pretty nice empty nodes
			[
			[v \\{[}u$\phi$: \_ {]}, name=objectmarker]
			[VP [ ]
			[
			[V ]  				
			[PersP
			[{[}i$\phi${]}, name=Goal]  
			[DP,name=DP [{[}i$\phi${]}, roof]]
			] ] ] ] 	
			\draw[->, thick,overlay] (objectmarker) to [out=south,in=west] node[near start,left]{\textit{Agree}} (Goal);	
	\end{forest}
	\end{minipage}\begin{minipage}{.5\textwidth}\centering
	 \forestset{pretty nice empty nodes/.style={
			for tree={
				calign=fixed edge angles,
				parent anchor=children,
				delay={if content={}{
						inner sep=0pt,
						edge path={\noexpand\path [\forestoption{edge}] (!u.parent anchor) -- (.children)\forestoption{edge label};}
					}{}}
			},
		},}
		\begin{forest} for tree={align=center}, for tree= pretty nice empty nodes
			[
			[v \\{[}u$\phi$: \_ {]}, name=objectmarker]
			[VP [ ]
			[
			[V ]  				
			[DP,name=DP [{[}i$\phi${]}, roof, name=Goal]]
			] ] ] 	
			\draw[->, thick,overlay] (objectmarker) to [out=south,in=west] node[near start,left]{\textit{Agree}} (Goal);	
		\end{forest}
    \end{minipage}
\end{figure}

The second aspect needed in this anaysis is head movement in the lower part of the clause, but not all the way to T. There is good morphological evidence for this head movement in the Bantu languages, since verbal derivation is visible as suffixes on the verb. This verbal morphology provides clear clues as to its underlying syntax. Following \citet{Myers1990}, \citet{Julien2002}, \citet{Kinyalolo2003}, and \citet{Buell2005}, and drawing on the explanation in \citet{Van_der_Wal2009}, I assume that the verb starts out as a root in V and incorporates the derivational and inflectional \textit{suffixes} by head movement in the lower part of the clause. It then terminates in a position lower than T. The inflectional \textit{prefixes} on the verb (apart from the object marker) represent functional heads spelled out in their base positions. The (derived) verb stem and prefixes form one word by phonological merger.\largerpage

To illustrate this derivation, consider first the Makhuwa example in (\ref{ex-vdwal:15}) and the proposed derivation in \figref{fig-vdwal:18}. 


\begin{exe}
\ex Makhuwa (P31, \citealt[169]{Van_der_Wal2009})\label{ex-vdwal:15}\\ 
	\gll nlópwáná o-h-oón-íh-er-íyá epuluútsa.\\
	1.man 1\textsc{sm}-\textsc{perf}.\textsc{dj}-see-\textsc{caus}-\textsc{appl}-\textsc{pass}-\textsc{fv} 9.blouse\\
	\glt `The man was shown the blouse.'
\end{exe}

\begin{figure}
\caption{Proposed derivation of \REF{ex-vdwal:15}\label{fig-vdwal:18}}
		\begin{forest} for tree={calign=fixed edge angles, calign primary angle=-60,calign secondary angle=60}
			[AgrSP\footnotemark
			[o-] 
			[TP 
			[-h-] 
			[AspP
			[{[} {[} {[} {[} {[} -oon{]} \textsubscript{i}ih{]}\textsubscript{j}er{]}\textsubscript{k}iy{]}\textsubscript{m}a{]}]
			[vP
			[t]
			[PassP
			[t\textsubscript{m}]
			[ApplP
			[t\textsubscript{k}]
			[CausP
			[t\textsubscript{j}]
			[VP
			[t\textsubscript{i}]
			[epuluutsa
			] ] ] ]	] ] ] ] ]			
	\end{forest}
\end{figure}

The verb stem \textit{-oon-} ‘to see’, head-moves to CausP and incorporates the caus\-a\-tive morpheme to its left: \textit{-oon-ih-}. This combined head moves on to ApplP, incorporating a further suffix to its right: \textit{-oon-ih-er-}. The next step adds the passive morpheme to form \textit{ooniheriy} and this complex moves once more to add the final suffix (also known as ``final vowel''). Since it can carry inflectional meaning, the final suffix has been posited in an aspectual projection just above vP. Crucially, these are all suffixes, and they surface in reversed order of structural hierarchy (the Mirror Principle, \citealt{Baker1985, Baker1988}; and see among others \citealt{Alsina1999}; \citealt{Hyman2003}; \citealt{Good2005}; and \citealt{Muriungi2008} for discussion of the relation between semantic scope, morpheme order and syntactic structure).

There is no reason to assume that a moved head will first incorporate morphemes to its right (the extensions and final inflectional suffix) and then to its left (the agreement and TAM markers). Therefore, the fact that inflectional morphemes for subject marking, negation, and tense surface as prefixes suggests that these are not incorporated into the verb in the same way as the derivational suffixes, and thus that the verb has not head-moved further in the inflectional domain.\largerpage

The prefixes do form one phonological unit with the verb stem, but are posited as individual heads that merely undergo phonological merger. Another argument for this analysis is found in the order of the prefixes, which matches the order of the corresponding syntactic heads, as shown in (\ref{ex-vdwal:17}) and \figref{fig-vdwal:20}. If the inflectional prefixes were also incorporated, like the suffixes, one would expect them to surface in the opposite order. And this is indeed what we find in a language like French, where there is independent evidence that the verb does move to T: the inflectional morphemes appear in the reverse order of the Makhuwa inflectional prefixes, and they appear as suffixes on the verb in (\ref{ex-vdwal:19}).
\footnotetext{The node AgrSP is represented here merely for expository reasons -- the subject marker is treated as a reflex of $\phi$ features on T.}

\begin{exe}
		\ex Makhuwa (P31, \citealt[169]{Van_der_Wal2009})\label{ex-vdwal:17} \\
		\gll kha-mw-aa-tsúwéla.\\
		\textsc{neg}-\textsc{2pl}-\textsc{impf}-know\\
		\glt `You didn’t know.'
\end{exe}

\begin{figure}
	\caption{Bantu verbal prefixes as individual heads\label{fig-vdwal:20}}
		\begin{forest}
			[NegP
			[kha-] 
			[AgrSP
			[-mw-] 
			[TAM
			[-aa-]
			[AspP
			[-tsuwela\textsubscript{i}]
			[vP [t\textsubscript{i}, roof
			]
			] ] ] ]	]			
	\end{forest}
\end{figure}
	\begin{exe}
		\ex French \label{ex-vdwal:19} \\
		\gll Nous aim-er-i-ons.\\
		1\textsc{pl}.\textsc{pro} love-\textsc{irr}-\textsc{past}-1\textsc{pl}\\
		\glt `We would love.'
\end{exe} 
The verbal morphology thus provides empirical evidence for movement of the verb in the lower part of the clause to a position just outside of vP, with the prefixes spelled out in their individual positions in the inflectional domain. Theoretically I assume this head movement proceeds as proposed in \citet{Roberts2010}, involving an Agree relation between higher and lower heads in the clausal spine; see also \citet{Adger2003} and \citet{Bjorkman2011}, among others. The higher heads have additional features with respect to the lower verbal heads, which again makes the lower heads into defective Goals, spelling out the features on the highest head (AspP). See \citet{Roberts2010} for details.

Returning to the status of the object marker, in the verbal template it sits right between the derived verb stem and the inflectional prefixes -- nothing can intervene between the object marker and the verb stem. Despite its prefixal appearance, the object marker is different from the other prefixes such as the tense marker. The object marker and the verb stem still behave as one unit, together forming what is known in Bantu studies as the ``macrostem''. The macrostem is the relevant unit for tone assignment and further phonological rules; see \citet{Hyman2003,Hyman_et_al2008,Marlo2015}. The object markers are thus somehow special within the verbal morphology. I propose that this is because they are the result of spelling out a set of $\phi$ features that is present on little v and therefore on the (derived) verbal head, as outlined above.

With these basics in place, we can proceed to multiple object marking and a featural account of the AWSOM correlation.

\section{ Multiple object markers as additional low phi probes} \label{sec-vdwal:5}
In the current analysis, object marking is due to v agreeing with a defective Goal. The presence of the object marker is thus dependent on having a $\phi$ probe on v. Taking as a starting point that the distribution of $\phi$ features on functional heads is parameterised, the presence of multiple object markers is -- for most languages; see \sectref{sec-vdwal:6} -- hypothesised to reflect the presence of $\phi$ features on multiple functional heads. The most straightforward analysis is to postulate $\phi$ features on the actual heads that introduce the ``extra'' arguments, i.e. the applicative and causative heads, as represented in \figref{fig-vdwal:21}.\footnote{I assume \citet{pylkkanen2008}'s structure of double object constructions, involving an Applicative head. The analysis presented here should in principle be applicable to high and low Applicatives, as well as Causatives (see also \citealt{Van_der_Wal2017a, Van_der_Wal2017b}). In the tree structures, ``BEN'' (for benefactive) represents any argument introduced by Appl, which may also have a Recipient, Malefactive or other role.}

\begin{figure}
\caption{Multiple $\phi$ probes in a double object construction\label{fig-vdwal:21}}
		\forestset{nice empty nodes/.style={for tree={calign=fixed edge angles}, delay={where content={}{shape=coordinate, for current and siblings={anchor=north}}{}}
			},
		} 
		\begin{forest}	for tree={align=center}
			[, nice empty nodes
			[v \\{[}u$\phi${]}, name=varphi]
			[ApplP 
			[BEN, name=BEN, inner xsep=0pt]
			[
			[Appl\\{[}u$\phi${]}, name=varphi2] 
			[VP 
			[V]
			[TH, name=TH]
			] ] ] ]	
			\draw[->, thick] (varphi) to [out=south,in=west] (BEN);	
			\draw[->, thick] (varphi2) to [out=south,in=south,looseness=2] (TH);		
	\end{forest}
\end{figure} 
\noindent If a language has $\phi$ features not just on v but also on Appl, under a default downward probing, the prediction is that Appl agrees with the Theme/lower argument, and that the shared features are spelled out on the Probe (Appl) if the features of the Goal are a subset of the features of the Probe. The head movement of V through the lower part of the clause picks up the $\phi$ features of Appl and v, resulting in multiple sets of $\phi$ features on the derived head, and hence the potential for multiple object markers.\footnote{As mentioned, the lower functional heads themselves incorporate as suffixes in the course of the verb’s head movement, and the sets of $\phi$ features are located on this complex head, spelling out as prefixes to this head if the Goal is defective.}\largerpage[2]

With this analysis of multiple object marking we can thus answer the first research question of why there is such a strong correlation between multiple object marking and symmetry (the AWSOM): it follows from the presence of lower $\phi$ probes that the Theme is always accessible to a $\phi$ probe, independent of the marking of the Recipient/Benefactive. Appl will agree with the Theme and may or may not spell out its $\phi$ features as an object marker, depending on the structure of the Goal, and v will agree with the higher argument, which again may or may not spell out as an object marker. 

\noindent To illustrate how the analysis works, consider the patterns in Luganda. In all sentences in (\ref{ex-vdwal:21}), Appl agrees in $\phi $ features with the Theme \textit{ssente} ‘money’ and v agrees in $\phi$ features with the Recipient \textit{taata} ‘father’. Via head-movement of the verb these sets of $\phi$ features end up on the head just above v. In (\ref{ex-vdwal:21a}) the objects are non-defective DPs and they will simply be spelled out as DPs (no object marker). In (\ref{ex-vdwal:21b}) and (\ref{ex-vdwal:21c}) only one of the objects is a defective $\phi$P Goal whose $\phi$ features will be spelled out on the Probe, resulting in the one or the other object marker being present. Finally, in (\ref{ex-vdwal:21d}) both objects are defective and therefore spelled out on the Probe as object markers.\footnote{The encountered cross-Bantu variation in the precise number of object markers (one, two, three, more) allowed in any particular language \citep{Polak1986, Marlo2015} can potentially be understood as variation in the presence of $\phi$ features on other lower heads (e.g. high/low causatives).}

\begin{exe}
\ex	Luganda (JE15, \citealt[67, 72]{Ssekiryango2006}) \label{ex-vdwal:21}
	\xlist
	\ex \label{ex-vdwal:21a}
		\gll Maama a-wa-dde \textbf{taata} \uline{ssente}.\\
		1.mother \textsc{1sm}-give-\textsc{pfv} 1.father 10.money \\ 
		\glt `Mother has given father money.'
	\ex  \label{ex-vdwal:21b}
		\gll  Maama a-\textbf{mu}-wa-dde \uline{ssente}. \\
		1.mother \textsc{1sm}-\textsc{1om}-give-\textsc{pfv} 10.money. \\
		\glt `Mother has given him money.'
	\ex  \label{ex-vdwal:21c}
		\gll Maama a-\uline{zi}-wa-dde \textbf{taata}.\\
		1.mother \textsc{1sm}-\textsc{10om-}give-\textsc{pfv} 1.father\\
		\glt `Mother has given it father.'
	\ex  \label{ex-vdwal:21d}
		\gll Maama a-\uline{zi}-\textbf{mu}-wa-dde.\\
		1.mother \textsc{1sm}-\textsc{10om-1om-}give-\textsc{pfv}\\  
		\glt `Mother has given it to him.'
	\endxlist
\end{exe}
A further question that may be asked is what determines the order of object markers when multiple objects are defective. In Luganda, object markers are ordered strictly according to their semantic role (which may reflect the structural hierarchy): the Recipient is always closest to the stem, in mirrored order of the order of postverbal elements (cf. \citealt{Baker1985, Baker1988}), as illustrated in (\ref{ex-vdwal:22}).

\begin{exe}
\ex	Luganda (JE15, \citealt[13]{Ranero2015}) \label{ex-vdwal:22}
	\xlist
		\ex	
			\gll Omusajja y-a-\uline{zi}-\textbf{ba}-wa.\\
			1.man \textsc{1sm-pst-10om-2om}-give\\
			\glt `The man gave them it.'
		\ex 
			\gll *Omusajja y-a-\textbf{ba}-\uline{zi}-wa.\\
			1.man \textsc{1sm-pst-2om-10om}-give\\
			\glt int. `The man gave them it.'
	\endxlist
\end{exe}
Neither person (\ref{ex-vdwal:23}) nor animacy (\ref{ex-vdwal:24}) can change this ordering or make it ambiguous.

\begin{exe}
\ex	Luganda (Judith Nakayiza \& Saudah Namyalo, p.c.) \label{ex-vdwal:23}\\ 
\textit{Context: My assistant is ill and Judith is happy for me to work with hers. I tell a colleague:}\\
		\gll Judith a-\uline{mu}-\textbf{nj}-aziseemu olwaleero.\\
		1.Judith \textsc{1sm-1om-1sg.om}-lend day.of.today\\
		\glt `Judith lends him/her to me for the day.'\\
		*`Judith lends me to him/her for the day.'
\end{exe}

\begin{exe}
\ex Luganda (Judith Nakayiza \& Saudah Namyalo, p.c.)\label{ex-vdwal:24}
	\xlist
	\ex
		\gll N-a-gul-i-dde ennimiro abaddu.\\
		\textsc{1sg.sm-pst}-buy-\textsc{appl-pfv} 9.garden 2.slaves\\
		\glt `I bought slaves for the garden/farm.'
	\ex 
		\gll N-a-\uline{ba}-\textbf{gi}-gul-i-dde.\\
		\textsc{1sg.sm-pst-2om-9om}-buy-\textsc{appl-pfv}\\
		\glt `I bought them for it.' (not common)\\
		*`I bought it for them.'
	\ex 	
		\gll N-a-\uline{gi}-\textbf{ba}-gul-i-dde.\\
		\textsc{1sg.sm-pst-9om-2om}-buy-\textsc{appl-pfv}\\
		\glt `I bought it for them.'\\	
		*`I bought them for it.'
	\endxlist
\end{exe}
In other Bantu languages with multiple object markers, however, the ordering does not necessarily follow the thematic roles but is either determined by animacy, or free.\footnote{See also \citet{Bresnan_Moshi1990} and \citet{Alsina1996} on morpheme order in Chaga.} To illustrate the first, consider Kinyarwanda, where morpheme order is primarily based on person and animacy: when one prefix refers to a human, this needs to be closest to the stem (\ref{ex-vdwal:25}), and 1\textsuperscript{st}/2\textsuperscript{nd} person pronouns take precedence over other referents for the verb-adjacent position (\ref{ex-vdwal:26}).\footnote{Some form of person restriction for 1\textsuperscript{st} and 2\textsuperscript{nd} person objects in DOCs is commonly present in Bantu languages, but this extends beyond multiple object markers -- see \citet{Riedel2009} for discussion of the strong and weak PCC; see \citet{Yokoyama2016} for a featural account of the PCC and ordering restrictions in Kinyarwanda. Further literature on the order of object markers in various Bantu languages includes \citet{Duranti1979}, \citet{Bresnan_Moshi1990}, \citet{Rugemalira1993}, and \citet{Alsina1996}.} As expected, this strict ordering results in ambiguity.

\begin{exe}
\ex Kinyarwanda (JD62, \citealt[211, 212]{Zeller_Ngoboka2015}) \label{ex-vdwal:25}
	\xlist
	\ex[]{
		\glll Umwáarimú yeeretse Muhiíre inká. \\
		 u-mu-aarimu a-a-eerek-ye Muhiire i-n-ka \\
		\textsc{aug}-1-teacher \textsc{1sm}-\textsc{pst}-show-\textsc{asp} 1.Muhire  \textsc{aug}-9-cow \\
		\glt `The teacher showed Muhire the cow.'}
	\ex[]{
		\glll Umwáarimú yaayimwéeretse.\\
		 u-mu-aarimu a-a-a-\uline{yi}-\textbf{mu}-eerek-ye \\
		\textsc{aug}-1-teacher \textsc{1sm-pst-dj-9om}-\textsc{1om}-show-\textsc{asp} \\
		\glt `The teacher showed it to him.'}
	\ex[*]{\glll Umwáarimu yaamuyiyéeretse. \\
		 u-mu-aarimu a-a-a-\textbf{mu}-\uline{yi}-eerek-ye \\
		\textsc{aug}-1-teacher \textsc{1sm-pst-dj-1om-9om}-show-\textsc{asp} \\
		\glt `The teacher showed it to him.'}
	\endxlist
\end{exe}

\begin{exe}
\ex \label{ex-vdwal:26}
	\xlist
	\ex[]{
		\glll	 Umwáarimú yaabányeeretse. \\
		 u-mu-aarimu a-a-a-\textbf{\uline{ba}}-\textbf{\uline{n}}-eerek-ye\\
		\textsc{aug}-1-teacher \textsc{1sm-pst-dj-2om-1sg.om}-show-\textsc{asp}\\
		\glt `The teacher showed them to me.' \textit{or}\\
		`The teacher showed me to them.'}
	\ex[*]{ 
		\glll	Umwáarimú yaambéeretse.\\
		 u-mu-aarimu a-a-a-\textbf{\uline{n}}-\textbf{\uline{ba}}-eerek-ye\\
		\textsc{aug}-1-teacher \textsc{1sm-pst-dj-1sg.om-2om}-show-\textsc{asp}\\
		\glt int. `The teacher showed them to me/me to them.'}
	\endxlist
\end{exe}
In contrast, there is no strict ordering for multiple object markers referring to non-human referents, as shown in (\ref{ex-vdwal:27}), where the authors report that there is no semantic or pragmatic difference between (\ref{ex-vdwal:27b}) and (\ref{ex-vdwal:27c}). The sets of $\phi$ features gathered on the verbal head can thus be spelled out in either order.\largerpage

\begin{exe}
\ex	Kinyarwanda (JD62, \citealt[212]{Zeller_Ngoboka2015}) \label{ex-vdwal:27}
	\xlist
	\ex	\label{ex-vdwal:27a}
		\glll	 Yahaaye ingurube ibijuumba.\\
		 a-a-ha-ye i-n-gurube i-bi-juumba \\
		\textsc{1sm-pst}-give-\textsc{asp} \textsc{aug}-9-pig \textsc{aug}-8-sweet\_potatoes \\
		\glt `He has given the pig sweet potatoes.' 
	\ex \label{ex-vdwal:27b}
			 Yabiyíhaaye.\\
		\gll a-a-a-\uline{bi}-\textbf{yi}-ha-ye.\\
		\textsc{1sm-pst-dj-8om-9om}-give-\textsc{asp}\\
		\glt `He has given them to it.' 
	\ex \label{ex-vdwal:27c}
			 Yayibíhaye.\\
		\gll a-a-a-\textbf{yi}-\uline{bi}-ha-ye. \\
		\textsc{1sm-pst-dj-9om-8om}-give-\textsc{asp} \\
		\glt `He has given them to it.' 
	\endxlist
\end{exe}
Some varieties of Setswana seems to be even less restricted in the order of prefixes, generally allowing either order, as in (\ref{ex-vdwal:28}).\footnote{However, \citet{Pretorius_et_al2012} suspect that discourse preferences may be of influence here, and \citet{Creissels2002} notes for the variety he describes that the order is determined first by animacy, and in case the arguments are equal in animacy, then semantic role dictates the order of the markers.} 

\begin{exe}
\ex	Setswana \citep[247]{Marten_Kula2012} \label{ex-vdwal:28}
	\xlist
	\ex	
		\gll Ke-\textbf{mo}-\uline{e}-ape-ets-e.\\
		\textsc{1sg.sm-1om-9om}-cook-\textsc{appl-pfv}\\
		\glt `I cooked it for him/her.'
	\ex 
		\gll  Ke-\uline{e}-\textbf{mo}-ape-ets-e.\\
		\textsc{1sg.sm-9om-1om}-cook-\textsc{appl-pfv}\\
		\glt `I cooked it for him/her.'
	\endxlist
\end{exe}
It seems likely, then, that the sets of $\phi$ features on the verbal head are spelled out either freely, or according to a morphological template that prioritises referents higher on the scales of person and animacy, or thematic role (cf. \citealt{Duranti1979}). Further research is needed to establish the details in variation, what this may tell us about the syntax involved (if anything), and the spell-out or readjustment rules for morphology.

To summarise, in the proposed analysis object marking involves an Agree relation between a $\phi$ probe on a low functional head (v, Appl) and a DP Goal. If the features on the Goal are a subset of the Probe (e.g. a $\phi$P), the $\phi$ features on the Probe spell out as an object marker. Postulating $\phi$ probes on multiple lower functional heads (v, Appl, Caus) as the underlying structure in languages that allow multiple object markers derives the AWSOM correlation successfully and straightforwardly: the lower $\phi$ probe can \textit{always} agree with the Theme argument. The analysis also fits the larger typological implicational hierarchy for the distribution of $\phi$ features \citep{Moravcsik1974, Givon1976}: lower licensing heads only have $\phi$ features if higher heads do so too. If a language has u$\phi$ on Appl (multiple OM), it has u$\phi$ on v (object marking), and if a language has u$\phi$ on v, it has u$\phi$ on T (subject marking).\footnote{The implicational relation does not automatically follow from the analysis presented here, but see \citet{Van_der_WalTA} for a parameter hierarchy from which the implicational relation does fall out; reminiscent of the Final over Final Condition \citep{Sheehan_et_al2017}.} However, the analysis does not account for symmetry with a single object marker, nor for the Sambaa data. Symmetric single object marking is discussed in \sectref{sec-vdwal:7}, and the exceptional parameter setting of Sambaa (research question 2, page~\pageref{vdw:researchquestions}) is addressed in the following section. 

\section{Multiple object markers as additional higher \texorpdfstring{$\phi$}{\textphi} probes} \label{sec-vdwal:6}

Sambaa object marking came out as exceptional in allowing multiple object markers but being asymmetric. The hierarchical strictness in Sambaa multiple object marking suggests that the u$\phi$ features responsible for object marking are located \textit{above} the highest object, with the Minimal Link Condition determining that the highest object be agreed with first. The difference between Sambaa and asymmetric object marking in languages with only one object marker would thus be the presence of an extra set of $\phi$ features on v \citep{Adams2010}. If Sambaa indeed has two $\phi$ probes on v, then the first Probe finds the closest Goal (Benefactive) and agrees with it, after which the second Probe finds the lower Goal (Theme), forming a second Agree relation for $\phi$ features. Little v thus has two sets of valued $\phi$ features that can be spelled out as object markers (see \figref{fig:vdw:2}).

\begin{figure}
\caption{Two $\phi$ probes on v in Sambaa\label{fig:vdw:2}}
	\forestset{nice empty nodes/.style={for tree={calign=fixed edge angles}, delay={where content={}{shape=coordinate, for current and siblings={anchor=north}}{}}
			},
		} 
		\begin{forest}	for tree={align=center}
			[, nice empty nodes
			[v \\{[}u$\phi${]} {[}u$\phi${]},name=v]
			[ApplP 
			[BEN, name=BEN,inner xsep=0pt]
			[
			[Appl] 
			[VP 
			[V]
			[TH, name=TH]
			] ] ] ]	
			\draw[->, thick,overlay] (v) to [out=240,in=south] (TH);	
			\draw[->, thick] (v) to [out=300,in=west] (BEN);
			\node[below=\baselineskip of TH] (phantom) {};
	\end{forest}
\end{figure}
However, remember that the current model assumes that spell-out of the object marker is dependent on the featural make-up of the Goal relative to the Probe \citep{Roberts2010, Iorio2014, Van_der_Wal2015}: there is always an Agree relation, but only defective Goals will spell out as an object marker. This means that the two sets of $\phi$ features could be spelled out independently of each other, which is the case in symmetric multiple object marking languages, but not asymmetric Sambaa. This could be repaired by specifying a phonological condition that the second Probe can only be spelled out if the first is. This, however, is an ad-hoc solution that should only be adopted as a last resort. 

The question is thus why the second Probe can only reach the Theme if the first Probe agrees with a defective Goal. I propose that this follows from the nature of defective Goals: once the first Probe has agreed with a defective Recipient, the relation cannot be distinguished from a chain, and the bottom of a chain (i.e. a trace) is invisible for further agreement \citep{Chomsky2000, Chomsky2001}. This allows the second Probe to ``skip'' the invisible higher Benefactive argument and agree with the Theme, as represented in \figref{fig:vdw:3}a.\footnote{Remember that the $\phi$ probes in this analysis are underspecified and therefore do not differ from each other.}

\begin{figure}
\caption{Agree with a defective (a) and non-defective (b) Benefactive goal\label{fig:vdw:3}}
\begin{minipage}[t]{.5\textwidth}
a. \begin{forest}	for tree={align=center}
			[, nice empty nodes
			[v \\{[}u$\phi${]} {[}u$\phi${]} ]
			[ApplP 
			[BEN-$\phi$P, name=BEN]
			[
			[Appl] 
			[VP 
			[V]
			[TH, name=TH]
			] ] ] ]	
			\draw[->, thick,overlay] (-1.4,-2.4) to [out=south west,in=south] (TH);	
			\draw[->, thick,overlay] (-0.6,-2.4) to [out=south west,in=south west] (-0.2,-3.8);
	\end{forest}\end{minipage}\begin{minipage}[t]{.5\textwidth}
	b. \begin{forest}	for tree={align=center}
			[, nice empty nodes
			[v \\{[}u$\phi${]} {[}u$\phi${]} ]
			[ApplP 
			[BEN-DP, name=BEN]
			[
			[Appl] 
			[VP 
			[V]
			[TH, name=TH]
			] ] ] ]	
			\draw[->, thick,overlay] (-1.4,-2.4) to [out=south west,in=south] node[strike out,draw,-]{} (TH);	
			\draw[->, thick,overlay] (-1.4,-2.4) to [out=south west,in=south] (-0.2,-3.8);	
			\draw[-, thick ,overlay] (-0.6,-2.4) to [out=south west,in=south west] (-0.2,-3.9);
			\node[below=\baselineskip of TH] (phantom) {};			
	\end{forest}
\end{minipage}\end{figure}

If, on the other hand, the first Probe agrees with a non-defective DP Benefactive, the DP will still be visible to the second Probe. The second Probe will thus also agree with the higher Recipient and cannot reach the lower Theme, as in \figref{fig:vdw:3}b. The (double set of the same) $\phi$ features on v will not be spelled out, because the Goal is not defective, resulting in no object marking.

We may now wonder how the Theme is licensed if v does not agree with it in \figref{fig:vdw:3}b, and also how the second $\phi$ probe cannot reach past the Benefactive if that is already licensed by the first Probe. The question behind both points is whether the extra u$\phi$ set is also a Case licenser.\footnote{Assuming that Bantu languages need Case licensing, which is debated; see \citet{Diercks2012}, \citet{Van_der_Wal2015} and \citet{Sheehan-vdwaltap}. However, the debatable status mostly concerns nominative Case.} I argue that it is not, and that instead Appl is still a licenser. This is the same as in the case of symmetric languages, and asymmetric languages with only one object marker. That is, v and Appl are always licensers if they introduce an argument (contra \citealt{Woolford1995}), and the distribution of $\phi$ probes is logically independent of this. We have already seen this in the derivation for languages with only one object marker, where Appl licenses an object but only v has a $\phi$ probe.\footnote{Similarly, \citet{Bhatt2005} proposes for Hindi that both T and v are Case assigners, but only T has a $\phi$ probe.} This is represented in \figref{ex-vdwal:32}, where dashed lines indicate licensing and the solid line is $\phi$ agreement.

\begin{figure}[t]
	\caption{$\phi$ agreement independent of Case licensing\label{ex-vdwal:32}}
	\forestset{nice empty nodes/.style={for tree={calign=fixed edge angles}, delay={where content={}{shape=coordinate, for current and siblings={anchor=north}}{}}
			},
		}
		\begin{forest}	for tree={align=center}
			[, nice empty nodes
			[v \\{[}u$\phi${]} {[}Case{]} ]
			[ApplP 
			[BEN, name=BEN, inner sep=0pt]
			[
			[Appl\\ {[}Case{]}, name=Case2] 
			[VP 
			[V]
			[TH, name=TH]
			] ] ] ]	
			\draw[->, thick] (-1.4,-2.4) to [out=south,in=south west] (BEN);	
			\draw[->, thick, dotted] (-0.6,-2.4) to [out=south,in=west] (BEN);
			\draw[->, thick, dotted] (Case2) to [out=south,in=south] (TH);
	\end{forest}
\end{figure}

Recent theoretical proposals have highlighted mismatches between (morphological) case and $\phi$ agreement and shown them to be separate, as \citet{Bhatt2005}, \citet{Baker2008b,Baker2008a,Baker2012,Baker2015}, \citet{Bobaljik2008}, \citet{Barany2015}, \citet{StegovecTA} argue (contra \citealt{Chomsky2000, Chomsky2001} who views case and agreement as two sides of the same coin). Therefore, case and agreement “cannot be two realizations of the same abstract Agree relation” \citep[272 on Amharic]{Baker2012}. Baker takes this to be an argument in favour of morphological case not being determined by an Agree relation at all (instead following from a Dependent Case algorithm, \citealt{Marantz1991,Baker2015}), but it also points towards the independence of abstract Case and $\phi$ features \citep{Keine2010, Barany2015}. If u$\phi$ and Case are logically separate, then we can understand the unique situation of Sambaa. In all the other combinations of object marking parameters in \tabref{tab-vdwal:3} (page~\pageref{tab-vdwal:3}), Case and u$\phi$ operate together, and Case can be present by itself, but Sambaa (asymmetric multiple OM) presents the exceptional situation of a $\phi$ probe independent of a Case feature, as shown in \tabref{tab-vdwal:4}.{\interfootnotelinepenalty=10000\footnote{The symmetric single object marking type is discussed in \sectref{sec-vdwal:7}.}}\largerpage

\begin{table}
\caption{Featural distribution in 4 types of languages for symmetry and number of object markers} 
\label{tab-vdwal:4}
	\begin{tabularx}{\textwidth}{XXX} 
	\lsptoprule	
		symmetry & multiple & single\\
	\midrule
		asymmetric &  v:       Case-$\phi$ \fbox{+ $\phi$}
		
		Appl: Case & { v:       Case-$\phi$} 
		
		Appl: Case\\ 
	\midrule
		symmetric & { v:       Case-$\phi$} 
		
		Appl: Case-$\phi$ & { v: Case-$\phi$} 
		
		Appl: Case\\
	\lspbottomrule		
	\end{tabularx}
\end{table} 

The derivation in Sambaa thus proceeds as follows. First, Appl licenses the Theme (as in other languages). Second, assuming that $\phi$ and Case licensing go together as much as possible (as discussed below with regard to acquisition), then the first $\phi$ probe on v licenses Case and agrees for $\phi$ features, whereas the second Probe only concerns u$\phi$ features. It would thus be expected that this second Probe is not restricted to arguments that are ``active'' for [uCase] (see \citeauthor{Chomsky2001}'s \citeyear{Chomsky2001} ``Activity Condition''), but can agree with any set of $\phi$ features. This is why the second $\phi$ probe will still find the non-defective Benefactive DP, as in \figref{fig:vdw:3}b, even if the Goal is already licensed by the first [Case+$\phi$] Agree relation and no longer active for [uCase]. The only exception, as explained earlier, is when the Benefactive is a defective Goal (a $\phi$P). In this case, the Benefactive is not visible for the second $\phi$ probe, which can thus agree with the Theme as in \figref{fig:vdw:3}a. The result is two differently valued sets of $\phi$ features on v, which spell out as multiple object marking if the Theme is defective too.

With this analysis of a second $\phi$ probe on v, the second and third research questions can now be answered: Sambaa has multiple object marking because it has multiple sets of u$\phi$ features, and it is asymmetric because the second set of u$\phi$ features is located not on Appl but on v. Case licensing is still taken care of by both v and Appl, as in all other languages. This split between Case licensing and u$\phi$ features is rare, making Sambaa appear as an exception to the AWSOM correlation.

The rarity of the split between Case and $\phi$ can potentially be understood from the point of view of acquisition. In order to set parameters and to discover the uninterpretable features in their language, acquirers need a certain amount of clear form-meaning correlations (see a.o. \citealt{Biberauer2017a,Biberauer2017b,Biberauer_Roberts2017,Fasanella_Fortuny2016}). In Bantu languages, morphology forms a strong clue to deduce the underlying structure and features. The mismatch between observed $\phi$ agreement and Case licensing would thus appear to be suboptimal for easy acquisition, explaining the tendency for Case and $\phi$ agreement to go together. This line of reasoning makes testable predictions for acquisition (on which we have no data whatsoever), as well as relative diachronic instability (where a comparison between earlier sources such as \citealt{Roehl1911} and \citealt{Riedel2009} could have given a small amount of time-depth, but Roehl does not provide conclusive data). I leave this for further research.

\section{Two ways of being symmetric?} \label{sec-vdwal:7}

A final question that arises if we look again at \tabref{Table 3} from page~\pageref{Table 3} concerns the category of languages that are symmetric despite only allowing one object marker. Symmetry in these languages can theoretically be modeled in at least two ways. The first assumes that these languages work exactly like languages that have multiple object markers, but there is a PF condition preventing all but one set of $\phi$ features from being spelled out (cf. \citealt{Adams2010} for Zulu). The second proposes a flexible licensing by lower functional heads, allowing the one set of $\phi$ features on v to probe past the higher argument \citep{Haddican_Holmberg2012, Haddican_Holmberg2015,Van_der_Wal2017a, Holmberg_Et_AlTA}.\largerpage

The first model is problematic for passives because an asymmetry appears, even in otherwise symmetric languages, when passivisation and object marking are combined (see \citealt{Holmberg_Et_AlTA} and references therein). The sensitivity to animacy and topicality is another aspect that does not follow from a PF condition on multiple object markers (cf. \citealt{Zeller2012}). The second model looks promising, also in deriving other typological properties of double object constructions.{\interfootnotelinepenalty=10000\footnote{Specifically, flexibility of licensing can account for an asymmetry in the passive of otherwise symmetrical languages (\citealt{Holmberg_Et_AlTA}), and in explaining the RANDOM correlation (the Relation between Asymmetry and Non-Doubling Object Marking, \citealt{Van_der_Wal2017b}).} A full discussion of the analysis goes beyond the scope and space-limit of the current paper (see the mentioned references for details), but the essence is that heads such as Appl (which has no u$\phi$) can license downwards (Theme, \figref{fig:vdw:5}a) or upwards (Benefactive, \figref{fig:vdw:5}b), depending on the relative animacy and topicality of the two objects (see also \citetv{chapters/08-dalessandro} and \citetv{chapters/09-mursell}). This leaves the other argument, be that the Benefactive or the Theme, for licensing by and $\phi$ agreeing with v (in an active clause) or T (in a passive clause). The single set of u$\phi$ features on v can thus agree with either argument, depending on which argument is first licensed by Appl.\footnote{Note that u$\phi$ on v in symmetrical single-OM languages combines with Case, which is why the argument licensed by Appl is not a Goal for v.} This accounts for the symmetry found in single object marking languages \citep{Van_der_Wal2017a}.}

This implies that languages can have two ways to show symmetric object marking: either an extra set of $\phi$ features on Appl (multiple object marking), or flexible licensing by Appl (single object marking). Note that the presence of an extra $\phi$ probe in the former type does not exclude the presence of flexible licensing, though: Appl may have a $\phi$ probe and also flexible licensing. In fact, this is essential in the derivation of symmetric passivisation, since the presence of extra $\phi$ probes explains how the Theme may be object-marked but not how it can become the subject of a passive. This too I have to leave for future research.

\begin{figure}
\caption{(a): v agrees with TH (and can object-mark it). (b): v agrees with BEN (and can object-mark it)\label{fig:vdw:5}}
% % \label{ex-vdwal:33}
\begin{multicols}{2}\raggedcolumns
\forestset{nice empty nodes/.style={for tree={calign=fixed edge angles}, delay={where content={}{shape=coordinate, for current and siblings={anchor=north}}{}}
			},
		} 
		a.
		\begin{forest}	for tree={align=center}
			[vP, nice empty nodes
			[]
			[
			[v {[}u$\phi${]}, name=varphi]
			[ApplP 
			[BEN, name=BEN, inner ysep=0pt]
			[
			[Appl, name=Case2] 
			[VP 
			[V]
			[TH, name=TH]
			] ] ] ] ]
			\draw[overlay, ->, thick, dotted] (varphi) to [out=south,in=south west] (TH);
			\draw[overlay, ->, thick, dotted] (Case2) to [out=west,in=south] (BEN);
        \end{forest}\columnbreak
	b. 	
% % % 	\label{ex-vdwal:34}
		\begin{forest}	for tree={align=center}
			[vP, nice empty nodes
			[ ]
			[
			[v {[}u$\phi${]}, name=varphi]
			[ApplP 
			[BEN, name=BEN, inner xsep=0pt]
			[
			[Appl, name=Case2] 
			[VP 
			[V]
			[TH, name=TH]
			] ] ] ] ]
			\draw[overlay, ->, thick, dotted] (varphi) to [out=south,in=west] (BEN);
			\draw[overlay, ->, thick, dotted] (Case2) to [out=south west,in=south] (TH);
	\end{forest}
\end{multicols}
\end{figure}

\section{Conclusions and further research} \label{sec-vdwal:8}

Although there is a wealth of microvariation in Bantu object marking, this variation is not random and unconstrained. On the basis of data from more than 50 Bantu languages, the current paper shows that there is an almost-gap in the distribution of languages according to the number of object markers and double object symmetry: of the four logical combinations of parameter settings, three are common and one comes out as exceptional. This can be described as the AWSOM correlation, according to which asymmetry wants a single object marker. Both the AWSOM correlation and the exception of Sambaa can be understood in a model of syntax where the distribution of $\phi$ features over clausal heads is parameterised. Multiple object markers are indicative of additional sets of u$\phi$ features. In symmetric languages, these extra $\phi$ probes are located on lower functional heads such as Appl, whereas in asymmetric Sambaa the additional $\phi$ probe is present on v. 

This approach is in line with the Borer-Chomsky conjecture (BCC, \citealt{Borer1984,Chomsky1995,Baker2008b,Baker2008a}), which states that all parameters of variation are attributable to differences in the features of heads in the lexicon. This is an attractive Minimalist point of departure, as it allows us to keep basic syntactic operations the same across languages. Specifically for the current proposal: all object marking involves an Agree relation, and Agree is kept constant, whether a language shows symmetry or asymmetry, single or multiple object markers (and doubling or non-doubling object marking).

Under the BCC, further variation in the subparts of $\phi$ features, specifically Person features, is expected to play an important role in restrictions on combinations of 1\textsuperscript{st}/2\textsuperscript{nd} person objects in double object constructions (see footnote 12), but also in the “1+” type of language. It is striking that these languages allow a second object marker when the first is a reflexive or 1\textsuperscript{st} person -- that is, precisely in case the higher object can value all of the subfeatures [person [participant [speaker]]] \citep{Bejar_Rezac2009}. Further research will have to confirm whether 1+ object marking can be accounted for in a relativised probing account like \citet{Bejar_Rezac2009}, where the Probe renews if it is successful for all subfeatures in the first search (a “phoenix probe”).

Since there are about 500 Bantu languages and this paper covers only 10\% of them, the research should of course be extended to further Bantu languages and languages beyond the Bantu family to see how the AWSOM correlation and the proposed model fare for a broader set of languages.


\section*{Abbreviations and symbols}
Numbers refer to noun classes, or to persons when followed by \textsc{sg} or \textsc{pl}. High tones are marked by an acute accent, low tones are unmarked or marked by a grave accent.
\begin{multicols}{2}
		\begin{tabbing}
			\textsc{fs/fv}\hspace{5mm} \= final suffix/final vowel\kill
		\textsc{appl} 	\> applicative \\
		\textsc{asp} 	\> aspect \\ 	
		\textsc{aug} 	\> augment 	\\					
		\textsc{ben} 	\> Benefactive \\	
		\textsc{cj} 	\> conjoint verb form 	\\		 
		\textsc{conn} \> connective \\	
		\textsc{dem}	\> demonstrative \\				 
		\textsc{dj} 	\> disjoint verb form \\ 
		\textsc{fs/fv} 	\> final suffix/final vowel 	\\	 
		\textsc{om} 	\> object marker \\	
		\textsc{opt}	\> optative \\						 
		\textsc{pass} \> passive \\	
		\textsc{pfv} \> perfective 	\\				 
		\textsc{poss} \> possessive \\	
		\textsc{pres} 	\> present 	\\					 
		\textsc{prog} \> progressive \\	
		\textsc{pst} 	\> past tense 	\\				 
		R 			\> Recipient \\	
		\textsc{sm} 	\> subject marker \\				
		T 			\> tense \\	
		TH 				\> Theme 										
	\end{tabbing} 
\end{multicols}

\section*{Acknowledgements}
The research reported here is part of the ReCoS project, European Research Council Advanced Grant No. 269752 ``Rethinking Comparative Syntax'' at the University of Cambridge. I am grateful to Nikki Adams, Koen Bostoen, Jean Chavula, Thera Crane, Denis Creissels, Gloria Cocchi, Michael Diercks, Thabo Ditsele, Hannah Gibson, Peter Githinji, Chege Githiora, David Iorio, Carolyn Harford, Claire Halpert, Rev. Tumaini Kallaghe, Jekura U. Kavari, Heidi Kröger, Nancy Kula, Michael Marlo, Joyce Mbepera, Lutz Marten, Paul Murrell, Peter Muriungi, Jean Paul Ngoboka, Judith Nakayiza, Saudah Namyalo, Rodrigo Ranero, Patricia Schneider-Zioga, Justine Sikuku, Ron Simango, Oliver Stegen, Mattie Wechsler, Nobuko Yoneda, Jochen Zeller, the ReCoS team (András Bárány, Timothy Bazalgette, Theresa Biberauer, Alison Biggs, Georg Höhn, Anders Holmberg, Ian Roberts, Michelle Sheehan, and Sam Wolfe), two reviewers, and the audiences at SyntaxLab (Cambridge, March 2016), ACAL 47 (Berkeley, March 2016), CamCoS~5 (Cambridge, May 2016), LingLunch (MIT 2016), IGRA Leipzig 2017 and UConn 2017 for sharing and discussing thoughts and data with me. Any errors and misrepresentations are my responsibility.

{\sloppy\printbibliography[heading=subbibliography,notkeyword=this]}
\end{document}
