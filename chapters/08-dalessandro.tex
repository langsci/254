\documentclass[output=paper
,modfonts
,nonflat]{langsci/langscibook} 


\ChapterDOI{10.5281/zenodo.3541757}
\title{Agreement across the board:\newlineCover{} Topic agreement in Ripano}
\author{Roberta D'Alessandro\affiliation{Utrecht University}}

\abstract{Ripano, an Italo-Romance variety spoken in Ripatransone in central Italy, exhibits a number of phenomena that are quite rare among the Romance languages. It shows dedicated gender marking on the finite verb, unlike any other Romance language. The same variety also exhibits adverbial and prepositional agreement. Furthermore, whenever a topic is present in the clause it triggers agreement: finite verbs, non-finite verbs (participles and gerunds), adverbs, and prepositions all show agreement with the topic. 

Topic-driven agreement, I argue, is the result of two co-occurring factors: 1. the presence of an extra item, a $\varphi $-probe, in the lexical inventory of the language; 2. A special setup of this probe, which requires agreement with the topic. I also present some cross-linguistic evidence for this analysis.} 

\begin{document}
\maketitle 

\section{Agreement in Ripano}  \label{sec-dalessandro:1} 
\subsection{Introduction}\label{sec-dalessandro:1.1}
Ripano is the name of a dialect spoken in Ripatransone\footnote{Unless otherwise stated, the data from Ripatransone were collected on fieldwork by the author, in 2007. These agreement patterns are largely confirmed by a recent investigation by the Zurich Agreement Database project.}, a village in the province of Ascoli Piceno, in central Italy. Ripatransone is situated on an isogloss bundle separating central and upper-southern Italo-Romance varieties. The fact that Ripatransone is in a language-transition area has probably triggered the emergence of a number of phenomena that are quite unusual in the rest of Italo-Romance. One of them is agreement, which has often attracted the interest of dialectologists. This paper addresses agreement in Ripano: it aims to provide a comprehensive description and analysis of the agreement patterns found in this variety.

A disclaimer is in order at the very beginning: there are several studies on Ripano agreement within the DP, most notably the ongoing research by the Zurich Agreement Database group (in particular \citealt{Paciaroni_Loporcaro2018}). These studies are mostly concerned with morphological agreement within the DP, which we will not explore here. The present article instead discusses agreement within a clause, and more specifically:

\begin{enumerate}[label=\alph*.]
\item Adverbial agreement
\item Prepositional agreement
\item Gerund agreement
\item Argumental agreement
\end{enumerate}

\noindent Agreement will be considered at clause level, and not within a DP. The agreement phenomena in Ripano are all quite exceptional, both with respect to the rest of Romance and in general. I will argue that they can be attributed to the same underlying cause: the presence of an extra feature bundle that is topic-oriented, in a way which will become clear in the rest of the article.

Before investigating agreement relations, it is worth introducing the paradigm of the finite verb, given that it is peculiar in and of itself. Ripano is in fact known to all dialectologists in Italy because of the presence of gender inflection on finite verbs, as well as gerunds (\citealt{Egidi1965}; \citealt{Parrino1967}; \citealt{Luedtke1976}, \citealt{Mancini1988/1997,Mancini1993}; \citealt{Harder1998}; \citealt{Jones2001}; \citealt{Ledgeway2006}; \citealt{Rossi2008}; \citealt{Ferrari_Bridgers2010}). Finite verbs have a full masculine paradigm and a full feminine paradigm, as exemplified in \tabref{tab-dalessandro:1} for the present tense.

\begin{table}
\caption{Verbal endings \citep[31]{Rossi2008}\label{tab-dalessandro:1}}
\begin{tabular}{lll}  
\lsptoprule
    \textsc{masculine} & \textsc{feminine} &\\
\midrule	
i’ ridu & ìa ride & `I laugh'\\     	
tu ridu & tu ride &   `you laugh' \\   	
issu ridu  & esse ride & `he laughs/she laughs'\\  
noja ridemi & noja ridema & `we laugh'\\   	
voja rideti  & voja rideta &  `you laugh' \\  	
issi ridi  & essa ride &  `they laugh'   \\
	\lspbottomrule
\end{tabular}
\end{table}

Observe that gender is never marked on finite verbs in Romance. In Ripano, however, every finite verb is marked for gender, in the present, imperfect, future, past, subjunctive (present and past) and conditional (see \citealt{Rossi2008} for full paradigms). Ripano is \textit{pro}-drop.

Nouns/adjectives are also marked for number and gender, according to the paradigm in \tabref{tab-dalessandro:2}.

	\begin{table}
		\caption{Noun/adjective endings}
		\label{tab-dalessandro:2}
		\begin{tabular}{lll}
			\lsptoprule
			 & \textsc{singular} & \textsc{plural}\\
			\midrule
			\textsc{masculine} & -u / -ə & -i / -a\\
			\textsc{feminine} & -e (-a in modern Ripano) & -a\\
			\textsc{neuter/mismatch} & -a/ -ə\footnote{Throughout the paper, I will use the gloss N (‘neuter’) to refer to lexical neuter and \textsc{mm}  to indicate the agreement mismatch ending. At this stage, it is not clear whether the two categories are coincident; the \textit{-ə} marking the mismatch could be the result of a phonological reduction process, or it could signal the fact that the language resorts to a different gender (neuter). In order to avoid confusion, and especially to signal when agreement mismatch has arisen, I will use two different glosses. 
Observe furthermore that \textit{-ə} can also be used as a masculine marker. In some inversion contexts, a process of vowel reduction is at play, as argued by \citet{Paciaroni2017}. An example of this masculine \textit{-ə} can be found in (15).} & -a/-ə\\
			\lspbottomrule
		\end{tabular}
	\end{table}

Ripano is special not only because it displays agreement endings on lexical items that don’t usually show any agreement inflection in Romance, but also because of the agreement patterns it presents in verb--argument agreement, and for the choice of what agrees with what.

In what follows I will try to give a detailed outline of agreement patterns, first concentrating on the description of those elements that show agreement inflection in a way that is different from the rest of Romance, and then illustrating agreement patterns between these items and the agreement controller, to use a term that is common in typological studies.

The article is organized as follows: we will first look at \textit{what} shows agreement inflection (agreement \textsc{targets}), then at \textit{what agrees with what}, and then we will try to answer the questions \textit{how} and \textit{why} these strange agreement patterns happen in Ripano. 

\subsection{Agreement targets: Adverbs} \label{sec-dalessandro:1.2}
Adverbs in Romance are invariant. In Ripano, though, adverbs display agreement with what we will for the time being call “the most prominent element” in the clause. The adverbs that show agreement are not only the usual quantificational adverbs like \textit{quanto} (‘how/as much’) and \textit{molto} (‘much’) in Italian, but also manner adverbs and temporal adverbs, as well as degree, spatial and quantity adverbs; numerals are also inflected for gender and number.

An example of manner adverbial agreement is (\ref{ex-dalessandro:1}):

\begin{exe}
\ex\label{ex-dalessandro:1} \citet[8]{Burroni_Et_Al2016} \xlist
	\ex \label{ex-dalessandro:1a}
	\gll Iss-u    ha    rispost-u     mal-u.\\
	he-3\textsc{sg.m}  have.\textsc{3sg}  answer.\textsc{ptc-sg.m}  badly-\textsc{sg.m}\\
	\glt `He answered badly.' 
	\ex\label{ex-dalessandro:1b}
	\gll Ess-e    ha    rispost-e     mal-e.\\
	she-\textsc{3sg.f}  have.\textsc{3sg}  answer\textsc{.ptc-sg.f}  badly-\textsc{sg.f}\\
	\glt `She answered badly.' 
	\ex\label{ex-dalessandro:1c}
	\gll Iss-i    ha     rispost-i    mal-i.\\
	they-\textsc{3pl.m} have.\textsc{3pl} answer.\textsc{ptc-pl.m}  badly-\textsc{pl.m}\\
	\glt `They answered badly.' 
	\ex\label{ex-dalessandro:1d}
	\gll Iss-a    ha    rispost-a    mal-a.\\
	they-\textsc{3pl.f} have.\textsc{3pl} answer.\textsc{ptc-pl.f}  badly-\textsc{pl.f}\\
	\glt `They answered badly.' 
\endxlist
\end{exe}
As you can see, in (\ref{ex-dalessandro:1a}) the adverb agrees with the masculine singular subject, while in (\ref{ex-dalessandro:1b}) it agrees with the feminine singular subject. \xxref{ex-dalessandro:1c}{ex-dalessandro:1d} show agreement in gender and number with the corresponding plural subjects. While these are cases of straightforward agreement with the subject, it will be shown later on that Ripano actually shows agreement with topics. For the moment, it is sufficient to observe that adverbs inflect for number and gender and undergo agreement.

The same agreement patterns are usually found with degree adverbs (\ref{ex-dalessandro:2}), temporal adverbs (\ref{ex-dalessandro:3}), spatial adverbs (\ref{ex-dalessandro:4}), and quantity adverbs (\ref{ex-dalessandro:5}):
\begin{exe}
	\ex \label{ex-dalessandro:2} \citet{Ledgeway2012} 
	\xlist{\multicolsep=1ex\begin{multicols}{2}
	\ex 
	\gll  È   quaʃʃ-u    muort-u.\\
	is.3\textsc{sg} almost-\textsc{sg.m} dead-\textsc{sg.m}\\
	\glt `He is almost dead.' 
	\ex
	\gll  È   quaʃʃ-e     mort-e.\\
	is.3\textsc{sg} almost-\textsc{sg.f} dead-\textsc{sg.f}\\
	\glt `She is almost dead.'\end{multicols}}
	\endxlist
\end{exe}
\begin{exe}
	\ex \label{ex-dalessandro:3}
	\xlist
	\ex \label{ex-dalessandro:3a}
	\gll  Cə   vac-u    sembr-u.\\
	there  go.1\textsc{sg-m} always-\textsc{sg.m}\\
	\glt `I\textsc{\textsubscript{m}} always go there.' 
	\ex \label{ex-dalessandro:3b}
	\gll Cə  vach-e    sembr-e.\\
	there  go.1\textsc{sg-f} always-\textsc{sg.f}\\
	\glt `I\textsc{\textsubscript{f}} always go there.' 
	\endxlist
\end{exe}
\begin{exe}
	\ex\label{ex-dalessandro:4} \xlist
	\ex 
	\gll Ne   macchene   l’-è=mmist-u    sott-u.\\
	\textsc{sg.f} car.\textsc{sg.f} him-is.\textsc{3sg=}put-\textsc{sg.m} under-\textsc{sg.m}\\
	\glt `A car ran him over.' 
	\ex
	\gll  Ne   macchene   l’-è=mmist-e    sott-e.\\
	\textsc{sg.f} car.\textsc{sg.f} her-is.\textsc{3sg=}put-\textsc{sg.f} under-\textsc{sg.f}\\
	\glt `A car ran her over.'  
	\endxlist
\end{exe}
\begin{exe}
	\ex \label{ex-dalessandro:5}\citet[45--46]{Lambertelli2003} \xlist
	\ex 
	\gll  Esse   e   magnat-e   tand-e.\\
	she.\textsc{sg.f} is  eaten-\textsc{sg.f} much-\textsc{sg.f}\\
	\glt `She ate a lot.' 
	\ex
	\gll Issu   e  magnat-u   tand-u.\\
	he.\textsc{sg.m} is  eaten-\textsc{sg.m} much-\textsc{sg.m}\\
	\glt `He ate a lot.' 
	\endxlist
\end{exe}
Observe that while in \xxref{ex-dalessandro:1}{ex-dalessandro:3} and (\ref{ex-dalessandro:5}) the adverbs agree with the subject, in (\ref{ex-dalessandro:4}) they agree with the object clitic. 

\subsection{Agreement targets: Prepositions} \label{sec-dalessandro:1.3}
In Ripano, prepositions and prepositional adverbs can also display $\varphi$-features; when they do, they usually agree with their complement:

\begin{exe}
	\ex \label{ex-dalessandro:6}\citet[309]{Ledgeway2012} \xlist
	\ex 
	\gll Sottu      lu     tavulì \\
	under.\textsc{sg}.\textsc{m}   the.\textsc{sg}.\textsc{m}  coffee.table. \textsc{sg}.\textsc{m}\\
	\glt `under the coffee table' 
	\ex
	\gll  sotte     le     sedie\\
	under.\textsc{sg}.\textsc{f}  the.\textsc{sg}.\textsc{f}  chair.\textsc{sg}.\textsc{f}\\
	\glt `under the chair' 
	\endxlist
\end{exe}
\begin{exe}
	\ex \label{ex-dalessandro:7}\citet[54]{Lambertelli2003} \xlist
	\ex 
	\gll   è   bianghe   comm-e   n-e   spos-e.\\
	be.\textsc{3}  white.\textsc{sg.f} as-\textsc{sg.f}    a-\textsc{sg.f}  bride-\textsc{sg.f}\\
	\glt `She is white like a bride.' 
	\ex
	\gll  è   nir-u     comm-u   l-u     cherv-ò.\\
	be.3  black-\textsc{sg.m} like-\textsc{sg.m} the-\textsc{sg.m} coal-\textsc{sg.m}\\
	\glt `He is black like coal.' 
	\ex
	\gll  è  biang-a    comm-a  l  spos-a.\\
	be.3  white-\textsc{pl.f} as-\textsc{pl.f}    the  bride-\textsc{pl.f}\\
	\glt `They are white like brides.' 
	\ex
	\gll  è   nir-i     comm-i  l-i     cherv-ù.\\
	be.3   black-\textsc{pl.m} like-\textsc{pl.m} the-\textsc{pl.m} coal-\textsc{pl.m}\\
	\glt `They are black like coals.'
	\endxlist
\end{exe}
While most prepositions are invariable, the presence of this agreement pattern (for those prepositional adverbs that inflect) is widespread.

\subsection{Agreement targets: Nouns} \label{sec-dalessandro:1.4}
In Ripano, like in other Romance languages, lexical nouns are specified as masculine or feminine. However, in some contexts lexical nouns can change their gender specification. As an example take the word \textit{fame} ‘hunger’. \textit{Fame} is lexically feminine. It is listed as feminine in the dictionary, it is feminine when uttered on its own. Its default determiner is also feminine.\footnote{An anonymous reviewer asks whether it is really the case that \textit{fame} is feminine. We could instead assume that there is just a root and that agreement is not part of the nominal extended projection; so \textit{fam-} would be a root and agreement would determine gender. There are several reasons why I wouldn’t want to follow that path: the first is that all nominals require a gender specification in Romance; the second is that Romance nouns have invariable gender, with the exception of a very small number of nouns in Ripano. Nouns that change gender in Ripano only do so when they are bare. Postulating such a mechanism would require some sort of default agreement ending assignment in Romance, which is clearly not in place given that nouns come with one gender specification, feminine or masculine, in the lexicon, and do not change it ever (except in Ripano).} 
However, in some cases, feminine gender is replaced/overwritten by the gender of what seems the most prominent element in the clause (in this case, the subject). Recall that Ripano is \textit{pro}-drop, so we know the gender of the subject by looking at the inflectional ending of the verb.

\begin{exe}
	\ex \label{ex-dalessandro:8}\xlist
	\ex 
	\gll C’\footnotemark=aj-u     fam-u. \\
	\textsc{expl}=have-\textsc{1sg.m} hunger.\textsc{f-sg.m}\\
	\glt `I\textsc{\textsubscript{m}} am hungry.' 
	\ex
	\gll  C’=aj-e    fam-e.\\
	\textsc{expl}=have-\textsc{1sg.f} hunger.\textsc{f-sg.f}\\
	\glt `I\textsc{\textsubscript{f}} am hungry.' (lit. `I have hunger')
	\endxlist
\end{exe}
\footnotetext{This expletive has no particular function. It is a locative that usually co-occurs with
possessive \textit{have} in many Italo-Romance varieties, including spoken Italian (averci `\textit{to have}', instead of \textit{avere}). \textit{Ci} is a locative particle and does not have gender.}

In (\ref{ex-dalessandro:8}), the gender of the object varies depending on the gender of the subject. It should be noted that this agreement change on lexical nouns is found in a very limited number of cases, mostly very frequent expressions referring to bodily needs or psychological states like \textit{to be hungry, to be thirsty, to be scared, to be in a hurry} (all of which are expressed by means of transitive/possessive \textit{have} plus a DP complement). Furthermore, these nouns must be bare.

In what follows, nominal agreement will refer to nouns changing their gender, not to argumental agreement. 

\subsection{Agreement targets: Gerunds and infinitives} \label{ex-dalessandro:1.5}
Gerunds and infinitives do not inflect in Romance. In Ripano, they do. Here are some examples of gerund:

\begin{exe}
	\ex \label{ex-dalessandro:9}	\citet[43]{Lambertelli2003} \xlist
	\ex 
	\gll stieng-u     jenn-u. \\
	stay.\textsc{pres-1sg.m} going-\textsc{sg.m}\\
	\glt `I\textsc{\textsubscript{m}} am going.' (lit. `I stay going')
	\ex
	\gll  stiv-e     jenn-e.\\
	stay.\textsc{impf-sg.f} going-\textsc{sg.f}\\
	\glt `You\textsc{\textsubscript{f}} were going.'
	\endxlist
\end{exe}
Infinitives are also reported as inflecting by \citet{Mancini1993} and \citet{Ledgeway2012}. I couldn’t find any inflected infinitives during my fieldwork, and the inflected infinitive is not reported in other grammars. Here’s an example from Ledgeway:

\begin{exe}
\ex \label{ex-dalessandro:10} \citet[302]{Ledgeway2012}\\
\gll Sai   scriv-u?\\
can.\textsc{2sg} write.\textsc{inf-sg.m}\\
\glt `Can you write?'
\end{exe}
Non-finite forms thus show gender and number agreement in Ripano. It is plausible that the existence of the inflected infinitive is a matter of intra-speaker variation, or this is perhaps an archaic feature of the language. 

\section{Argumental agreement} \label{sec-dalessandro:2}
\subsection{Agreement mismatch with transitive verbs} \label{sec-dalessandro:2.1}
We have seen that some lexical items show gender and number inflection, unlike their corresponding forms in the rest of Romance, which cannot be inflected. Ripano also displays some unusual verb--argument agreement patterns.

Ripano is in fact known for exhibiting agreement mismatch patterns in transitive constructions, whereby the finite transitive verb, as well as the participle in the present perfect, show a dedicated ending in cases where the external argument and the internal argument display different gender or number \citep{D`Alessandro2017}. One such agreement pattern is in (\ref{ex-dalessandro:11}):

\begin{exe}
	\ex \label{ex-dalessandro:11}Ripano \citep[107]{Mancini1988/1997} \xlist
	\ex \label{ex-dalessandro:11a}
	\gll Babbu dic-ə     l-e     vərità. \\
	dad.\textsc{sg.m} say-\textsc{3sg.mm}     the-\textsc{sg.f}   truth.\textsc{sg.f}\\
	\glt `Dad tells the truth.' 
	\ex \label{ex-dalessandro:11b}
	\gll  So magnat-ə   l-u       panì.\\
	be.\textsc{1sg} eaten-\textsc{mm}\footnotemark {} the-\textsc{sg.m}   breadroll.\textsc{sg.m}\\
	\glt `I\textsc{\textsubscript{f}} have eaten the breadroll.'
	\endxlist
\end{exe}
\footnotetext{\textsc{mm} indicates an agreement mismatch marker. See footnote to \tabref{tab-dalessandro:2}.}
In (\ref{ex-dalessandro:11}) the ending \textit{-ə} signals gender mismatch between the subject and the object. Agreement with the subject in (\ref{ex-dalessandro:11a}) would have triggered the ending \textit{-u} to appear on the verb. Agreement with the object would have triggered the ending \textit{-e} to appear on the verb. Agreement with the subject in (\ref{ex-dalessandro:11b}) would have required the insertion of an \textit{-e} ending, for feminine; agreement with the object would have triggered the insertion of an \textit{-u}.

This system is attested in many old grammars, and has also been found during fieldwork. It is, however, restricted to SVO sentences, in which no obvious information structure is visible. Mismatch mostly arises when the two arguments have different gender, but it also often arises with different number, and between countable and uncountable arguments.

\subsection{What agrees with what: Topic-oriented agreement} \label{sec-dalessandro:2.2}
Ripano intransitive finite verbs follow the agreement patterns of the rest of Romance: the intransitive verb agrees with the subject.

We saw in the previous section that in transitive SVO declarative sentences the finite verb shows a mismatch agreement marker. This marking is limited to declarative clauses with canonical word order. One interesting feature that offers the beginning of an explanation of Ripano agreement is the observation that agreement mismatch disappears the moment one of the arguments is topicalized. 

Take for example sentence (\ref{ex-dalessandro:12}). 

\begin{exe}
\ex \label{ex-dalessandro:12}\citet[51]{Rossi2008}\\
\gll Issu   e   rott-ə     l-u     vitria  e   l-e     corb-e   l’-e     ddussat-e     a me.\\
he.\textsc{sg.m} is  broken-\textsc{n} the-\textsc{sg.m} glass.\textsc{sg.m.mass} and  the-\textsc{sg.f} fault-\textsc{sg.f} it.\textsc{sg.f-}be.3  attributed-\textsc{sg.f} to me\\
\glt `It was HIM who broke the glass, and now he is saying it’s my fault.'
\end{exe}
Both arguments in the first clause are masculine singular, but the object is a mass noun, ending in \textit{-a}. We hence find the expected agreement mismatch pattern (between countable and uncountable arguments). The second clause features agreement with a clitic left-dislocated object. Clitic-left dislocation characterizes topics in Italo-Romance; the presence of the clitic is what distinguishes a topic from a focus (see, for instance, \citealt{Cinque1983}). When an argument is left-dislocated, it usually lands in a left peripheral position preceding the finite verb. This position is often referred to as TopP, following \citegen{Rizzi1997} seminal work on the fine structure of the left periphery. 
(\ref{ex-dalessandro:12}) suggests that if a topicalized element is present in the clause, it will become the agreement controller, and the finite verb will agree with it, independent of sentence structure. Topic agreement overwrites other forms of agreement. This is not the case for focused elements, as shown by the following minimal pair:

\begin{exe} \settowidth\jamwidth{(New Information Focus)}
	\ex  \label{ex-dalessandro:13}\citet[9]{Paciaroni2017} \xlist
	\ex   Chi ride{?} 
	\glt `Who laughs{?}' 
	\ex
	\gll  Rid-ə     Gianni\\ 
	laughs-\textsc{3sg.m} Gianni.\textsc{sg.m}\\ \jambox{(New Information Focus)}
	\glt `Gianni is laughing.'
	\endxlist
\end{exe}
\begin{exe} \settowidth\jamwidth{(Topic)}
	\ex \label{ex-dalessandro:14} \citet[9]{Paciaroni2017} \xlist
	\ex   Che fa Gianni? 
	\glt `What does Gianni do?' 
	\ex
	\gll  Eh,   rid-u     Gianni.\\ 
	{ } laughs-\textsc{sg.m} Gianni.\textsc{sg.m}\\ \jambox{(Topic)}
	\glt `Eh, Gianni laughs.'
	\endxlist
\end{exe}
In (\ref{ex-dalessandro:14}) \textit{Gianni} is a known element, introduced in the previous clause. Following \citet{Frascarelli2012}, it is a Familiar Topic: “Familiar (Fam-) Topics constitute given information in the discourse context and are used either for topic continuity or to resume background information” \citep[181]{Frascarelli2012}. 

Finally, we also have examples of agreement with topics that are in situ objects:

\begin{exe}
	\ex \label{ex-dalessandro:15}	\citet[93]{Rossi2008}\\
	\gll L-u     petrò e   mannat-a l-e   disdett-e a lu cuntedì.\\
	the-\textsc{sg.m}  lord  be.3  sent-\textsc{sg.f} the-\textsc{sg.f} cancellation-\textsc{sg.f} to the farmer\\
	\glt `The owner sent the cancellation to the farmer.'
\end{exe}
The ending of the past participle is \textit{-a}, not \textit{-e}. We will return later in the paper to the different agreement endings and the contact-induced paradigm shift. In any case, there is no doubt that the past participle agrees with the object in situ. The choice of \textit{-a} instead of \textit{-e} might also be due to the fact that this sentence, in the archaic dialect, was probably uttered with an agreement mismatch marker, which is \textit{-ə}. Agreement is no longer with both arguments here, but with the most prominent, which is not the subject. 

Another example of topic-oriented agreement with an in situ object is in (\ref{ex-dalessandro:16}):

\begin{exe}
	\ex \label{ex-dalessandro:16}	\citet[140]{Rossi2008}\\
	\gll L-u   nonna   e   lasciat-a tutta   l-e   robb-e.\\
	the-\textsc{sg.m} grandpa  be.3  left-\textsc{f} all.\textsc{sg.f} the-\textsc{sg.f} things-\textsc{sg.f}\\
	\glt `Grandpa has left all his belongings.'
\end{exe}
In what follows, we will try to analyse these data as a unified phenomenon. First, a general introduction to the agreement system of upper-southern varieties will be provided. It will be shown that these varieties feature an extra $\varphi $-bundle of unvalued features, which dock on different hosts.

I will then outline an analysis that builds on \citet{Miyagawa2017}, who shows that discourse features enter agreement relations just like $\varphi$-features. Building on \citet{Miyagawa2017}, I will argue that in Ripano topics exhibit a $\delta$-feature (discourse feature). Furthermore, the finite verb and all other agreeing elements exhibit a feature bundle which includes both unvalued $\varphi $ features and unvalued $\delta$-features.  These feature sets are linked and features cannot probe on each separately. 

The agreement system of Ripano is similar to that found in Chichewa and other Bantu languages \citep{Bresnan_Mchombo1987} and partially recalls Dinka wh-agreement \citep{Van_Urk2015}.

\subsection{Some microvariational evidence} \label{sec-dalessandro:2.3}
While agreement mismatch and gender-marking on the finite verb is only found in Ripano, topic-oriented agreement is quite widespread in the northernmost part of the upper-southern language area. The dialect of Arielli, about 100\,km south of Ripatransone, shows omnivorous number agreement \citep{D`Alessandro_Roberts2010,D`Alessandro2017}. This means that the verb will agree with the plural argument, independent of whether it is a subject or an object. An example is given in (\ref{ex-dalessandro:17}):

\begin{exe} 
	\ex \label{ex-dalessandro:17} \xlist
	\ex   \label{ex-dalessandro:17a}
	\gll Seme   fitte   lu   pane.\\
	are.\textsc{pl} made.\textsc{pl.m} the.\textsc{sg.m} bred.\textsc{sg.m}\\
	\glt `We baked bread.' 
	\ex \label{ex-dalessandro:17b}
	\gll  So     f\textbf{i}tte   li         sagne.\\ 
	am.\textsc{1.sg} made.\textsc{pl} the.\textsc{pl} tagliatelle.\textsc{pl}\\ 
	\glt `I made tagliatelle.'
	\endxlist
\end{exe}
In (\ref{ex-dalessandro:17a}), the verb agrees with the plural subject of the transitive verb \textit{fa’} (‘do’); in (\ref{ex-dalessandro:17b}) it agrees with the plural object of the same verb. 

In Italo-Romance in general, the verb does not agree with an indirect object. Ariellese is no different in that respect. A sentence like (\ref{ex-dalessandro:18}), where the finite verb agrees with the indirect object, is ungrammatical:

\begin{exe}
\ex[*]{
\gll So   \textbf{mannite}   na lettere   \textbf{a} \textbf{quille}\\
am\textsc{.1sg} sent.\textsc{pl} a letter     to them\\
\glt `I sent a letter to them.'} \label{ex-dalessandro:18}
\end{exe}
If, however, the indirect object is topicalized through clitic left-dislocation, agreement is suddenly possible:

\begin{exe}
\ex Arielli\\
\gll A quille, je   so mannite   na   lettere\\
to them  them.\textsc{cl.dat} am sent.\textsc{pl} a.\textsc{sg.f} letter.\textsc{sg.f}\\
\glt `To them, I sent (them) a letter.'
\end{exe}
Topic-oriented agreement is spreading across the whole area. This is why this seems a promising starting point to explain the Ripano patterns. One important point is that Ripano is in clear decline with respect to Italian, which has spread in the area, as in the rest of Italy, over the last 60 years. Many of the agreement endings are alternating between an Italian and an original Ripano version. Furthermore, there are some morpho-phonological rules that apply within the DP, studied among others by \citet{Paciaroni_Loporcaro2018}, and that make its morphology less transparent. 

In what follows, I will try to provide a unified analysis of the agreement phenomena in Ripano.

\section{Agreement in upper-southern Italian varieties} \label{sec-dalessandro:3}
\subsection{A unified analysis} \label{sec-dalessandro:3.1}
The agreement system of upper-southern Italo-Romance varieties is seemingly quite complex. We can start from the general observation that we see agreement in contexts in which we don’t usually see it (i.e. adverbs and pronominal roots). In the case of Ripano, we need to ascertain at least two things. The first of these is what makes this agreement possible. This seems like quite a naïve question, the answer to which is ``some $\varphi $-features'', but if we wish to have a uniform theory of agreement we need to find evidence that these extra features exist.

The second issue is what is agreeing with what. We have already seen, in \sectref{sec-dalessandro:1.2}, conflicting evidence regarding what the adverbs are agreeing with. They seem to be able to agree both with the internal and with the external argument. This is an indication of the fact that agreement is not structure-driven, but rather information-structure driven. That this is the right approach is also suggested by the cross-dialectal data presented in \sectref{sec-dalessandro:2.2}.

To analyse the Ripano data I propose the following:
\begin{itemize}
	\item
		Agreement on adverbs, gerunds, and other lexical items that are usually uninflected in Romance is possible in Ripano because of an extra set of unvalued $\varphi $-features, which are visible in most Italian varieties in different forms.
	\item 
		Agreement, at least in Ripano, is information-driven. Specifically, for Ripano and neighbouring varieties, agreement is topic-oriented, i.e. driven by a $\delta $-feature on the Topic. 
\end{itemize}
In what follows, I will first show how the extra feature set which has been proposed by \citet{D`Alessandro2017} can account for these data. Then, I will show that agreement in Ripano is information-driven. Finally, I will propose a tentative unified analysis for argumental agreement as well as adverbial, verbal and prepositional agreement.

\subsection{An extra \texorpdfstring{$\varphi$}{\textphi}-set} \label{sec-dalessandro:3.2}
In the Minimalist Program \citep{Chomsky1995} and in much of the subsequent literature, features are considered to be either interpretable or uninterpretable\footnote{
I leave aside here the discussion of the relation between interpretability and valuation. For that, see \citet{Pesetsky_Torrego2007}, \citet{Zeijlstra2008} and many others.}, where interpretability needs to be understood roughly as “legible at the interfaces”. The only interface that actually counts for interpretability is LF, and much of the discussion on interpretability concerns \textit{semantic} interpretation (see, for instance, \citealt{Zeijlstra2008}). According to \citet{Zeijlstra2008} et seq., a formal feature that also has semantic content will be interpretable; a formal feature that does not have semantic content will be uninterpretable, and will therefore need to be eliminated before it reaches the interface with the C-I system.

Interpretability is crucially linked to the feature host: number is for instance interpretable on nouns but not on verbs. A verb carrying uninterpretable features must get rid of them before it reaches the interface, or the derivation will crash.

This formulation of interpretability stems from the early Minimalist Program, which had at its core a morphemic view of the Numeration items, inherited from the Government and Binding Theory era. Derivations applied to morphemes, elements moved to incorporate morphemes (for example, early MP made use of \citeauthor{Belletti1990}'s 1990 Generalized Verb Movement as a model). In contemporary views of syntax, uninterpretable features identify phase heads, and join the derivation in bundles, but work individually. Each feature probes by itself (but see Case, which is assigned under full Match/Agree according to \citealt{Chomsky2000}). The phonological realization of features is a matter which is defined post-syntactically, at PF. Morphemes no longer come into the picture, in narrow syntax. Nevertheless, the problem of hosting features has remained, and selection and mapping are more than ever proving a difficult issue.

One issue which remains unaddressed is what happens to a bundle of unvalued/uninterpretable features without a host. Selection is usually category-driven. For instance, a verb selects a DP as its complement. But what happens if we have a bundle of uninterpretable features that are acategorical and yet need to be merged in the derivation? What is it merged to? At which point does it join the derivation?

Where there is no obvious merging locus for a bundle of features, because they are acategorical, languages choose by themselves where to locate these elements. What I would like to propose here is that the merging locus of a bundle of unvalued features (which I call $\pi $ following \citealt{D`Alessandro2017}), is determined parametrically, as illustrated in \figref{ex-dalessandro:20}.

\begin{figure}[h]
	 \forestset{nice empty nodes/.style={for tree={calign=fixed edge angles}, delay={where content={}{shape=coordinate, for current and siblings={anchor=north}}{}}
		},
	} 
\caption{\label{ex-dalessandro:20} The locus of merger of $\pi$ (for a similar idea in terms of parametric distribution of phi features see \citealt{Van_der_WalTA}.)}
	\begin{forest} for tree={l sep=2\baselineskip,if n=1{edge label={node[midway,above,sloped]{no}}}{edge label={node[midway,above,sloped]{yes}}}}
	[Does the language have a $\pi$?, calign primary angle=-65,calign secondary angle=65
	[, nice empty nodes]
	[Can $\pi$ dock on a lexical item?, calign primary angle=-40,calign secondary angle=40
	[Can it be merged\\ with the vP?
	[Can it be merged\\ with the TP?
	[Can it be merged\\ with VP?
	[, nice empty nodes]
	[\textit{person-driven}\\ \textit{differential object marking}] ]
	[\textit{Subject clitics in}\\ \textit{northern Italian varieties}] ]
	[\textit{Person-oriented} \\ \textit{auxiliaries}] ]
	[\textit{Dock it on}\\ \textit{a suitable host}\\ \textit{(inflected adverbs,}\\ \textit{prepositions,}\\ \textit{gerunds)}] ]	
	] 
	\end{forest}
\end{figure}

In a recent article, \citet{D`Alessandro2017} shows that some subject clitics (those that are not pronominal in nature) in northern Italian varieties and person-ori\-ent\-ed auxiliaries in upper-southern varieties are two faces of the same coin: an extra uninterpretable $\varphi$ bundle, which is realized as a subject clitic in northern varieties and as an auxiliary root in southern varieties. We leave aside the cases of $\pi $-merger with functional projections in this paper, as well as the multiple occurrence of $\pi$, and concentrate on the cases in which $\pi$ docks on lexical items and on T. This picture is quite rare in Romance, but it seems to be exactly what is going on in Ripano, as suggested by the agreement patterns we saw in \sectref{sec-dalessandro:2}.

\section{Topic-oriented agreement}\label{sec-dalessandro:4}
\subsection{Directionality of agreement}\label{sec-dalessandro:4.1}
Theories of agreement come in many different forms. In general, one feature that characterizes all theories of agreement regards its structural dependency: what agrees with what depends on the structure in which the elements appear. One of the hottest debates of recent years regards the directionality of the Agree operation: according to Chomsky’s first formulation, Agree can be both probe-oriented and goal-oriented. In the early MP, agreement is linked to movement. The conditions for both to apply are the following:

\protectedex{
\begin{exe}
\ex \citet[257]{Chomsky1995}\\ $\alpha$ can target K only if: \xlist
\ex a feature of $\alpha $ is checked by the operation
\ex a feature of either $\alpha $ or K is checked by the operation
\ex the operation is a necessary step toward some later operation in which a feature of $\alpha $ will be checked
\endxlist
\end{exe}}
When Agree is first conceived as an operation, no precise direction is established for it. In (\citeyear{Chomsky2000}) with \textit{MI}, Chomsky dissociates movement from agreement, and establishes that Agree takes place under c-command (in short, downwards). According to \citet{Zeijlstra2012}, \citet{Bjorkman_ZeijlstraTA} and \citet{Wurmbrand2012, Wurmbrand2014, Wurmbrand2017}, however, this is wrong; data from negative concord, fake indexicals, and other phenomena show that Agree should take place “upwards”: not simply under a Spec-Head relation but in a reverse manner with respect to Chomsky’s Agree, with a Goal c-commanding a Probe. Agreement under c-command is assumed also by the \textit{feature sharing} model developed by \citet{Pesetsky_Torrego2007} and \citet{Preminger2012, Preminger2013, Preminger2014}.

Many other definitions of agreement have been proposed through the years, from \textit{Cyclic Agree} \citep{Bejar_Rezac2009}, which is substantially agreement à la Chomsky plus a re-projection that in practice allows probing upstairs, to bidirectional agreement (\citealt{Boskovic2007} et seq.). In general, it is safe to say that unless we are dealing with a special set of data that require special extra postulations, general theories of agreement select one direction and stick to it.

Directionality is of course closely related to structural relations between agreeing elements. What matters is, for instance, that the probe and the goal are in a c-command relation.

In general, given that the arguments of the verb occupy well-defined positions at the moment of Transfer, one can safely link argumental positions to agreement sites.

One systematic exception to this agreement taxonomy is the topic position. While topics have been mentioned with respect to agreement phenomena in various ways \citep{Bresnan_Mchombo1987, Lambrecht1981}, agreement with topics has rarely been discussed. Topic agreement is identified as a kind of agreement, for instance, by \citet{Miyagawa2017}, according to whom $\delta $-features (discourse features) behave like $\varphi $-features: they enter agreement, and trigger movement. $\delta $-features are, however, different and distinct from $\varphi $-features. According to Miyagawa, there are languages in which $\delta $-features and $\varphi $-features appear/are percolated to the same head (for instance, in Spanish), but there is no causality between topichood and $\varphi $-agreement. 

Miyagawa also discusses cases of $\varphi $-agreement with topics (most notably, the case of northern Italian varieties in (\ref{ex-dalessandro:22}).

\protectedex{
\begin{exe}
	\ex \label{ex-dalessandro:22}\citet[90]{Miyagawa2017}\\
\gll Gli   è   venut-o     dell-e     ragazz-e\\
\textsc{scl} is.2\textsc{sg} come.\textsc{prt-sg.m} some-\textsc{pl.f} girls-\textsc{pl.f}\\
\glt `Some girls have come.'
\end{exe}}
He analyses these data as involving movement of the subject to Spec,TP for $\varphi $-agreement reasons, which then results in agreement with the DP that has moved to Spec,TP, a topic position. If the DP subject stays in situ, the finite verb will fail to show agreement. Agreement is a consequence of movement, but there is no direct causality, for Miyagawa, between topichood and $\varphi $-agreement.

Regarding Romance, \citet{Jimenez_Fernandez2016} convincingly shows that \textit{pro} is licensed in Spanish only if co-referential with a Topic (an aboutness shift topic\footnote{
I have not carried out a finer-grained analysis of the kinds of topics involved in agreement in this paper. I intend to perform some fieldwork looking at intonation, but at the moment I only have written sources and insufficient recordings to be able to ascertain the different kinds of topics in the clauses.}). I take this to be also a form of agreement with the topic. 

An interesting proposal regarding agreement with wh- or topicalized phrases in Bantu comes from \citet{Carstens2005a}, who observes the following agreement patterns in Bantu (see also \citetv{chapters/07-van-der-wal}):

\begin{exe}
\ex \citet[220]{Carstens2005a} \xlist
\ex
\gll Bábo bíkulu   b-á-kás-íl-é     mwámí bíkí mu-mwílo?\\
	2that   2woman   \textsc{2sa-s}{}-give-\textsc{perf-fv} 1chief 8what  18-3village\\ 
\glt `What did those women give the chief in the village?'
\ex
\gll Bikí   bi-á-kás-íl-é   bábo bíkulu   mwámí   mu-mwílo?\\
	8what   \textsc{8ca-a-}give-\textsc{perf-fv} 2that 2woman 1chief 18-3village \\
\glt `What did those women give the chief in the village?'
\endxlist
\end{exe}
Carstens proposes a correlation between the presence of $\varphi $-features and the presence of an EPP on C, disentangling wh-agreement from movement. C has u$\varphi $-features, which enter Agree under c-command with a wh-phrase, which is subsequently moved to Spec,C. More precisely, she proposes the following generalization:

\begin{exe}
\ex Bantu $\varphi $\textsubscript{EPP}: u$\varphi $-features have EPP features, in Bantu \citep[222]{Carstens2005a}
\end{exe}
Similar reasoning is found in Van Urk’s work on agreement in Dinka. Like Carstens, \citet{Van_Urk2015} examines cases of C-agreement, and attributes them to the presence of $\varphi $-features on C (see also \citealt{Haegeman_Van_Koppen2012}). One important element of Van Urk’s analysis is the V2 status of Dinka, allowing C to agree with wh-elements and topics. The verb shows agreement with these elements because C is $\varphi $-rich. This richness allows for V2 and also for wh- and topic agreement with the verb, which c-commands these elements (at some point in the derivation). The right configuration is required for agreement of this sort.

We can construct an analysis of Ripano along the same lines, bearing in mind that Ripano is a Romance dialect, and as such has V-to-T for finite verbs. An interesting piece of data regards agreement in C when the verb moves there (for instance, in interrogatives): we will show (in \sectref{sec-dalessandro:4.6}) that there used to be agreement between wh- and C, much like in Dinka. This agreement has now almost completely disappeared, but it has been documented in older attestations.

In what follows, it will be shown that:

\begin{enumerate}
\item[1.]The finite verb, as well as the participle, agrees with the topic, be it preverbal or postverbal (\sectref{sec-dalessandro:4.2}).
\item[2.]Preverbal and postverbal foci do not trigger agreement (\sectref{sec-dalessandro:4.3}).
\item[3.]If no topic is identifiable in the clause and the sentence has the canonical SVO order, agreement mismatch emerges (\sectref{sec-dalessandro:4.4}).
\item[4.]Topic agreement overwrites the featural specification of DPs (\sectref{sec-dalessandro:4.5}).
\item[5.]In the archaic variety, there used to be wh- agreement with the topic. That is no longer the case (\sectref{sec-dalessandro:4.6}).
\item[6.]Elements that usually don’t inflect in Romance, like adverbs and gerunds, can inflect in Ripano (\sectref{sec-dalessandro:5}).
\end{enumerate}
I argue that all these facts can be explained if we assume that the extra $\varphi $-bundle on agreeing items comes with a Topic feature (a $\delta $-feature in Miyagawa’s terms) in Ripano. The features of this bundle (saying “I want to agree with the topic”) must be valued at once, all together, and overwrite any other agreement. The Topic holds a valued $\delta $-feature. Much like in \citeauthor{Carstens2005a}’ system, agreement with the Topic takes place because of this $\delta $-feature; the condition of simultaneous agreement with the whole bundle creates agreement with the Topic as a byproduct.

\noindent Let us first go back and examine the data more closely, and then see how they can be accounted for in \sectref{sec-dalessandro:5}. In this paper, we will follow the standard assumption that unvalued features are uninterpretable (but see  \citealt{Pesetsky_Torrego2007} for a different view). 

\subsection{Agreement with preverbal and postverbal topic}\label{sec-dalessandro:4.2}
Wherever the topic is located, the verb will agree with it. Consider the following sentences:\largerpage

\begin{exe}
	\ex\label{ex-dalessandro:25} \citet[71]{Rossi2008}\\
	\gll So   magnat-u   l-e     mənestr-e   də paste e cicia.\\
	am  eaten-\textsc{sg.m}  the-\textsc{sg.f} soup-\textsc{sg.f} of pasta and chickpeas\\ 
	\glt `I\textsc{\textsubscript{m}} ate a pasta and chickpea soup.'
\end{exe}
\begin{exe}
	\ex\label{ex-dalessandro:26} \citet[394]{Harder1998} \xlist{\multicolsep=1ex\begin{multicols}{2}
	\ex
	\gll  Io   tə     ved-u.\\
	I.\textsc{m}  you.2\textsc{sg}  see-\textsc{sg.m}\\ 
	\glt `I see you.'
	\ex
	\gll Io   və     ved-i.\\
	I\textsc{.m} you.2\textsc{pl} see-\textsc{pl.m}\\ 
	\glt `I see you\textsc{\textsubscript{pl}}.'\end{multicols}}
	\endxlist
\end{exe}
\begin{exe}
	\ex \label{ex-dalessandro:27}\citet[93]{Rossi2008}\\
	\gll L-u     petrò e   mannat-a l-e   disdett-e     a lu cuntedì.\\
	the-\textsc{sg.m}  lord  be.3  sent-\textsc{sg.f} the-\textsc{sg.f} cancellation-\textsc{sg.f} to the farmer\\ 
	\glt `The owner sent the cancellation to the farmer.'
\end{exe}
In (\ref{ex-dalessandro:25}), the subject is a topic, and both the copula and the participle agree with it (observe that in other Romance varieties the past participle does not agree with the subject; \citeauthor{Belletti2005}'s (2005) generalization, but see \citealt{D`Alessandro_Roberts2010}).

In (\ref{ex-dalessandro:26}), the verb agrees in gender and number with the cliticized object. This kind of agreement in only found in participles in the rest of Romance, and never on finite verbs. (\ref{ex-dalessandro:27}) illustrates instead the case of agreement with an object topic. 

\subsection{Focus does not trigger agreement} \label{sec-dalessandro:4.3}
Recall that focused elements do not trigger agreement in Ripano. Consider again this contrast, reported by \citet{Paciaroni2017}:

\begin{exe} \settowidth\jamwidth{(New Information Focus)}
	\ex \label{ex-dalessandro:28}\citet[9]{Paciaroni2017} \xlist
	\ex Chi ride?
	\glt Who laughs?'
	\ex
	\gll Rid-ə     Gianni. \\
	laughs-\textsc{3sg.mm} Gianni.\textsc{sg.m}\\ \jambox{(New Information Focus)}
	\glt `Gianni is laughing.'
	\endxlist
\end{exe}
\begin{exe} \settowidth\jamwidth{(Topic)}
	\ex \label{ex-dalessandro:29}\citet[9]{Paciaroni2017} \xlist
	\ex Che fa Gianni?
	\glt`What does Gianni do?'
	\ex
	\gll Eh,   rid-u     Gianni \\
	{} laughs-\textsc{sg.m} Gianni.\textsc{sg.m}\\ \jambox{(Topic)}
	\glt `Eh, Gianni laughs.'
	\endxlist
\end{exe}
The contrast between these two sentences is very neat; in both cases the subject is postverbal, but in (\ref{ex-dalessandro:28}) it is a New Information Focus \citep{Lambrecht1981,Cruschina2012,Frascarelli2007}, while in (\ref{ex-dalessandro:29}) it is a topic. Agreement does not take place simply via Agree, but has an extra component, linked to information structure.

Further evidence that Focus does not trigger agreement is offered again by \citet{Paciaroni2017}:

\begin{exe}
\ex
\gll Manga   N-U   FRIKÍ   a pagat-a/-ǝ/*-u  l-u    bijetta. \\
{not even}   a-\textsc{sg.m} boy.\textsc{sg.m} has paid-\textsc{n/mm/sg.m} the-\textsc{sg.m} ticket.\textsc{sg.m}\\
\glt `Not one single boy has paid the ticket.'
\end{exe}
Neither new information focus nor contrastive focus trigger agreement in Ripano.
\subsection{Agreement mismatch}\label{sec-dalessandro:4.4}
If no topic is identifiable, or no element is salient in the clause, agreement in transitive clauses features a mismatch ending \citep{D`Alessandro2017}. The data are illustrated in (\ref{ex-dalessandro:31}).

\begin{exe} 
	\ex \label{ex-dalessandro:31}\citet[107]{Mancini1988/1997} \xlist
	\ex 
	\gll Babbu   dicə   le   vərità.\\
	dad.\textsc{sg.m}   say.\textsc{3sg.mm} the.\textsc{sg.f}   truth.\textsc{sg.f}\\
	\glt`Dad tells the truth.'
	\ex
	\gll So   magnatə   lu     pani’.\\
	am   eaten.\textsc{mm}   the.\textsc{sg.m}   breadroll.\textsc{sg.m}\\ 
	\glt `I\textsc{\textsubscript{f}} have eaten the breadroll.'
	\endxlist
\end{exe}
We will start from the syntax of these clauses to outline an analysis for Topic-oriented agreement. This pattern will be analysed along the lines proposed in \citet{D`Alessandro2017}, which will be illustrated in detail in \sectref{sec-dalessandro:5}. \citet{D`Alessandro2017} argues for the existence of a Complex Probe in Ripano, a sort of scattered v head simultaneously probing for both arguments. This Complex Probe will receive its exponent at PF.\footnote{One reviewer asks what mechanism is at play at PF telling morphology that two or three heads constitute a complex Probe. I would like to argue that the mechanism is the same that is at work in inflectional languages when morphology has one exponent for more than one head (like, in Italian, \textit{-o} is the exponent for \textsc{1sg.pres.ind}).} If the feature values on the two heads that form the Complex Probe are conflicting, a reduced ending (\textit{-ə}) will be inserted at PF, signalling that there is feature mismatch between the arguments.  
\subsection{Agreement stacking} \label{sec-dalessandro:4.5}
Topic-oriented agreement overwrites the agreement ending of a noun. In Ripano, like in the rest of Romance, nouns are morphologically marked for number and gender. When agreement with a Topic is involved, it overwrites these endings, in what I wish to call \textsc{agreement stacking}.

\begin{exe} 
	\ex  \xlist
	\ex 
	\gll C'-aju     fam-u / set-u / furj-u.\\
	\textsc{expl}{}-have.\textsc{1sg.m}  hunger.\textsc{f-sg.m} {} thirst-\textsc{sg.m} {} hurry-\textsc{sg.m}\\
	\glt`I\textsc{\textsubscript{m}} am hungry/thirsty/in a hurry.'
	\ex
	\gll  C'-aje     fam-e / set-e / furj-e.\\
	\textsc{expl}{}-have.\textsc{1sg.f} hunger.\textsc{f-sg.f} {} thirst-\textsc{sg.f} {} hurry-\textsc{sg.f}\\ 
	\glt `I\textsc{\textsubscript{f}} am hungry/thirsty/in a hurry.'
	\endxlist
\end{exe}
Agreement stacking is particularly interesting as it offers an insight into the featural composition of lexical items. As stated above, the words \textit{hunger, hurry, thirst,} etc. are feminine. We can see this when they are uttered in isolation; they are also listed as feminine in the dictionary, and they are feminine in Italo-Romance languages. The fact that they inflect suggests that $\pi $ agreement with the Topic (in this case, the subject) overwrites the lexical ending. We can think of this process in two ways: the first is to posit some sort of embedded DP, along the lines of Case-stacking \citep{McCreight1988, Nordlinger1998, Merchant2006, Richards2013, Pesetsky2013}. The other is to assume that inflection works at head level (or at word level), not at phrase level. This second option correctly accounts for DP agreement stacking cases, as well as for PP and numeral phrases. There, agreement is clearly visible on the head: 

\begin{exe} 
	\ex \citet[30]{Lambertelli2003} \xlist
	{\multicolsep=1ex\begin{multicols}{2}\ex 
	\gll Sema   ott\footnotemark  {}  femmen-a.\\ 
	are.\textsc{pl.f} eight.\textsc{pl.f} woman-\textsc{pl.f}\\
	\footnotetext{The author elides the final vowel, probably because of phonological clash with the next vowel. The underlying vowel is arguably \textit{-a}.}
	\glt`We are eight women.'
	\ex
	\gll Semi     ott-i     maschia.\\
	are.\textsc{pl.m} eight-\textsc{pl.m} men.\textsc{pl.m}\\ 
	\glt `We are eight men.'\end{multicols}}
	\endxlist
\end{exe}
Agreement is thus merged directly with the word root, not at DP level.

\subsection{C-agreement in the archaic variety}\label{sec-dalessandro:4.6}
In the archaic version of the dialect, cases of agreement with wh- elements are attested. Some speakers still use these forms:

\begin{exe} 
	\ex Mancini (1993 in \citealt[4]{Ledgeway2006}) \xlist
	{\multicolsep=1ex\begin{multicols}{2}\ex 
	\gll Ndov-u   va?\\
	where-\textsc{sg.m} go\\
	\glt`Where are you\textsc{\textsubscript{m}} going?'
	\ex
	\gll   Ndov-i     va?\\
	where-\textsc{pl.m} go.3\\ 
	\glt `Where are they\textsc{\textsubscript{m}} going?'\end{multicols}}
	\endxlist
\end{exe}
\begin{exe} 
	\ex  \xlist
	{\multicolsep=1ex\begin{multicols}{2}\ex 
	\gll Komm-u   te     siend-u?\\
	how-\textsc{sg.m} you.\textsc{acc} feel-\textsc{sg.m}\\
	\glt `How do you\textsc{\textsubscript{sg.m}} feel?'
	\ex
	\gll Komme   te     siend-e? \\
	how-\textsc{sg.f} you.\textsc{acc} feel-\textsc{sg.f}\\ 
	\glt `How do you\textsc{\textsubscript{sg.f}} feel?'\end{multicols}}
	\endxlist
\end{exe}
\largerpage[-1]\pagebreak
\begin{exe} 
	\ex \citet[54]{Lambertelli2003} \xlist
	{\multicolsep=1ex\begin{multicols}{2}\ex 
	\gll Quand-u   cost-u?\\
	how.much-\textsc{sg.m}  cost-\textsc{3sg.m}\\
	\glt `How much does it\textsc{\textsubscript{sg.m}} cost?'
	\ex
	\gll Quand-i   custet-i? \\
	how.much-\textsc{pl.m}  cost-\textsc{2pl.m}\\ 
	\glt `How much do you\textsc{\textsubscript{pl.m}} cost?'
	\ex
	\gll Quand-e   cuost-e? \\
	how.much-\textsc{sg.f}  cost-\textsc{3sg.f}\\ 
	\glt `How much does it\textsc{\textsubscript{sg.f}} cost?'
	\ex
	\gll Quand-a   custet-a? \\
	how.much-\textsc{pl.f}  cost-\textsc{2pl.f}\\ 
	\glt `How much do you\textsc{\textsubscript{pl.f}} cost?'
	\end{multicols}}\endxlist
\end{exe}
\noindent These sentences are recognized as Ripano by most speakers, but are also no longer in use in the modern dialect. 
These data are of great value, as they suggest a situation similar to that found in Dinka or in various Bantu languages, with C featuring a $\varphi $-set. For Ripano, this was probably also the case historically. The Ripano system has now moved towards a more ``standard'' Romance system, exhibiting $\varphi $-features on T.
These data also show some sort of diachronic proof for feature inheritance. Ripano is not so exceptional (if compared with non-Romance languages).
What we know so far is that agreement takes place with topics in Ripano; that there used to be at least some form of C-agreement; and that foci do not trigger agreement.
Let us now turn to the analysis of the data, in \sectref{sec-dalessandro:5}.


\section{Agreeing with topics} \label{sec-dalessandro:5}
This section outlines an analysis of topic-oriented agreement. However, we will summarize first the analysis for declarative SVO sentences in Ripano that present agreement mismatch. We will then use that analysis as a basis for an account of topic-oriented agreement.
Recall that Ripano displays agreement and mismatch marking (\textit{-ə}) on the finite verb, as well as the participle, when the internal and external argument have conflicting feature values. \citet{D`Alessandro2017} proposes that this is due to two factors:

\begin{enumerate}
\item[1.]The presence of an extra unvalued feature bundle ($\pi $), which is present in all Italo-Romance varieties.
\item[2.]The fact that this $\pi $ forms a Complex Probe with v. 
\end{enumerate}
A Complex Probe is defined as follows:

\begin{exe}
\ex	Complex probe: Given two heads F\textsubscript{1} and F\textsubscript{2}, where F\textsubscript{1} immediately dominates F\textsubscript{2}, F\textsubscript{1} and F\textsubscript{2} constitute a complex probe if they share their $\varphi $-features and these $\varphi $-features are unvalued. \citep[24]{D`Alessandro2017}.
\end{exe}
In a sentence like (\ref{ex-dalessandro:38}), $\pi $ and v form a Complex Probe (meaning that they are a discontinuous head which will receive one exponent at PF). 

\begin{exe}
\ex \label{ex-dalessandro:38}Ripano \citep[107]{Mancini1988/1997}\\
\gll Babb-u   dicə     le   vərità\\
dad\textsc{-sg.m}     says.\textsc{3sg.n}   the.\textsc{sg.f} truth.\textsc{sg.f}\\
\glt `Dad tells the truth.'
\end{exe}
\noindent (\ref{ex-dalessandro:38}) is analysed as follows:

\begin{exe}
\ex (from \citealt[27]{D`Alessandro2017})\\\label{ex-dalessandro:39}
\begin{forest}
[TP
[T]
[$\pi$P
[$\pi $\textsubscript{[ug, un]}, name=pi]
[vP
[EA, name=EA]
[v
[v\textsubscript{[ug, un]}, name=v]
[VP
[V]
[IA, name=IA] ] ] ] ] ]
\draw[->, blue] (pi) to [out=south west,in=south] (EA);	
\draw[->] (v) to [out=south west,in=south] (IA);	
\node at (7,0) {PF};
\node [red] at (7,-3) {MISMATCH!};
\draw [->, blue] (6,-3.5) to (8, -3.5);
\node at (7,-4) {$\pi$\textsubscript{[}\textsc{\textsubscript{m.sg}}\textsubscript{]} + \textsubscript{[}\textsc{\textsubscript{f.sg]} }};
\node [draw] at (9, -3.5){-ə};
\draw [-, blue] (4,1) to (4,-7);
\end{forest}
\end{exe}

\noindent In (\ref{ex-dalessandro:39}), $\pi $ and v probe the internal and the external argument simultaneously. This happens in SVO sentences with no clearly marked topic. At PF, if the features are conflicting (like in the case of \ref{ex-dalessandro:41}), an agreement mismatch marker is inserted. What is also crucial is that the Complex Probe docks on T. In Ripano, like in the rest of Romance, and as proposed for Abruzzese in \citet{D`Alessandro_Roberts2010}, T also has an unvalued feature set. This set is overwritten by the values of the Complex Probe.\footnote{In \citet{D`Alessandro2017} $\pi $ incorporates on T, in a clitic-like incorporation. This kind of clitic-like docking is proposed because of a parallel between some auxiliary selection patterns and subject clitics. For the present purposes, it does not really matter how the Complex Probe docks on T. What matters is that its feature values overwrite those that would normally appear on T.} 
In the case of sentences with a Topic, I propose that $\pi $ includes an unvalued $\delta $-feature, which needs to be valued by the $\delta $-feature of the Topic. This feature combination ($\delta +\varphi $) overwrites other endings. Let us see how this happens in more detail.

\subsection{The topic is the external argument}\label{sec-dalessandro:5.1}
Topic agreement and $\varphi $-agreement are strictly linked in Ripano if, as we have argued, agreement is Topic-oriented. If an argument is not marked as Topic, agreement will follow the structural Agree pattern, with agreement mismatch or agreement of the finite verb with the subject, depending on the verb class. Following \citet{Miyagawa2017} I assume that topics involve a $\delta $-feature (discourse).\footnote{Observe that Miyagawa uses the shorthand $\delta $-features to refer to any discourse-related feature. Here, I use $\delta $-feature to refer to topics, but not to foci.} Miyagawa shows that in what he calls group-D languages, like Spanish, $\delta $-features and $\varphi $-features are both inherited by T from C. Ripano shows diachronic evidence for this kind of inheritance (see \sectref{sec-dalessandro:4.6}). 
I have proposed that in a sentence where a Topic is present $\pi $ has an extra $\delta $-feature. This $\delta $-feature, a discourse feature, forces the resolution of agreement at PF as “agree with the Topic”. \citet{D`Alessandro2017} presents the following data, which are left unexplained in that paper.

\begin{exe}
\judgewidth{*/$\#$}
\ex[]{\label{ex-dalessandro:40}
\gll \textit{pro} So rlavatə le camisce.\\
pro.1\textsc{sg.m} be.\textsc{1sg} washed.\textsc{n} the.\textsc{sg.f} shirt.\textsc{sg.f}\\
\glt `I\textsubscript{\textsc{m}} washed the shirt.'}
\ex[*/$\#$]{ 
	\gll \textit{pro} So rlavatu le chemisce.\\
	pro.\textsc{1sg.m} be.\textsc{1sg} washed.\textsc{sg.m} the.\textsc{sg.f} shirt.\textsc{sg.f}\\
	\glt `I\textsubscript{\textsc{m}} have washed the shirt.'} \label{ex-dalessandro:41}
\end{exe}
The sentence in (\ref{ex-dalessandro:41}) does not present a structural agreement mismatch, but shows the finite verb agreeing with the subject. If the subject is a Topic, this will mean that the $\pi $ bundle contains a $\delta $-feature, which is forced to agree with the Topic. 
It is crucial that $\pi $ in Ripano also includes an unvalued topic feature u$\delta $, which needs to be valued against a topic. When this probe enters agreement with a topic, the whole bundle will be valued according to the topic features. $\pi $ in Ripano is hence [u$\delta $, u\textsc{n,} u\textsc{g].} This is what happens in the case of (\ref{ex-dalessandro:41}).

Let us consider the sentence in (\ref{ex-dalessandro:42}), where the subject is clearly a topic:

\begin{exe}
	\ex\label{ex-dalessandro:42}\citet[86]{Rossi2008}\\
	\gll Tu   nghe mme   ti   pij-u     tropp-e   cunfidenz-e.\\
	you.\textsc{m}  with me  \textsc{refl} take-\textsc{sg.m} too.much-\textsc{sg.f} confidence-\textsc{sg.f}\\
	\glt `You take too much liberty with me.'
\end{exe}
The featural setup of the relevant elements is as follows:

\begin{exe}
\ex 
tu [\textbf{$\delta$}, N:\textsc{sg}, P:2, G:\textsc{m}]\\
T [T:pres, M:indicative, uN, uP]\\
$\pi $ [\textbf{u}\textbf{$\delta$}\textbf{, uN, uG}]\\
v  [uN, uG]
\end{exe}


	\begin{exe}
		\ex\label{ex-dalessandro:44}
			\resizebox{\linewidth}{!}{\begin{forest}
				[TP
				[T]
				[$\pi$P
				[$\pi $\textsubscript{[u$\delta$, ug, un]}, name=pi]
				[vP
				[tu\textsubscript{[$\delta$, 2sg]}, name=tu]
				[v
				[v\textsubscript{[ug, un]}, name=v]
				[VP
				[V]
				[IA, name=IA] ] ] ] ] ]
				\draw[->, blue] (pi) to [out=south west,in=south] (tu);	
				\draw[-] (pi) to (tu);	
				\draw[->] (v) to [out=south west,in=south] (IA);	
				\node at (7,0) {PF};
				\node [red] at (7,-3) {MISMATCH!};
				\draw [->, blue] (6,-3.5) to (8, -3.5);
				\node at (7,-4) {$\pi$\textsubscript{[}\textsc{\textsubscript{\textbf{\textsc{top}.} m.sg}}\textsubscript{]} + V\textsubscript{[}\textsc{\textsubscript{f.sg]} }};
				\node [draw] at (9, -3.5){\textsc{2sg.m}};
				\draw [-, blue] (4,1) to (4,-7);
		\end{forest}}%
	\end{exe} 
\noindent The relevant part of the derivation is illustrated in (\ref{ex-dalessandro:44}). In (\ref{ex-dalessandro:44}) the featural mismatch is resolved in terms of the Topic. If a Topic is present, the Complex Probe will agree with it. T also probes for the external argument, and its features are valued accordingly. Once the Complex Probe docks on T, however, its features will overwrite those of T (as happens normally in Ripano). The only difference between Topic-oriented agreement and argumental agreement mismatch is that the $\pi $-v complex will have a different resolution for the featural value: that of both arguments, if they have conflicting values and no information structure is available, or that of the Topic, if they have conflicting values and information structure is available.  
It is not obvious why a feature bundle containing a Topic $\delta $-feature should override the one signalling featural mismatch. One explanation might lie in the fact that the marker of agreement mismatch is \textit{{}-ə}, which is a neutralized vowel. Agreement with the Topic is instead agreement with a full(er) feature set, and has a more specific marker. This competition between two morphemes is probably won by the most specific marking by virtue of an Elsewhere Principle. 

\subsection{The topic is the internal argument}\label{sec-dalessandro:5.2}
Let us now look at the case in which the topic is an object in situ. 

Consider the examples in (\ref{ex-dalessandro:45}) and (\ref{ex-dalessandro:46}):

\begin{exe}
	\ex \label{ex-dalessandro:45}\citet[59]{Rossi2008}\\
	\gll C’-eviè   set-u     e   mə   so   fatt-a     n-e   bbəvut-e ...\\
	\textsc{expl-}had  thirst-\textsc{sg.m} and  \textsc{refl.1} am  made-\textsc{sg.f} a-\textsc{sg.f} drink-\textsc{sg.f}\\
	\glt `I was thirsty and I drank...'
\end{exe}
\begin{exe}
	\ex\label{ex-dalessandro:46}\citet[87]{Rossi2008}\\
	\gll L-u   preta     cunzacr-e     ll’-ostia.\\
	the-\textsc{sg.m} priest.\textsc{sg.m} consecrate-\textsc{3sg.f} the-host.\textsc{sg.f}\\
	\glt `The priest consecrates the Host.'
\end{exe}
In these examples there is agreement with the object in situ, which is a topic. The topic object exhibits a $\delta $-feature, which is c-commanded by everything else. Once again, the Complex Probe probes for both the external argument and the internal argument simultaneously. Since there is a $\delta $-feature on the object, the featural conflict at PF will be resolved in its favour, and the Complex Probe will result in an inflectional ending agreeing with the topic (the object, in this case).

\subsection{Feature spreading/vowel harmony?}\label{sec-dalessandro:5.3}
The featural setup of the Topic and the Complex Probe could be only one part of the story. We have convincing evidence that Topic-oriented agreement overwrites the featural values of T. It should be added that the situation is not as straightforward as has been presented here, as the inflectional endings of Ripano are shifting significantly through contact with Italian.
Contact with Italian has, for instance, caused the gender marking on some nouns to shift. Feminine singular nouns, marked with \textit{-e} in Ripano, sometimes appear with \textit{-a}, which is the Italian marker for feminine singular. Furthermore, some masculine nouns end in \textit{-a} in Ripano. While these nouns always take the masculine singular article \textit{lu}, the agreement they trigger when they are topicalized sometimes surfaces as \textit{-a,}  as shown in (\ref{ex-dalessandro:47}):

\begin{exe}
	\ex \label{ex-dalessandro:47}\citet[113]{Rossi2008}\\
	\gll Giggì   m’-è   data     ‘ne   bbòtte     de   ommeta senze    ccorge-s-a.\\
	Gigi.sg.\textsc{m} me-is  given.\textsc{sg.f}  a.\textsc{sg.f} blow.\textsc{sg.f} of  elbow.\textsc{sg.m} without  realizing-\textsc{refl-sg}\\
	\glt `Gigi has given me a blow with his elbow without realizing.'
\end{exe}
In (\ref{ex-dalessandro:47}) it is very difficult to understand what is agreeing with what. \textit{Data} is definitely feminine, but with a “modern”/Italian ending. It is agreeing with the object \textit{bbotte}. \textit{Ommeta} is masculine (\textit{lu ommeta)} but it has an \textit{-a} ending (probably a residue of a neuter/dual). The reflexive verb shows an \textit{-a} ending, as a result of the spreading from \textit{ommeta} or from \textit{botte}. In any case, we would expect a masculine there, but we find a feminine or, at best, a neuter/masculine. This piece of data shows, I think, that there is a sort of inflectional harmony going on in the language. While structural agreement explains most of the agreement patterns we find, we sometimes see a sort of spreading of the morphological ending rather than of the feature values. 
Spreading of a surface form, rather than copying of feature values, is a sort of agreement which has been postulated for adjectives, for instance, and is often called concord. It could be the case that Ripano agreement is shifting forward, because of contact with Italian.
\tabref{tab-dalessandro:3} shows the stages of agreement systems that we seem to find in Ripano.

\begin{table}
\righthyphenmin2
\caption{The stages of agreement systems in Ripano\label{tab-dalessandro:3}}
\begin{tabularx}{\textwidth}{X@{ }c@{ }X@{ }c@{ }X}
\lsptoprule
{Structural agreement}  & {>} & \multicolumn{1}{Q}{Topic-oriented agreement}  & {>}    & \multicolumn{1}{Q}{Vowel harmony and Concord}\\ 
\midrule
\textit{Stage A}. wh-agreement/ T agreement ($\varphi $-features on both C and T, agreement under closest c-command) >

\textit{Stage B}. argumental agreement ($\varphi $-features on T and $\pi $, no longer on C; agreement mismatch marking) &  & \textit{Stage C}. $\varphi $-features on T-v, but different resolution of agreement mismatch; the Topic wins; \newline argument structure is less relevant than information structure. &  & \textit{Stage D.} Feature spread from the Topic in a sort of adjectival agreement concord. It is not the featural value which is copied, however, but the morpheme/vowel itself.\\
\lspbottomrule
\end{tabularx}
\end{table}

\noindent Regarding Stage D: adjectival agreement/concord usually takes place within a given domain. In the Italian sentence in (\ref{ex-dalessandro:48}) the feminine singular of \textit{casa} spreads in all directions but only within the DP; the participle \textit{comprato} is masculine singular, and is not affected by concord. 

\begin{exe}
\ex\label{ex-dalessandro:48} Ho comprato [una bella casa nuova]
\end{exe}
In the case of Ripano, $\delta $ is present on the Topic head by definition, and therefore the topic agreement domain includes at least the head marked as Topic. In general, from sentences like (\ref{ex-dalessandro:45}) here repeated as (\ref{ex-dalessandro:49}), it seems safe to say that the spreading domain is the clause, although PPs constitute an exception, as will be shown in \sectref{sec-dalessandro:5.4}.

\protectedex{
\begin{exe}
	\ex\label{ex-dalessandro:49} \citet[59]{Rossi2008}\\
	\gll C’-eviè   set-u     e   mə   so   fatt-a     n-e   bbəvut-e ...\\
	\textsc{expl-}had  thirst-\textsc{sg.m} and  \textsc{refl.1} am  made-\textsc{sg.f} a-\textsc{sg.f} drink-\textsc{sg.f}\\
	\glt `I was thirsty and I drank...'
\end{exe}}
Recall that in archaic varieties it was possible to find wh-agreement as well (and Ripano, like all Italo-Romance varieties with some exceptions, has wh-move\-ment). For that grammar, the agreement domain was the entire clause. For modern Ripano, agreement spans from the VP to at least the TP. 

\begin{exe}
\ex C   \framebox{T  v  V}
\end{exe}
Ripano agreement is in continuous evolution; while the first stages of evolution were mainly due to syntax-internal reasons, the loss of $\varphi $-features in C and the creation of an extra $\varphi $-head ($\pi $), the last stage, spreading, should probably be attributed to intense contact with Italian. For a thorough discussion of Concord and agreement within the DP, as well as harmony, see \citet{Paciaroni2017}. 

\subsection{Adverbial agreement} \label{sec-dalessandro:5.4}
We have seen that adverbs can also show agreement in Ripano, as illustrated in example (\ref{ex-dalessandro:1}), here repeated as (\ref{ex-dalessandro:51}).

\begin{exe}
	\ex \label{ex-dalessandro:51}\citet[8]{Burroni_Et_Al2016} \xlist
	\ex 
	\gll Iss-u    ha    rispost-u     mal-u\\
	he-3\textsc{sg.m}  have.\textsc{3sg}  answer.\textsc{ptc-sg.m}  badly-\textsc{sg.m}\\
	\glt `He answered badly' 
	\ex
	\gll Ess-e    ha    rispost-e     mal-e\\
	she-\textsc{3sg.f}  have.\textsc{3sg}  answer\textsc{.ptc-sg.f}  badly-\textsc{sg.f}\\
	\glt `She answered badly' 
	\ex
	\gll Iss-i    ha     rispost-i    mal-i\\
	they-\textsc{3pl.m} have.\textsc{3pl} answer.\textsc{ptc-pl.m}  badly-\textsc{pl.m}\\
	\glt `They answered badly' 
	\ex \label{ex-dalessandro:51d}
	\gll Iss-a    ha    rispost-a    mal-a\\
	they-\textsc{3pl.f} have.\textsc{3pl} answer.\textsc{ptc-pl.f}  badly-\textsc{pl.f}\\
	\glt `They answered badly' 
	\endxlist
\end{exe}
(\ref{ex-dalessandro:51}) is an impressive example of agreement across-the board. Every element in the clause agrees with the topic.
This tells us two things. First, $\pi $ is not limited to only one instance, but can appear on several elements. Second, once again the value of $\pi $ overwrites the value of the lexical endings.
Let us see how adverbial agreement works. A sentence like (\ref{ex-dalessandro:51d}), for instance, is derived as follows, assuming once again that the feature spreading/Concord takes place within the TopP:


\begin{exe}
\ex The values of the external argument (topic) get copied onto all $\pi $s present in the clause\\
			\begin{forest} for tree={fit=tight}
				[TP
				[EA \textsubscript{\textbf{\textsc{m.sg}.\textbf{$\delta$}}}]
				[T
				[$\pi $+T\textsubscript{uN.uP;}\\\textsubscript{\textbf{u\textbf{$\delta$}.uN.uG}}, name=pi]
				[vP
				[\sout{EA} \textsubscript{\sout{\textsc{m.sg}.$\delta$}}, name=tu]
				[v
				[v\textsubscript{[uG.uN]}, name=v]
				[VP
				[V+$\pi$\textsubscript{uG.uN.u$\delta$}, name=V]
				[$\pi$+Adv\textsubscript{\textbf{uG.uN.u\textbf{$\delta$}}}] ] ] ] ] ] ]
				\draw[-, blue] (tu) to [out=south west,in=south] (pi);	
				\draw[-, blue] (tu) to [out=south west,in=south west] (V);	
				\draw[-, blue] (tu) to [out=south west,in=south] (v);	
		\end{forest}
\end{exe}
Once a topic has been identified, its feature values will be copied on all elements in the clause that host a $\pi $.

\subsection{Agreement within the PP} \label{sec-dalessandro:5.5}

The same mechanism accounts for agreement on gerunds and wh-elements (in the archaic variety). An interesting exception seems to be offered by prepositions; while it is often the case that PPs are inflected for gender and number, their Concord domain seems to be the PP itself. 
We have seen examples (\ref{ex-dalessandro:6}) and (\ref{ex-dalessandro:7}) (here repeated as \ref{ex-dalessandro:53}--\ref{ex-dalessandro:54}) where prepositions agree with their complement. From these examples, however, it is not clear whether agreement happens within the clause:

\begin{exe}
	\ex \label{ex-dalessandro:53}\citet[309]{Ledgeway2012} \xlist
	\ex 
	\gll Sottu      lu     tavulì \\
	under.\textsc{sg}.\textsc{m}   the.\textsc{sg}.\textsc{m}  coffee.table. \textsc{sg}.\textsc{m}\\
	\glt `under the coffee table' 
	\ex
	\gll  sotte     le     sedie\\
	under.\textsc{sg}.\textsc{f}  the.\textsc{sg}.\textsc{f}  chair.\textsc{sg}.\textsc{f}\\
	\glt `under the chair' 
	\endxlist

\ex\label{ex-dalessandro:54} \citet[54]{Lambertelli2003} \xlista
	\ex 
	\gll   è   bianghe   comm-e   n-e   spos-e.\\
	be.\textsc{3}  white-\textsc{sg.f} as-\textsc{sg.f}    a.\textsc{sg.f}  bride-\textsc{sg.f}\\
	\glt `She is white like a bride.' 
	\ex
	\gll  è   nir-u     comm-u   l-u     cherv-ò.\\
	be.3  black-\textsc{sg.m} like-\textsc{sg.m} the-\textsc{sg.m} coal-\textsc{sg.m}\\
	\glt `He is black like coal.' 
	\ex
	\gll  è  biang-a    comm-a  l  spos-a.\\
	be.3  white-\textsc{pl.f} as-\textsc{pl.f}    the  bride-\textsc{pl.f}\\
	\glt `They are white like brides.' 
	\ex
	\gll  è   nir-i     comm-i  l-i     cherv-ù.\\
	be.3   black-\textsc{pl.m} like-\textsc{pl.m} the-\textsc{pl.m} coal-\textsc{pl.m}\\
	\glt `They are black like coals.'
	\endxlista
\end{exe}
The examples in (\ref{ex-dalessandro:54}) are especially unfortunate, as there is no mismatch between the features of the subject and those of the complement of the prepositional adverb. That PPs constitute their own agreement domain is shown, however, by sentences like the following:

\begin{exe}
\ex \citet[78]{Lambertelli2003}\\
	\gll Comma   sembra,   er-e     chepit-e     d-e   sol-e     chǝ sǝ   vǝliev-u     scallà     n-e   occ-e     dǝ   vì dop-a   magnat-a\\
	as always   be.\textsc{impf-3.f} understood-\textsc{sg.f} of-\textsc{sg.f}   alone-\textsc{sg.f} that \textsc{refl} wanted.\textsc{impf-3sg.m} warm.\textsc{inf} a-\textsc{sg.f} drop-\textsc{sg.f} of   wine.\textsc{sg.m} after-\textsc{n}   eating-\textsc{n}\\
	\glt `As usual, she had understood that he wanted to warm up some wine for himself after dinner.'
\end{exe}
This sentence is a masterpiece of Ripano agreement. The prepositional adverb \textit{comma} (‘like’) agrees with \textit{sembra} (‘always’), which is found here in its basic citation form, although it can inflect (see example \ref{ex-dalessandro:3}). The subject is feminine (it is a woman called \textit{Cucumm’le}, in the story), and it triggers agreement on the auxiliary, on the participle, and on the adpositional noun. The finite verb \textit{vǝlievu} agrees with the subject (\textit{Giuseppe}, a man, in the story). The object is feminine and does not trigger agreement with anything. The prepositional adverb \textit{dopa} agrees with \textit{magnata}, a “neuter” form. We know that \textit{dopa} can inflect from the existence of sentences like (\ref{ex-dalessandro:56}), mentioned in \citet{Ledgeway2012}.

\begin{exe}
	\ex\label{ex-dalessandro:56} \citet[309]{Ledgeway2012} \xlist
	\ex 
	\gll dop-u     l-u     ddì\\
	after-\textsc{sg.m} the-\textsc{sg.m} day.\textsc{sg.m}\\
	\glt `after the day' 
	\ex
	\gll   dop-e     l-e     nott-e\\
	after-\textsc{sg.f} the-\textsc{sg.f} night-\textsc{sg.f}\\
	\glt `after the night' 
	\endxlist
\end{exe}
Prepositions thus probe within their own domain, and tend to agree with their complement.

\section{Conclusions} \label{sec-dalessandro:6}
This paper has presented some novel data on an agreement pattern found in Ripano, a dialect spoken in central Italy. In Ripano, lexical items that are usually invariable in Romance show inflection. These elements tend to agree with the topic, within the clause.
Ripano also shows peculiar argument agreement patterns: the verb does not agree with a fixed argument depending on the verbal class, but with the topic. If no topic/most prominent element in the clause is identifiable, the verb shows an agreement mismatch ending in transitive sentences, and with the subject in intransitive sentences.
These agreement facts have been accounted for by showing that, like in other Italo-Romance varieties, Ripano exhibits lexical items made up of pure $\varphi $-feature bundles, which in the case of this particular language can dock on many elements in the clause (gerunds, prepositions, wh-elements, finite verbs). The existence of this extra probe, which also contains an unvalued topic feature forcing it to agree with the topic, creates these surprising (for Romance, at least) agreement patterns.

\section*{Acknowledgements}
This article is dedicated to the memory of Mr Antonio Giannetti, whose help with Ripano data was invaluable. 
I would like to take the opportunity to thank all the people who commented on this paper, in particular Hedde Zeijlstra, Sandhya Sundaresan, Jan Casalicchio, and the participants of the Frankfurt Agreement Workshop. I wish to thank two anonymous reviewers for their excellent feedback and advice.

The research leading to these results has received funding from the European Research Council under the European Union's \textit{Horizon 2020 Research and Innovation Program}\slash ERC Grant Agreement n. 681959 [Microcontact].

{\sloppy\printbibliography[heading=subbibliography,notkeyword=this]}
\end{document}
