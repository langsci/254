\documentclass[output=paper
,modfonts
,nonflat]{langsci/langscibook} 


\title{Long distance agreement and locality: A reprojection approach}

\author{Kristin Börjesson\affiliation{Universität Leipzig}\and Gereon Müller\affiliation{Universität Leipzig}} 

\abstract{Based on the assumption that all cases of long distance
	agreement  should be analyzed as involving only local agreement,
	and based on the observation that all existing approaches to long-distance agreement in
	terms of local agreement face substantial empirical and/or conceptual
	problems, the present paper sets out to develop a radically new
	approach to long-distance agreement. We suggest that long-distance agreement can and should be analyzed as a
	strictly local operation taking place early in the derivation, and
	giving rise to a counter-bleeding effect (i.e., apparent non-locality)
	later in the derivation as a consequence of regular syntactic
	structure building. More specifically, we argue that long-distance agreement involves (a)
	(what will become) a matrix verb V\sub{1} which enters the syntactic
	derivation as part of a complex predicate V\sub{1}-V\sub{2} that is merged with
	the embedded internal argument, agreeing with it locally early in the
	derivation; and (b) subsequent reprojection movement of V\sub{1} out of
	V\sub{2}'s clause, which eventually produces a biclausal structure (and
	thereby leads to a counter-bleeding effect with agreement). Empirical
	evidence for the new approach mainly comes from Nakh-Daghestanian
	languages, among them Hinuq, Khwarshi, and Tsakhur.
}

\begin{document}
	
	\maketitle
	
	\newcommand{\refprefix}{ex:mueller:}
	\section{Introduction} \label{sec-bjoe-muel:1}
	
	Long-distance agreement is a phenomenon where agreement seems to take
	place in a non-local configuration, that is, across a clause boundary.
	More specifically, in cases of long-distance agreement, the verb in
	the matrix sentence agrees with respect to $\phi$-features with an
	argument of the verb in an embedded sentence. A language that shows
	this phenomenon is Hindi-Urdu (see \citealt{Mahajan:90},
	\citealt{Butt:95,Butt:08}, \citealt{Bhatt:05}, and \citealt{Chandra:05}, among
	others). A relevant pair of examples is given in (\ref{ex:mueller:1}). In Hindi, a DP
	qualifies as an agreement controller if it is not overtly case-marked (if
	both an external and an internal argument fail to be overtly
	case-marked, agreement is with the subject). Long-distance agreement
	is optional here: either there is agreement of both the matrix verb
	and the embedded verb with the embedded absolutive object DP (as in
	(\ref{ex:mueller:1b})), or the verbs show default agreement (as in (\ref{ex:mueller:1a})).
	
	\ea\label{ex:mueller:1}
	\ea \label{ex:mueller:1a}
	\gll  Raam-ne [\sub{$\alpha$} rotii khaanaa~] chaahaa \\ 
	Ram.{\scshape masc}.-{\scshape erg} {} bread.{\scshape fem} eat.{\scshape inf}.{\scshape masc} want.{\scshape perf.pst.masc} \\ 
	\glt  `Ram wanted to eat bread.'
	\ex \label{ex:mueller:1b}
	\gll Raam-ne [\sub{$\alpha$} rotii khaanii~] caahii \\ 
	Ram.{\scshape masc}-{\scshape erg} {} bread.{\scshape fem} eat.{\scshape inf}.{\scshape fem} want.{\scshape perf.pst.fem} \\ 
	\glt `Ram wanted to eat bread.'
	\z
	\z
	Long-distance agreement is also widespread in Nakh-Daghestanian
	languages. A relevant pair of examples from Tsez is given in (\ref{ex:mueller:2})
	(see \citealt{PolinskyPotsdam:01}). Agreement with respect to gender
	(III, in the case at hand) is controlled by
	absolutive DPs in Tsez. It always shows up on the embedded verb (if
	that verb can host overt agreement morphology in principle), and may
	then optionally also show up on the matrix verb (as in \ref{ex:mueller:2b});
	alternatively, there is no long-distance agreement, and the matrix
	verb exhibits default (IV) agreement marking (as in (\ref{ex:mueller:2a})).\footnote{As
		noted by \citet{Bhatt:05}, some varieties of Hindi behave similarly
		in that they also exhibit an asymmetry between agreement with matrix
		as opposed to embedded verbs (such that embedded verbs can agree
		while matrix verbs do not have to), whereas other varieties of Hindi show a
		strict one-to-one correspondence (such that embedded verb agreement
		implies matrix verb agreement), as presupposed in the main text above.}

	\ea\label{ex:mueller:2}
	\ea \label{ex:mueller:2a}
	\gll     Eni-r [\sub{$\alpha$} u\v{z}-\={a} magalu b-\={a}c'-ru-\l i~] r-iy-xo \\
	mother-{\scshape dat} {} boy-{\scshape erg} bread.{\scshape iii.abs} {\scshape iii}-eat-{\scshape pstprt-nmlz} {\scshape iv}-know-{\scshape prs} \\
	\glt     `The mother knows that the boy ate the bread.'\newpage
	\ex \label{ex:mueller:2b}
	\gll     Eni-r [\sub{$\alpha$} u\v{z}-\={a} \label{2-b}magalu b-\={a}c'-ru-\l i~] b-iy-xo \\
	mother-{\scshape dat} {} boy-{\scshape erg} bread.{\scshape iii.abs} {\scshape iii}-eat-{\scshape pstprt-nmlz} {\scshape iii}-know-{\scshape prs} \\
	\glt      `The mother knows that the boy ate the bread.'
	\z
	\z
	Another example for long-distance agreement in Nakh-Daghestanian
	languages comes from Hinuq; see (\ref{ex:mueller:3a} and \ref{ex:mueller:3b}) (from
	\citealt{Forker:11}). Again, gender agreement is controlled by an
	absolutive DP, and gender-based long-distance agreement is optional.
	
	\ea\label{ex:mueller:3}
	\ea \label{ex:mueller:3a}
	\gll Sa\textbarglotstop ida-z r-eq'i-yo [\sub{$\alpha$} Madina-y \textgamma i ga:-s-\textbeltl i~]\sub{V} \\
	Saida-{\scshape dat} {V}-know.{\scshape prs} {} Madina-{\scshape erg} milk(IV).{\scshape abs} drink-{\scshape res-abst} \\ 
	\glt `Saida knows that Madina drank milk.'
	\ex \label{ex:mueller:3b}
	\gll  Sa\textbarglotstop ida-z y-eq'i-yo [\sub{$\alpha$} \label{3-b}Madina-y \textgamma i ga:-s-\textbeltl i~]\sub{V} \\
	Saida-{\scshape dat} {IV}-know.{\scshape prs} {} Madina-{\scshape erg} milk({IV}).{\scshape abs} drink-{\scshape res-abst} \\
	\glt  `Saida knows that Madina drank milk.'
	\z
	\z
	Other languages exhibiting long-distance agreement are Itelmen (see
	\citealt{BobaljikWurmbrand:05}; a relevant pair of examples is given in
	(\ref{ex:mueller:4})), Innu-aim\^{u}n (see \citealt{BraniganMacKenzie:02}, with
	relevant examples in (\ref{ex:mueller:5})), Passamaquoddy (see \citealt{Bruening:01}),
	Chukchee (see \citealt{Boskovic:07}) and Blackfoot (see \citealt{Bliss:09}).
	
	\ea\label{ex:mueller:4}
	\ea 
	\gll Na netxa-in [\sub{$\alpha$} kma jeβna-s~] \\
	he forget-{{\scshape 3sg.subj(intrans)}} {} me meet-{\scshape inf} \\
	\glt  `He forgot to meet me.'
	\ex
	\gll Na əntxa-βum=nm [\sub{$\alpha$} kma jeβna-s~] \\
	he forget-{{\scshape 1sg.obj=3.cl}} {} {me} meet-{\scshape inf} \\
	\glt `He forgot to meet me.'
	\z
	\z
	
	\ea\label{ex:mueller:5}
	\ea 
	\gll Ni-tshissenit-en [\sub{$\alpha$} P\^{u}n k\^{a}-m\^{u}pisht-\^{a}shk~] \\
	1-know-{{\scshape ti}} {} Paul {\scshape prt}-visited-2/{\scshape inv} \\
	\glt `I know that Paul visited you.'
	\ex
	\gll Ni-tshissenim-{\^a}u [\sub{$\alpha$} P\^{u}n k\^{a}-m\^{u}pisht-\^{a}shk~] \\
	1-know-{3} {} {Paul} {\scshape prt}-visited-2/{\scshape inv} \\
	\glt `I know that Paul visited you.'
	\z
	\z
	While long-distance agreement is optional in all these languages,
	there are also some languages where long-distance agreement is
	obligatory in certain environments; this holds, e.g., for Icelandic,
	Kutchi Gujarati and Chamorro.  In this paper, we will concentrate on
	those languages in which long-distance agreement seems
	optional. Still, as with many other cases of syntactic optionality, it
	turns out that in all these cases, the choice of long-distance
	agreement in a sentence goes along with an interpretation of the
	controller of long-distance agreement as having a particular
	information structural status, i.e., an interpretation that the other
	member of the sentence pair -- that with (only) local agreement --
	lacks. 
	
	The central challenge posed by long-distance agreement for syntactic
	theory is, of course, the apparent non-locality of the operation. More
	specifically, the embedded DP that controls the agreement would seem
	to be separated by a clause-like constituent ($\alpha$ in the examples
	above) from the matrix verb, and therefore be too far away to permit
	establishing a local relation.  This is potentially problematic
	because most current syntactic theories do indeed postulate that
	syntactic operations (like agreement) are highly local (e.g., this
	holds for the Minimalist Program, HPSG, Categorial Grammar, and Optimality
	Theory); in line with this, in all these theories, apparently
	non-local dependencies like long-distance movement, long-distance
	reflexivization, long-distance case assignment, sequence of tense, and
	switch reference have successfully been reanalyzed as involving only
	fairly local operations (see \citealt{Alexiadouetal:12} for an
	overview).
	
	Against the background of the Minimalist Program, the question raised
	by long-distance agreement is how it can be ensured that the basic
	locality requirement imposed by the Phase Impenetrability Condition
	(PIC, \citealt{Chomsky:00,Chomsky:01,Chomsky:08,Chomsky:13}) in (\ref{ex:mueller:6}) is
	respected by the operation. 
	
	\ea\label{ex:mueller:6} {\itshape Phase Impenetrability Condition} \label{pic1}(PIC):\\
	In phase $\alpha$ with head H, the domain of H is not accessible to
	operations outside $\alpha$; only H and its edge are accessible to such
	operations.
	\z
{\noindent}If $\alpha$ in (\ref{ex:mueller:1})--(\ref{ex:mueller:5}) qualifies as a phase in the sense of
	(\ref{ex:mueller:6}), the very existence of long-distance agreement seems to be at
	variance with the PIC.\footnote{\label{fn1}As a matter of fact, simple cases of T
		agreeing with a nominative object DP in VP in a language like
		Icelandic are already problematic from the point of view of the PIC
		in (\ref{pic1}). For this reason, \citet{Chomsky:01} also envisages a
		second, somewhat more liberal version of the PIC, where a phase
		domain becomes opaque only when the next phase is reached. However, even
		this less restrictive version of the PIC would not suffice to
		straightforwardly derive the possibility of long-distance
		agreement. Furthermore, one might argue that the head actually
		responsible for this agreement into VP is not T but v. We will
		abstract away from this issue in what follows, and presuppose the
		PIC in (\ref{pic1}) for the remainder of the paper.}
	
\noindent In what follows, we will argue that the PIC-induced locality problem
	with long-distance agreement is real, and has not been
	convincingly solved yet in any of the approaches to long-distance
	agreement that have been developed so far. In view of this state of
	affairs, we will propose a radically new approach in terms of complex
	predicate formation plus reprojection where long-distance agreement
	can be analyzed as a strictly local operation: The new analysis  involves (i) a verb
	V\sub{1} which enters the syntactic derivation as part of a complex
	predicate V\sub{1}-V\sub{2} that is merged with the embedded internal argument,
	agreeing with it locally early in the derivation; and (ii) subsequent
	reprojection movement of V\sub{1} out of V\sub{2}'s clause, which eventually
	produces a biclausal structure (and thereby leads to a
	counter-bleeding effect with agreement). 
	
	The article is structured as follows. In section \ref{sec-bjoe-muel:2}, we will sketch the
	different types of existing approaches to the phenomenon of
	long-distance agreement. Given the PIC-based premise of strictly local
	application of all syntactic operations, we will point out individual
	problems the different approaches face as well as introduce some new
	data that turn out to be problematic for almost all of them. We will
	then go on to introduce the new approach in terms of reprojection in
	section \ref{sec-bjoe-muel:3}. In section \ref{sec-bjoe-muel:4}, which concludes the article, we will
	summarize the main features of the new approach and provide an outlook
	into how it might also be put to use in other contexts involving
	extraction from DPs, where there is evidence for an extremely local
	relation of two heads that show up in two separate domains in
	syntactic output structures.
	
	
	\section{Existing analyses} \label{sec-bjoe-muel:2}
	
	At least for our present purposes, four different kinds of analysis
	of long-distance agreement can be distinguished. We will refer to
	these as (i) non-local analyses (where long-distance agreeement can
	apply in a non-local fashion),  (ii) small structure analyses (where
	there is no phase boundary), (iii) cyclic Agree analyses (where the
	information relevant for agreement is locally passed on through the
	tree, originating with the controller and ultimately reaching the
	matrix verb), and (iv) feeding analyses (where movement
	of the controller makes 
	local agreement possible). We will discuss these four analysis types in
	turn.\footnote{We will not consider a fifth type of analysis, where
		the matrix verb locally agrees with a covert pronoun, which in turn
		is coindexed (and therefore shares $\phi$-features) with an
		embedded DP. As shown in \citet{PolinskyPotsdam:01} and
		\citet{BhattKeine:16:lon}, such `proxy agreement' is not a viable
		alternative in general. (Also, it is worth noting that such an
		analysis solves one locality problem (seemingly non-local agreement)
		by shifting it to another, well-known locality problem (seemingly
		non-local binding chains).}
	
	\subsection{Type (i): Non-local analyses}
	
	Non-local analyses either assume that the locality constraint on Agree
	is weaker than the original PIC (see \citealt{Chomsky:01} and footnote
	\ref{fn1}), or that Agree is not, in fact, subject to such a strict
	locality constraint in the first place (see \citealt{Sells:06}, \citealt{Boskovic:07},
	\citealt{Keine:16}).
	
	For example, Bo\v{s}kovi{\'c}'s (2007) analysis relies on a revised
	Agree operation, one which is not subject to either the  Activity
	Condition ((\ref{AC}) in the original Agree definition in
	(\ref{Agree})) or the PIC. 
	
	\ea\label{ex:mueller:7} \label{Agree}{\itshape Agree} \\ $\alpha$ may Agree with $\beta$ iff
	(a)--(d) hold.
	\ea  $\alpha$ carries at least one unvalued and uninterpretable
	feature and $\beta$ carries a matching interpretable and valued
	feature 
	\ex  $\alpha$ c-commands $\beta$
	\ex \label{AgreeClosest}$\beta$ is the closest goal to $\alpha$
	\ex \label{AC}$\beta$ bears an unvalued uninterpretable feature.
	\z
	\z
	Thus, the matrix verb can look all the way
	down to the embedded absolutive DP to check its
	$\phi$-features. \citet{Boskovic:07} assumes that finite complement
	clauses can in principle be CPs or TPs. Thus, in complement clauses in
	which there is no evidence for a CP layer, one might as well asume
	that those clauses actually lack it.
	The idea, then, is that CPs block long-distance agreement in Tsez while TPs allow it. This
	is because, Bo\v{s}kovi{\'c} (2007) claims, CPs, in contrast to TPs,
	may carry $\phi$-features that need to get checked. In the case of Tsez,
	CPs (and those of the languages that lack long-distance agreement altogheter), he assumes
	that they do carry such (default-valued) $\phi$-features, which makes it possible for
	the matrix verb to locally agree with the CP as such, leading to local
	agreement. Long-distance agreement in cases that involve CPs is impossible due to the
	condition on Agree to involve the potential goal closest to the probe
	(see (\ref{AgreeClosest})).
	
	Whatever the merits of this proposal, it is clear that it is
	incompatible with the strict PIC in (\ref{pic1}): 
	Since the potential long-distance agreement controller  in
	transitive sentences is the internal argument bearing absolutive case, which is base-generated
	in VP, the PIC will predict that long-distance agreement should not be
	possible, independently of whether the embedded vP phase has both a TP
	and a CP projection on top of it, or just a TP projection. 
	
	\subsection{Type (ii): Small structure analyses}
	
	In small structure analyses, it is argued that the configuration of matrix verb
	and long-distance agree\-ment-trigger is local after all. The structure
	of the complement clause is assumed to be smaller than might be
	thought from the surface data (e.g., a VP in \citealt{Boeckx:04}; an
	InflP in \citealt{Bhatt:05}).
	
	Like \citegen{Boskovic:07} approach, \citegen{Bhatt:05} analysis is
	based on a revised Agree operation -- in this case, it is one which also is not
	subject to the {\itshape Activity Condition}, but which does respect the PIC.
	To account for apparent long-distance agreement in Hindi, \citet{Bhatt:05} assumes
	that complement clauses to  matrix verbs that allow long-distance agreement are in fact
	only InflPs/VPs, which lack an external argument (i.e., they have
	no PRO) and are thus not phases. In contrast, long-distance
	agreement out of finite clauses is not possible in Hindi because
	finite complement clauses are CPs; the same prediction arises
	for infinitival structures that contain a PRO subject. 
	In line with this, as far as the optionality of long-distance agreement in Hindi is
	concerned, 
	\citet{Bhatt:05} explains it by
	assuming that matrix verbs that allow long-distance agreement have an option of selecting
	either a restructuring infinitive or a non-restructuring infinitive, where the latter involves a syntatically projected PRO subject.
	This PRO intervenes between the matrix verb and the embedded
	object and, thus, blocks long-distance agreement (in the same way,
	a problem with the PIC would arise in this context). 
	
	\citegen{Bhatt:05} analysis may work well for a language like Hindi,
	but it does not carry over to long-distance agreement in languages of
	the Nakh-Daghestanian type, where the external argument is clearly
	present (bearing ergative) in the embedded clause; see, e.g.,
	(\ref{2-b}) from Tsez and (\ref{3-b}) from Hinuq. Given the strict version
	of the PIC in (\ref{pic1}), it is clear that the presence of an external
	argument DP uncontroversially implies the presence of a vP phase, and
	this should make agreement of a matrix V with an internal argument DP
	included in the complement domain of v impossible, independently of
	whether vP qualifies as $\alpha$ (in the above sense) or not (i.e.,
	independently of whether there is additional structure on top of vP in
	the complement of the matrix V).\footnote{As a matter of fact, to
		account for evidence of the Nakh-Daghestanian type, 
		\citet[791]{Bhatt:05} ultimately concludes that a feeding account of
		along the lines of \cite{PolinskyPotsdam:01}  is independently called
		for; see below.}
	
	\subsection{Type (iii): Cyclic agree analyses}
	
	In cyclic Agree analyses, it is assumed that what looks like long-distance
	agreement actually is to be decomposed into a series of shorter
	agreement steps, all of of which obey strict locality. On this view,
	first the embedded verb agrees with the embedded agreement controller
	DP; second, the matrix verb agrees with the embedded verb; third, by
	transitivity, this implies that the matrix verb will eventually agree
	with the embedded DP, albeit indirectly. This kind of analysis has
	been pursued by \cite{Butt:95}, \cite{Legate:05:pha}, \cite{Keine:08},
	\cite{Preminger:09}, and \cite{Lahne:12}, among others. As an
	illustration, consider the specific approach developed in
	\cite{Legate:05:pha}.
	
	The basic premise of this approach is that at no stage of the
	derivation is there an Agree relation between the matrix verb and the
	embedded DP.  Rather, the agreement controller DP's $\phi$-features
	first valuate an $[u\phi]$ probe feature of a phase head, which by
	definition (cf. the PIC in (\ref{pic1})) is also part of the higher
	phase.  The matrix verb then probes the embedded phase head's
	$\phi$-features.  Thus, the embedded phase head acts as a hinge
	between the matrix and embedded domains.  This accounts for the
	observation that long-distance agreement presupposes the existence of
	local agreement in the embedded clause. 
	
	The cyclic Agree approach solves the locality problem with long-distance agreement in a very simple manner that directly
	corresponds to the analogous (and by now well-established) treatment
	of long-distance movement in terms of successions of smaller movement
	steps.\footnote{As a matter of fact, an alternative local analysis of long-distance agreement that mimicks {\scshape slash} feature percolation as it has been proposed for movement dependencies (see \citealt{Gazdar:81}) might in principle also be an option. This would then express a similarity of the two operations in long-distance contexts (viz., movement and agreement) even more straightforwardly. However, to the best of our knowledge, such an analysis has not yet been proposed. That notwithstanding, it would be subject to the same empirical problem mentioned in the main text below.}  However, there are both conceptual and empirical problems
	raised by cyclic Agree approaches to long-distance agreement. On the
	one hand, cyclic Agree is conceptually problematic from a minimalist
	perspective, given standard assumptions about probe features, goal
	features, and the Agree operation: It looks as though one and the same
	set of $\phi$-features (on the phase head in the middle) must act as a
	probe in one case, and as a goal in another (see \citealt{Bhatt:05}). On
	the other hand, there is an empirical problem (see
	\citealt{PolinskyPotsdam:01}, \citealt{BhattKeine:16:lon}) that is due to the
	fact that cyclic Agree approaches rely on transitivity. The problem is
	that if two verbs V\sub{1} and V\sub{2} can in principle participate in local
	agreement in some long-distance agreement constructions, and V\sub{2} and
	DP\sub{\textit{abs}} can participate in local agreement, then long-distance
	agreement involving V\sub{1} and DP\sub{\textit{abs}} must also be possible. However,
	this is not always the  case: For instance,  in Tsez, long-distance
	agreement, unlike local agremeent, requires DP\sub{\textit{abs}} to
	be a topic (see \citealt{PolinskyPotsdam:01}). This is shown by the
	examples in (\ref{ex:mueller:8}), where (i) local agreement of the embedded predicate and
	the absolutive DP as the agreement controller is possible throughout,
	(ii) the matrix and embedded predicates can participate in
	long-distance agremeent in principle, but (iii) long-distance
	agreement in this configuration is blocked nevertheless because the
	absolutive DP is not interpreted as a topic: It is interpreted as a
	focus in (\ref{ex:mueller:8a}), and it shows up as an -- inherently
	non-topicalizable -- reflexive pronoun in (\ref{ex:mueller:8c}). In both cases, the
	matrix verb can only carry out agreement with the embedded clause
	($\alpha$) itself; cf. (\ref{ex:mueller:8b})-(\ref{ex:mueller:8d}). 
	
	\ea\label{ex:mueller:8}
	\ea{\label{ex:mueller:8a}}
		\gll   *Eni-r [\sub{$\alpha$\sub{\rm IV}} t$'$ek-kin y-igu y\={a}\l -ru-\l i~] y-iy-xo\\
		mother-{\scshape dat} {} bookII.{\scshape abs}-{\scshape foc} II-good be-{\scshape pstprt-nmlz} II-know-{\scshape pres}\\
		\glt   `The mother knows that the {\scshape book} \rm is good.'
	\ex{\label{ex:mueller:8b}}
		\gll Eni-r [\sub{$\alpha$\sub{\rm IV}} t$'$ek-kin y-igu y\={a}\l -ru-\l i~] r-iy-xo\\
		mother-{\scshape dat} {} bookII.{\scshape abs}-{\scshape foc} II-good be-{\scshape  pstprt-nmlz} IV-know-{\scshape pres}\\
		\glt `The mother knows that the {\scshape book} \rm is good.'
	\ex{\label{ex:mueller:8c}}
		\gll *Eni-r [\sub{$\alpha$\sub{\rm IV}} u\v{z}-\={a} {nes\={a} \v{z}e}  \v{z}\={a}k$'$-ru-\l i~] {} {\O}-iy-xo \\
		mother-{\scshape dat} {} boy-{\scshape erg} {\scshape refl}.I.{\scshape abs} beat-{\scshape  pstprt-nmlz} {} I-know-{\scshape pres}\\
		\glt `The mother knows that the boy beat himself up.'
	\ex{\label{ex:mueller:8d}}
		\gll Eni-r [\sub{$\alpha$\sub{\rm IV}} u\v{z}-\={a} {nes\={a} \v{z}e} \v{z}\={a}k$'$-ru-\l i~] r-iy-xo \\
		mother-{\scshape dat} {} boy-{\scshape erg} {\scshape refl}.I.{\scshape abs} beat-{\scshape pstprt-nmlz} IV-know-{\scshape pres}\\ \vspace{-0.9cm}
		\glt `The mother knows that the boy beat himself up.'
	\z
	\z
	
	\subsection{Type (iv): Feeding analyses} \label{sec-bjoe-muel:2.4}
	
	\subsubsection{Movement feeds agreement}
	
	Fourth and finally, \cite{PolinskyPotsdam:01} have argued that a
	local approach to long-distance agreement in Tsez is both technically
	feasible and empirically supported according to which long-distance
	agreement involves feeding of local agreement by movement. The basic
	assumption is that the agreement controller (an absolutive DP in Tsez)
	moves to a position in which it can locally agree with the matrix
	verb. However, there are two complications to this simple
	picture. First, Polinsky \& Potsdam present strong arguments against
	the assumption that displacement of the agreement controller DP ends
	up in the matrix clause itself. For one thing, all established
	movement operations in Tsez are strictly clause-bound, so the 
	operation that feeds long-distance agreement would be the only type of
	movement that could leave a clause. For another,
	\cite{PolinskyPotsdam:01} observe that long-distance
	agreement in Tsez never co-occurs with scope reversal; in other words,
	a DP that participates in long-distance agreement with a matrix verb
	can never take scope over quantifed items in the matrix clause.
	This latter property is illustrated in (\ref{ex:mueller:9}): Independently of whether
	long-distance agreement takes place (see (\ref{ex:mueller:9b})) or not (see
	(\ref{ex:mueller:9a})), the embedded absolutive DP (with a universal quantifier in
	(\ref{ex:mueller:9})) cannot take scope over a  matrix subject (with an existential
	quantifier in the case at hand). 
	
	\ea \label{ex:mueller:9}  \label{scope32}
	\ea \label{ex:mueller:9a}
	\gll Sis u\v{c}iteler [\sub{$\alpha$} \v{s}ibaw u\v{z}i \o -ik'ixosi-\textbeltl i] r-iy-xo \\
	one teacher {} every boy{\scshape i}.{\scshape abs} {\scshape i}-go-{\scshape nmlz} {\scshape iv}-know-{\scshape prs}  \\
	\glt `Some teacher is such that he knows that every boy is going.'\\
	\char42`Every boy is such that some teacher knows that he is going.'
	\ex \label{ex:mueller:9b}
	\gll Sis u\v{c}iteler [\sub{$\alpha$} \v{s}ibaw u\v{z}i \o -ik'ixosi-\textbeltl i] \o -iy-xo \\
	one teacher {} every boy{\scshape i}.{\scshape abs} {\scshape i}-go-{\scshape nmlz} {\scshape i}-know-{\scshape prs}  \\
	\glt      `Some teacher is such that he knows that every boy is going.'\\
	\char42`Every boy is such that some teacher knows that he is going.'
	\z
	\z
	From these considerations it follows that the postulated movement operation
	cannot actually end up in the matrix clause in cases of long-distance
	agreement. The second complication involves the overt/covert
	distinction of movement operations. Since the absolutive DP that participates in
	long-distance agreement does not have to be overtly
	displaced and typically shows up in its in situ position (or, more generally, given
	that Tsez exhibits variable word order: it shows up in its unmarked position),
	it is clear that the movement operation that feeds long-distance
	agreement must be a covert one. 
	
	Against this background, \citegen{PolinskyPotsdam:01} proposal is
	the following: There is  covert, information structure-driven
	movement of the long-distance agreement-controller into a higher domain (phase) of the
	same clause, and the position thus reached provides a  local enough
	configuration with the matrix verb to make  Agree with it possible.
	More specifically, \citegen{PolinskyPotsdam:01} analysis works as
	follows. 
	
	A crucial basic assumption is that the size of embedded clauses in
	Tsez is variable. (\ref{ex:mueller:10}) gives the maximal syntactic structure for a
	clause in Tsez. This structure is only fully  built up when needed;
	i.e., clauses are CPs if they exhibit material that belongs in this
	layer (e.g., a C head) but not otherwise; a TopP is projected if the
	clause contains a topic; and so forth. 
	
	\ea\label{ex:mueller:10} {\itshape Clause \label{Tsezstructure}structure for Tsez}:\\
	{[}\sub{CP} [\sub{$'$} C [\sub{TopP} [\sub{Top$'$} Top [\sub{TP} DP\sub{\textit{erg}} [\sub{T$'$} T [\sub{VP} DP\sub{\textit{abs}} V]]]]]]{]}
	\z
	On this  basis,  \cite{PolinskyPotsdam:01} assume that, generally, long-distance agreement-allowing matrix
	verbs can select a TP as complement. 
	However,  to derive long-distance agreement in Tsez, it is postulated
	that a   topic-marked long-distance agreement-controlling element is covertly moved to the
	specifier of TopP in the left periphery of the complement clause. 
	This movement brings the triggering element into a sufficiently
	local relation to the matrix verb to allow the latter to check its
	uninterpretable $\phi$-features against those of the covertly
	topicalised element, resulting in long-distance agreement.
	Long-distance agreement is, thus, taken to be a reflex of the topic-status of the
	triggering element. 
	
	A CP is predicted to block long-distance agreement, as is a bare TP,
	given that a DP\sub{\textit{abs}} agreement controller does not occupy SpecT (at
	this point the analysis is not fully PIC-compatible). The relation
	that \cite{PolinskyPotsdam:01} assume to underlie long-distance
	agreement is  Head Government (not Agree, as in (\ref{Agree})),
	which, following \cite{Rizzi:90}, is understood as in (\ref{ex:mueller:11}).
	
	\ea\label{ex:mueller:11} {\itshape Head Government}:\\
	X head-governs Y iff (a)--(d) hold:
	\ea \label{list1}X $\in$ $\{$A, N, P, V, H[+{\rm tense}]$\}$.
	\ex X m-commands Y.
	\ex No barrier intervenes.
	\ex Relativized Minimality is respected. 
	\z
	\z
	It is then postulated that it can be derived from (\ref{ex:mueller:11}) that ``a head
	governs its specifier, its complement, an element adjoined to its
	complement, and the specifier of its complement''
	\citep[627]{PolinskyPotsdam:01}, but not, say, the specifier of
	the complement of the head's complement. However, as a matter of fact,
	it is not quite clear why this should be the case. Consider (\ref{ex:mueller:12}).
	
	\ea\label{ex:mueller:12} {[}\sub{XP} X [\sub{ZP} \label{vcp}Z [\sub{WP} YP [\sub{W$'$} ...~]]]] \z
	Does X head-govern YP in (\ref{ex:mueller:12})? First, suppose that X is one of the
	possible items mentioned in (\ref{list1}). Second, X clearly m-commands
	YP. Third, Relativized Minimality is respected in (\ref{ex:mueller:12}) because there
	is no intervening phrase that could induce a Relativized Minimality
	effect for X and YP (there is no phrase that c-commands YP and is
	c-commanded by X -- ZP and WP both dominate YP). That leaves the
	presence of a barrier as the only possible source of a failure of
	head government of YP by X. Whether ZP or WP is a barrier in (\ref{ex:mueller:12})
	depends on the exact definition of this concept (which Polinsky and
	Potsdam do not provide). The first thing to note is that both ZP and
	WP are complements, i.e., sisters of X$^0$ categories. This will
	suffice to exempt them from barrier status in most of the available
	conceptions of barriers (see, e.g., \citealt{Cinque:90}). In contrast,
	according to the more complex, two-stage definition of barrier in
	\cite{Chomsky:86}, ZP might in fact emerge as a barrier in (\Last) if WP
	can be classified as a blocking category that passes on its status as
	a ``virtual barrier'' to the phrase immediately above it. So it seems
	that only under this complex approach, based on blocking categories
	vs. real barriers, can it be derived that X does not head-govern YP in
	(\Last). 
	
	Based on the assumption that all agreement relations are subject to
	(\LLast) (or at least the general consequences that (\LLast) is supposed
	to have), a number of restrictions that Polinsky and Potsdam observe for
	long-distance agreement in Tsez follow. First, a CP can never be
	projected in long-distance agreement contexts because ``it would block
	government of SpecTop by the verb''
	\citep[638]{PolinskyPotsdam:01}. Note  that this presupposes that
	ZP (= CP) would indeed qualify as a barrier in (\Last) that makes
	head government of YP (= DP\sub{\textit{{abs}}} in SpecTop) by matrix V (= X)
	impossible. (As we have just seen, this consequence is far from
	straightforward.) However, with this qualification, it can be derived
	that long-distance agreement is impossible (i) in the presence of a
	wh-phrase in a clause (which inherently activates the CP layer,
	whether or not wh-movement takes place overtly), and (ii) in the
	presence of the element {\itshape \textcrlambda in}, which is assumed to be
	a designated C element.\footnote{\label{whabs}As noted by
		\citet[fn. 20]{PolinskyPotsdam:01}, long-distance agreement in the
		presence of a 
		wh-phrase would ceteris paribus be expected to be possible if the
		wh-phrase is itself the absolutive argument and occupies SpecC.}
	A third prediction is that long-distance
	agreement with D\sub{\textit{{abs}}} is impossible if some other XP functions as
	the topic in a clause.\footnote{Again, though, details of the account
		are somewhat unclear. One might think that this effect is due to
		intervention, i.e., the Relativized Minimality part of the
		definition of Head Government. However,  if there can only be one topic per clause in
		Tsez, and that is not DP\sub{\textit{{abs}}}, then DP\sub{\textit{{abs}}} can never reach the
		position (viz., SpecTop) it needs to reach to enable long-distance
		agreement, and a resort to intervention is not necessary. If, on the other hand, there can be more than one topic
		per clause, then it is not obvious why DP\sub{\textit{{abs}}} should not qualify
		as the structurally highest one of them, thereby circumventing an
		intervention effect.}
	
	These qualifications notwithstanding, \citegen{PolinskyPotsdam:01}
	analysis would seem to derive licit and illicit cases of long-distance
	agreement in a very simple and elegant way that furthermore respects
	locality considerations. Still, as shown in the following subsection,
	there are conceptual and empirical problems with this approach.
	
	
	\subsubsection{Problems with the feeding approach}	
	\subsubsubsection{\textbf{The nature of covert topic movement}}
	
	First, as noted by \cite{Boskovic:07}, the crucial postulation of a
	covert topicalisation operation for Tsez is far from innocuous. There
	is virtually no independent evidence that such an operation
	exists. Also, there is a real danger of an ordering paradox: It is not
	really clear how {\itshape covert} movement at LF can trigger {\itshape overt}
	agreement -- if the movement takes place at LF, it comes too
	late.\footnote{This problem  can in principle be solved by assuming
		that covert movement is actually movement taking place in the
		narrow syntax, with the only difference to overt movement being that
		the lowest copy of a complex chain is subject to phonological
		realization (rather than the highest member, as with overt movement).
		However, even if one were to adopt such an approach based on the copy
		theory of movement, the intended effect does not seem to arise
		anywhere else. As far as we can tell, other instances of covert
		movement that have been suggested in the literature (e.g., for certain
		cases of wh-in situ) never feed Agree operations. Thus, compare (\ref{ex:mueller:Fn8a}),
		where overt wh-movement gives rise to new options for reflexivization
		(assumed here to be an instance of Agree, see \citealt{Reuland:11} for
		discussion), with (\ref{ex:mueller:Fn8b}), where covert wh-movement fails to produce the
		same effect (see \citealt{Barss:86}). 
		
		\ea\label{ex:mueller:Fn8}
		\ea\label{ex:mueller:Fn8a}{John\sub{1} wonders [\sub{CP} [\sub{DP} which book about himself\sub{1}] Bill bought~]}
		\ex\label{ex:mueller:Fn8b}*{John\sub{1} wonders [\sub{CP} why Bill bought [\sub{DP} a book about himself\sub{1}]]}
		\z
		\z
		In the same way, covert wh-movement (unlike overt wh-movement) never seems to feed case
		assignment or agreement in the world's languages. -- 
		All  that said, \citet[626]{PolinskyPotsdam:01} would seem to 
		exclude a reinterpretation of covert movement as overt movement plus
		pronunciation of the lower copy when they explicitly state that the
		``syntactic agreement configuration between the probe and the
		absolutive trigger is created at LF''.}
	
	
	\subsubsubsection{\textbf{Complementizers}}
There are two comp\-lement\-izer-like items in Tsez viz. {\itshape {\l}i},
	which permits long-distance agreement (and shows up in all Tsez
	examples exhibiting long-distance agreement disussed above), and {\it
		\textcrlambda in}, which blocks long-distance agreement. As noted,
	\citet{PolinskyPotsdam:01} assume that {\itshape \textcrlambda in} is
	indeed a regular C item, and assuming that the presence of a CP makes
	head government impossible, it is correctly predicted that there is no
	long-distance agreement across {\itshape \textcrlambda in}.
	However,  for the same reason, it must be assumed that {\itshape {\l}i} is
	not a C element. The problem here is that this is exactly what
	it looks like, given that, like {\itshape \textcrlambda in}, it is the
	outermost head in the word containing V. What is more, it does not yet suffice to
	assume that  {\itshape {\l}i} not is not a C element -- {\itshape {\l}i} must  be
	assumed not to be structurally represented {\itshape at all}. The reason is
	that if {\itshape {\l}i} were the head of a phrase (outside of TopP), it would block
	long-distance agreement in the same way as {\itshape \textcrlambda in}. 
	It remains unclear whether there is any independent evidence for such
	a radically different treatment (projecting complementizer
	vs. structure-less morphological marker) of the two items. (In the
	analysis to be developed in section \ref{sec-bjoe-muel:3} below, we will presuppose that
	both  {\itshape \textcrlambda in} and {\itshape {\l}i} are regular C items.)
	
	We take these first two problems to be potentially worrisome but
	certainly not decisive. Arguably, things are different with the next
	two issues raised by Polinsky \ Potsdam's analysis, concerning a
	semantic problem based on the assumed covert DP movement, and an
	incompatibility of the analysis with what look like clear cases of
	long-distance agreement across a CP boundary.
	
	
	\subsubsubsection{\textbf{Topic interpretation within the embedded clause}}
	
\citet{PolinskyPotsdam:01} assume that the landing site of the
abstract movement is in the left periphery of the embedded
	clause. Accordingly, the long-distance agreement-controlling DP is
	interpreted as the topic of the embedded clause. A problem with this
	analysis is that information structure phenomena -- and root phenomena
	in general -- are usually confined to clauses that have some
	illocutionary force; see \citet{HooperThompson:73},
	\citet{Ebertetal08}, \citet{Krifkainprep}, and \citet{Maticetal14}. Most long-distance agreement-allowing
	matrix verbs, however, are factives, which semantically take sentence
	radicals (see \citealt{Stenius67}), i.e.~propositions, as complements (see \citealt{Krifka04})
	and also syntactically involve smaller
	structures (see \citealt{deCubaUrogdi10}). Thus, complements of
	factives do not involve any illocutionary operator in their
	syntax/semantics -- only the matrix clause does. Under a structured
	proposition approach  (see \citealt{Krifka92}) to information structural
	phenomena, this leads to a semantic representation in which the topic
	can only be understood as the topic of the whole sentence. That is,
	the predicted structure for (\ref{2-b})  (repeated in
	(\ref{ex:mueller:13}) below) is as in (\ref{ex:mueller:14}).
	
	
	\ea\label{ex:mueller:13}
	\gll    Eni-r [\textsubscript{{$\alpha$}} u\v{z}\={a} \label{2d-b}magalu b-\={a}c'-ru-\l i~] b-iy-xo \\
	mother-{\scshape dat} {} boy-{\scshape erg} bread.{\scshape iii.abs} {\scshape iii}-eat-{\scshape pstprt-nmlz} {\scshape iii}-know-{\scshape prs} \\
	\glt      `The mother knows that the boy ate the bread.'
	\z
	
	\ea\label{ex:mueller:14} {\scshape Assert}($<$\sub{T} $\lambda$P.{\bfseries mother}($\lambda$z.{\bfseries boy}($\lambda$x.P($\lambda$y.eat(x,y))) ($\lambda$p.know(z,p))), {\bfseries bread}$>$)\z
	The felicity conditions of the {\scshape Assert}-operator for topic-comment
	structures make reference to the first part of the structured
	proposition as a whole. Thus, it is unclear whether the embedded topic interpretation
	advocated by \cite{PolinskyPotsdam:01}  is actually available;
	i.e.,  whether (\LLast)\ {\itshape is} actually understood as paraphrased in (\ref{embeddedtopicint}), or not rather as in (\ref{matrixtopicint}).
	
	\ea\label{ex:mueller:15} \textit{Readings for topics in long-distance agreement}
	\ea \label{embeddedtopicint}The mother knows that, as for the bread, the boy ate it.
	\ex \label{matrixtopicint}As for the bread, the mother knows that the boy ate it.
	\z
	\z
	More specifically, complex sentences with factive matrix verbs
	presuppose the truth of the proposition denoted by the respective
	complement clause. In terms of information structure, factive
	presuppositions belong to the ``background'' of an utterance and they
	are ``taken for granted''. Thus, they are not ``at-issue'' or ``under
	discussion''. This characterisation is inconsistent with
	\citegen{PolinskyPotsdam:01} assumption that a DP embedded under a factive verb
	can act as topic of the embedded clause.
	
	\subsubsubsection{\textbf{Long-distance agreement across a CP boundary}}
	
	\noindent A severe empirical problem for the analysis developed in
	\cite{PolinskyPotsdam:01} (but also for analyses of the small
	structure type) is that there is evidence from other Nakh-Daghestanian
	languages that strongly suggests that long-distance agreement is in
	principle possible across a CP boundary, and without movement to SpecC
	(recall footnote \ref{whabs}). 
	
	Thus, Khwarshi (see
	\citealt{Khalilova09}) and Hinuq (see \citealt{Forker:11}), two
	Nakh-Daghe\-stanian languages closely related to Tsez,
	also exhibit long-distance agreement. Similarly to Tsez, this
	also goes along with a prominent information structural status of the
	triggering DP. In contrast to Tsez, however, in these languages, the
	triggering NP can have either topic or focus status.
	For instance, in the Khwarshi examples in (\Next), the 
	absolutive DP is interpreted as the topic of the embedded clause, and
	long-distance agreement is possible.
	However, long-distance agreement is also possible in answers to
	information questions, as in (\NNext), suggesting that the
	long-distance agreement-controlling DP\sub{\textit{{abs}}} may also function as the focus of the embedded clause.
	\vspace{0.3cm}
	\ea\label{ex:mueller:16}
	\ea
	\gll I\v{s}et'u-l l-iq'-\v{s}e goli u\v{z}a bataxu y-acc-u \\
	mother.{\scshape obl-lat} IV-knows-{\scshape prs} {\scshape cop} boy.{\scshape erg} bread(V) V-eat-{\scshape pst.ptcp} \\
	\glt `Mother knows that the boy ate bread.'
	\ex  
	\gll I\v{s}et'u-l \label{16-b}y-iq'-\v{s}e goli u\v{z}a bataxu y-acc-u \\
	mother.{\scshape obl-lat} V-knows-{\scshape prs} {\scshape cop} boy.{\scshape erg}  bread(V) V-eat-{\scshape pst.ptcp} \\
	\glt `As for the bread, mother knows that  the boy ate it.'
	\z
	\z
	
	\ea\label{ex:mueller:17}
	\ea (Which cow does the boy know came?)
	\ex\label{17-b}
	\gll  U\v{z}a-l l/b-iq'-\v{s}e k\textsuperscript{ʕ}aba zihe b-ot'uq'q'-u \\
	boy.{\scshape obl-lat} IV/III-know-{\scshape prs} black cow(III) III-come-{\scshape pst.ptcp} \\
	\glt `The boy knows that the black cow has come.'
	\z
	\z
	Similar facts obtain in Hinuq.  
	
	Importantly, long-distance agreement in Khwarshi and Hinuq is also less
	restricted in another respect:
	There are cases in which a {\em wh}-element occurs in
	an interrogative  complement clause, and long-distance agreement is nevertheless
	available. This is shown for Khwarshi in (\Next) (see \citealt{Khalilova09}). 
	
	\ea\label{ex:mueller:18}
	\gll U\v{z}a-l l/b-iq'-\v{s}e [\sub{CP\sub{{\rm (IV)}}} \l \label{cp1}u\sub{\rm [foc]} zihe b-iti-xx-u] \\
	boy.{\scshape obl-lat} {\scshape iv/iii}-know-{\scshape prs} {} who.{\scshape erg} cow({\scshape iii}) {\scshape iii}-divide-{\scshape caus-pst.ptcp} \\
	\glt `The boy knows who has stolen the cow.'
	\z
	Here a wh-phrase bearing ergative case shows up in the embedded
	interrogative clause. Given standard assumptions about the
	semantics of questions (see, e.g., \citealt{Stechow:96:aga}),
	the interrogative interpretation of a clause is inherently, and
	invariably, tied to the presence of a C element. Therefore, (\Last)
	proves that long-distance agreement across a CP boundary is possible
	in Khwarshi independently of whether the ergative wh-phrase can be
	assumed to be located in SpecC in the syntax, and of whether or not
	there is an overt C item present. 
	
	Next, (\Next) and (\NNext) illustrate the possibility of long-distance
	agreement across a CP boundary in Hinuq (see \citealt{Forker:11}). In
	(\Next), the embedded interrogative clause contains a subject
	wh-phrase that does not block long-distance agreement with the
	(non-wh) absolutive DP. 
	
	\ea\label{ex:mueller:19} 
	\gll\relax [\sub{CP} \textbeltl u rek'$^{\textrm{w}}$e go\textbeltl \label{cp2}i\v{s}~] di\v{z} \o -eq'i-yo gom \\
	{} who man.{\scshape i} be.{\scshape cvb} I.{\scshape dat} {\scshape i}-know-{\scshape prs}  be.{\scshape neg} \\ 
	\glt `I do not know who he was.'
	\z
	In contrast, (\Next) shows that long-distance agreement is also possible with
	an absolutive DP that is a wh-element itself; but there is no reason
	to assume that DP\sub{\textit{{abs}}} is not in its base position here. 
	
	\ea\label{ex:mueller:20} \label{cp3}
	\gll Debez r/\o -eq'i-ye [\sub{CP(V)} k'a\v{c}a\textgamma -za-y \textbeltl u \o -uher-i\v{s}-\textbeltl i~]~? \\
	you.{\scshape sg.dat} {\scshape v/i}-know-{\scshape q} {} bandit-{\scshape obl.pl-erg}   who {\scshape i}-kill-{\scshape res-abst} \\ 
	\glt `Do you know whom the bandits killed?'
	\z
	Note that (\LLast) and (\Last) contain clause-final elements that
	would seem to correspond to {\itshape {\l}i} rather than  {\it
		\textcrlambda in} in Tsez. However, notwithstanding the problems
	raised by \citegen{PolinskyPotsdam:01} analysis of {\itshape {\l}i}
	mentioned above, and notwithstanding the arguments that Forker
	presents for a uniform CP analysis of these contexts in Hinuq after
	all, it is clear that the status of {\itshape {\l}i} has no bearing on the
	question of whether there is a CP present in (\ref{cp2}) and (\ref{cp3})
	(and (\ref{cp1}), for that matter): There must be a CP boundary here
	because of the combination of interrogative semantics and a full
	clausal structure, including assignment of ergative (which by itself
	is not yet decisive, given that there are good arguments for assuming
	that the ergative is assigned within vP in Nakh-Daghestanian
	languages; see \citealt{Gagliardietal:14}, \citealt{Polinsky:16:arc}). 
	
	Thus, there is strong evidence from Khwarshi and Hinuq for the general
	availability of long-distance agreement across what must qualify as a
	CP.
	
	What is more, unlike Tsez, the non-Tsezic Nakh-Dahgestanian language
	Tsakhur  also permits what can be called
	``super-long-distance agreement'', i.e., long-distance agreement 
	across two clause boundaries; see \citet{Kibrik:99}.
	
	\ea\label{ex:mueller:21} \label{tsa1}
	\gll I\v{c}\={\textsci}-s w=u\={k}\={\textsci}k\textbari n-na [\sub{CP} ji\v{c}o-j-s [\sub{CP} gaba  h\={a}\textglotstop-as~] XaIr-qi=w=x-es~] \\
	girl-\textsc{dat} {3}=want.\textsc{pf-{aa}} {} sister-\textsc{obl-dat} {} {carpet.3} 3.do-\textsc{pot} learn-{3}=become-\textsc{pot} \\
	\glt `The girl wants her sister to learn to make a carpet.'
	\z
	Similar phenomena have also been reported by \citet{Forker:11} for
	Hinuq; see (\Next).
	
	\ea \label{ex:mueller:22}\label{tsa2}
	\gll Iyo-z b-eq$'$i-yo [\sub{CP} Pat$'$imat-ez [\sub{CP} tort  b-ac$'$-a~] b-eti-\v{s}-\l i~]]\\
	mother-{\scshape dat} III-know-{\scshape prs} {} Patimat-{\scshape dat} {} cake(III) III-eat-{\scshape inf} III-want-{\scshape res}-{\scshape abst}\\
	\glt `The mother knows that Patimat wanted to eat the cake.''
	\z
	If there are two CP boundaries present, there is no way for the feeding approach
	to account for the option of long-distance agreement since the covert
	movement postulated in this approach always has to be
	clause-bound.
	
	\subsection{Interim conclusion}
	
	We take it that the conclusion that can be drawn on this basis is that
	all four existing approaches face significant problems with
	long-distance agreement from the point of view of a grammar that
	incorporates a strict locality principle like the PIC. Thus, there is
	every reason to pursue a  new approach; and given the general
	availability of long-distance agreement across a CP, this approach
	must be such that it preserves strict locality even if there can be no
	denying the fact that the matrix verb and the embedded agreement
	controller DP can be far away from one another in structural terms in
	syntactic surface representations. 
	
	As a basic premise, we will assume that the only way to locally model
	non-local dependencies is via {\itshape movement} (see
	\citealt{Hornstein:01,Hornstein:09} for this general point, and
	\citealt{Mueller:14:buf} for some specific proposals in \textit{a priori}
	recalcitrant domains). Given the PIC, a matrix V and an embedded
	(agreement-controlling) DP have to enter a local relation at some
	point of the derivation. As we have seen, there is evidence against
	the assumption that DP moves to the matrix V domain (or to a position
	of the embedded domain that is accessible from it); this excludes feeding
	analyses. The only remaining possibility then is that it is actually
	V that moves to the matrix domain: If the mountain won't come to the
	prophet, the prophet will go to the mountain.
	
	For concreteness, we would like to propose that locality in
	long-distance agreement is not established {\itshape late} in the
	derivation (as in \citealt{PolinskyPotsdam:01}'s approach, where
	movement feeds long-distance agreement), but, in fact, {\it
		early}. This approach thus involves counter-bleeding (rather than
	feeding): Agreement with the embedded internal argument DP takes place
	at a stage in the derivation when DP and the two verbs involved are
	all clause-mates. It is only due to subsequent reprojection movement
	of what will eventually become the matrix verb 
	that on the surface it looks as if agreement takes place
	long-distance; reprojection movement of V thus comes too late to bleed
	(i.e., it counter-bleeds) Agree.
	
	
	\section{A new analysis} \label{sec-bjoe-muel:3}
	
	\subsection{Head movement as reprojection}
	
	Let us begin by sketching the outlines of a general approach to head
	movement in terms of reprojection, a concept that has been widely
	pursued for various empirical domains over the last decades (see
	\citealt{Pesetsky:85}, \citealt{StechowSternefeld:88},
	\citealt{Sternefeld:89}, \citealt{Holmberg:91}, \citealt{Ackemaetal:93},
	\citealt{Kiss:95}, \citealt{Koeneman:00}, \citealt{Haider:00:bra},
	\citealt{Bhatt:02}, \citealt{HornsteinUriagereka:02},
	\citealt{Fanselow:03,Fanselow:09:boo}, \citealt{Bury:03}, \citealt{Suranyi:05},
	\citealt{Donati:06}, \citealt{BayerBrandner:08},
	\citealt{GeorgiMueller:10:rep}, \citealt{Mueller:11:loc}, and
	\citealt{SMueller:15}, among others).\footnote{\citet{Pesetsky:85}
		suggests that reprojection after head movement at LF serves to
		circumvent bracketing paradoxes. As far as we can tell, this
		qualifies as the first instance of a reprojection approach to head
		movement in the literature.} The basic idea behind head movement as
	reprojection is that an X$^0$ head is moved out of a projection that dominates it 
	and takes this projection as its own complement by merging with it,
	projecting anew in the derived position. This solves the notorious
	c-command and Extension Condition (cf. \citealt{Chomsky:95}) problems with head movement as
	adjunction to an X$^0$ category: In a head-movement-as-adjunction structure like (\Next),
	the moved head Y fails to extend the tree (since XP must, be
	definition, have been in place before movement of Y), and Y does not
	c-command its trace (because the next branching node containing Y is
	the higher X segment, which does not dominate Y's trace).
	
	\ea\label{ex:mueller:23}\relax [\sub{XP} [\sub{X} Y\sub{1} X~] [\sub{WP} ... t\sub{1} ...~]]\z
	These problems disappear under a reprojection approach:  Head movement has now extended the
	tree, and the moved item is able to c-commands its
	trace. Furthermore, this approach does not 
	necessitate (i) a relocation of head movement to PF (see
	\citealt{Chomsky:00}), (ii) a
	reinterpretation as XP movement (see \citealt{KoopmanSzabolcsi:00}, \citealt{Mahajan:01},
	and \citealt{Nilsen:03:diss}, among many others), or (iii) the
	postulation of a complex
	operation integrating both regular syntactic movement and
	syntactically irregular morphological merger (see
	\citealt{Matushansky:06}). 
	
	There are basically three different reprojection scenarios. A first
	possibility is that a head moves out of its own projection, merges
	with the XP of which it was the head prior to the movement, and
	projects anew. Such local reflexive reprojection is shown in (\Next). 

	\begin{figure}[!h]
		\begin{exe}
			\ex	\label{ex:mueller:24} {\textit{Local reflexive reprojection}}:\\
				\begin{forest}	
					[XP
					[X\sub{1}]
					[XP
					[WP]
					[X$'$
					[t\sub{1}]
					[ZP]
					] ] ] 	
			\end{forest}
		\end{exe} \vspace{-1.1cm}
	\end{figure}
	\newpage \noindent A second possibility is that a reprojection movement is still highly
	local (in the sense that the moved head attaches to the minimal phrase
	that dominated it before the movement step was carried out), but not
	reflexive. In this scenario, the moved head excorporates from a complex
	head structure that was formed by an earlier (possibly pre-syntactic) 
	operation combining two primitive X$^0$ categories (in accordance with
	c-command and Extension Condition requirements), or that is stored as
	such in the lexicon; after the movement, the moved head projects its own
	XP in the derived position.\footnote{Thus, strictly speaking, this is
		not actually an instance of {\itshape re}-projection: X in (\ref{aux12})
		projects for the first time in the derived position.} Local non-reflexive
	reprojection is illustrated in (\Next). 
	\begin{figure}[!h]
		\begin{exe}
			\ex	\label{ex:mueller:25} {\textit{Local \label{aux12}non-reflexive reprojection}}:\\
				\begin{forest}	
					[XP
					[X\sub{1}]
					[YP
					[WP]
					[Y$'$
					[Y-t\sub{1}]
					[ZP]
					] ] ] 	
			\end{forest}
		\end{exe} \vspace{-0.8cm}
	\end{figure}
\newline	Finally, reprojection can be non-local (by definition, it is then also
	non-reflexive). In (\Next), the moved head skips over two maximal
	projections and reprojects in the derived position. 
	
	\begin{figure}[!h]
		\begin{exe}
			\ex	\label{ex:mueller:26} {\textit{Non-local \label{25}non-reflexive reprojection}}:\\
				\begin{forest}	
					[XP
					[X\sub{1}]
					[YP
					[WP]
					[Y$'$
					[Y]
					[XP
					[ZP]
					[X$'$
					[t\sub{1}]
					[UP]
					] ] ] ] ] 	
			\end{forest}
		\end{exe} \vspace{-1.1cm}
	\end{figure}
	\newpage\noindent Assuming these three scenarios to be available in the world's
	languages, it can be concluded that head movement can involve
	excorporation (see \citealt{Roberts:91,Roberts:97:res}), and that head
	movement does not obey the Head Movement Constraint (see
	\citealt{Roberts:09:hea,Roberts:10} vs. \citealt{Travis:84} for arguments to
	this effect). Given that the data that originally motivated
	stipulation of the excorporation and Head Movement Constraint
	restrictions can be derived otherwise, this would seem to permit a
	simpler, more attractive theoretical approach, and to correspond to
	the null hypothesis. Furthermore, one should expect that head movement
	as reprojection obeys the same constraints that hold of all movement
	operations; this includes the PIC (see (\ref{pic1})). Thus, for the
	operation to be legitimate, it can be concluded that YP is not a phase
	in (\Last); and that, more generally, head movement as reprojection can
	cross phases by carrying out intermediate movement steps to phase
	edges, in accordance with the PIC.
	
	As for the concrete mechanics of reprojection movement, we will make
	the following assumptions. First,  all syntactic operations are
	feature-driven: On the one hand, there are designated   structure-building features (edge features, subcategorization
	features) that trigger (external or internal) Merge; we will refer to
	these as [{\small $\bullet$}F{\small $\bullet$}] features. On the
	other hand, there are   probe features that trigger Agree. To simplify
	exposition and simultaneously avoid commitment to one of the existing options in
	various domains (e.g., valuation vs. checking, interpretability
	vs. uninterpretability), we will refer to probes as 
	[$*$F$*$] features throughout.
	All these features triggering syntactic Merge and Agree operations are
	ordered on lexical items; and they are discharged (i.e., rendered
	syntactically inactive) one after the other
	after having induced the respective operations that they encode. 
	Finally (although this assumption will not actually be crucial), we
	postulate that all phrases are phases. As a consequence, movement must take place via all
	intermediate phrase edges  that intervene between a base position and
	the ultimate landing site of some moved item (except for the minimal
	specifier domain if the item is already part of the phase edge, as
	is the case with reprojection movement of heads). Given this
	assumption, YP in (\ref{25}) must be a phase, and X\sub{1} must therefore
	carry out an intermediate step to SpecY on its way to its ultimate
	position.\footnote{One might think that allowing heads to move to
		specifier positions might give rise to various over-generation
		problems. However, it is worth bearing in mind that specifier
		positions can only ever be used as {\itshape intermediate} escape
		hatches (required by the PIC) by head movement under present
		assumptions; thus, the situation is completely analogous to, say,
		wh-movement via an intermediate Specv position in English -- the
		wh-phrase can use this position as an intermediate escape hatch,
		but can never ultimately show up in it (for essentially the same
		reason, viz., that the trigger can only be saturated in the final
		landing site).}
	
	Suppose further that Featural Cyclicity holds, as in
	(\Next).\footnote{This constraint can plausibly be derived as a theorem
		under various conceptions of cyclic spell-out of complements of
		phase heads.}
	
	\ea\label{ex:mueller:27} {\itshape Featural \label{fc}Cyclicity}:\\
	A non-root XP cannot contain a feature $\delta$ in the non-edge domain of X that is supposed to trigger an operation ([{\small $\bullet$}F{\small $\bullet$}] or [$*$F$*$]).\z
	In the normal course of events, the head X of some XP has discharged
	all the Merge-inducing features ([{\small $\bullet$}F{\small $\bullet$}])
	and Agree-inducing features ([$*$F$*$]) it contains before XP is merged with some
	other category. However, suppose that the head X has not been able to
	discharge a [{\small $\bullet$}F{\small $\bullet$}] or [$*$F$*$]
	(plus, possibly, other features that are lower on the list of
	operation-inducing features of the head, and that can only be accessed
	if the topmost feature has been discharged). In such a situation, one
	of two Last Resort operations may take place: Either the [{\small
		$\bullet$}F{\small $\bullet$}] or [$*$F$*$] feature is deleted (see 
	\citealt{BejarRezac:09},
	\citealt{Preminger:14}, and  \citealt{Georgi:14} for proposals along these
	lines); or the item containing the incriminating feature is moved to the edge domain of
	the current phrase, so as not to violate Featural Cyclicity
	in (\Last). The two Last Resort options for [{\small
		$\bullet$}F{\small $\bullet$}] and [$*$F$*$] features are stated
	in (\Next). 
	\protectedex{
	\ea\label{ex:mueller:28} {\itshape Last Resort}:\\
	If a feature $\delta$ on X that triggers an operation cannot be discharged
	in XP, there are two basic options:
	\ea $\delta$ is deleted.
	\ex $\delta$ is moved to the edge of XP, pied-piping the minimal
	category containing it. 
	\z
	\z}
	Thus, a head X with a non-discharged $\delta$ ([{\small
		$\bullet$}F{\small $\bullet$}] or [$*$F$*$]) feature undergoes
	intermediate movement to phrase edges for as long as it takes to reach
	a position in which $\delta$ can eventually be discharged. Following
	\citet{Fanselow:03,Fanselow:09:boo}, \citet{Suranyi:05},
	\citet{Matushansky:06}, and \citet{GeorgiMueller:10:rep}, these kinds
	of features can then be viewed as triggers for reprojection
	movement.\footnote{\citet{Fanselow:03,Fanselow:09:boo} and
		\citet{GeorgiMueller:10:rep} refer to these kinds of features as
		{\itshape M\"unchhausen} features, based on the literary character
		Baron M\"unchhausen who escapes from a swamp (where he is
		trapped on the back of his horse) by pulling himself up by his
		hair.} Note that it can in principle be both probe features on
	some head X that trigger (intermediate or final) reprojection movement
	(e.g., if there is no matching goal for a probe in the structure, or
	if the goal is not c-commanded by the probe feature on X), and
	structure-building features (e.g., if there is no accessible matching
	category, or if two heads simultaneously need to discharge their
	[{\small $\bullet$}F{\small $\bullet$}] feature but only one can do
	this at any given stage of the derivation). However, in the
	reprojection approach to long-distance agreement to be developed in
	the next section, it is the need to discharge a structure-building
	feature that triggers the movement of (what thereby becomes) the
	matrix verb.
	
	
	\subsection{Long-distance agreement by reprojection}
	
	\subsubsection{Complex predicates}
	
	Long-distance agreement typically encompasses verbs that in many
	languages are restructuring verbs.  In fact, for another
	Nakh-Daghestanian language, Godoberi, \citet{Haspelmath99} shows with a
	series of tests that apparent long-distance agreement in the language
	actually involves only a monoclausal structure with a complex
	predicate.  However, \citet{Forker:11} and \citet{Khalilova09} show
	with similar tests that this is not the case for Hinuq or Khwarshi,
	both of which involve truly biclausal structures, with an embedded CP.
	
	In view of this state of affairs, we would like to suggest that
	despite this biclausal character, long-distance agreement in Hinuq and
	Khwarshi (and Tsez, and perhaps more generally) does indeed involve
	some form of restructuring, albeit in the form of a special type of
	complex predicate formation. In standard lexical approaches to complex
	predicate formation (see, e.g., \citealt{Haider:93,Haider:10},
	\citealt{Kiss:95}, \citealt{Stiebels:96}, and \citealt{SMueller:02} on
	German, or \citealt{Butt:95} on Hindi/Urdu), all lexical
	subcategorization information of the verbs that participate in the
	operation is unified by functional composition.  This results in one
	featural array for the complex predicate and monoclausality
	throughout. Against the background of the present approach, this would
	imply a unique list of structure-building and probe features
	associated with the complex predicate, with the features discharged
	one after the other. In contrast, we adopt a version of pre-syntactic
	complex predicate formation where two predicates (two verbs, in the
	case at hand) are combined into a complex category in a way that, crucially,
	leaves the verbs' individual lexical information intact -- i.e., there
	are still two separate lists of features triggering syntactic
	operations.\footnote{For present purposes, it is immaterial whether
		this pre-syntactic component is conceived of as the lexicon, or as a
		pre-syntactic morphology domain; for concreteness, we will generally assume
		the former here.}
	
	\subsubsection{Derivations}
	
	Let us now look at how long-distance agreement in Nakh-Dahestanian
	languages (and possibly elsewhere) can be derived on the basis of an
	approach in terms of reprojection and pre-syntactic complex predicate
	formation. Throughout, we will assume a CP status of the embedded
	clause, with both  {\itshape \textcrlambda in}-type and {\itshape {\l}i}-type
	markers qualifying as C heads. The definition of Agree that we will
	adopt is similar but not identical to the one in (\ref{Agree}) above; it
	is given in (\Next).\footnote{Since the PIC holds for all syntactic
		operations, the fact that Agree is also subject to this constraint
		does not have to be mentioned explicitly. As noted above, we will
		not address the question here of how exactly other cases of Agree that
		would at first sight seem to violate the PIC can be accounted for;
		but note that this issue is even more prominent (though not
		categorially different) in an approach where all phrases are
		phases. 
		
		The requirement in (\ref{agua2}) permits both upward and downward Agree; see
		\citet{Zeijlstra:12} and \citet{BjorkmanZeijlstra:14} vs. \citet{Preminger:13:tha}.
		(The local Agree operation initiated by (what will become) the matrix verb in
		long-distance agreement will involve upward Agree.) (\ref{agua3})
		ensures minimality, with closeness definable in terms of minimal path
		length. There is no defective intervention here: Discharged
		features on intervening  heads and checked features on intervening phrases can be ignored. Finally,
		(\ref{agua4}) encodes the Activity Condition: An active feature is one
		that has not participated in Agree.}
\newpage	\protectedex{
	\ea\label{ex:mueller:29} {\itshape Agree}:\\
	$\alpha$ can Agree with $\beta$ if (a)--(d) hold.
	\ea $\alpha$ carries a probe feature [$*$F$*$], and $\beta$ carries a
	maching goal feature [F].
	\ex $\alpha$ \label{agua2}c-commands $\beta$, or $\beta$ c-commands $\alpha$.
	\ex There is \label{agua3}no $\delta$ that is closer to $\beta$ than $\alpha$ and
	also carries [$*$F$*$], and there is no $\gamma$ that is closer to
	$\alpha$ than $\beta$ and carries an active [F].
	\ex $\beta$ bears an \label{agua4}active feature. 
	\z
	\z}
	The syntactic derivation of a sentence such as (\ref{cp3}) in Hinuq,
	where the matrix verb undergoes long-distance agreement with the
	embedded absolutive wh-phrase, starts with the complex predicate in
	(\Next); (\ref{cp3}) is repeated here as (\NNext) (with the default
	agreement option ignored).\footnote{\label{vari2}One may ask why it is
		that V\sub{2} (which will eventually become the embedded verb) projects
		in this structure, rather than V\sub{1} (which will end up as the matrix
		verb).  As a matter of fact, there does not seem to be a good reason
		why the alternative representation where V\sub{1} projects -- {\it
			[\sub{V\sub{1}}[\sub{V\sub{1}}know]-[\sub{V\sub{2}}kill]]} -- should be excluded as such:
		With two bare X$^0$ heads forming a complex structure, labelling can
		be expected to be free. However, as will become clear when we look
		at the derivation for (\ref{cp3d}), choosing an initial representation
		{\itshape [\sub{V\sub{1}}[\sub{V\sub{1}}know]-[\sub{V\sub{2}}kill]]} of the complex predicate
		(rather than {\itshape [\sub{V\sub{2}}[\sub{V\sub{1}}know]-[\sub{V\sub{2}}kill]]}) can never lead
		to a well-formed derivation: V\sub{2} ultimately needs to combine with a
		DP, but after (extended) projection of V\sub{1} has been completed, only
		a CP is available for V\sub{2}, and this makes it impossible to discharge
		the [{\small $\bullet$}D{\small $\bullet$}] feature of V\sub{2}. In the
		same way, V\sub{1} needs to combine with a CP (due to [{\small
			$\bullet$}C{\small $\bullet$}] on its feature list), but such a
		CP is not available at the beginning of the derivation.}
	
	\ea\relax\label{ex:mueller:30} {[}\sub{V\sub{2}} [\sub{V\sub{1}} know] [\sub{V\sub{2}} kill]]\z
	
	\ea\label{ex:mueller:31}  \label{cp3d}
	\gll  Debez \o -eq'i-ye [\sub{CP(V)} k'a\v{c}a\textgamma -za-y \textbeltl u \o -uher-i\v{s}-\textbeltl i~]~? \\
	you.{\scshape sg.dat} {\scshape i}-know-{\scshape q} {} bandit-{\scshape obl.pl-erg}   who {\scshape i}-kill-{\scshape res-abst} \\ 
	\glt `Do you know whom the bandits killed?'
	\z
	In the first step, {\itshape [\sub{V\sub{2}}[\sub{V\sub{1}}know]-[\sub{V\sub{2}}kill]]} is merged
	with  the internal argument DP, triggered by  [{\small
		$\bullet$}D{\small $\bullet$}] on V\sub{2}. The resulting
	representation is shown in (\ref{Baum}).
	\begin{figure}[!h]
	\begin{exe}
		\ex	\label{ex:mueller:32} {\textit{Long-distance agreement by reprojection, first stage}\label{Baum}}:\\
			\begin{forest}	
				[VP
				[V\sub{2}
				[V\sub{1{,}{[}$\bullet$C$\bullet${]}{,}{[}$\ast$$\phi$$\ast${]}{,}{[}\textit{$\ast$inf--st$\ast$}{]}} ]
				[V\sub{2{,}\underline{{[}$\ast$$\phi$$\ast${]}}}] ]
				[DP\sub{\textit{abs}{,}{[}$\phi${]}{,}{[}\textit{inf--st}{]}}]
				] 	
		\end{forest}
	\end{exe} \vspace{-0.8cm}
\end{figure}
\newpage \noindent	Each of the verbs involved has its own $\phi$-probe
	(see \citealt{BejarRezac:09}), which is checked through Agree with the
	$\phi$-feature on the internal argument DP. Since V\sub{2} is the head of
	the complex predicate, its $\phi$-probe intervenes between V\sub{1}'s
	$\phi$-probe and the internal DP.  Thus, V\sub{2}'s $\phi$-probe has to be
	discharged first; afterwards, V\sub{1} can discharge its  $\phi$-probe via
	Agree with DP. This
	derives the generalization that long-distance agreement (i.e., under
	present assumptions, extremely local agreement of V\sub{1} and the absolutive DP)
	is possible only if embedded agreement (i.e., agreement of V\sub{2} and the
	absolutive DP) has taken place.\footnote{Given that only absolutive
		DPs can act as agreement controllers in the languages currently
		under consideration, the question arises whether this information is
		already locally available in the structure in (\ref{Baum}). There are
		two possibilities, both of which strike us as viable. First, the
		absolutive (vs. lexically case-marked) nature of an internal DP
		argument might indeed already be visible at this stage (e.g.,
		because V\sub{2} does not have a lexical case feature). Second, if
		absolutive is not identifiable yet at this stage, agreement could
		simply take place in the hope that it will later emerge (e.g., be
		assigned by a functional head like T) -- if it does not, the
		derivation will eventually crash.}
	
	In addition to the dependence of long-distance agreement on local
	agreement, a second generalization about long-distance can be derived
	at this point: There must be an obligatory information-structural
	reflex on the DP participating in long-distance agreement (with an
	interpretation as topic in Tsez, as topic or focus in Hinuq and
	Khwarshi, etc.); this is simply signalled by [inf-st] in (\Last) (i.e.,
	[inf-st] stands for [topic], [topic, focus], or other
	information-structural features). Here is why: 
	Given  \citegen{Chomsky:01} {\itshape Activity Condition}, after $\phi$-Agree
	with V\sub{2}, DP\sub{\textit{{abs}}} in (\ref{Baum}) can only undergo $\phi$-Agree
	with V\sub{1} if it still has {\itshape a different}, {\itshape active} feature
	that V\sub{1} is looking for; and [inf-st] fulfills this role.
	This explains the presence of [$\ast$inf-st$\ast$] on V\sub{1} that needs to undergo
	Agree with [inf-st] on  DP\sub{\textit{{int}}}. As a consequence of this second
	Agree operation involving V\sub{1} and DP\sub{\textit{{int}}}, V\sub{1} is equipped with the
	information that DP\sub{\textit{{int}}} is a topic. 
	
	In the further course of the derivation, V\sub{2} first discharges all its
	structure-building features (if it has any such features
	left). Subsequently, v merges with VP; after that it merges 
	with an external argument DP; and then it assigns ergative case to it. At
	this point, V\sub{1} has not yet had a chance to discharge its [{\small
		$\bullet$}C{\small $\bullet$}] feature.\footnote{This
		feature is either lower on the list of operation-triggering
		features of V\sub{1} than the probe features for agreement with
		DP\sub{\textit{{int}}}, or there are actually two separate stacks involved here
		(as indicated in (\ref{Baum})):
		one for structure-building features, and one for probe
		features. This second option might be preferable on conceptual and empirical
		grounds; see \cite{Mueller:04:arg,Mueller:09:eao} for discussion.}
	Therefore, before
	the vP is completed, V\sub{1} needs to move to v's specifier position, so
	as to comply with Featural Cyclicity (cf. (\ref{fc})). The resulting
	representation is shown in (\Next).
	
	\begin{figure}[!h]
		\begin{exe}
			\ex	\label{ex:mueller:33} {\textit{Long-distance agreement \label{Baum2}by reprojection, second stage}}:\\
				\forestset{nice empty nodes/.style={for tree={calign=fixed edge angles}, delay={where content={}{shape=coordinate, for current and siblings={anchor=north}}{}}
					},
				} 
				\begin{forest} 	for tree={fit=rectangle}
					[vP
					[V\sub{1{,}{[}$\bullet$C$\bullet${]}{,}{[}\underline{$\ast$$\phi$$\ast$}{]}{,}{[}\underline{\textit{$\ast$inf--st$\ast$}}{]}} ]
					[v$'$
					[DP\sub{\textit{erg}}]
					[v$'$
					[v] 
					[VP
					[V\sub{2}
					[t\sub{1}]
					[V\sub{2{,}\underline{{[}$\ast$$\phi$$\ast${]}}}  ] ]
					[DP\sub{\textit{abs}{,}{[}$\phi${]}{,}{[}\textit{inf--st}{]}}]
					] ] ] ]	
			\end{forest}
		\end{exe} \vspace{-0.6cm}
	\end{figure}
\noindent In further steps, the TP and CP structures of (what will become) the
	embedded clause are generated by Merge and Agree operations, while V\sub{1} moves up
	the developing syntactic structure, via intervening phase
	edges. Finally, when the CP is completed, and V\sub{1} has moved to C's
	edge domain because it still has not been able to discharge its
	structure-building feature [{\small
		$\bullet$}C{\small $\bullet$}], V\sub{1} is in a position from which it
	can undergo reprojection movement, take the CP generated so far as
	its complement (thereby discharging [{\small
		$\bullet$}C{\small $\bullet$}]), and create a matrix
	VP.\footnote{In addition to subcategorization, V\sub{1} carries out an
		Agree operation with C that reflects the embedding of an
		interrogative ([+wh]) clause.} This is
	shown in (\Next).\footnote{The mechanics here are similar to 
		\citegen{Martinovic:15} analysis of the left periphery of Wolof
		in terms of head splitting and reprojection.}   
\begin{figure}[!h]
	\begin{exe} 
		\ex	\label{ex:mueller:34} {\textit{Long-distance agreement \label{Baum3}by reprojection, third stage}}:\\
			\forestset{nice empty nodes/.style={for tree={calign=fixed edge angles}, delay={where content={}{shape=coordinate, for current and siblings={anchor=north}}{}}
				},
			} 
		\scalebox{0.8}{
			\begin{forest} 	for tree={fit=rectangle}
				[VP
				[V\sub{1{,}{[}\underline{$\ast$$\phi$$\ast$}{]}{,}{[}\underline{\textit{$\ast$inf--st$\ast$}}{]}}]
				[CP
				[t$^{'''}_{1}$]
				[C$'$
				[C]
				[TP
				[t$^{''}_{1}$]
				[T$'$
				[T]
				[vP
				[t$^{'}_{1}$]
				[v$'$
				[DP\sub{\textit{erg}}]
				[v$'$
				[v] 
				[VP
				[V\sub{2}
				[t\sub{1}]
				[V\sub{2{,}\underline{{[}$\ast$$\phi$$\ast${]}}}] ]
				[DP\sub{{[}\textit{inf--st}{]}}]
				] ] ] ] ] ] ] ] ]	
		\end{forest}}
	\end{exe} \vspace{-0.8cm}
\end{figure} 		
\newpage \noindent The resulting representation is opaque in \citegen{Kiparsky:73:abs}
	sense as it involves a counter-bleeding interaction of operations
	(also cf.  \citealt{Chomsky:51}, \citealt[25-26]{Chomsky:75:the}):
	Reprojection movement of V\sub{1} would bleed Agree with DP\sub{\textit{{abs}}}
	(which requires strict locality, due to the PIC) but fails to do so
	because it applies too late: When V\sub{1} has left the local domain in
	which agreement  with DP\sub{\textit{{abs}}} can legitimately be carried out, this
	agreement has already taken place. 
	
	From this point onwards, everything happens exactly as one would
	expect it to (with matrix vP, TP, and CP generated by Merge and Agree
	operations), and there is basically no difference anymore to
	derivations in which there is no complex predicate formation to begin
	with.  Of course, given that pre-syntactic (lexical) complex predicate
	formation is an optional process, this second kind of derivation can
	be assumed to underlie minimally different sentences in which
	long-distance agreement does not occur. Thus, the two strategies
	differ substantially as far as earlier stages are concerned, but they
	end up with exactly the same structures once the matrix domain has
	been reached. There is one qualification, though. 
	As a consequence of reprojection movement of V\sub{1}, [inf-st] of
	DP\sub{\textit{{int}}} is
	transported into the matrix clause.\footnote{Note that this implies
		that {\itshape discharged} features, while syntactically inert, are not
		actually deleted. This assumption
		must independently be made for
		discharged probe features more generally that give rise to
		morphological realization; see \cite{Adger:03}.} 
	The information that the embedded DP\sub{{\rm\textit{ [inf-st]}}}
	is interpreted as a topic is therefore shifted to the matrix
	sentence, and consequently, a DP that is affected by long-distance
	agreement is interpreted as the topic of the entire complex sentence.
	The analysis is thus consistent with usual assumptions concerning the impossibility
	of information-struc\-tural elements in clauses without
	illocutionary force (like non-assertive, presuppositional
	declarative clauses). Whereas information-structural features of an
	embedded DP can thus be interpreted in the matrix clause, there is
	no way how an embedded DP could take relative scope in the matrix clause as
	well (cf. the sentences in (\ref{scope32})): Relative scope is
	determined by the position of an item, not by features, and there is no stage of
	the derivation where the embedded DP\sub{\textit{{abs}}} would show up in the
	matrix clause. 
	
	\subsubsection{Further consequences}
	
	The example of long-distance agreement that we have considered here on
	the basis of the sample derivation in (\ref{Baum}), (\ref{Baum2}) and
	(\ref{Baum3}), involves a DP\sub{\textit{{abs}}} controller that is also a
	wh-phrase. However, it should be clear that the approach generalizes
	to all the other cases of long-distance agreement mentioned
	above. For instance, an account in terms of reprojection of the part
	of a complex predicate straightforwardly derives long-distance
	agreement as in (\ref{2-b}) in Tsez and in (\ref{3-b}) in Hinuq (where it can
	now be assumed that $\alpha$ stands for a full CP). Similarly,
	examples like (\ref{16-b}) in Khwarshi (with a DP\sub{\textit{{abs}}} controller
	acting as a topic, i.e., [inf-st] representing [topic]) and (\ref{17-b})
	in Hinuq (with a DP\sub{\textit{{abs}}} controller acting as a focus, i.e., [inf-st]
	representing [focus]) are directly accounted for under the present
	analysis. Examples (\ref{cp1}) (from Khwarshi) and (\ref{cp2}) (from Hinuq) have
	subject wh-phrases (one marked by ergative, one not) that do not block
	long-distance agreement with the absolutive DP. Again, this is
	expected under present assumptions: Independently of whether the
	wh-phrase here occupies SpecC in overt syntax or not, reprojection
	movement of the verb to the matrix domain is possible (given the
	general option of multiple specifiers, particularly for intermediate
	movement steps). 
	
	Next, instances of of super-long-distance agreement where the
	agreeing verb and the agreement controller DP are separated by two
	intervening CP boundaries, like (\ref{tsa1}) in Tsakhur or (\ref{tsa2}) in
	Hinuq, can also be addressed under the reprojection approach: Here a complex
	predicate is formed pre-syntactically where V\sub{1} (which will become the
	highest verb) and V\sub{2} (which will become the intermediate verb) are
	first combined, with V\sub{2} projecting (in a successful derivation;
	cf. footnote \ref{vari2}), and then the complex V\sub{2} category is
	combined with V\sub{3} (which will become the most deeply embedded verb),
	with V\sub{3} projecting, as shown in (\Next).
	
	\ea\relax\label{ex:mueller:35} {[}\sub{V\sub{3}} [\sub{V\sub{2}} V\sub{1} V\sub{2}] V\sub{3}]\z
	Here V\sub{3}, V\sub{2}, and V\sub{1} first carry out Agree operations with V\sub{3}'s
	internal argument (DP\sub{\textit{{abs}}}), and then a CP is generated on top of
	VP\sub{3}, with the complex V\sub{2} moving successive-cyclically to
	intermediate phase edge positions, until it finally merges with the
	CP. Then, the second, intermediate, CP is generated, with V\sub{1}
	excorporating from the complex {\itshape [\sub{V\sub{2}} V\sub{1} V\sub{2}]} category and
	moving via the intermediate CP's phases edges until, finally, the
	intermediate CP has been completed and V\sub{1} can take this CP as its
	internal argument, via reprojection. The Tsakhur example in (\ref{tsa1})
	and the Hinuq example in (\ref{tsa2}) 
	fully correspond to this scenario, with V\sub{1}, V\sub{2}, and V\sub{3} all
	participating in agreement with DP\sub{\textit{{abs}}}. However, examples
	involving super-long-distance agreement like the one in 
	(\ref{tsa3}) (from Hinuq) can also be found. 
	
	\ea\label{ex:mueller:36} 
	\gll Di\v{z} \label{tsa3}y-eq$'$i-yo [\sub{CP} \textglotstop umar-i [\sub{CP} Madina  y-aq$'$-es=t\l en~] ese-s-\l i~]\\
	I.{\scshape dat} II-know-{\scshape prs} {} Umar-{\scshape erg} {} Madina(II) II-come-{\scshape pst}={\scshape quot} tell-{\scshape res}-{\scshape abst}\\
	\glt `I know that Omar said that Madina came.'
	\z
	In (\Last), the intermediate verb V\sub{2} does in fact not exhibit overt
	agreement marking even though both the matrix verb V\sub{1} and the most
	deeply embedded verb V\sub{3} do. Still, (\Last) does not call into question
	the present approach: It can plausibly be assumed that $\phi$-feature
	agreement is indeed present on V\sub{2}, but fails to be registered overtly
	(there are many verbs that fail to exhibit visible agreement marking
	despite showing up in the proper syntactic context in
	Nakh-Daghestanian languages, and the reason for this is presumably
	simply a morphological one). Thus, all in all, super-long-distance
	agreement can be derived.\footnote{It should be mentioned that there
		are two further complications, though. First, recall that the
		present account of the obligatory information-structural reflex of
		long-distance agreement in terms of the Activity Condition would,
		strictly speaking, require two different additional features (next
		to the $\phi$-probes) on V\sub{1} and V\sub{2}, and not just one, as in the
		cases discussed so far. It is not a priori clear what
		this extra feature might be. However, it has been argued that
		information-structural features like topic and focus do not qualify
		as primitives, but are rather composed of more primitive binary
		features (so as to capture natural classes of information-structural
		categories), like [$\pm$new], [$\pm$prom] (with, say, topic emerging
		as [--new,+prom]); see \citet{Choi:99}, based on
		\citet{Vallduvi:92}). If so, V\sub{2} and V\sub{1} can be equipped with
		separate pieces of [inf-st] information.
		
		Second, \citet{Forker:11} also maintains that it is not completely
		impossible in Hinuq to have super-long-distance agreement involving
		V\sub{1}, V\sub{3}, and D\sub{\textit{{abs}}} in the most deeply embedded clause, not merely
		in the absence of agreement on V\sub{2} (as in \ref{tsa3}), but in the presence of a {\it
			different} agreement on V\sub{2}.  If such sentences (which  Forker assigns
		an intermediate status, signalled by ``?'') can be substantiated as
		grammatically well formed, additional assumptions that complement the
		present analysis will be called for.}
	
	\noindent A further property of long-distance agreement that needs to be
	accounted for concerns \citegen{PolinskyPotsdam:01} observation
	that the C element {\itshape \textcrlambda in} blocks the operation in Tsez
	(cf. section \ref{sec-bjoe-muel:2.4} above). Given that there is good evidence that
	long-distance agreement across CP is possible in principle in
	Nakh-Daghestanian languages, and given that we have analyzed the
	transparent morpheme {\itshape {\l}i} as a C item, too, a recourse to a
	general blocking nature of C is not available in the present
	approach. Also, it is not possible to claim that a reprojecting V\sub{1}
	cannot merge with a CP headed by {\itshape \textcrlambda in}: First, the
	[{\small $\bullet$}C{\small $\bullet$}] feature responsible for
	reprojection movement is not sensitive to a difference between C
	heads, and an additional selection relation (mediated by Agree) would
	have to be stipulated; second (and more importantly), [{\small
		$\bullet$}C{\small $\bullet$}] on a reprojecting V is exactly the
	same feature as [{\small $\bullet$}C{\small $\bullet$}] on a V that
	fails to undergo complex predicate formation, and successfully takes
	CP complements headed by {\itshape \textcrlambda in} in environments
	without long-distance agreement. In view of this, we would like to
	suggest that the blocking effect of a C head {\itshape \textcrlambda in} is
	due to the fact that it does not permit a specifier. Thus, the problem
	with long-distance agreement in these contexts can be traced back to
	the unavailability of intermediate movement of a V\sub{1} that is initially
	part of a complex predicate, to SpecC: As a consequence, the final
	reprojection step of V\sub{1} will have to fatally violate the PIC (V\sub{1}
	can only reach SpecT, which is not accessible anymore once CP has been
	completed).
	
	Finally, we would like to point out that the present approach in terms
	of reprojection makes a very simple prediction: Reprojection movement
	of a verb by definition creates a head-complement structure; there is
	no way how a specifier or adjunct could be involved (since this would
	require a non-X$^0$ category to move). Therefore, long-distance
	agreement is expected never to occur into subject clauses or adjunct
	clauses.  This prediction is borne out: Long-distance agreement always
	involves complement clauses.
	
	
	\section{Conclusion} \label{sec-bjoe-muel:4}
	
	We have argued that from the point of view of a model of syntax where
	all operations apply in strictly local domains (as defined by the
	Phase Impenetrability Condition (PIC)), and in the face of empirical
	evidence showing that long-distance agreement can involve a matrix
	verb and an agreement-controlling DP separated by a CP, none of the
	existing approaches to long-distance agreement (non-local analyses,
	small structure analyses, cyclic Agree analyses, and analyses where
	movement to the edge feeds agreement) work satisfactorily. In view of
	this, we have developed a new approach in terms of pre-syntactic
	complex predicate formation and reprojection: The derivation starts
	out with a complex verb {V\sub{1}-V\sub{2}} headed by V\sub{2}, so that agreement
	of V\sub{1} with DP can apply early in the derivation (not late, as in
	other approaches), in an extremely local domain, and subsequent
	reprojection movement of V\sub{1} turns the latter into a matrix verb, thereby
	masking the locality of agreement and creating opacity (viz.,
	counter-bleeding) in syntax.  
	
	This approach may at first sight look quite radical. However, it is
	worth bearing in mind that it suggests itself without further ado once
	two widely employed operations are adopted and combined, viz., (i)
	pre-syntactic complex predicate formation, and (ii) head movement as
	reprojection. The properties that these two operations must have for
	the analysis to work all qualify as independently motivated, and they
	often correspond to standard assumptions in the field (in analyses
	that adopt the operations). As a matter of fact, the only innovative
	assumption that we have come up with is that pre-syntactic complex
	predicate formation does not (or does not have to) result in a single
	list of structure-building and agreement-inducing features (via a
	process of functional composition), but can maintain the integrity and
	independence of the two individual lists of structure-building and
	agreement-inducing features.
	
	Nevertheless, ideally there should be independent evidence for the
	type of interaction of complex predicate formation and reprojection
	movement that is at the heart of the present analysis of long-distance
	agreement. To end this paper, we would like to briefly sketch an
	approach to an entirely different phenomenon that works in the same
	way, viz., extraction from DPs in German. 
	
	As for the empirical evidence, extraction is impossible from subject
	DPs and indirect object (dative-marked) DPs. This is shown (with
	wh-movement as the extraction operation) in (\ref{35a}) and (\ref{35b}),
	respectively.
	
	\protectedex{
	\ea\label{ex:mueller:37}
	\ea
		\gll*\relax [\sub{PP} \"Uber wen]\sub{1} \label{35a}hat [\sub{DP} ein Buch t\sub{1}] den Karl beeindruckt?\\
		{} about whom has {} a book.{\scshape nom} {} the Karl.{\scshape acc} impressed\\
		\glt `For which person is it the case that a book about that person impressed Karl?'
	\ex
		\gll *[\sub{PP} \"Uber wen] hat \label{35b}sie [\sub{DP} einem Buch t\sub{1}] keine Chance gegeben?\\
		{} about whom has she.{\scshape nom} {} a book.{\scshape dat} {} no chance.{\scshape acc} given\\
		\glt `For which person is it the case that she gave a book about that person no chance?'
	\z 
	\z}
	With extraction from direct object (accusative-marked) DPs, things are
	somewhat more variable: With some combinations of V and N, extraction
	is possible (see (\ref{36a})), with other combinations, it is not (see
	(\ref{36b})). 
	
	\ea\label{ex:mueller:38}
	\ea[]{
		\gll\relax [\sub{PP} \"Uber wen]\sub{1} \label{36a}hat Karl [\sub{DP} ein Buch t\sub{1}] gelesen~?\\
		{} about whom has Karl.{\scshape nom} {} a book.{\scshape acc} {} read\\
		\glt `Who did Karl read a book about?'}
	\ex
		\gll*\relax [\sub{PP} \"Uber wen]\sub{1} \label{36b}hat Karl [\sub{DP} ein Buch t\sub{1}] geklaut~?\\
		{} about whom has Karl.{\scshape nom} {} a book.{\scshape acc} {} stolen\\
		\glt `For which person is it the case that Karl stole a book about that person?'
	\z
	\z
	Thus, both structural and lexical factors play a role: On the one
	hand, extraction from DP can be well formed in German if DP is a complement
	(as with direct objects in (\ref{36a}) and (\ref{36b}), but not if it is a specifier (as in
	with subjects and indirect objects in (\ref{35a}) and (\ref{35b}), which can be assumed
	to occupy Specv and SpecAppl positions, respectively). On the other
	hand, extraction from DP also requires V and N to form a tight unit,
	or a ``natural predicate''. This latter status is arguably determined
	both by semantic considerations and by extralinguistic factors
	(frequency, entrenchment), and it may to some extent vary from speaker
	to speaker. Still, it must be modelled in the grammar in some way. In
	\citet{Mueller:91:abs} and \citet{MuellerSternefeld:95}, it is proposed
	that the relevant concept is that of abstract incorporation (in
	\citealt{Baker:88}'s sense, conceived of as incorporation at LF that
	is signalled already by co-indexation of heads in overt syntax): V
	({\itshape read}) and N ({\itshape book}) in (\ref{36a})) undergo abstract
	incorporation and thus form a natural predicate, whereas V ({\it
		steal}) and N ({\itshape book}) in (\ref{36b}) do not (for most speakers). Given that
	the theory of locality constraints on movement is sensitive to this
	difference (as well as to the structural difference between
	complements and specifiers), the data in (\LLast) and (\Last) can then all
	be accounted for. Similar approaches in terms of abstract
	incorporation have subsequently been developed by
	\citet{DaviesDubinsky:03} and \citet{Schmellentin:06}. However, there
	are problems with this kind of approach. In particular, on this view
	abstract incorporation of N into V must either be able to apply
	non-locally, across an intervening DP projection (plus, possibly,
	other functional projections in the DP that may intervene between D
	and N); or the analysis must abandon the DP-over-NP hypothesis. To be
	sure, there are ways out for the abstract incorporation
	approach.\footnote{For instance, in \citet{Mueller:11:loc}, abstract incorporation is
		viewed as a regular syntactic Agree operation, with no actual
		movement involved.} Still, it can be noted that an approach based on
	complex predicate formation plus reprojection can account for the data
	in a very simple way. 
	
	Thus, suppose that some combinations of V and N can undergo
	pre-syntactic (lexical) complex predicate formation whereas others
	cannot do so. This means that complex heads like the one in (\Next) can
	be primitive inputs of Merge operations in the syntax.\footnote{As
		before, the head that will ultimately come to occupy a higher
		(c-commanding) position must be the one that fails to project initially in a
		well-formed derivation; cf. footnote \ref{vari2}.}
	
	\ea\relax\label{ex:mueller:39} {[}\sub{N\sub{2}} V\sub{1} N\sub{2}] \z
	In the ensuing derivation, N\sub{2} first discharges its structure-building
	and probe features (thereby undergoing Merge with a PP); see (\ref{ex:mueller:40a}).
	Then
	DP is added on top of NP (and possibly also other functional projections
	before that), with V\sub{1} undergoing intermediate, Last Resort-driven
	movement to SpecD; see (\ref{ex:mueller:40b}). After that, V\sub{1} undergoes the
	final  movement step to take DP as its complement and thereby
	discharge its [{\small $\bullet$}D{\small $\bullet$}] feature; see
	(\ref{ex:mueller:40c}). From this point onwards, everything proceeds exactly as in
	a derivation where there is no complex predicate formation as in
	(\Next); such a derivation produces the VP in (\NNext). 
	
	\ea\label{ex:mueller:40}
	\ea\label{ex:mueller:40a}\relax [\sub{NP} [\sub{N\sub{2}} V\sub{1} N\sub{2}] PP]
	\ex\label{ex:mueller:40b}\relax [\sub{DP} V\sub{1} [\sub{D$'$} D [\sub{NP} [\sub{N\sub{2}} t\sub{1} N\sub{2}] PP]]]
	\ex\label{ex:mueller:40c}\relax [\sub{VP} V\sub{1} [\sub{DP} t$'$\sub{1} [\sub{D$'$} D [\sub{NP} [\sub{N\sub{2}} t\sub{1} N\sub{2}] PP~]]]]
	\z
	\z
	
	\ea\label{ex:mueller:41} {[}\sub{VP} V\sub{1} [\sub{DP} D [\sub{NP} N\sub{2} PP]]]\z
	Importantly, both the NP and the PP in (\ref{ex:mueller:40c}) (based on complex
	predicate formation of V and N) have been in an extremely local
	(object-like) relation with V whereas the NP and PP in (\Last) (based
	on regular, separate projection of V and N) have never been in a local
	relation with V. Without going into the details of how exactly this
	will best be implemented in a given theory of locality restrictions on movement, it
	seems plausible to assume that it is the extremely local relation of V
	and NP/PP at an earlier derivational step that makes extraction of PP
	from DP possible in  (\ref{36a}), and it is the absence of such a relation that blocks
	the movement in (\ref{36b}). As before, reprojection movement of V
	by definition cannot take place from specifiers, which then accounts
	for the illformedness of (\ref{35a}) and (\ref{35b}).
	
	\section*{Abbreviations}
	\begin{multicols}{2}
		\begin{tabbing}
			\textsc{intrans}\hspace{5mm} \= information structure\kill
			\textsc{abs} \> absolutive\\ 
			\textsc{abst} \> abstract suffix\\ 
			\textsc{acc} \> accusative\\
			\textsc{caus} \> causative\\ 
			\textsc{cl} \> clitic\\ 
			\textsc{cop} \> copula\\ 
			\textsc{cvb} \> narrative converb\\ 
			\textsc{dat} \> dative\\ 
			\textsc{erg} \> ergative\\ 
			\textsc{fem} \> feminine\\ 
			\textsc{foc} \> focus\\ 
			\textsc{inf} \> infinitive\\ 
			\textsc{inf-st} \> information structure\\ 
			\textsc{intrans} \> intransitive\\ 
			\textsc{lat} \> lative\\ 
			\textsc{masc} \> masculine\\ 
			\textsc{nom} \> nominative\\ 
			\textsc{nmlz} \> nominalizer\\ 
			\textsc{obj} \> object\\ 
			\textsc{obl} \> oblique\\
			\textsc{perf} \> perfective\\
			\textsc{pl} \> plural\\ 
			\textsc{pr(e)s} \> present\\ 
			\textsc{pst} \> past\\ 
			\textsc{pstprt} \> past participle\\
			\textsc{ptcp} \> participle\\ 
			\textsc{q} \> question particle\\ 
			\textsc{quot} \> quotative enclitic\\
			\textsc{res} \> resultative participle\\ 
			\textsc{sg} \> singular\\ 
			\textsc{subj} \> subject\\ 
			\textsc{ti} \> transitive inanimate
		\end{tabbing} 
	\end{multicols}
	
	\section*{Acknowledgments}
	For comments and discussion, we are grateful to Josef Bayer, Elena
	Pyatigorskaya, Tonjes Veenstra, Martina Martinovi\'c, Johannes Mursell
	(and the other editors of the present volume), and audiences at the
	{\itshape GGS} meeting at Universit{\"a}t Konstanz (2014), the workshop on
	{\itshape Agreement} at the University of York (2014), and the workshop
	{\itshape Heads, Phrases, Ts and Nodes} at Humboldt-Universit{\"a}t Berlin (2017).
	
	{\sloppy
		\printbibliography[heading=subbibliography,notkeyword=this]}
\end{document}



