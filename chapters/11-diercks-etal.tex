\documentclass[output=paper
,modfonts
,nonflat
]{langsci/langscibook} 




\title{Agree probes down: Anaphoric feature valuation and phase reference
} 

\author{Michael Diercks\affiliation{Pomona College}\and 
 Marjo van Koppen\affiliation{Utrecht University/Meertens Institute}\lastand 
 Michael Putnam\affiliation{Penn State University}
}


% \chapterDOI{} %will be filled in at production
% \epigram{}

\abstract{This paper investigates the question of the directionality of Agree in the domain of complementizer agreement (CA). Germanic and Bantu patterns of CA provide \textit{prima facie} evidence of both downward and upward-probing relations, as Germanic complementizers are valued by the subject of the embedded clause, whereas the relevant Lubukusu complementizers are valued by the subject of the main clause. We argue, however, that all feature valuation relations can be explained by a downward-probing Agree operation. Apparent instances of upward-probing feature-valuation are analyzed as anaphoric feature valuation, which is a composite operation consisting of movement of the relevant (unvalued, interpretable) features followed by probing of their c-command domain for valuation. We propose that the behavior of anaphoric features can be derived from more fundamental syntactic properties using a model of syntax that relies on the referential properties of phases: more rigid reference of a phase is derived by movement of phase-internal elements to the edge of that phase.
}

\begin{document}
\maketitle

%\setcounter{tocdepth}{3}
%\tableofcontents

\section{Introduction} \label{Introduction}

The Minimalist Program (MP) (\citealt{Chomsky2000} et seq.) posits the notion of Agree: a local feature valuation relation that is constrained by a c-command relation between a Probe bearing an unvalued feature [uF] and a structurally lower Goal bearing an interpretable variant of this feature [iF]. Agree has come to be the principal mechanism for various kinds of feature-matching relationships in syntactic theory, and as such the subject of intensive research and interesting debates. Recent literature provides (at least) three different theoretical approaches to the Agree operation:

\ea
Theoretical Approaches to Agree \label{TheoryApproaches}
\begin{xlist}
\ex Agree is the result of a structurally higher Probe probing down (\citealt{Chomsky2000}, \citealt{Chomsky:2001}, \citealt{Preminger:2013}, \citealt{Polinsky:2015}) 
\ex Agree is the result of a structurally lower Probe probing up \citep{Zeijlstra:2012, Wurmbrand:2011, Bjorkman:toappearb} 
\ex Agree can probe up or down (\citealt{Bejar:2009}\footnote{\citet{Bejar:2009} do not propose that a Probe can probe upwards, but they argue that unvalued features of a Probe can be reintroduced higher up in the tree and Probe down from there again. This gives the surface appearance of Upward Probing, but is in effect downward probing. We will argue for something similar.}, \citealt{Baker:2008}, \citealt{Putnam:2011}, \citealt{Carstens:2016})
\end{xlist}
\z
\noindent On the surface there is a strong case for the existence of both upward and downward probing in the grammar of complementizer agreement (CA). One set of data motivating the downward-probing operation comes from the familiar West Germanic instances of CA where C agrees with the embedded subject.

\ea \label{ProbeDownSchematic}
Probing Down: [\textsubscript{XP} Probe\textsubscript{[uF]} [\textsubscript{YP} Goal\textsubscript{[iF]} ]]
\z

\ea \label{WestFlemFirstExample}
\langinfo{West Flemish}{}{\citealt{Haegeman:1992}} \\
\gll K peinzen 	da-\circled{n} /   *da\circled{-\O}        	\circled{ze} 	morgen 	goan. \\
I think that-\textsc{pl} / that-\textsc{sg} they tomorrow go.\textsc{pl} \\
\glt `I think that they will go tomorrow.' 	
\z						
\noindent \textit{Prima facie} evidence for upward probing can also be found in the complementizer domain, this time in various languages of Africa. The best-described case comes from Lubukusu, a Bantu language spoken in western Kenya \citep{Diercks:2010,Diercks:2013,Wasike:2007}; as shown in (\ref{BasicLubukusuExample}) the class 2 agreement \textit{ba-} on the complementizer \textit{-li} is triggered by the class 2 matrix subject, and not by any other potential agreement trigger in the embedded clause.\footnote{
	Every Lubukusu noun phrase in this paper is glossed for its noun class, for which we follow the Bantuist tradition of labeling by number, where odd numbers are singulars (e.g. 1) and the immediately ascendant even number is that noun class’ plural form (e.g. 2 is the plural of 1).  S or O following a verbal noun class agreement indicates ‘subject’ or ‘object’ verbal agreement.  Person features are represented by the person together with the number, for example 1sg, 2pl. Tone marking is not provided for the Lubukusu examples.}

\ea \label{ProbingUpSchematic}
Apparent Probing Up: [\textsubscript{XP} Goal\textsubscript{[iF]} [\textsubscript{YP} Probe\textsubscript{[uF]} ]]
\z

\ea \label{BasicLubukusuExample}
\langinfo{Lubukusu}{}{\citealt{Diercks:2013}}\\
\gll \circled{Ba-ba-ndu}	ba-bol-el-a	Alfredi	\circled{ba}-li	a-kha-khil-e.\\
	2-2-people 	2\textsc{sa}-said-\textsc{ap}-\textsc{fv}	1Alfred	2-that	1\textsc{sa}-\textsc{fut}-conquer-\textsc{fv} \\
\glt `The people told Alfred that he will win.'	
\z
\noindent Despite the apparent `upward' agreement in (\ref{BasicLubukusuExample}), we argue that the data in (\ref{WestFlemFirstExample}) and (\ref{BasicLubukusuExample}) can both be accounted for by the widely accepted theory that unvalued features probe their c-command domains for a Goal by which to be valued (\citealt{Chomsky2000} et seq.).  We claim that the Lubukusu φ-features on C have anaphoric properties (e.g. subject-orientation). We follow \citet{Rooryck:2011} in identifying anaphoric features as \textit{interpretable}, \textit{unvalued} features, which necessarily move to a position higher than their antecedent and undergo a standard Agree operation (Rooryck \& Vanden Wyngaerd \citeyear{Rooryck:2011} propose this is the derivation of self-reflexives). Therefore feature-valuation may either be non-anaphoric (where pure Agree results in downward-oriented syntactic agreement, \textit{contra} \citealt{Zeijlstra:2012}, \citealt{Wurmbrand:2011}, and \citealt{Bjorkman:toappearb}) or anaphoric (where an Agree relation 
is preceded by a movement operation). Therefore while only one feature-valuation operation is a primitive of the grammar (Agree), there are multiple derivative patterns: non-anaphoric agreement where the Goal is structurally lower than the Probe (`pure' Agree) and anaphoric agreement where the Goal may appear to be structurally higher than the Probe due to (covert) movement of the Probe (Internal Merge + Agree). 

Sections \ref{GermanicCA}-\ref{AnalysisSection} deal with the empirical grounds for the discussion above, and the core proposal set forward in this paper. Section \ref{ExplainPAPA} addresses why unvalued yet interpretable features should undergo internal merge by linking this movement to the Phase Reference model of \citet{Hinzen:2012} (and related work). Section \ref{CA in Kipsigis} discusses CA-data from another language spoken in Kenya\textemdash Kipsigis\textemdash that provides additional evidence for our analysis. Section \ref{sectioncarstens} compares our approach to Carstens' (\citeyear{Carstens:2016}) analysis of Lubukusu CA.

\section{Germanic CA: Agree probing down} \label{GermanicCA}

Various Dutch and German dialects display CA in which a declarative-embedding complementizer carries inflectional morphology that agrees with the φ-features of the embedded subject. The West-Flemish examples illustrate that the complementizer \textit{da} `that' displays overt plural agreement morphology (-\textit{n}) when there is a plural embedded subject, shown in (\ref{WestFlemishPluralAgr}) (with no overt agreement otherwise). 

\ea \label{InitialWestFlemishExample}
\langinfo{West Flemish}{}{\citealt{Haegeman:2012}}\\
\begin{xlist}

\ex \label{WestFlemishPluralAgr}
\gll K peinzen da\circled{-n} / *da{-\O} \circled{die venten} Marie kenn-en.\\
I 	think	that-\textsc{pl}	/	*that-\textsc{sg}	{those men}	Marie	know-\textsc{pl} \\
\glt `I think that those men know Marie.' 	

\ex \label{WestFlemishNo3rdAgr}
\gll K peinzen da\circled{-\O} / *da-n \circled{dienen vent} Marie kenn-t.\\
I think that-\textsc{sg} / *that-\textsc{pl} {that man} Marie know-\textsc{sg} \\
\glt `I think that that man knows Marie.' 	

\end{xlist}

\z
\noindent Tegelen Dutch complementizers show a slightly different pattern, displaying overt inflection (\textit{-s}) with second person singular subjects (\textit{doow} `you' in (\ref{TegelenDutchBasic})) and a bare form otherwise.\footnote{We only describe the basic properties of CA in West Germanic here. We refer the reader to the extensive literature on CA in Germanic for a more in depth description of this phenomenon (see \citealt{vanKoppen:2017} and references cited there).}

\ea \label{TegelenDutch} \label{TegelenDutchBasic}
\langinfo{Tegelen Dutch}{}{\citealt{Haegeman:2012}} \\
\begin{xlist}
\ex 
\gll Ich denk de\circled{-s} / *det \circled{doow} Marie ontmoet-s.\\
I think that-2\textsc{sg} / *that you.2\textsc{sg} Marie meet-2\textsc{sg} \\
\glt `I think that you will meet Marie.'					
\ex 
\gll Ich denk det\circled{-\O} / *de-s \circled{geej} Marie ontmoet-e. \\
I think that / *that-2\textsc{sg} you.\textsc{pl} Marie meet-\textsc{pl} \\
\glt `I think that you will meet Marie.'	 
\end{xlist}
\z
\noindent The analysis of Germanic CA that we advocate here is the same as that proposed by \citet{Carstens:2003}, \citet{vanKoppen:2005}, and \citet{Haegeman:2012}. Following this literature, we assume that C° in dialects with CA has a set of uninterpretable φ-features, which probe C°’s c-command domain for a set of matching interpretable φ-features. The first potential Goal it encounters is the embedded subject, which values the φ-features on C° that are then spelled out as CA. This derivation is represented in (\ref{BasicGermanicCATree}):

\ea \label{BasicGermanicCATree}
\begin{forest}
[CP
	[C°\\\textit{de-s}\\{[uφ]},name=source]
    [TP
    	[\textit{doow}\textsubscript{\textit{i}}\\{[φ:2\textsc{sg}]},name=target]
        [TP  
        	[T°\\{[uφ]},name=T]
        	[VP
            	[\sout{\textit{doow}}\textsubscript{\textit{i}}\\\sout{{[φ:2\textsc{sg}]}},name=BaseSubj]
            	[VP [NP\\\textit{Marie}] [V°\\\textit{ontmoets}] ]
			]
		]
	]
]
\draw[->] (source) to[out=south, in=south west] (target);
\draw[->] (T) to[out=south, in=south west] (BaseSubj);
\end{forest}
\z \vspace{-0.1cm}
\noindent We will briefly consider two different alternative analyses of the Germanic CA pattern, demonstrating that a downward-probing Agree analysis is the most probable (though we mainly point the reader to the relevant literature for discussion).\footnote{Another possible approach is that Germanic CA is non-syntactic, occurring at PF as a morphological process (cf. e.g. \citealt{Ackema:2004}, \citealt{Fuss:2008}). We refer the reader to \citet{vanKoppen:2005} and \citet{Haegeman:2012} for counter-arguments.} One alternative has been to argue that the φ-features on C° and T° have the same origin (the `shared-source' analysis).\footnote{See also \citet{Haegeman:2012} for an extensive discussion of these proposals.} One implementation of this idea is for the φ-features that arise on C° to have originated in T° (an approach amenable to a Spec,Head agreement analysis: cf. \citealt{denBesten:1983,denBesten:1989,Zwart:1993,Zwart:1997,Hoekstra:1989,Watanabe:2000}, among others). On this approach the φ-features of T° are valued by the subject in Spec,TP, after which T° (or the φ-feature set of T°) raises to C° and are realized as CA. A second implementation of the `shared source' analysis adopts a Feature Inheritance approach, which also leads to a configuration in which the subject c-commands the φ-features of Cº. More specifically, Chomsky (2008, et seq.) argues that the φ-features on C° cannot remain on Cº (because it is a phase head) and therefore have to be passed on to a non-phase head, Tº in this instance (see also \citealt{Richards:2007a}). CA can then be taken as an additional morphological reflex of agreement between T° and the subject, spelled out on Cº at the base position of those φ-features. 

\citet{Haegeman:2012} argue extensively against the `shared source' approach, showing that a key prediction is not upheld\textemdash that the φ-feature set on Tº be identical or a subset of the feature set on the Cº phase head.\footnote{This is not a claim that the morphological forms must be identical, only that (after morphological analysis) the φ-feature distinctions shown on Tº should demonstrably be the same ones shown on Cº.}  Haegeman \& van Koppen point out two key empirical problems for this hypothesis: CA with coordinated subjects in Tegelen Dutch and CA with external possessors in West Flemish, both of which result in Cº and Tº having distinct sets of φ-features. For the sake of space, we consider only the first here. A basic example of CA in Tegelen Dutch is provided in (\ref{TegelenDutchConjoinedSubj}):

\ea \label{TegelenDutchConjoinedSubj}
\langinfo{Tegelen Dutch}{}{\citealt{Haegeman:2012}} \\
\gll Ich denk de\circled{-s} \circled{doow} Marie ontmoet-\textbf{s}.\\
I think that-2\textsc{sg} you.\textsc{sg} Marie meet-2\textsc{sg} \\
\glt `I think that you will meet Marie.'		

\z
\noindent In an example with a conjoined subject like (\ref{TegelenDutchConjoinedSubjNoResolved}), the verb (i.e. Tº) agrees with the plural-feature of the entire coordinated subject \textit{doow en ich} `you and I,' but CA is solely with the person and number features (2nd singular) of the first conjunct in this coordinated subject.

\ea \label{TegelenDutchConjoinedSubjNoResolved}
\langinfo{Tegelen Dutch}{}{\citealt{Haegeman:2012}} \\
\gll … de\circled{-s} \circled{doow} en ich ôs kenn\circled{-e} treffe.\\
{} that-2\textsc{sg} [you.\textsc{sg} and I]1\textsc{pl} each.other.1\textsc{pl} can-\textsc{pl} meet \\
\glt `… that you and I can meet.'										
\z

\noindent As argued by \citeauthor{Haegeman:2012}, it is clear then that CA differs from agreement on T (TA) in (\ref{TegelenDutchConjoinedSubjNoResolved}), which is unexpected if CA and TA have a shared source (in the sense we introduced earlier). 

Having set aside the `shared source' approach to Germanic CA, there is a second alternative analysis available: C° and T° probe separately, but embedded subjects raise into the CP-domain and trigger agreement on a CP-level Agr head (AgrC). AgrC° proceeds to raise over the subject in Spec,AgrCP, producing the expected word order where the complementizer (and agreement features) precede the embedded subject (see \citealt{Shlonsky:1994} and \citealt{Zwart:1993} for a discussion of this kind of approach). Although descriptively adequate, this split-CP implementation of an upward-probing analysis of Germanic CA poses some challenges, particularly regarding first conjunct agreement (FCA) patterns in Tegelen Dutch: it is problematic that AgrC would agree with a first conjunct that is not in its complement. Upward probing accounts of CA predict this type of agreement to be impossible (i.e. agreement with an element in the specifier of the Goal), because in order for Agree to take place the Goal has to c-command the Probe, which is not the case in the FCA examples (see \citealt{Baker:2008,Zeijlstra:2012,Wurmbrand:2011}). As such, upward-probing accounts would never expect agreement with the first conjunct of a coordinated subject, contrary to fact. There are additional empirical problems, but for brevity's sake we refer the reader to \citet{vanKoppen:2005,vanKoppen:2017}.

The preceding discussion of Germanic CA patterns has shown that CA and TA are best analyzed as resulting from distinct φ-feature probes, one on Cº and one on Tº, and that Cº probes down in the structure, finding the embedded subject in its canonical position (as argued by \citealt{Carstens:2003,vanKoppen:2005,Haegeman:2012}). The facts from Germanic CA argue against an account where Agree only probes up (cf. \citealt{Zeijlstra:2012,Wurmbrand:2011,Bjorkman:toappearb}), though the case remains to be made that all feature valuation operations are the result of a downward-probing Agree. 

\section{Lubukusu CA: Agree probing up?} \label{LubukusuCA}

In contrast to the Germanic patterns, Lubukusu (Bantu, J.30, Kenya) displays a CA relation where a declarative-embedding complementizer shows full φ-feature agreement (gender, number, and person) with the subject of the matrix clause:\footnote{For discussion of similar constructions, see \citet{Kawasha:2007} (five central Bantu languages), \citet{LetsholoSafir:2017} (Ikalanga), \citet{DiercksRao:2017} (Kipsigis), \citet{Torrence:2016} (Ibibio), and \citet{Idiatov:2010} (various Mande languages). Ongoing work by Diercks has shown the same phenomenon in Lwidakho and Luwanga (Bantu languages of the Luyia subgroup, related to Lubukusu).}

\ea \label{FirstLubukusuExample}
\langinfo{Lubukusu}{}{\citealt{Diercks:2013}} \\

\begin{xlist}

\ex
\gll \circled{Ba-ba-ndu} ba-bol-el-a Alfredi \circled{ba-}li	a-kha-khil-e. \\
2-2-people 2\textsc{sa}-said-\textsc{ap}-\textsc{fv} 1Alfred 2-that 1\textsc{sa}-\textsc{fut}-conquer-\textsc{fv} \\
\glt `The people told Alfred that he will win.' 

\ex
\gll \circled{Alfredi} ka-bol-el-a ba-ba-ndu \circled{a-}li ba-kha-khil-e. \\
1Alfred 1\textsc{sa}-said-\textsc{ap}-\textsc{fv} 2-2-person 1-that 2\textsc{sa}-\textsc{fut}-conquer-\textsc{fv} \\
\glt `Alfred told the people that they will win.'

\end{xlist}

\z

\noindent As we mentioned above, this CA pattern appears on the face of it to be a case of Agree Probing Up, with a probe structurally lower than its goal, though we will show in what follows that this approach cannot be maintained.

\noindent First, example (\ref{BukusuCAMorphCaus}) gives a morphological causative construction; despite the fact that the causee \textit{Alfredi} in (\ref{BukusuCAMorphCaus}) triggers CA in a periphrastic causative context, when it is not the subject of the sentence it cannot trigger agreement on the complementizer:

\ea \label{BukusuCAMorphCaus}
%\langinfo{LANGUAGE}{}{} \\
\gll N-a-suubi-sya \circled{Alfredi} n-di/\circled{*a-li} ba-keni khe-be-ech-a. \\
1\textsc{sg}.\textsc{sa}-\textsc{pst}-believe-\textsc{caus} 1Alfred 1\textsc{sg}-that/*1-that 2-guests \textsc{prog}-2\textsc{sa}-come-\textsc{fv} \\
\glt `I made Alfred believe that the guests are coming.'
\z

\noindent Similarly, in a ditransitive the complementizer can only agree with the subject, not with the intervening indirect object.

\ea \label{BukusuCADitransitive}
%\langinfo{LANGUAGE}{}{} \\
\gll W-a-bol-el-a \circled{Nelsoni} o-li/\circled{*a-li} ba-keni ba-a-rekukh-a. \\
2\textsc{sg}.\textsc{sa}-\textsc{pst}-say-\textsc{ap}-\textsc{fv}	1Nelson	2\textsc{sg}-that/*1-that	2-guests	2\textsc{sa}-\textsc{pst}-leave-\textsc{fv} \\
\glt `You told Nelson that the guests left.'

\z
\noindent As can be seen in (\ref{MorphCause=Obj}) and (\ref{IndirectObject=Obj}) below (equivalents of (\ref{BukusuCAMorphCaus}) and (\ref{BukusuCADitransitive}) respectively), both the causee and the indirect object can be object-marked on the verb; object marking in Lubukusu is restricted to structural arguments of the verb \citep{Diercks:2011,SikukuEt:2017}.\footnote{Non-accusative objects like locative phrases may be marked on the verb, but are marked with a post-verbal locative clitic, as demonstrated by \citet{Diercks:2010,Diercks:2011} and \citet{SikukuEt:2017} (`accusative' here is used as an expository mechanism, as DPs are not case-marked in Lubukusu, like in other Bantu languages, and the status of case-marking in general is a larger issue: \citealt{Harford:1985,Halpert:2012,Diercks:2012,vanderwal:2015}). And as \citet{Diercks:2011} shows, even for locatives in Lubukusu it is only possible to mark them on the verb when they are selected by the verb, locative-marking is unavailable for adjunct locative phrases.}\textsuperscript{\textrm{,}}\footnote{For an elaborate discussion on object marking in Bantu, see \textcitetv{chapters/07-van-der-wal}.} This is reason enough to believe them to be DP objects of the verb and therefore potential interveners in any Agree relationship between the complementizer and the superordinate subject.\footnote{Example (\ref{MorphCause=Obj}) is translated as verum focus because doubling an object marker with an overt object is only possible in Lubukusu in a set of pragmatic contexts akin to those that elicit verum focus in English.}
\newpage 

\ea \label{MorphCause=Obj}
%\langinfo{LANGUAGE}{}{} \\
\gll N-a-\circled{mu-}suubi-sya \circled{Alfredi} n-di/*a-li ba-keni khe-be-ech-a.	\\
1\textsc{sg}.\textsc{sa}-\textsc{pst}-1\textsc{om}-believe-\textsc{caus} 1Alfred 1\textsc{sg}-that/*1-that 2-guests \textsc{prog}-2\textsc{sa}-come-\textsc{fv} \\
\glt `I DID make Alfred believe that the guests are coming.'
\z

\ea \label{IndirectObject=Obj}
%\langinfo{LANGUAGE}{}{} \\
\gll W-a-\circled{mu-}bol-el-a o-li/*a-li ba-keni ba-a-rekukh-a. \\
2\textsc{sg}.\textsc{sa}-\textsc{pst}-1\textsc{om}-say-\textsc{ap}-\textsc{fv} {2\textsc{sg}-that/*1-that} 2-guests 2\textsc{sa}-\textsc{pst}-leave-\textsc{fv} \\
\glt `You told him that the guests left.'
\z
\noindent Following \citet{Diercks:2013}, our conclusion is that the Lubukusu CA construction cannot be explained under an account of uφ on Cº probing upwards, given the lack of intervention effects with intervening DPs. Coupled with the evidence from Germanic CA, this leads us to conclude that downward probing is a central component of the syntax, whereas upward probing is not necessarily so.  

%\subsection{Lubukusu CA is Anaphoric}
\citet{Diercks:2013} proposes that agreement on the complementizer is triggered locally in the embedded CP by a null subject-oriented anaphor, so the agreement is in fact only triggered indirectly by the matrix subject. As a result of the subject-oriented properties of the null anaphor, CA in Lubukusu is determined by the features of the matrix subject. Abstracting away from the details for the moment, Diercks claims that the strict subject orientation of Lubukusu CA is enforced by LF clitic-movement of the null anaphor to T° (following Safir’s \citeyear{Safir:2004} analysis of long-distance subject-oriented anaphors). 

Support for the proposal that Lubukusu CA is anaphoric in nature comes from predictable sources, mainly, that the locality constraints for anaphoric relations are known to be distinct from those for morphosyntactic agreement (formalized by Chomsky’s \citeyear{Chomsky:2001} Agree). First, CA is clause-bounded, only agreeing with the most local super-ordinate subject (cf. Chomsky’s \citeyear{Chomsky:1973} Tensed Sentence Condition). In (\ref{BukusuCAClosestClause}) the lower complementizer only agrees with the intermediate class 2 subject and not with the class 1 matrix subject.

\ea \label{BukusuCAClosestClause}
\gll \circled{Alfredi} ka-a-lom-a a-li ba-ba-andu ba-mwekesia ba-li/\circled{*a-li} o-mu-keni k-ol-a. \\
1Alfred 1\textsc{sa}-\textsc{pst}-say-\textsc{fv} 1-that 2-2-people 2\textsc{sa}-revealed {2-that/*1-that} 1-1-guest 1\textsc{sa}.\textsc{pst}-arrived-\textsc{fv} \\
\glt `Alfred said people revealed that the guest arrived.'
\z

\noindent In addition, Lubukusu CA has a strict subject orientation\textemdash indirect objects and causes do not trigger agreement, agentive by-phrases in passives do not either, nor do other plausible agreement triggers like source-adjuncts in perception predicates (e.g. \textit{hear from X}). We refer the reader to \citet{Diercks:2010,Diercks:2013} for additional empirical argumentation for an anaphoric analysis of Lubukusu CA.

The proposal to be set forward here maintains the core generalizations and analysis from \citet{Diercks:2013}, namely, that Lubukusu CA is at its heart an anaphoric relation.\footnote{See section 4.3 below for an alternative analysis from \citet{Carstens:2016}.} The contributions that we will make here are 1) to utilize the Lubukusu CA facts as evidence for a generalizable theory of anaphoric relations, and 2) to follow recent work like \citet{Hicks:2009}, \citet{Reuland:2005,Reuland:2011}, and \citet{Rooryck:2011} (among others) to derive anaphoric relations from more basic elements of the grammar. And, to bring this back even further to the broadest purposes of this paper, these conclusions present crucial evidence on the question of the directionality of probing of Agree. 

\section{Anaphoric vs. non-anaphoric feature valuation} \label{AnalysisSection}

\subsection{Setting the stage for the analysis}

The Agree operation is a natural, parsimonious account of feature-valuation, and is particularly useful for explaining West-Germanic CA constructions. CA in Lubukusu and other Bantu languages, however, cannot be licensed solely by Agree without significantly altering notions of locality and Agree.  
This sets up an interesting dichotomy that lies at the heart of our proposals in this paper. On the one hand, inflectional agreement relations (like subject-verb agreement) are derived by a feature-valuation operation with specific generalizable properties (like strict structural locality). On the other hand, basic anaphoric relations (e.g. subject-oriented anaphors in object position) also show matching of features, but take on a different set of characteristics with respect to locality and other constraints (as documented in a long line of generative literature, e.g. \citealt{Chomsky:1981,Safir:2004,Reuland:2011} and \citetv{chapters/13-sundaresan}). While recent generative work (e.g. \citealt{Reuland:2011}, \citealt{Hicks:2009}, comments in \citealt{Wurmbrand:2011}) has made significant progress reducing anaphoric relations to Agree relations (along with basic chain formation), the Lubukusu CA facts are a \textit{prima facie} case of precisely the opposite situation. Here, an instance of morphosyntactic agreement does not in fact accord with the predictions of agreement by Agree, instead showing the properties of an anaphoric relationship. The paradox, of course, is that the argument that Lubukusu CA is best analyzed as anaphoric instead of a syntactic agreement relation is nonsensical if anaphora and agreement are both explained by the same underlying syntactic operation (Agree). The logical conclusion, then, is either that Lubukusu CA is not in fact anaphoric (\textit{contra} \citealt{Diercks:2013}), or that anaphora and agreement do not reduce to identical syntactic operations.

Our conclusion is that Lubukusu CA is an example of an anaphoric feature-valuation relationship that cannot reduce to Agree alone. If this is in fact the case, then any efforts to reduce all feature sharing/strict reference relationships in the syntax to identical Probe-Goal relations (= Agree) are misguided, and there needs to be some principled way to distinguish anaphoric feature valuation from non-anaphoric feature valuation on a theoretical level. Our claim, as we’ve discussed above (following \citealt{Rooryck:2011}), is that anaphoric feature valuation relations derive from a compound operation of Move + Agree.

\subsection{Deriving Lubukusu CA}

\subsubsection{Step 1: Reducing anaphoric relations to Agree}

We follow \citet{Hicks:2009}, \citet{Reuland:2011}, and \citet{Rooryck:2011} (henceforth, R\&VW) in assuming that binding is not a primitive of grammar. In particular, R\&VW propose that intensifiers and reflexives must raise out of their base positions to adjoin to \textit{v}P. This movement is necessary in order for these units to be in a position from which they can probe their c-command domain and are valued by the subject (equating reflexives with Doetjes’ \citeyear{Doetjes:1997} analysis of floating quantifiers). Example (\ref{BasicR&VWDerivationB}) derives the sentence \textit{Peter invited himself}, where features marked with a * are those that are shared with the subject DP (R\&VW: 89 ex (2)).\footnote{This feature sharing/valuation occurs via the Agree relation \citep{FramptonGutmann:2000,Pesetsky:2007}.}

\begin{figure}
\ea \label{BasicR&VWDerivationB}
\begin{forest}
[\textit{v}P  
	[DP\textsubscript{2} [himself\\{[P:3*,N:sg*,G:m*]}, roof, name=MoveTarget] ]
	[\textit{v}P
    	[DP\textsubscript{1} [Peter\\{[P:3,N:sg,G:m]}, roof, name=AgreeTarget] ]
		[\textit{v}P
			[\textit{v}°]
			[VP [V°\\invited] [DP\textsubscript{2} [himself\\{[P:3*,N:sg*,G:m*]}, roof, name=MoveSource] ] 
            ]
		]
	]
]
\draw[->] (MoveSource) to[out=south west,in=south] (MoveTarget);
\draw[<-] (AgreeTarget) to[out=west,in=south east] (MoveTarget);
\end{forest}
\z \vspace{-2cm}
\end{figure}

Under this view, Agree is hypothesized to exclusively search in the probe's c-command domain. Anaphors are analyzed as consisting of a set of unvalued φ-features that are valued (via Agree) by moving the reflexive over its antecedent. Subsequent subject and verb movement then obscure this reflexive movement (in R\&VW’s account). In order to be able to distinguish this agreement from other φ-feature valuation (which is presumably deleted or not interpreted at LF), they claim that the φ-features on reflexive pronouns are interpretable, unvalued features. A major prediction of R\&VW’s approach (and others like theirs) is that self-anaphora are at their heart an instance of feature valuation in the syntax.\footnote{This is opposed to an approach like that of \citet{Reinhart:1993b}, where self-reflexives are the product of constraints on licensing reflexivity of predicates (i.e. multiple arguments of a predicate being saturated by the same semantic variable).}  If this is the case, there ought to be feature valuation operations that show the properties of anaphora while having little to do with reflexivity of predicates. The claim that we advance in the remainder of this paper is that Lubukusu CA exemplifies precisely this prediction: an instance of a feature bundle with the same values as anaphoric features\textemdash interpretable and unvalued\textemdash that shows the same syntactic behavior, despite not being an instance of predicate reflexivity.

\subsubsection{Step 2: The interpretative effects of CA in Lubukusu vs. CA in Germanic}

\citet{Diercks:2010,Diercks:2013} observes that the agreeing complementizer in Lubukusu has an interpretation that appears to be evidential in nature: an agreeing complementizer signals the speaker’s assessment that the reported information is relatively reliable, and is ruled out in instances where the reliability of the reported information is in question.  In those cases, a non-agreeing complementizer (here \textit{bali}) is necessary:

\ea
\gll Mosesi a-lom-ile \underline{\hspace{0.4in}} Sammy k-eb-ile chi-rupia. \\
1Moses 1\textsc{sa}-say-\textsc{prf} that 1Sammy 1\textsc{sa}-steal-\textsc{prf} 10-money \\
\glt `Moses has said that Sammy stole the money.'

\begin{xlist}
\ex Moses saw the event, and the speaker believes him: \textit{*bali/ali}
\ex Moses didn't see the event, but reported hearsay: \textit{bali/*ali}
\ex Moses says he saw the event, but the speaker doubts him: \textit{bali/*ali}

\end{xlist}

\z
\noindent
Here we observe a noticeable contrast between CA in Germanic and Lubukusu/ Bantu; whereas the agreeing complementizer appears to have an interpretive effect in Lubukusu, Germanic CA does not have any semantic contribution (see \citealt{vanKoppen:2005,vanKoppen:2017}). Based on these patterns, we hypothesize that the φ-features on Cº in Lubukusu have an effect on semantic interpretation, and are therefore interpretable, unvalued features. The φ-features on Cº in Germanic do not have an interpretation and are hence uninterpretable, unvalued features. This key contrast is noted in (\ref{unintint}):\footnote{On the distinction between (un)interpretable and (un)valued features, also see \citet{Pesetsky:2007}.}

\ea \label{unintint}

\underline{φ-features on Cº} \\
Lubukusu: interpretable, unvalued \\
Germanic: uninterpretable, unvalued

\z
\noindent Note, at this point we have not given a precise account of what the interpretation of these interpretable features is, only that the presence of these features leads to an interpretation that is different from the one where these features are absent.   

\subsubsection{Step 3: Deriving Lubukusu complementizer agreement}

As a point of departure, we analyze the φ-features originating on a higher CP-projection than the rest of the complementizer, following the same proposal in \citet{Carstens:2016}.\footnote{Our thanks goes to Vicki Carstens (p.c.) for invaluable comments and feedback on this analysis. See \citet{Carstens:2016} for a different approach to these same data that (like our approach) seeks to explain Lubukusu CA under a general analysis of feature valuation (agreement), but which does so without the anaphoric analysis pursued here.}

\ea 
{[}\textsubscript{ForceP} Force[iφ:\_] ... [\textsubscript{FinP} Fin[-li] [\textsubscript{TP} ...]]]\\
\z
\noindent Cº is merged with unvalued, interpretable φ-features. At present, we will simply stipulate that because these features are interpretable, unvalued features, they are not valued immediately by Agree (\S \ref{ExplainPAPA} discusses why). The derivation proceeds until the \textit{v}º phase head is merged, at which point the subject is merged, and Force is adjoined to \textit{v}P in a movement operation. It is from this adjoined position that the interpretable, unvalued φ-features of Forceº probe the subject, and are specified as sharing its φ-features.\footnote{This mechanism is reminiscent of the reprojection analysis discussed by \textcitetv{chapters/10-boerjesson-mueller}.} On this analysis, Forceº will always agree with the highest Goal in the \textit{v}P, namely the subject. We assume that Forceº has  morphophonological requirements stating that it must undergo morphological merger with a Cº head (following standard Distributed Morphology assumptions that morphological exponents state the morphosyntactic contexts in which they are realized); therefore, the \textit{v}P-adjoined copy of Force cannot be spelled out, only the lower copy can be phonologically-realized.\footnote{An anonymous reviewer suggests that "we might have expected instead, though, that merger would either force the higher copy to be pronounced, or would break the link between the two copies of the chain and result in doubling.” These are indeed additional logical options which might indeed apply in other circumstances. However,  these options do not apply in this case, since we assume that the Force head has to undergo morphological merger with a C head.}\textsuperscript{\textrm{,}}\footnote{This assumes a feature-sharing model of Agree, wherein valuation of one copy's features values all copies' features (because features are in fact \textit{shared} between copies, rather than being distinct): see \citet{FramptonGutmann:2000,Rooryck:2011,Pesetsky:2007}. So there is no transmission of features to the lower copy, but rather valuation on one copy in fact is valuation on all. Thanks to a reviewer for comments on this question.}

\ea \label{upwardagreelubukusu}

\begin{forest}
[\textit{v}P
	[\sout{Force}°\textsubscript{\textit{k}}\\\sout{{[φ:*β]}},name=source]
    [\textit{v}P
    	[Subject \\{[φ:β]},name=target]
        [...\\ForceP  
        	[Force°\textsubscript{\textit{k}}\\{[φ:*β]}\\AGR-]
        	[FinP
            	[Fin\\-li]
            	[...]
			]
		]
	]
]
\draw[->] (source) to[out=south, in=south west] (target);
\end{forest}
\z \vspace{-0.2cm}
\noindent Here we have assumed for expository purposes that the Forceº head itself has raised to the edge of \textit{v}P, though it is not critical that it does so; it may well be that only the anaphoric φ-features themselves move in a form of feature-splitting merge \citep{Obata:2011}. This approach may well be preferable given that this movement does not obey expected constraints on head-movement. An alternative is to claim that the φ-features percolate to the maximal category of CP, and the entire CP raises to the edge of \textit{v}P: \citet{LetsholoSafir:2017} propose just this to account for Ikalanga complementizer agreement patterns, and \citet{Moulton:2015} suggests that all CPs may do so to resolve type-theoretic semantic concerns (and in the process explaining a variety of puzzles about similarities and differences between CP and DP verbal complements). At present we simply focus on the φ-features themselves and leave these details for future work: what is critical for us is that unvalued, interpretable φ-features raise to the edge of \textit{v}P. Whether they do so alone (feature-splitting Merge), pied-pipe the Forceº head, or pied-pipe the entire CP, the core claims of our account here will still hold.\footnote{A reviewer questions whether there is independent evidence that adjuncts can serve as probes: we refer the reader to \citet{CarstensDiercks:2013} for discussion of a Lubukusu pattern where the manner wh-word \textit{how} probes and agrees with the subject of the clause, not dissimilar to the analysis proposed here.} 

We therefore claim that CA in Lubukusu is derived by the very same mechanism that we find for CA in Germanic: downward-probing Agree. The crucial difference between CA observed in these two languages is not the mechanism(s) employed, but rather, the moment of the valuation of these φ-features:


\ea \underline{Derivations of CA}
\begin{itemize}
\item Germanic: φ-features on Cº are valued at Merge of Cº via Agree with the embedded subject
\item Lubukusu: φ-features on Cº are valued after Internal Merge with \textit{v}P and Agree with the matrix subject
\end{itemize}
\z

\noindent The critical component of our analysis, then, can be reduced to this general principle (which is directly based on R\&VW, but generalizes beyond argument anaphors):\footnote{A reviewer points out that \citet{Pesetsky:2007} propose that T bears interpretable, unvalued features which are valued by valued tense features on verbs. We can avoid disagreeing by limiting this proposal to φ-features, but if (like Pesetsky \& Torrego) we want to explain tense on verbs via Agree, the PAPA could be extended via the assumption that tense is interpretable and unvalued on verbs, which become valued by tense on T via a procedure similar to what we propose here (though perhaps with verb movement to Cº). It is not clear that tense on verbs ought to be explained in this way, however, since semantic tense seems more likely to be a component of Tº than Vº. Instead, tense may well come to be inflected on verbs post-syntactically. The more likely extension of these ideas to Tense in our eyes is to phenomena of sequence of tense (i.e. agreement between T heads), though we have not explored this in any depth.} 

\ea	\label{PAPA}
\underline{Principle for the Anaphoric Properties of Agreement  (PAPA)} \\
Anaphoric φ-features (i.e., interpretable, unvalued φ-features) adjoin to the edge of \textit{v}P.

\z
\noindent In the case of Lubukusu CA, the anaphoric φ-features of the agreeing complementizer adjoin to \textit{v}P and are then valued by Agree. A welcome result of this analysis is that our assertion that Agree always probes downward can be upheld. The difference between Bantu and West-Germanic (to speak metaphorically) is that uninterpretable φ-features are impatient, probing their c-command domain at first-merge, whereas anaphoric φ-features are patient: they do not probe their c-command domains when merged, but are instead (eventually) adjoined to \textit{v}P and probe from that position.\footnote{An anonymous reviewer questions whether there is independent evidence that a movement operation of a probe can feed valuation of that probe. While we do not have such independent evidence to offer here, we are in fact claiming that all valuation of interpretable features should be upward-oriented in this way: see discussion of the Anaphoric Agreement Corollary in (\ref{AnaphAgrCorollary}) for some predictions of this account. The same reviewer also notes some conceptual similarities between this proposal and the long-distance agreement analysis of \citet{Potsdam:2001}, where covert movement enables otherwise-unexpected agreement relations.}  

The principle in (\ref{PAPA}) is presented as axiomatic, but this raises many important issues. What exactly is the nature of the interpretation of interpretable, unvalued features? And more pressing for our current concerns, what evidence is there that these anaphoric features must raise to the edge of \textit{v}P, rather than probing their own c-command domain? Furthermore, it is important to the current discussion whether the PAPA is in fact axiomatic, or if it can be derived from more basic principles. We now turn to these questions.

\section{Toward an explanation of the PAPA} \label{ExplainPAPA}

After briefly discussing relevant previous work on anaphors in the next subsection, we engage in three levels of argumentation to work our way back to a discussion of the PAPA: 1) why syntactic elements move to the \textit{v}P edge in general, then 2) why object anaphors specifically move to the \textit{v}P edge, and 3) the extension back to our concerns, of why anaphors in our particular context (anaphoric features at CP) move to the \textit{v}P edge. 

\subsection{Movement of anaphors}
The idea that reflexives covertly raise to a position local to their antecedents is a long-standing explanation for anaphoric properties in generative grammar. \citet{Safir:2004}, \citet{Pica:1987}, and \citet{Cole:1990} all rely on this kind of analysis of long-distance anaphors, raising into a local relationship with their antecedents, and while \citet{Reuland:2011} does not argue that self-reflexives universally raise into their predicate, he does conclude that they do in at least a subset of cases due to general economy constraints in interpretation. 

R\&VW propose that complex reflexives adjoin to \textit{v}P, but they leave open the question of what motivates movement of self-reflexives to the edge of \textit{v}P: 

\begin{quote}
It is not clear to us at this point what drives the movement of self-reflexives to the edge of \textit{v}P. It might be that this movement is driven by the need for valuation of unvalued features. \citet{Boskovic:2007a} suggests something along these lines, in that he argues that the uninterpretable features present on a constituent X may trigger the movement of X. Alternatively, there is another feature of self-reflexives that requires satisfaction and that triggers their movement. (R\&VW: 106, fn. 14).
\end{quote}
\noindent R\&VW do not offer a motivation for this movement, and leave the question for future research. In general, the notion that the phase is the source of binding domains is implicit in the work of both \citet{Reuland:2011} and R\&VW, who utilize such independently motivated locality constraints to derive the properties of binding. In fact, a wide range of work focuses on the role of phase boundaries as delimiting binding domains in a variety of specific construction types \citep{Wurmbrand:2011,Lee-Schoenfeld:2008,Canac-Marquis:2005,Heinat:2006,Hicks:2009,Quicoli:2008,CharnavelSportiche:2016}.

\subsection{On movement to the edge of the \textit{v}P phase}

The PAPA (\ref{PAPA}) proposes that interpretable, unvalued features move to the edge of \textit{v}P: this accounts for the core Lubukusu CA facts, but \textit{why} do anaphoric features behave in this way? We believe that this raising of anaphoric φ-features to the phase edge is a plausible proposal if evaluated in the light of recent work on the meaning of grammatical categories by Wolfram Hinzen and his collaborators \citep{Hinzen:2012,SheehanHinzen:2011,HinzenSheehan:2013,ArsenijevicHinzen:2012}, who claim that phases have both syntactic and semantic properties, specifically, phases enable \textit{reference}. In short, we will argue that the anaphoric features move to the edge of the phase because they have to become referential, and in order for the \textit{v}P itself to be capable of referring to an event. 

\subsubsection{Phases as a unit of semantic significance}

\citet{Hinzen:2006, Hinzen:2012} and \citet{HinzenSheehan:2013} challenge the notion that the semantic ontology and semantic principles are independent of syntax. This abandons the approach developed in a long history of Chomsky’s work that claims that language is simply a tool to express thought, but that language and thought are fundamentally distinct (e.g. \citealt{Chomsky:2000b}). Hinzen adopts a framework that is in fact closely linked with the syntactic architecture of the Minimalist Program (\citealt{Chomsky2000}, \citealt{Chomsky:2001}, \citealt{Chomsky2008}) that claims that the syntactic derivation proceeds by phase, and each phase must necessarily be legible at the C-I (Conceptual-Intentional) interface. However, Hinzen contests the traditional syntax-semantics disjunct and instead claims that grammar is in fact the principal factor that allows for organization of meaning in language. Therefore, ``rather than being `autonomous' and merely `interfacing' with the semantic component, … grammar is a way of carving up semantic spaces'' (\citealt[311]{Hinzen:2012}).  That is to say, grammar ``creates the semantic ontology of language,'' such that grammar in fact is meaningful, and meaningful contribution of grammar is reference (\citealt[311]{Hinzen:2012}). Specifically, the phase is the referential component of grammar, with different phases referring to different entities\textemdash DPs refer to individuals, \textit{v}Ps to events, and CPs to propositions/truth \citep{HinzenSheehan:2013, SheehanHinzen:2011}.
A phase's semantic contribution is to take the conceptual/predicational content of the phase (e.g. the concept of \textsc{dog}, or \textsc{banana}) and to enable linguistic \textit{reference} to relevant entities. Phases themselves are composed of a phase interior and a phase edge, as shown in (\ref{PhaseSchematic}), a notion with which syntacticians are now long familiar (Chomsky 2001 and subsequent work).\footnote{The formulation in terms of edge/interior presented here is adopted from \citet{HinzenSheehan:2013}.} 



\ea	 \label{PhaseSchematic}
{[} EDGE [ INTERIOR ]]
\z

\ea 
{[}\textsubscript{DP} the [\textsubscript{NP} man ]]
\z

\noindent A DP phase, for example, will refer to an object. On the approach developed in this collection of work, the interior of a phase is the descriptive content of the phase and the edge of the phase (head+extended material) enables reference. In this sense lexical content cannot refer on its own–reference is only possible in grammatical contexts. 

\begin{quote}

Lexemes by contrast [to animal calls] not only can be used referentially in the physical absence of their referent, but are also very incomplete in their meaning. The word ‘eagle’ by itself does not denote anything in particular: not this eagle or that, not all eagles or some, not a kind of bird as opposed to another, not the property of being an eagle, etc.—things that it can denote only once it appears in the right grammatical configurations. It is also used for purposes of reference and predication, in addition to being used as a directive for action, and it again requires a phrasal context, hence grammar, when it is so used. (\citealt{HinzenSheehan:2013}: 42-3). 

\end{quote}

\noindent On this approach, then, linguistic meaning is \textit{reference} (to objects, events, and propositions), and reference is determined grammatically, via a syntactic derivation by phase. For ease of exposition, we will refer to this general framework as the Phase Reference (PR) model. In one sense the PR model is an inconsequential shift for syntacticians’ everyday sort of analysis–this does not change the nature of our grammatical architecture much, retaining derivation by phase, Merge, Agree, and the kinds of functional structure we are familiar with at present. In another sense, however, the PR model is a dramatic shift, as we suddenly have incorporated reference\textemdash a central semantic notion\textemdash into the syntax itself. The PR model introduces a new range of predictions for a given syntactic analysis (involvement of phase edges in a derivation ought to predict referential consequences for the relevant referent). It also incorporates an additional kind of explanatory mechanism for solving linguistic puzzles, given that the referential properties of language is now a central aspect of the syntax. Let us look at some specific examples of how syntax and semantics are intertwined by looking at Sheehan and Hinzen’s (\citeyear{SheehanHinzen:2011}) (henceforth S\&H) discussion of the referential possibilities of DPs and CPs, before exploring the consequences for \textit{v}P structure that we will rely on in our approach to valuation of anaphoric features.  

As for the DP-level, S\&H point to Longobardi’s (\citeyear{Longobardi:1994,Longobardi:2005}) proposals regarding the range of interpretations available for DPs, particularly the proposal that proper names raise to D. Modifying and building on Longobardi’s approach, they propose that there is a threefold ontology of DPs in terms of their referential capabilities: 

\ea \underline{Referential capabilities of DPs (S\&H: 415)}

\begin{xlist}
\ex Indefinite existential nominal reference 
\ex Definite descriptions (contextually bound free variables)
\ex Proper names (maximally specifically referential, with rigid reference)

\end{xlist}

\z

\noindent One illustration that they rely on here draws on data from \citet{Elbourne:2008}: 

\ea	

\begin{xlist}
\ex The Pope is usually Italian.
\ex (Pointing at the Pope) He is usually Italian.
\ex \#Joseph Aloisius Ratzinger is usually Italian.
\end{xlist}

\z
\noindent Both definite descriptions and pronouns can refer to different individuals (as specified by context), whereas proper names have much more rigid reference to a specific individual. 

S\&H claim that these three sorts of DP reference are syntactically derived, that is to say, there are syntactic correlates of all three interpretive possibilities. 

\begin{quote}
``When the D-position is empty (there is no determiner and there is no movement to D), a default existential interpretation is derived, where reference is to an arbitrary instance of the predicate. In short, reference is restricted merely in virtue of the predicate’s content, or by the interior of the nominal phase''. (S\&H: 421).  

``Definite reference, in contrast, involves both the Dº position and the base predicate position, such as an instance of a definite determiner in Dº and the noun occurring in Nº. In this case, both the phase interior and phase edge determine reference”. (S\&H: 421). 

\end{quote}
\noindent Proper names, in contrast, consist of movement from Nº to Dº with Nº substituting for Dº, such that 


\begin{quote}

``reference is unmediated by descriptive content and only the phase edge determines reference, resulting in the rigid referential properties of proper names.'' (S\&H: 421) \footnote{The proposal of movement of proper names to D is adopted directly from \citet{Longobardi:1994,Longobardi:2005}.}  

``Broadly speaking, then, the three referential possibilities nicely correlate with the three logically possible ways in which the phase edge and interior can contribute to the determination of reference: only the phase interior mediates reference, or both the interior and edge do, or only the edge is involved''. (S\&H: 421).

\end{quote}
\noindent S\&H then extend this threefold ontology of phases, correlating the three referential possibilities of DPs for reference to individuals to a threefold ontology of reference by CPs to facts. Specifically, they claim that CPs may be indefinite, representing propositions, definite, yielding facts, or rigid in their reference, denoting truth. 

\ea \underline{Referential capabilities of CPs (S\&H: 424)} \\

\begin{xlist}
\ex \textbf{Reference to Propositions}: Cº is empty or underspecified, through a quantificational operator (optionally null in English), yielding an indefinite interpretation; 
\ex \textbf{Reference to Facts}: Cº is pro-form (obligatorily overt in English) with a TP-restriction, yielding a referential interpretation;
\ex \textbf{Reference to Truth}: Cº is substituted by Vº/Tº overtly or covertly (covertly in English, overtly in V2 languages), yielding a rigid interpretation unmediated by a descriptive condition.
\end{xlist}
\z
\noindent S\&H correlate these referential possibilities with the various interpretations of clauses in embedded contexts in particular, discussing non-factive clauses as indefinite reference, factive clauses as definite reference, and root clauses and embedded clauses with root clause properties as those with the rigid interpretations that come from a truth-conditional (i.e. truth-referring) clause. 

There are two relevant conclusions for our purposes here\textemdash the first is that there are particular interpretive (referential) properties of phases, and the second that the syntactic realization of a phase (specifically, the relationship between the phase-internal material and the phase edge) has specific referential consequences depending on the phase in question. Sheehan and Hinzen conclude their paper with the following statement: 

\begin{quote}
 ``... reference in human language is an `edge phenomenon': it depends on the extent to which a phase edge is involved in the determination of reference. The more edge-heavy the phase becomes (through Determiner or Complementizer phasal heads, or movement of phase internal material into these positions), the more referential the phase becomes, giving rise to object reference and fact reference in nominals and clauses, respectively.'' (S\&H: 451)

\end{quote}
\noindent 
These proposals are set forth as relevant to all phases (DP nominal reference, \textit{v}P event reference, and CP fact reference).  To our knowledge they have only developed in-depth analyses of DP and CP, however, and our discussion here that extends their ideas to the realm of \textit{v}P is a new contribution; we adopt their claim that \textit{v}Ps refer to events, and rely on their connection of movement to the edge of a phase with increased specificity of reference so that we can motivate the movement of anaphoric φ-features to the edge of \textit{v}P. 

\subsubsection{Toward an ontology of \textit{v}P structure}

\citet{SheehanHinzen:2011} and \citet{HinzenSheehan:2013} do not extend a detailed analysis of the reference of phases to \textit{v}Ps. Their comments are mainly restricted to the notion that \textit{v}Ps refer to events, though \citet{SheehanHinzen:2011} do comment that more specific reference with respect to \textit{v}Ps may well have to do with the boundedness of events (i.e. the aspectual properties of predicates). We develop this idea here in more depth; specifically, we propose that there is also generally a threefold ontology of \textit{v}P phases based on the aspectual properties of predicates, as shown in (\ref{vPReferentialProperties}):

\ea \label{vPReferentialProperties} \underline{Referential capabilities of \textit{v}Ps} (to be expanded on below)

\begin{xlist}
\ex \textbf{Existential event reference} (e.g. existential/presentational clauses)
\ex \textbf{Atelic events} (boundedness of event is addressed but is not rigid)
\ex \textbf{Telic events} (maximally specifically reference, with rigid reference to bounded event)
\end{xlist}

\z
\noindent Here telic events are those where the predicate dictates a specific culmination point; atelic predicates do not (\citealt{Beavers:2012} offers a good overview of the relevant issues). Existential clauses, on the other hand, are the most unspecified sort of event that does not refer to a bounded event at all, but rather a state of existence. For this ontology to hold in the PR model it should be demonstrable that telic events show maximal involvement of the edge of the phase in the syntactic derivation, with atelic events showing less, and existential reference to events showing the least involvement of the edge. As we will see, the involvement of both verbs and objects in \textit{v}P-based event reference complicates this threefold ontology, though notably in exactly the ways predicted by the PR model. \footnote{It is important to note that the proposals here have broad-reaching implications that cannot possibly be defended sufficiently in this paper, and would take us too far afield of our overall goals of the exploration of anaphoric feature valuation. But we will provide evidence from existing work on telicity and aspectual properties of predicates in order to at least show that the ontology in (\ref{vPReferentialProperties}) is well-founded empirically, and shows exactly the kinds of intersections of syntactic structure and referential results that are predicted by the PR model.} 

Perhaps the classic English diagnostic for telicity of predicates is the distinction in application of \textit{in/for} modifying PPs, for example \textit{in an hour} (compatible with telic predicates) and \textit{for an hour} (compatible with atelic predicates) (\citealt{Vendler:1967,Dowty:1979,Thompson:2006,Beavers:2012}, among many others). 

\ea
\langinfo{English}{}{\citealt{Thompson:2006}: 213} \\
\begin{xlist}
\ex Mary ate an apple in an hour/??for an hour.   
\ex Mary walked ??in an hour / for an hour.
\end{xlist}

\z
\noindent As noted by a variety of work, verbs alone do not determine the aspectual properties of a predicate, which are instead determined by the combined verb phrase material \citep{Verkuyl:1972,Verkuyl:1989,Verkuyl:1993,Verkuyl:1999,Pustejovsky:1991,Zagona:1993,Garey:1957,Tenny:1987,Tenny:1992,Tenny:1994,Krifka:1989,Krifka:1998a,Krifka:1992,Dowty:1991,Jackendoff:1991,Jackendoff:1996,Travis:2010}. For example, bare plurals in English yield atelic readings of predicates (\ref{BarePluralAtelic}), and objects with quantized reference yield telic predicates (\ref{QuantizedTelic}), whereas objects with non-quantized reference yield atelic predicates (\ref{Non-QuantizedAtelic}).\footnote{Aspectual inflections (e.g. progressive vs. perfective) also influence the aspectual interpretation of predicates (Mary has written the book vs. Mary is writing the book). }\textsuperscript{,}\footnote{Likewise, in English paths/goals represented in PPs can influence the interpretation of an event with respect to telicity, where specific goals of directed motion generate telicity whereas paths of motion alone do not, showing that it is not only objects that play a role in telicity of events, though we focus on object properties here (\citealt{Thompson:2006}: 214).} 

\ea \label{TelicityVerbPlusObject}
\langinfo{English}{}{\citealt{Thompson:2006}: 212, \citealt{Beavers:2012}: 24} \\

\begin{xlist}
\ex Mary ate an apple in an hour/??for an hour. \label{TelicIndefinite}  
\ex Mary ate apples ??in an hour / for an hour. \label{BarePluralAtelic}
\ex John drank wine ??in an hour / for an hour \label{Non-QuantizedAtelic}
\ex John drank a glass of wine in an hour/??for an hour	\label{QuantizedTelic}
\end{xlist}

\z
\noindent What we see, then, is that the properties of multiple components of a \textit{v}P can influence the aspectual properties of a predicate. \citet{Thompson:2006} shows a variety of evidence (including word order of manner adverbs, among others) that there is movement of DP objects to the edge of \textit{v}P in telic contexts, proposing that telicity is produced by checking [bounded] features at an aspect projection.
	Thompson’s proposal, therefore, is precisely that movement to the edge of \textit{v}P correlates with telicity. Rather than adopt the proposal that this is the result of checking a [bounded] feature, we propose that this is a direct result of the fundamentals of the PR model: 1) phases are referential, 2) \textit{v}P phases refer to events, 3) most specific reference to an event corresponds to telicity, and 4) the general strategy for achieving more specific reference within a phase is moving to the edge of the phase. Given this general PR approach, and following on Sheehan and Hinzen’s (\citeyear{SheehanHinzen:2011}) suggestion that boundedness is the correlate of ``referential specificity'' with respect to events, a finding like Thompson’s (that telicity corresponds with enrichment of the phase edge) is exactly what we would predict. The one new component here that is not directly suggested in Sheehan and Hinzen's work is that raising of the DP object (rather than just the verb) can correlate with higher specificity of reference. 

As mentioned above, \citet{SheehanHinzen:2011} focus on predicational lexical heads (N, V) raising to the edge of their phase in instances of more specific reference. Nothing in their account claims, however, that some other descriptive content of the phase interior ought not contribute to the `greater referential specificity' of the phase in question.\footnote{One potentially problematic aspect of this proposal is that it may challenge somewhat their proposal for threefold ontologies of each phase, which assumes that the predicate is either in the phase interior or in the edge, but doesn’t directly deal with the idea that a portion of the phase's descriptive content could remain in the interior, and a portion raise to the edge. This does not undermine their account, as much as it potentially makes the available ontologies more complex than originally predicted, or perhaps even non-discrete.}  And in fact, Arsenijevi{\'c} \& Hinzen's \citeyear{ArsenijevicHinzen:2012} (henceforth AH) discussion of the PR model gives reason to think that movement of either a verb or the DP object to the edge of the vP phase should in fact be expected as part of greater specificity of phase reference. AH in particular focus on how derivation-by-phase generates the specific sorts of recursivity and intensionality that occur in natural language. They make the argument that all lexical items begin their syntactic lives as predicates, essentially\textemdash that is, as the descriptive content of some phase, which becomes referential when a phase head is merged and when descriptive content is raised to the phase edge. Lexical items themselves are not predicates or arguments, but rather, `predicate' and `argument' are grammatical notions. The descriptive content of a phase becomes referential when that phase is complete–the lexical concept \textsc{man} becomes referential when embedded in a DP phase: \textit{this man} or \textit{the old man} or even kind-referring structures like \textit{men}. Phases are necessarily ordered, then, as parts of a whole (objects are participants in events, which are the foundation of propositions when embedded in a temporal frame).\footnote{This claim of the PR model (that phases induce reference) is also meant to derive the general intensionality of language \citep{Hinzen:2014b,ArsenijevicHinzen:2012,HinzenSheehan:2013}. The interpretation of any phrase or constituent\textemdash even of a proposition\textemdash is dependent on the grammatical structures it occurs within. This accords with a model where any phase-internal material makes up the descriptive content of the reference of the phase that is currently being built, even if part of that phase-internal material is a previous phase.} 

As is clear at this point, the interior of a phase makes up the descriptive content of the higher phase, such that reference to an external object by a DP necessarily must be an object that is described by the lexical N and any other descriptive content (e.g. adjectives or PPs). Likewise, a DP object of a verb is part of the descriptive content of a VP, essentially forming part of the predicate\textemdash the descriptive content\textemdash of the \textit{v}P phase. So in the sentence \textit{Linus ate the pretzels} it is the object DP \textit{the pretzels} and the verb \textit{eat} that make up the descriptive content of \textit{v}P, as they both belong to the phase interior of \textit{v}P. And as such, raising of either the object or the verb itself in a \textit{v}P ought to contribute to the degree of specificity of reference of the phase being built.

What we see, then, is that specificity of reference of an event is governed by (at least) two distinct components of events: the lexical predicate itself, and the arguments of the relevant predicates referring to that event. Specific reference to an event must necessarily include full specification of the participants in the event (e.g. a verb and its arguments) in addition to boundedness. Event Specificity therefore is composed of two distinct but clearly mutually dependent factors: reference to event participants (a/b below), and reference to boundedness/durativity (aspect) of the event (c). 

\ea \underline{Degree of Event Specificity is determined by}:

\begin{xlist}
\ex inclusion of all participants in the event, including \label{IncludeParticipants}
\ex the degree of specific reference to those participants, and %\label{}
\ex aspectual distinctions (telicity)
\end{xlist}

\z
\noindent Intuitively this is relatively uncontroversial following on the discussion of telicity: an event of eating cannot be complete without (implicit or explicit) reference to the agent and the theme. And given the degree to which objects and PPs are tied into (a)telic interpretations of predicates, it is clear that specificity of reference to events includes the properties of the participants in the event. In essence, then, \textit{v}P phases without reference to all the participants of an event are incomplete, a notion that we build on below.

\subsubsection{Anaphora and underspecification of \textit{v}P events}

Recall the PAPA, repeated here as (\ref{PAPArepeat}): 

\ea \label{PAPArepeat} \underline{Principle for Anaphoric Properties of Agreement  (PAPA):} \\
Anaphoric φ-features (i.e. interpretable, unvalued φ-features) adjoin to the edge of vP.
\z
\noindent We have claimed that moving the anaphor into the edge of \textit{v}P provides the anaphoric feature bundle with a value and hence with a reference (which is in turn critical for determining the referential properties of the \textit{v}P, the entire event).

Let us first look at the anaphoric feature set of a reflexive object of a verb: as proposed previously, they are interpretable, unvalued φ-features. Interpretable, valued φ-feature sets are usually referential, i.e. they can be linked to an entity in the discourse. Uninterpretable φ-feature bundles, for instance on Tº or on Cº in Germanic CA-languages, are not referential. They simply reflect the syntactic relationship between, in this example, the verb or the complementizer and the subject. A feature bundle that is unvalued yet interpretable is somewhat of a paradox: it is interpretable, so it should be referential, yet it is unvalued, so it is unclear to what entity it refers exactly. 

We suggest that the presence of this sort of feature set, i.e. referential features that are unspecified with respect to their antecedent, renders the reference of a \textit{v}P event incomplete, underspecified. \citet{Hinzen:2012} and \citet{SheehanHinzen:2011} argue that referentiality is an edge phenomenon. Our proposal is that referential arguments of an event that do not have a value must necessarily raise to the edge of \textit{v}P to be identified, as it were, to become referentially specified. The intuition here is that underspecified \textit{v}Ps are not capable of referring to events. The solution to this paradox is to raise the phase-internal material (the descriptive content of the phase: the anaphoric object here) to the edge of \textit{v}P, where independent operations (i.e. Agree) allow the φ-features to attain a value. In essence, anaphoric φ-features (i.e. interpretable unvalued features) are a syntactic element in search of a referent, and as reference happens at phase edge, anaphoric features raise to the edge of the phase from which position they are valued, by probing the subject in its base position in Spec,\textit{v}P. Therefore, movement of anaphors is not explained soley by the needs of the anaphor, but also by the needs of the event-referring \textit{v}P that the anaphor is embedded within.

From their position at the edge, an anaphoric feature bundle is valued by the syntactic mechanisms generally utilized for feature valuation (Agree), leading to its anaphoric interpretation (a referential DP identified as sharing reference with the subject in Spec,\textit{v}P). The event participants as a result are now fully identified, and the \textit{v}P can be considered sufficiently referentially specific (i.e. able to refer to an event in time). The interpretability of anaphoric features plays a role here, in the sense that uninterpretable features would never enter the calculation of determining referentiality (either of a DP or consequently of a \textit{v}P); they are by definition irrelevant for referentiality and will not participate in this kind of movement-to-edge.

\subsubsection{Movement of anaphoric φ-features}

We are now at the final stage in our discussion toward deriving the PAPA. We established the properties of \textit{v}P phases as (degrees of) specific reference to events, where event specificity depends on two distinct but related notions\textemdash telicity and reference to all event participants\textemdash both of which have been previously described to interact in ways relevant to our proposals here. We then showed how this view of event reference dovetails with approaches to anaphora, providing a possible explanation for the movement of anaphors to the edge of \textit{v}P (as proposed by R\&VW). The PAPA of course extends this proposal to all anaphoric φ-features (not simply object anaphors), which brings us to the present question: why do anaphoric φ-features, even when they are not the object arguments of a verb (and, therefore, not always appearing in the same structural position as objects), show these same PAPA properties of valuation after movement to the edge of \textit{v}P?  That is to say, why do anaphoric φ-features behave like anaphors, even when they are not arguments themselves?

In what preceded we built the argument that event completeness is a key to why anaphors are raised to the edge of \textit{v}P. That is to say, it is not just that anaphors need a referent, but also that unvalued anaphoric φ-features lack reference, and therefore events containing anaphoric feature sets are incomplete, underspecified events. This leaves us at the following set of conclusions regarding anaphoric φ-features: they are probes, being sets of unvalued features that will be valued by Agree, but they are not \textit{just} probes. They are in fact an instruction to the grammar of the event to ``become more referential''. Or, better, to ``find a referent'', or more so, ``become referentially complete''.\footnote{Note that we do not mean to imply that all anaphoric predicates are telic\textemdash telicity effects are dependent both on the semantic properties of lexical verbs as well as on higher aspect. Rather, we mean to say that more specific event reference is triggered by movement to the edge of the syntactic phase referring to that event (\textit{v}P), and that anaphora can be explained by the same movement. A reviewer points out that this account predicts that \textit{v}Ps with anaphoric objects ought to be telic, at least in comparison to \textit{v}Ps with objects that demonstrably remain in their base position. This certainly deserves further exploration\textemdash this may be true, or it may simply be that movement to the edge must increase referential specifity, and moving from an unspecified event to a specified event is the result (i.e. that telicity effects only emerge when movement to the edge occurs within a \textit{v}P that is already complete). We leave these explorations to future work.} And we claim that the human language faculty's universal operation for resolving such instances of referential incompleteness is to raise the relevant structures to the edge of the phase. We suggest that ``reference resolution'' is necessarily an edge phenomenon (for all the reasons we discuss above, following the rich work of Hinzen, Sheehan, and others), and therefore immediate probing of anaphoric φ-features is in fact unexpected (in contrast to non-anaphoric φ-features, which are uninterpretable). In this way, the PAPA captures the syntactic patterns that are the result of the only way that unvalued interpretable features can be valued: at the edge of the \textit{v}P phase.

The extension we have to make is to claim that interpretable, unvalued φ-features at the edge of any phase (not just at the edge of DPs) that are accessible to the higher \textit{v}P results in that \textit{v}P being interpreted as referentially incomplete. We presume that this is because in these instances there is some kind of unresolved interpretive question in the descriptive content of that \textit{v}P that is underspecified (that will therefore make up the descriptive content of the event). In this sense the movement of anaphoric φ-features to the edge of \textit{v}P is altruistic movement. 

\section{Supporting evidence: CA in Kipsigis}\label{CA in Kipsigis}

Support for this analysis comes from recent work on a similar construction in Kipsigis, a Nilotic language of Kenya. Kipsigis is a verb-initial language with canonical VSO word order, but with relatively flexible word order after the verb.\footnote{Most of the data reported here come from \citet{Rao:2016} and \citet{DiercksRao:2017}, data that do not come from those works are noted as coming from field notes.} As in Lubukusu, a declarative-embedding complementizer in Kipsigis can agree with the matrix subject: 

\ea \label{BasicKipsigisSubj-CA}
\langinfo{Kipsigis}{}{\citealt[4]{DiercksRao:2017}} \\
\gll Ko-o-mwaa \textbf{o-lɛ} ko-\O-ɾuuja tuɣa amut. \\
\textsc{pst}-2\textsc{pl}-say \textbf{2\textsc{pl}-C} \textsc{pst}-3-sleep cows yesterday \\
\glt `You (pl) said that the cows slept yesterday.'

\z
\noindent \citeauthor{DiercksRao:2017} refer to this as Subj-CA (CA targeting the subject) for reasons that will become clear momentarily. Kipsigis Subj-CA generally displays similar patterns to Lubukusu: agreement is controlled by the matrix subject and not the embedded subject, matrix non-subjects cannot control the agreement, and only the most local superordinate subject can control agreement \citep{Rao:2016,DiercksRao:2017}. Also, as in Lubukusu, there is a complementizer drawn from the paradigm of agreeing complementizers that can be used in non-agreeing contexts:   

\ea \label{KipsigisNon-AgreeingCA}
\langinfo{Kipsigis}{}{\citealt{DiercksRao:2017}} \\
\gll	Ko-ɑ-mwaa \textbf{kɔlɛ} ko-\O-ɾuuja tuɣa amut. \\
\textsc{pst}-1\textsc{sg}-say that \textsc{pst}-3-sleep cows yesterday \\
\glt `I said that the cows slept yesterday.'
\z

\noindent Kipsigis Subj-CA also carries an interpretive effect as compared to the non-agree-\newline ing complementizer, which \citet{DiercksRao:2017} analyze as signaling that the proposition denoted in the embedded clause is the main point of the utterance (MPU). 

Kipsigis offers several interesting facts that are well-explained by the anaphoric agreement analysis offered here (and quite puzzling otherwise). First, complementizers may overtly raise in the main clause, and second, there is an object-oriented agreeing morpheme that can also occur on the complementizer that is mysterious under an approach like that of \citet{Diercks:2013}, but well-explained under the approach set forward here. To illustrate the first, we point to a phenomenon that \citet{Kawasha:2007} refers to as ``verb ellipsis,'' where the matrix verb {\justifying can be dropped, with only the complementizer introducing the complement\\ clause.} 

\ea
\langinfo{Luvale}{}{\citealt{Kawasha:2007}: 187}\\
\begin{xlist}

\ex 
\gll Etu tu-na-tachikiz-a \circled{ngwetu} ve-ez-anga zau. \\
we 1\textsc{pl}.\textsc{sa}-\textsc{tam}-know-\textsc{fv} \textsc{comp}.1\textsc{pl} 2.\textsc{sa}-come-\textsc{pst} yesterday \\
\glt `We know that they came yesterday.'

\ex 
\gll Etu \circled{ngwetu} mw-a-hasa vene. \\  
we \textsc{comp}.1\textsc{pl} \textsc{fut}-1.\textsc{sa}-be.able indeed\\
\glt `We (think) that he will be able.'\footnote{The interpretation of the elided verb is determined by context.}

\end{xlist}
\z 

\noindent \citet{Kawasha:2007} notes that this occurs in Chokwe (K.10), Luchazi (K.10), Lunda (L.50), and Luvale (K.14); the same occurs in Kipsigis. The verb-initial nature of Kipsigis gives us more insight into what is going on in this construction. As can be seen in (\ref{KipsigisRaisedComp}), the complementizer may occur in the main clause, replacing the matrix verb and preceding matrix arguments:\footnote{There is a main clause verb of speech that is homophonous with the agreeing complementizer, but the verb and the complementizer inflect differently for Obj-CA (vs. verbal object clitics) so the relevant agreement paradigms show that in constructions like this the clause-initial element is indeed the complementizer.} 

\ea \label{KipsigisRaisedComp}
\langinfo{Kipsigis}{}{\citealt{DiercksRao:2017}}\\
\gll Kɔ-lɛ-ndʒin Kiproono (*kɔ-lɛ-ndʒin) ko-\O-ɾuuja tuɣa amut. \\
3-C-2\textsc{sg}.\textsc{obj} Kiproono 3-C-2\textsc{sg}.\textsc{obj} \textsc{pst}-3-sleep cows yesterday \\
\glt `Kiproono told you that the cows slept yesterday.'
\z
\noindent Additional evidence shows that the raised C behaves like a verb of sorts when raised, but not when in its normal position. In (\ref{KipsigisRaisedCompNegation}) a complementizer in its canonical position cannot be negated (in contrast to main clause verbs), as evidenced by (\ref{No Low Neg 1}) and (\ref{No Low Neg 2}). But as is shown in (\ref{Neg High Up}), the complementizer can bear negation when it is functioning as the main verb.\footnote{We do not attempt to explain the lack of negation on the complementizer element, only to show that bearing negation is a main-verb property that complementizers may adopt when appearing in these ``verb ellipsis'' constructions.} 

\protectedex{
\ea \label{KipsigisRaisedCompNegation}
\langinfo{Kipsigis}{}{fieldnotes}\\
\begin{xlist}

\ex \label{Neg on Verb} 
\gll Maa-mwaa-un ɑ-lɛ-ndʒin ko-\O-ɾuuja tuɣa amut. \\
\textsc{neg}.1\textsc{sg}-tell-2\textsc{sg}.\textsc{obj} 1\textsc{sg}-C-2\textsc{sg}.\textsc{obj} \textsc{pst}-3-sleep cows yesterday \\
\glt `I didn't tell you that the cows slept yesterday.'

\ex \label{No Low Neg 1}
\gll *Ko-ɑ-mwaa-un \circled{mɑɑ-lɛ-ndʒin} ko-\O-ɾuuja tuɣa amut. \\
\textsc{pst}-1\textsc{sg}-tell-2\textsc{sg}.\textsc{obj} \textsc{neg}.1\textsc{sg}-C-2\textsc{sg}.\textsc{obj} \textsc{pst}-3-sleep cows yesterday \\
\glt `I didn't tell you that the cows slept yesterday.'

\ex \label{No Low Neg 2}
\gll *Maa-mwaa-un \circled{maa-lɛ-ndʒin} ko-\O-ɾuuja tuɣa amut. \\
\textsc{neg}.1\textsc{sg}-tell-2\textsc{sg}.\textsc{obj} \textsc{neg}.1\textsc{sg}-C-2\textsc{sg}.\textsc{obj} \textsc{pst}-3-sleep cows yesterday \\
\glt `I didn't tell you that the cows slept yesterday.'

\ex \label{Neg High Up}
\gll \circled{Maa-lɛ-ndʒin} ko-\O-ɾuuja tuɣa amut. \\
\textsc{neg}.1\textsc{sg}-C-2\textsc{sg}.\textsc{obj} \textsc{pst}-3-sleep cows yesterday \\
\glt `I didn't tell you that the cows slept yesterday.'

\end{xlist} 
\z}
\noindent We assume that there is a null verb of speech on Kipsigis that occurs in these constructions (and other languages with similar constructions). When the complementizer undergoes movement into the main clause, if the main verb is null it presumably allows for m-merger with the raised complementizer (at the edge of \textit{v}P), resulting in the complementizer appearing in the main clause. This kind of analysis is confirmed by the fact that when the complementizer behaves verb-like and appears clause-initially, it is impossible for the complementizer to appear in its canonical position (as shown in (\ref{KipsigisRaisedComp})). This complementary distribution is corroborating evidence that the clause-initial element is in fact the complementizer. The details are not important for our present purposes, however\textemdash the fact that agreeing complementizers can appear \textit{overtly} in the main clause is strong evidence that the subject-agreeing complementizers can agree with matrix subjects precisely because they have raised into the main clause (as we have proposed above).  

A second argument comes from the fact that agreeing complementizers in Kipsigis may also bear \textit{object}-oriented agreeing morphemes as well (Obj-CA). 

\ea \label{ObjCAExample} 
\langinfo{Kipsigis}{}{\citealt{DiercksRao:2017}}\\
\gll Ko-i-maa-\circled{ɑn} i-lɛ-\circled{ndʒɑn} ko-\O-ɪt laɣok. \\
\textsc{pst}-2\textsc{sg}-tell-1\textsc{sg}.\textsc{obj} 2\textsc{sg}-C-1\textsc{sg}.\textsc{obj} \textsc{pst}-3-arrive children \\ 
\glt `You (sg) DID tell me that the children arrived.'
\z

\noindent Obj-CA can only be triggered by matrix objects, not matrix subjects, and it is `optional' in the sense that it is not always present. There is no default form of Obj-CA; when Obj-CA does not occur the morpheme is simply absent (notably, this is different from Subj-CA, which shows default agreement in impersonal constructions). And most notably for our point here, Obj-CA can \textit{only} occur when Subj-CA is present; Obj-CA is unacceptable on a non-subject-agreeing complementizer, as shown in (\ref{Obj-CA w. Subj-CA}): 

\ea \label{Obj-CA w. Subj-CA}
\gll Ko-ɑ-mwaa-un ɑ-lɛ(-ndʒin)/*kɔlɛ-ndʒin ko-\O-ɪt tuɣa amut. \\
\textsc{pst}-1\textsc{sg}-tell-2\textsc{sg}.\textsc{obj} 1\textsc{sg}-C(-2\textsc{sg}.\textsc{obj})/*C-2\textsc{sg}.\textsc{obj} \textsc{pst}-3-arrive cows yesterday \\
\glt `I told you that the cows arrived yesterday.'
\z
\noindent These facts raise hard questions\textemdash even if the properties of Subj-CA in the languages that have it can be explained via an anaphoric explanation, to our knowledge there are not any purely object-oriented anaphors. How, then, can Obj-CA be explained? \citet{DiercksRao:2017} suggest that this set of facts is consistent with an analysis that Obj-CA is a clitic-doubling operation (a clitic on the complementizer doubling the matrix object), whereas Subj-CA is simply an agreement morpheme. But it is completely unclear how a clitic-doubling operation is possible on a complementizer embedded within a complement clause, unless that complementizer at some level of the derivation raises to a level higher than the matrix object (which is precisely what we have suggested in this paper). Notably, Obj-CA is only possible on complementizers with Subj-CA, which is what is expected if it is only those complementizers that have raised into the main clause.  

A full exploration of the mechanics of the Kipsigis Obj-CA construction go beyond the scope of this paper. It should be clear that these two sets of Kipsigis facts\textemdash the possibility of complementizers overtly raising into the main clause, and Obj-CA patterns\textemdash are largely consistent with an analysis where upward-agreeing complementizers raise into the main clause, and quite difficult to explain otherwise. 

\section{Other analyses of Lubukusu CA }\label{sectioncarstens}

In recent work, \citet{Carstens:2016} has argued against Diercks’ (\citeyear{Diercks:2013}) analysis that Lubukusu CA is anaphoric, claiming instead that upward-orientation is a standard and generalizable property of Agree. She proposes that the $ \phi $-features on the Lubukusu C° head are forced to seek valuation higher in the structure because probing of their own c-command domain has failed. Carstens terms this process delayed valuation, and posits two different mechanisms by which it may happen:

\begin{exe}
\ex \underline{Directionality-Free Mechanics of Delayed Valuation (\citealt{Carstens:2016}: 3)}\\
	uF with no match in its c-command domain can be valued:
	\begin{enumerate}
		\item Ex situ, by raising into locality with a matching feature, OR
		\item In situ, by the closest matching feature within the same phase
	\end{enumerate}
\end{exe}

{\noindent}The \textit{ex situ} valuation is similar to R\&VW's proposal that we utilize here, and is a version of Bo\v{s}kovi\'{c}'s (\citeyear{Boskovic:2007b} and \citeyear{Boskovic:2011}) proposal where unvalued features of a moving item drive its movement. The \textit{in situ} valuation, on the other hand, shares much conceptually with Bobaljik and Wurmbrand's (\citeyear{BobaljikWurmbrand:2005}) notion of feature valuation within agreement domains. Carstens uses this notion of delayed valuation of features to explain a range of feature-valuation operations.

With respect to Lubukusu CA, this is a similar sort of proposal to the one that we advocate here. The difference boils down to whether this Lubukusu CA is viewed as anaphoric, and as such bears distinct qualities from non-anaphoric feature valuation, or whether Lubukusu CA is instead indicative of the general properties of non-anaphoric feature valuation. \citet{Carstens:2016} connects the Lubukusu CA facts with a broad variety of other feature-valuation facts like Case-valuation, concluding (like we do here) that there is simply one feature valuation operation, namely, Agree.  In order to explain the upward-orientation of Agree, however, she adopts a view similar to \citet{Bejar:2009}, that a failure of downward probing triggers an upward-oriented valuation operation, which may include either movement or valuation by a higher element within the same phase. Our analysis, on the other hand, proposes a particular kind of behavior of unvalued, interpretable feature sets that is connected to anaphoric phenomena; interpretable, unvalued features will move to a phase edge and probe from that position. Essentially, while both Carstens’ proposal and the one advanced here maintain that only a single feature-valuation mechanism is necessary in the syntax, Carstens liberalizes the Agree operation more generally, whereas we link the movement and valuation to a distinct, derivative kind of feature valuation\textemdash anaphoric feature valuation\textemdash which is a composite of two (already-available) syntactic operations.

What evidence could distinguish these proposals? One relevant area is the availability of CA in Lubukusu in instances of raising to object, as shown in the example below: 

\ea	\label{RtOwithCA}
\langinfo{Lubukusu}{}{Justine Sikuku, pc} \\
\gll N-eny-a Barack Obama n-di a-khil-e.\\
1\textsc{sg}.\textsc{sa}-want-\textsc{fv} 1Barack Obama 1\textsc{sg}-that 1\textsc{sa}-win-\textsc{fv}\\
\glt `I want Barack Obama to succeed.'
\z
\noindent If we are to adopt a relatively uncontroversial assumption that the embedded subject raises to an object-licensing position in the main clause (perhaps AgrO below \textit{v}P), the account we propose here explains the non-intervention of the raised object in the CA relation naturally because the unvalued, interpretable features of the complementizer adjoin to \textit{v}P. However, this example is problematic for Carstens as on her account upward probing is only the result of the failure of downward probing. Presumably, however, if the lower clause is permeable for raising of the object, it should not be a phase and hence should also be permeable for probing by the complementizer head. Carstens claims that objects in raising to object (RtO) constructions like those in (\ref{RtOwithCA}) (i.e. those that raise across an agreeing complementizer) are A’-moved into the matrix clause (following Bruening's \citeyear{Bruening:2001} analysis of RtO), and that the lower clause is indeed a phase in these instances. We assume, in contrast, that such elements are in fact A-moved, which is supported by the fact that such objects can participate in standard object marking constructions (assumed to be an A-relation, as only arguments can be object marked; \citealt{Diercks:2011,SikukuEt:2017}). The example in (\ref{OMedRtOObject}) shows that an RtO object can be object marked, and (\ref{RtO,OM-doubling}) shows that a DP object in an RtO construction may be (clitic-)doubled by an OM.%\footnote{Note again here, as in the previous examples in the paper, that co-occurrence of an OM with an in situ object is only possible in specific pragmatic contexts, which \citet{SikukuEt:2017} analyze as verum (focus).} 

\ea
\langinfo{Lubukusu}{}{Justine Sikuku, pc} \\

\begin{xlist}

\ex \label{OMedRtOObject}
\gll E-mu-eny-a n-di a-khil-e. \\
1\textsc{sg}.\textsc{sa}.\textsc{prs}-1\textsc{om}-want-\textsc{fv} 1\textsc{sg}-that 1\textsc{sa}-win-\textsc{fv}\\
\glt `I want him to succeed.'

\ex \label{RtO,OM-doubling}
\gll E-mu-eny-a Barack Obama n-di a-khil-e.\\
1\textsc{sg}.\textsc{sa}.\textsc{prs}-1\textsc{om}-want-\textsc{fv} 1Barack Obama 1\textsc{sg}-that 1\textsc{sa}-win-\textsc{fv} \\
\glt `I DO want Barack Obama to succeed.'

\end{xlist}
\z
\noindent The availability of object marking objects in RtO contexts argues against an A’-movement account of raised objects. Rather, this suggests that raising to object is in fact A-movement, in which case the embedded CP should not be a phase boundary and should not cause failure of a downward-oriented probe on Cº, raising questions for Carstens’ account as to why the Cº head still is upward-oriented in its valuation in (\ref{RtO,OM-doubling}).

\section{Conclusions and open questions}

The primary issue we sought to explore in this paper was whether or not a universal direction of probing in Agree-relations could be established cross-linguistical-\newline ly. Recent proposals have suggested that constructions exist in various languages exhibiting both upward- and downward-oriented probing of Agree, and others have suggested that only upward probing exists. This paper makes a broad argument from a narrow empirical domain—complementizer agreement—considering the properties of CA in Dutch dialects (Germanic) and Lubukusu (Bantu). Pre-theoretically, there are clearly both upward- and downward-oriented agreement patterns; the question becomes what feature valuation mechanisms are necessarily a part of Universal Grammar. In \S \ref{GermanicCA} and \S \ref{LubukusuCA} we demonstrated that these agreement phenomena cannot reduce to a single, unified syntactic operation \oneline{(= Agree);} however, in \S \ref{AnalysisSection} we make the case that this situation does not necessitate the inclusion of new grammatical operations to license CA in Lubukusu. We propose, following Rooryck \& Vanden Wyngaerd (2011), that anaphoric relations such as those found in Lubukusu CA are realized via a composite operation of Internal Merge + downward-probing Agree. On this account, clearly divergent agreement relations can be explained using the same feature valuation operation, with the added component that anaphoric feature bundles must move before they can be valued (the PAPA = Principle for the Anaphoric Properties of Agreement). 

In \S \ref{ExplainPAPA} we proposed a motivation for raising interpretable, unvalued features to the edge of a phase; this discussion called on recent approaches to the referential interpretation of phases and the effects on specificity of reference by movement to the edge of phases. 
        
There remain many questions that we are unable to address in a paper of this size. Empirically, it is becoming clear that there is variation in (upward-oriented) CA patterns cross-linguistically which will be relevant to the best analysis of CA and consequently the best theoretical approach to Agree. For example, \citet{LetsholoSafir:2017} show that Ikalanga complementizer agreement patterns, while agreeing with the matrix subject, can reflect the tense and voice (active/passive) of the matrix clause. Likewise, \citet{Nformi:2017} documents a defective intervention pattern where the upward-oriented subject-agreeing complementizer agreement relation in Limbum can be disrupted by a matrix indirect object, despite being unable to agree with that intervening DP. Both patterns pose challenges to the current account that would require additional work to accommodate under our claims here. And besides these patterns, it is clear from the growing range of work on similar phenomena that we do not yet know the full range of empirical patterns that are possible on upward-agreeing complementizers, so additional empirical work will surely prove an important testing ground to the claims here.\footnote{\citet{Nformi:2017} claims that the Limbum patterns require an upward-probing account, which more naturally accommodates the defective intervention pattern of indirect objects in Limbum complementizer agreement. Our account as presented here would clearly require some revision to explain this Limbum pattern, but we do not engage the Limbum question in depth here because it appears to us that more work is necessary to fully understand the Limbum patterns. \citet{Nformi:2017} claims that CA is case-discriminatory and requires nominative case (following \citealt{Bobaljik:2008}), but Bobaljik's claim is that Agree is postsyntactic following assignment of morphological case, and tracks morphological case, whereas all of the Bantu patterns under consideration lack morphological case at all. Therefore it is quite unclear under any available account how to accommodate these data (especially since intervening DPs in morphological causatives in Limbum are \textit{not} interveners, and the case-based approach is insufficient). Adopting an analysis that agreeing complementizers must agree with a nominative DP also assumes the outcome of what we are trying to derive from more fundamental principles in this work. The Limbum patterns raised by \citet{Nformi:2017} are certainly important empirical complications for the account raised here, but we leave the question for future work.} 

Theoretically, there also remain a variety of open questions. For example, while we have specifically claimed that the interpretive effects of upward-oriented CA are a consequence of the anaphoric feature sets containing interpretive features, we have not provided a specific outline of \textit{how} these are derived.\footnote{In previous versions of this paper we proposed that the interpretation of interpretable, unvalued φ-features is essentially that of an intensifier, and proposed a way in which intensifiers on CP might create similar kinds of interpretive effects to Lubukusu CA when they arise on a specific indefinite CP (whose interpretation is generated via choice function). Space does not allow us to lay those ideas out here.} And perhaps the largest standing question in our proposals is the issue of delayed valuation. The PAPA requires that anaphoric φ-features be adjoined to \textit{v}P and being valued by the subject in that position, and we have laid out an extensive line of reasoning based on the PR model of syntax for why this is a reasonable proposal. But we did not fully explain why Lubukusu φ-features on C cannot probe their c-command domains from their base positions. A possible explanation may arise from the relative economy of doing this valuation at the superordinate \textit{v}P edge, with the result that there is some sense in which underspecified reference must be resolved at phase edge by the very nature of the syntactic architecture of phases (as this is where reference is established/managed; this is not inconsistent with Chomsky's \citeyear{Chomsky2008} claims about other phi feature valuation). At present, however, the precise issue of delayed valuation remains among the standing questions. 

Looking forward to future work, there is a clear testable prediction that arises from this account, which is that anaphoric feature valuation (i.e. instances of surface downward valuation or apparent upward probing) ought to have interpretive effects, as they are rooted in interpretable, unvalued features. We showed that this was the case for Lubukusu/Kipsigis vs. Germanic CA, where Lubukusu/Kipsigis CA influenced interpretation of a sentence whereas Germanic CA is simply a case of feature covariance. This prediction is laid out in (\ref{AnaphAgrCorollary}): 

\ea \label{AnaphAgrCorollary}
\underline{Anaphoric Agreement Corollary} \\
Upward-oriented agreement relations will have interpretive effects.
\z
\noindent The Anaphoric Agreement Corollary could well explain the tendency of the Upward Agree theorists to rely on evidence from domains such as negative concord and sequence of tense, whereas the downward Agree theorists tend to focus on issues of (uninterpretable) phi-feature agreement, though we leave a full evaluation of this prediction for future work (see, for example, \citealt{Bjorkman:toappearb}, and \citealt{Preminger:2013}). 

With respect to the discussion of the directionality of Agree, we conclude that unvalued features only probe down.  This does not deny, however, that there are instances of feature valuation where the valuer is structurally higher than the valuee, only that such instances are not instances of `pure' Agree, but instead are derived by movement followed by Agree. The result, therefore, is wide-reaching in providing support to a feature-valuation analysis of anaphors, in providing theoretical backstopping to the relatively common proposal that anaphors raise into their predicate (or into a local relationship with their antecedent) in order to ensure valuation/co-reference with that antecedent, and also in arguing that upward probing of Agree is an unnecessary component of the grammar, accomplished instead by anaphoric mechanisms that are quite general. The second part of the paper provides a first proposal for a Phase Reference model of \textit{v}Ps as event reference and proposed a range of ideas regarding both how this applies in basic instances, but also how this relevantly explains aspects of the Lubukusu CA puzzle. Clearly much research remains in all these domains\textemdash theories of Agree, documentation of CA cross-linguistically, and the development of the Phase Reference model\textemdash but the proposals here contribute to our current understanding of all three. 

\section*{Abbreviations}

\begin{multicols}{2}
	\begin{tabbing}
		\textsc{fs/fv}\hspace{5mm} \= Principle for the Anaphoric Properties of Agreement\kill
\textsc{ap} \> applicative \\ 
\textsc{ca} \> complementizer agreement \\ 
\textsc{caus} \> causative \\ 
\textsc{comp} \> complementizer \\
\textsc{fv} \> final vowel \\ 
\textsc{fca} \> first conjunct agreement \\
\textsc{fut} \> future \\
\textsc{obj} \> object \\
\textsc{om} \> object marker \\
PAPA \> Principle for the Anaphoric Properties of Agreement \\
\textsc{pass} \> passive \\
\textsc{pl} \> plural \\
\textsc{prf} \> perfective \\ 
\textsc{prs} \> present \\ 
\textsc{pst} \> past \\ 
\textsc{sa} \> subject agreement \\ 
\textsc{sbj} \> subjunctive \\
\textsc{sg} \> singular \\
\textsc{ta} \> tense agreement \\
\end{tabbing}
\end{multicols}

\section*{Acknowledgements}
The authors would like to express their gratitude to audiences at WCCFL 30 at UC-Santa Cruz and at the The Minimalist Program: Quo Vadis? – Newborn, Reborn, or Stillborn? Workshop in Potsdam for their helpful feedback. In particular, our thanks to Jeff Lidz, Johan Rooryck, Ken Safir, and Susi Wurmbrand for their comments, and our thanks to Michael Clausen and Meredith Landman at Pomona College for serving as sounding boards at various times. Our discussions with Vicki Carstens and her work on this issue have been particularly challenging and helpful as we have developed this paper. We also thank a total of six anonymous reviewers over the years, who posed significant and stimulating challenges to initial versions of this work that have quite significantly raised the quality of the resulting paper. Support from this project has come from Pomona College, from an NSF Collaborative Research Grant (Structure and Tone in Luyia: BCS-1355749) and from the NWO-grant (The universality of linguistic variation: 276-89-002). The authors blame each other for any remaining errors or inconsistencies.



\sloppy
\printbibliography[heading=subbibliography,notkeyword=this]


\end{document}
