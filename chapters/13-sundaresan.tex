\documentclass[output=paper, modfonts, nonflat]{langsci/langscibook} 

\title{Distinct featural classes of anaphor in an enriched person system} 
\author{Sandhya Sundaresan\affiliation{Universität Leipzig}}

% \chapterDOI{} %will be filled in at production

% \epigram{}

\abstract{This paper tackles the fundamental question of what an
  anaphor actually is --- and asks whether the label ``anaphor'' even
  carves out a homogenous class of element in grammar.  While most
  theories are in agreement that an anaphor is an element that is
  referentially deficient in some way, the question of how this might
  be encoded in terms of deficiency for syntactic features remains
  largely unresolved. The conventional wisdom is that anaphors lack
  some or more \ph-features. A less mainstream view proposes that
  anaphors are deficient for features that directly target
  reference. Here, I present different types of empirical evidence
  from a range of languages to argue that neither approach gets the
  full range of facts quite right. The role of \person, in particular,
  seems to be privileged. Some anaphors wear the empirical properties
  of a \person-defective nominal; yet others, however, are sensitive
  to \person-restrictions in a way that indicates that they are
  inherently specified for \person. Orthogonal to these are anaphors
  whose distribution seems to be regulated, not by \ph-features at
  all, but by perspective-sensitivity. Anaphors must, then, not be
  created equal, but be distinguished along featural classes. I
  delineate what this looks like against a binary feature system for
  \person{} enriched with a privative [sentience] feature. The
  current model is shown to make accurate empirical predictions for
  anaphors that are \emph{in}sensitive to \person-asymmetries for the
  PCC, animacy effects for anaphoric agreement, and instances of
  non-matching for \num{} and \person{}.}

\begin{document}

\maketitle

\section{Overview}

The conventional wisdom is that an anaphor like ``himself'' is
anaphoric because it lacks independent reference. At the same time, it
differs from a pronoun like ``him'' because an anaphor must already be
bound in the syntax, in a way that the pronoun need not, and indeed
\textit{cannot}, be (Conditions A and B of the Binding Theory,
respectively of \citealt{Chomsky:1981}).  In Minimalism, this idea is
captured by proposing that the anaphor lacks some feature in the
syntax. Valuation or checking of this feature under Agree by another
element (a nominal or functional head), triggers anaphoric binding at
LF. Construing binding in terms of Agree has the advantage that the
characteristic distributional properties of local anaphora (Binding
Condition A of \citealt{Chomsky:1981}), falls out epiphenomenally
\citep{Hicks:2009}. What still remains very much an open question,
however, is the featural content of what the anaphor and its
antecedent Agree for. The mainstream view is that anaphors are
\ph-deficient nominals \citep{heinat:2008, kratzer:2009, reuland:2001,
  Reuland:2011, roorwyn:2011}. But there is another, less central
view, which proposes to capture the referential dependency of anaphors
by arguing that they directly lack referential features
\citep{adgerramchand:2005, Hicks:2009}.\footnote{See also \textcitetv{chapters/11-diercks-etal} for very interesting discussion of the featural make-up of anaphors and their agreement behavior.}

The goal of this paper is to show that, while these views tell us
parts of the story, they crucially obscure others. Once we broaden our
field of scrutiny to include a range of empirical phenomena from a
number of different languages, a more nuanced pattern emerges. The
\person-feature, in particular, is shown to play a rather central
divisive role with respect to anaphora. Based on their
antecedence-taking properties, the Anaphor Agreement Effect (AAE)
\citep{rizzi:1990}, and certain types of morphological
underspecification and \ph-matching, some anaphors seem to lack the
\person{} feature. However, \person-restrictions reflected in
anaphoric agreement, sensitivity to PCC effects, and a rarely
discussed 1st, 2nd, vs.\ 3rd asymmetry in anaphoric antecedence
\citep{comrie:1999}, suggest that certain other anaphors are
inherently specified for \person. Running orthogonal to both is a
class of perspective-sensitive anaphora \citep[including so-called
logophora][]{clements:1975} whose antecedence is regulated by
perspective-holding with respect to some predication containing the
anaphor \citep{sells:1987, kuno:1987, koopmansportiche:1989,
  giorgi:2010, pearson:2013}. Recent work has argued that such
relationships must also be implemented in terms of a syntactic
dependency between the anaphor and its antecedent
\citep{sundaresan:2012, pearson:2013, nishigauchi:2014,
  charnavel:2016}. If this is correct, then anaphora thus doesn't
target a single homogenous class of nominal. Rather, it picks out
nominals that all end up being referentially bound by featurally
distinct routes. This then begs the question of what an anaphor
actually is, and whether it even makes sense to talk about an anaphor
as a coherent class of grammatical elements.

\noindent Standard theories classify \person{} into three categories: 1st, 2nd,
and 3rd. I argue here that such a classification is not fine-grained
enough to capture all the referential distinctions the full range of
anaphors in language needs recourse to. We need (at least) six
referential categories, as illustrated in Table \ref{per1}.
\begin{table}[h]
	\caption{Person Cross-Classification}
	\label{per1}
	\begin{tabularx}{\textwidth}{lll}
		\lsptoprule
		\textbf{Features} & \textbf{Category} & \textbf{Exponents}\\  
		\midrule
		{[+Author, +Addressee, sentience]} & \textsc{1incl.} & \textit{naam}
		(Tamil, \textsc{1incl.pl})\\
		{[+Author, -Addressee, sentience]} &  \textsc{1excl.} &
		\textit{naaŋgaɭ}
		(Tamil, \textsc{1excl.pl})\\
		{[-Author, +Addressee, sentience]} & \textsc{2} & \textit{you}\\
		{[-Author, -Addressee, sentience]} & \textsc{3} & \textit{him,
			sie} (German), \textit{si} (Italian)\\
		\midrule
		{[sentience]} & \textsc{Refl} & Anaphors in Bantu\\
		$\emptyset$ & \nul{} & \textit{ziji} (Chinese),
		\textit{man} (German)\\
		\lspbottomrule
	\end{tabularx}
\end{table}
\newline\noindent   Table \ref{per1} shows that there is not one, but three, non-1st and
  non-2nd \person-categories. The [sentient] feature is marked on
  nominals that denote individuals that have the ability to be
  mentally aware and bear a mental experience, and in turn entails
  semantic animacy. Categories that are contentful for \person,
  [$\pm$Author] and [$\pm$Addressee], thus automatically bear this
  feature.  Articulated \person-classifications involving similar
  binary features have, indeed, been previously proposed (see
  e.g. \citealt{nevins:2007, anag:2005}, a.o.). The novel contribution of this
  paper is that it provides empirical support for such a feature
  system from a relatively untested empirical phenomenon, namely that
  of anaphora.\footnote{This said, it should be clarified from the
    outset that one of the central goals of this paper is to provide
    empirical evidence from anaphora for the greater articulation of
    \person-features, and not for the claim that such features should
    necessarily be modelled in terms of a binary feature
    structure. Put another way, such a level of articulation may well
    also be modelled through feature hierarchy systems such as
    \citet{harleyritter:2002} or a lattice-based model of
    \person-partitions like \citet{harbour:2016}.}

  Against such a featural system, we have the typology of anaphors
  given in Table \ref{anaph3}. This will be shown to capture the full
  range of empirical properties demonstrated by anaphors, discussed in
  the course of the paper.
  \begin{table}[!h]
  	\caption{Four classes of anaphor}
  	\label{anaph3}
  	\begin{tabularx}{\textwidth}{llX}  
  		\lsptoprule
  		\textbf{Class} &  \textbf{\person-Features} & \textbf{Exponents}\\  
  		\midrule
  		\textsc{3}rd-anaphor &  {[-Author, -Addressee, sentience]} & \textit{taan} (Tamil),
  		\textit{zich} \textit{(zelf)} (Dutch)\\
  		% \mid
  		\refl{}  & [sentience] & Bantu anaphors\\
  		%\mid
  		\nul-anaphor &  $\emptyset$ & \textit{ziji} (Chinese),
  		\textit{zibun} (Japanese)\\
  		\midrule
  		\textbf{Class} & \textbf{Non-\ph-Feature} & \textbf{Exponents}\\
  		\midrule
  		Perspectival anaphors & [\dep] & \textit{taan},
  		\textit{ziji}, 
  		\textit{sig} (Icelandic)\\  		
  		\lspbottomrule
  	\end{tabularx}
  \end{table}
The model developed here makes testable empirical predictions with
  respect to the PCC, \ph-matching, sentience effects in anaphoric
  agreement, and the AAE. I show that these are positively confirmed,
  attesting to the validity of the current approach.

\newpage

\section{Phi-based views of anaphora}
\label{secphi}

One of the main advantages of the \ph-deficiency approach is its
  theoretical parsimony. All the approaches predicated on this idea
  build on the fundamental assumption that an anaphor is defined by
  its lacking one or more \ph-features. \ph-features are independently
  motivated in language --- be it as an inherent property of nominal
  elements or as an acquired property on verbal ones. Such an approach
  thus avoids the inelegant pitfall of positing features that are
  peculiar to anaphors alone. The theoretical motivation for such a
  view may be traced back (at least) to an observation by
  \citet{bouchard:1984} that a nominal needs a a full set of
  \ph-features to be LF-interpretable. As such, any nominal that lacks
  a full \ph-feature specification must get its missing \ph-features
  checked in syntax, on pain of being subsequently uninterpretable at
  LF.

  Theories that are based on the \ph-deficiency view do not form one
  homogenous class: in fact, they differ significantly with respect to
  ancillary assumptions regarding the internal structure and overall
  feature-composition of an anaphor and, in some cases, also the
  nature of the Agree dependency between the anaphor and its
  antecedent. A fundamental variation arises with respect to
  assumptions concerning what \ph-featural deficiency actually
  means. For one thing, is the anaphor simply unvalued for
  \ph-features or does it lack them altogether (and how can we tell)?
  For another, does it lack some \ph-features or all (and again, how
  can we tell)?  \citet{kratzer:2009} proposes, for instance, that
  anaphors are ``minimal pronouns'' --- they lack not just the values,
  but also the attributes, for all \ph-features. Agree (or feature
  unification, in Kratzer's system) allows an anaphor to acquire all
  and only those features it actually surfaces with, yielding a
  transparent mapping between syntax and
  morphology. \citet{roorwyn:2011}, alternatively, propose that
  anaphors are merely lacking in \ph-values, which get valued in the
  course of the derivation via Agree. An issue that crops up in this
  context is what formally distinguishes an anaphor from a pronoun
  bearing identical \ph-features in the same structural position, once
  the anaphor's \ph-features have been valued. Rooryck and van den
  Wyngaerd suggest a brute-force solution: inherited features must be
  distinguished from inherent features by their bearing a ``*''
  featural diacritic. Yet others \citep{heinat:2008,reuland:2001,
    Reuland:2011, dechainewiltschko:2012} present independent
  arguments to distinguish anaphors from other nominals, not
  featurally, but in terms of their internal structure. Regardless of
  how this is formalized, however, this is a central problem that
  \textit{any} account that anaphors are deficient for a feature that is
  assumed to underlie \textit{all} nominals: the anaphor must continue
  to be distinguished from other nominals at the interfaces after this
  deficiency has been ``cured'' via Agree.

  The fundamental motivation of the reference deficiency view, in
  contrast, is that while the \ph-features of a nominal restrict its
  domain of reference (in the evaluation context), they crucially
  don't exhaust it. \ph-features introduce presuppositions that
  restrict, via partial functions, the lexical entry of nominals
  \citep{heimkratzer:1998}, as in (\ref{she}) below:
\begin{exe}  
\ex {$\llbracket$}she$\rrbracket^{c,g}$ = $\lambda$x: \textit{x is female \& x is an atom.x} \label{she}
\end{exe}

\noindent \citet{Hicks:2009} further notes that, under a \ph-deficiency view,
anaphors that are overtly specified for all their \ph-features, like
reflexives in English, would be predicted to behave like deictic
pronouns. While conceding that ``One possibility could be that the
morphological features are only assigned to the reflexive once they
receive a value from the Agree relation'', he rightly points out that,
``as soon as we allow this we lose the original diagnostic for
determining what is an anaphor and what is a pronoun according to
their overt \ph-morphology'' (\citealp{Hicks:2009}, 111).  Hicks
proposes, instead, that anaphoric dependence is built on
operator-variable features, along the lines of
\citet{adgerramchand:2005}. An anaphor is a semantically bound
variable: this is transparently reflected in its syntactic profile,
with an unvalued \var{} feature. An R-expression or a (deictic)
pronoun, in contrast, is born with an inherently valued \var, with
values being integers or letters that are arbitrarily assigned in the
course of the derivation.\footnote{This is not a trivial
  assumption. If Hicks were to assume, instead, that R-expressions and
  pronouns were lexically distinguished in terms of their \var-values,
  a valued \var{} would simply reduce to a referential index, in turn
  violating the Inclusiveness Condition in \citet[][381]{chomsky:1995}. Hicks assumes, therefore, that a pronoun or R-expression is
  born with a feature whose value is simply a \emph{pointer} or
  \emph{instruction} to be converted to an arbitrary integer or letter
  upon Merge.}  Quantifiers, like `all' and `some' have \op{} features
[\op: $\forall$] and [\op: $\exists$], respectively, yielding
derivations like (\ref{every1}) and (\ref{every2}) for
(\ref{everyhicks}):

\ea\label{everyhicks} Every toddler injures herself.
\ea\label{every1} Every$_{[\op:\forall]}$ toddler$_{[\var:x]}$ injures
herself$_{[\var:\ul{~~}]}$
\ex\label{every2} Every$_{[\op:\forall]}$ toddler$_{[\var:x]}$ injures
herself$_{[\var:x]}$ \z \z

\noindent Hicks also assumes that \emph{every} nominal has a \var{} feature:
this in turn ensures that an anaphor will be bound by the closest
c-commanding nominal that has a valued \var{} feature, yielding
Condition A epiphenomenally.
  
Below, I discuss some of the empirical properties that may be taken to
support the mainstream \ph-deficiency approach. But the notion of
referential defectiveness in an approach like \citet{Hicks:2009} is
itself crucially predicated on \ph-defectiveness, given the
afore-mentioned idea that \ph-features presuppositionally restrict
nominal reference. As such, many of the empirical properties below may
arguably be captured under the referential-deficiency view, as well. I
will henceforth use the term ``\ph-based'' to subsume both
\ph-deficiency and reference-deficiency approaches to anaphora.

\subsection{Anaphora and phi-matching}

Anaphors must typically match their antecedents for \ph-features, a
crosslinguistic tendency that has been explicitly noted as a required
condition on binding in syntax textbooks and elsewhere
\citep{sagwasbend:2003, carnie:2007, heim:2008}. Thus, (\ref{engbad})
is ungrammatical because the anaphor has \textsc{1sg} \ph-features
which don't match the \textsc{3msg} features of its binder:
\ea\label{engbad} *He$_i$ saw myself$_i$.  \z

\noindent Such \ph-matching seems to be a restriction on simplex anaphors as
well, as illustrated by the ungrammaticality of the German counterpart
to (\ref{engbad}) in (\ref{germbad}):

\ea\label{germbad} *Er$_i$ sah mich$_i$.
\z

\noindent Under a \ph-deficiency approach, this falls out for free. If an
anaphor must have one or more unvalued \ph-features and anaphoric
binding is triggered by the anaphor having its \ph-features valued,
via Agree, then such \ph-matching is, indeed, precisely what is
predicted. But this arguably also falls out naturally under a
reference-deficiency approach as in \citet{Hicks:2009} (or
\citealt{adgerramchand:2005}). The difference is that, here, such a
restriction would be the result of a \emph{semantic} incompatiblity
between the projected presuppositions of the individual nominals in
the binding relation. 

There are, of course, cases where no \ph-matching can be discerned, as
in Albanian, Chinese, Yiddish or Russian. This is illustrated for the
Albanian examples below (\citealt[270-271]{woolford:1999}, see also
\citealt[91]{hubbard:1985}):
\ea\label{alb2}\gll {Drites$_i$} dhimset vetja$_i$.\\
Drita.{\sc dat=3sg.dat} pity.{\sc 3sg.past.nact} \anaph.{\sc
  nom}\\
\glt `Drita$_i$ pities herself$_i$.'
\ex\label{alb3}\gll {Vetja$_i$=me$_i$} dhimset.\\
\anaph.{\sc nom=1sg.dat} pity.{\sc 3sg.prs.nact}\\
\glt `I$_i$ pity myself$_i$.'  \z

\noindent However, what such examples show is the \emph{absence} of overt
\ph-matching, not the \emph{presence} of overt
\emph{non-}matching. Under \citet{kratzer:2009}, a minimal pronoun (or
anaphor) is bound by a dedicated reflexive \lilv{} which, in addition
to its \ph-features, will transmit its ``signature'' reflexive feature
to the anaphor. This means that ``sometimes the signature feature is
all that is ever passed on to a minimal pronoun''
(\citealp{kratzer:2009}, 198).  It is when this happens, Kratzer
proposes, that the anaphor is spelled out as an invariant form, as in
the Albanian examples above. Note, however, that this is already a
deviation from a purely \ph-deficient approach to anaphora. An
alternative that stays truer to its \ph-deficiency premise might be to
posit that there is a single anaphoric form that is syncretic for all
\person, \num, and \gender{} combinations. In contrast, far from
posing a problem for the reference-deficiency view, such patterns
might be taken to be evidence in favor of it. Under an analysis like
\citet{Hicks:2009}, such invariant forms might simply be taken to be
the transparent spell-out of anaphors that have a \textsc{var} feature
(that has been valued under Agree) and nothing else.

Explicit cases of non-\ph-matching could involve some sort of
mismatch between the semantic and grammatical \ph-features on the
antecedent and the anaphor. Such a situation obtains in the minimal
pair (\ref{impost1}) and (\ref{impost2}), involving so-called
``imposters''\footnote{\citet[5, Ex. 10]{collinspostal:2012} define
  an imposter as ``a notionally X person DP that is grammatically Y
  person, X $\neq$ Y.''} (\citealt[97, 15-17]{collinspostal:2012}) : \ea \label{impost1} {[}The present authors]$_i$ are proud of
ourselves$_i$.  \ex\label{impost2}{[}The present authors]$_i$ are proud
of themselves$_i$.  \z

\noindent As Collins \& Postal show, a sentence like (\ref{impost1}) is only
grammatical when `the present authors' has a notional 1st-person
feature, i.e.\ is used by the speaker to refer to themselves in the
3rd-person. This indicates that (\ref{impost1}) doesn't really involve
a \ph-mismatch at all: rather, the antecedent has two distinct types
of \person-feature, a grammatical one that is 3rd-\person, and a
semantic one that is 1st-\person, and the anaphor is free to Agree
with either.

To sum up then, antecedence \ph-matching for anaphora falls out for
free under \ph-based views --- albeit syntactically in the
\ph-deficiency view, and semantically in the reference-deficiency
one. One might take this to mean that \ph-matching doesn't by itself
constitute a particularly strong empirical argument for either
approach.\footnote{I thank an anonymous reviewer for bringing up this
  point.} Yet, whereas \ph-featural matching entails strict
\ph-feature identity, semantic matching yields \ph-feature identity in
the \emph{default case}, but crucially not always. The requirement in
the case of the latter is \ph-feature consistency, not \ph-feature
matching. In Section \ref{phiabs}, I discuss a case where there is
featural consistency in the absence of feature-matching: this could
only have been achieved via a semantic route.

\subsection{Morphological underspecification of anaphors}
\label{under}

Going by restrictions placed on their antecedence, a remarkable number
of anaphors crosslinguistically seem to fail to mark the full range of
\ph-distinctions in the given language.  The identity and range of
these features is parametrized. Thus, Korean \textit{caki} and
Dravidian \textit{taan} are underspecified for gender alone: i.e.\ can
take antecedents of any gender, but these must be \textsc{3sg}; German
\textit{sich} (and its Germanic relatives) seem to be underspecified
for both gender and number; Japanese \textit{zibun} is unmarked for
person and gender; and Chinese \textit{ziji} seems to be maximally
underspecified.

Under a \ph-deficiency view, these distinctions can be captured in one
of two ways. Assuming that a bound variable starts out \ph-minimal
\citep{kratzer:2009}, we could propose that an anaphor acquires all
and only those \ph-features it actually surfaces with. Concretely,
then, Tamil \taan{} or Korean \textit{caki} would receive \person{}
and \num{} features alone but not \gender; Japanese \textit{zibun}
would receive \num{} alone, while \textit{ziji} would receive
``signature'' feature [reflexive] and thus remain unspecified for all
\ph-features. The morphology, then, straightforwardly spells out this
featural state-of-affairs. Of course, this implies that an anaphor be
born, not just lacking values for \ph-features, but lacking the
relevant \ph-attributes themselves. Notice, incidentally, that such a
solution is not obviously available for the reference-deficiency view
since the relationship to \ph-features is not encoded directly in the
syntax. 

Nevertheless, under both views, morphological underspecification could
simply be relegated to the morphological component, in particular to
rules of exponence for the anaphors in question. Let us assume that
the anaphor has all its \ph-features valued at the time of
SpellOut. The Vocabulary Insertion rule for the exponent \taan{} in
Tamil might then look like that in (\ref{tamvi}):

\ea\label{tamvi} \textsc{[3, sg, D]} $\leftrightarrow$ \taan{} \z

\noindent Under (\ref{tamvi}), all \textit{m, f, n} \gender{} combinations that
are \textsc{3sg} will be spelled out syncretically as \taan. Chinese
\textit{ziji}, in contrast, might have a maximally underspecified
SpellOut rule, as in (\ref{zijivi}):

\ea\label{zijivi} \textsc{[D]} $\leftrightarrow$ \textit{ziji}
\z

\noindent Since (\ref{zijivi}) makes reference to no \ph-features whatsoever, we
would get syncretism across all \person, \num, and \gender{}
categories for this anaphoric form.\footnote{Of course, the anaphor
  would still need to be distinguished from a deictic pronoun with the
  same features in that position: e.g.\ either via featural diacritics
  \citep{roorwyn:2011} or structurally \citep{heinat:2008,
    dechainewiltschko:2012}, as discussed.}

While a system like Kratzer's can directly capture the crosslinguistic
robustness of morphological underspecification, a purely morphological
solution would have to seek independent explanations, e.g.\ a
functionalist explanation \citep{roorwyn:2011}, for its
universality.\footnote{ ``The more specific a form is in terms of its
  feature makeup, the more restricted (i.e. effective) its
  reference. The situation is quite different for reflexive forms:
  since they have a local antecedent by definition and derive their
  reference from that antecedent, there is no need for them to be
  referentially restricted themselves. This does not exclude a
  situation where a reflexive has a rich set of distinctions \ldots
  but it does predict that underspecified forms, if they occur, will
  be found in the reflexive paradigm rather than in the nonreflexive
  one'' \citep[][45]{roorwyn:2011}.} Finally note that, under a
\ph-valuation approach, it is perfectly possible for an anaphor to be
exponed with all its \ph-features (as in Zapotec, Thai, or even
English), as well. Such an anaphor would have to satisfy the condition
that it have \emph{all} its \ph-features valued at the time of
SpellOut; additionally, it would have to be ensured that the SpellOut
rule itself not be underspecified for any \ph-feature. Such data, of
course, don't pose a challenge for the reference-deficiency view
either.



\subsection{Anaphor Agreement Effect (AAE)}


One of the strongest arguments for the \ph-deficiency view is,
perhaps, the Anaphor Agreement Effect (AAE). This refers to the
observation, going back to \citet{rizzi:1990}, and revised
periodically since \citet{woolford:1999}, \citet{tucker:2011},
  \citet{sundaresan:2016a}, that anaphors cannot trigger ``normal'' (i.e.\
covarying) \ph-agreement. Rizzi's original observation was motivated
by minimal pairs like the one below, from Italian
(\citealt[3]{rizzi:1990}):

\ea\label{aae-noanaph}\gll A loro interess-ano solo i ragazzi.\\
to them interest-{\sc 3pl} only the boys.{\sc nom}\\
\glt `They$_i$ are interested only in the boys$_i$.'
\ex\label{aae-agr}\gll *A loro interess-ano solo se-stessi.\\
to them interest-{\sc 3pl} only them-selves.{\sc nom}\\
\glt `They$_i$ are interested only in themselves$_i$.'  (Intended) \z

\noindent Italian has a nominative-accusative case system: \ph-agreement is
triggered by a nominative argument.  Thus, in (\ref{aae-noanaph}), the
nominative object `the boys' triggers 3rd-person plural agreement on
the verb. But if we replace this object with a plural nominative
anaphor, as in (\ref{aae-agr}), the sentence becomes ungrammatical. In
contrast, a sentence like (\ref{aae-noagr}) \citep[][33]{rizzi:1990}
where the anaphor appears in the genitive such that the co-occurring
verb surfaces with default 3rd-person singular agreement, is fully
licit:
%\vspace{-1ex}
   \ea\label{aae-noagr}\gll A loro import-a solo di se-stessi.\\
  to them matters-{\sc 3sg} only of them-selves\\
  \glt `They$_i$ only matter to themselves$_i$.'
\z
%\vspace{-1ex}
A key difference between (\ref{aae-agr}) and (\ref{aae-noagr}) is that
the anaphor triggers verb agreement in the former, but doesn't do so
in the latter. Strikingly, the grammaticality of these sentences seems
to be directly conditioned by this contrast: (\ref{aae-agr}), where
the anaphor should trigger agreement is ungrammatical whereas
(\ref{aae-noagr}) where the anaphor doesn't trigger agreement is
fine. Patterns such as these suggest that languages avoid structures
where an anaphor directly triggers agreement on its clausemate
verb. As such, \citet[28]{rizzi:1990}, proposed that ``[T]here is a
fundamental incompatibility between the property of being an anaphor
and the property of being construed with agreement.'' Subsequent
analyses \citep{woolford:1999, haegeman:2004, tucker:2011}
have tested the validity of the AAE against a wider range of
languages.

\noindent These investigations reveal that languages may choose to circumvent an
AAE violation in a number of additional ways. Some, like Inuit, may
simply detransitivize the predicate in question \citep{woolford:1999,
  bokbennema:1991}. Others, like the Malayo-Polynesian language
Selayerese, Modern Greek and West Flemish have been reported to
``protect'' the anaphor from triggering agreement by embedding it
inside another nominal \citep{woolford:1999, haegeman:2004}. In
\citet{sundaresan:2016a}, I argue that Tamil adopts an ``agreement
switch'' strategy. When the anaphor occurs in the agreement-triggering
case (nominative), co-varying \ph-agreement is exceptionally triggered
by \emph{some other nominal} with valued \ph-features in the local
domain. Such a strategy is arguably also reported for Kutchi Gujarati
in \citet{patelgrosz:2014} and \citet{murugrayn:2017}. Based on
such patterns, I update Rizzi's AAE as follows in
\citet[23]{sundaresan:2016a}: ``Anaphors cannot directly trigger
covarying \ph-agreement which results in covarying \ph-morphology.''

While it remains far from clear why a particular language adopts the
particular repair strategy it does, the AAE itself emerges as a
crosslinguistically robust constraint. It should be obvious that the
AAE is a clear argument in favor of any analysis that defines anaphora
in terms of \ph-feature deficiency. If an anaphor itself lacks
\ph-features, then such an anaphor should not be able to serve as a
Goal to value the \ph-features on a probing T or \lilv, yielding the
AAE (as argued by \citealt{kratzer:2009}). Under the
reference-deficiency approach, \ph-feature defectiveness is
presupposed but not featurally encoded. Given that agreement is a
featural dependency, however, the AAE doesn't come for free under such
a view.

\section{Complicating the picture}
\label{secref}

The previous section has presented two main ideas regarding the
feature composition of anaphora. We have also seen the anaphoric
phenomena that constitute the main empirical arguments, to a greater
or lesser degree, for these views. Here, I bring arguments to bear
showing that the anaphoric landscape is actually more nuanced and
complex, in a way that neither view can adequately capture by
itself. To this end, I present two main types of evidence:
\begin{itemize}
\item[(i)] Perspectival anaphora which are defined by a deficiency of
  a perspectival feature.
  \item[(ii)] Anaphors that are sensitive to \person{} asymmetries. 
  \end{itemize}
  The first type of evidence shows that \ph-features (or features that
  are built on \ph-features, like referential features) are not enough
  to capture the full range of anaphoric patterns in language. The
  second shows that the \person{} feature is privileged over other
  types of \ph-feature for purposes of anaphora --- something that a
  simple \ph- (or reference) deficiency view is not articulated enough
  to handle. 
  

 \subsection{When phi-features aren't enough: Perspectival anaphora}

 Perspectival anaphora have been reported for a number of languages, e.g.\ Malayalam, \citep{jayaseelan:1997}, Japanese,
 \citep{kuno:1987, nishigauchi:2014}, Icelandic,
 \citep{hellan:1988, sigurdsson:1991}, French,
 \citep{charnavel:2016}, Italian, \citep{giorgi:2010}, Abe,
 \citep{koopmansportiche:1989}, and Ewe, \citep{pearson:2013},
 a.o. Such anaphors are defined by their sensitivity to grammatical
 perspective, as noted. Concretely, the antecedent of such an anaphor
 must denote a perspective holder, mental or spatial, towards some
 predication containing the anaphor.

 Evidence showing that such perspective-holding is syntactically
 regulated --- which I discuss below --- suggests that
 perspective-sensitivity must be directly encoded in the featural
 make-up of such anaphors.  For instance, I propose in
 \citet{sundaresan:2012, sundaresan:2018} that a perspectival anaphor
 is born with an unvalued ``\dep'' feature, the valuation of which
 feeds semantic binding. The \dep-feature is formally identical to
 Hicks' \var: it is an attribute-value pair that takes arbitrarily
 assigned integers/letters as value.  The fundamental difference from
 Hicks' system lies in the notion that not every deictic pronoun and
 R-expression is born with a \emph{valued} \dep-feature. Rather, in a
 given phase, only one other nominal, by virtue of its dedicated
 structural position in the specifier of a Perspectival Phrase, is
 born with a valued \dep.


 \subsubsection{Sentience, sub-command, subject-orientation}

 In cases of perspectival anaphora, certain nominals are
 systematically excluded from potential antecedence. Non-sentient
 antecedents are ruled out, for instance, as illustrated below for the
 Chinese anaphor \textit{ziji} \citep{huangliu:2001}:

 \ea\label{zijisent}\gll Wo bu xiaoxin dapo-le ziji de
 yanjing.\\
 I not careful break-{\sc asp} \anaph{} {\sc poss} glasses\\
 \glt `Not being careful, I broke my own glasses.'
 \ex\label{zijiunsent}\gll *Yanjing$_i$ diao-dao dishang dapo-le
 ziji$_i$.\\
 glasses drop-to  floor break-{\sc asp} \anaph\\
 \glt `[The glasses]$_i$ dropped to the floor and broke
 themselves$_i$.'  (Intended) \z

\noindent Under a simple \ph-deficiency view, both `the glasses' with
 \textsc{3pl} features in (\ref{zijiunsent}) and `I' with \textsc{1sg}
 features in (\ref{zijisent}) should qualify as potential Goals for
 valuing the \ph-features on the anaphor, thus both (\ref{zijisent})
 and (\ref{zijiunsent}) should be grammatical. A possible way out might
 be to propose that the sentience restriction applies only later, at
 LF. The syntax would thus \emph{overgenerate}; at LF, non-sentient
 nominals involved in the Agree relation would be systematically
 filtered out, leaving only sentient nominals as potential antecedents
 behind.

 While this initially looks promising, we have nevertheless
 weakened the link between \ph-features and reference by bringing in
 sentience through the back door. Second, the fact that the English
 counterpart to (\ref{zijiunsent}) is perfectly grammatical suggests
 that a proposal that is predicated on the notion that the anaphors in
 both languages are featurally identical may be misguided. Finally,
 patterns of so-called ``sub-command'', like those in
 (\ref{sub})-(\ref{nosub}), reported also for Italian
 \citep{giorgi:2006} and Malayalam \citep{jayaseelan:1997}, suggest
 that the LF filtering account is too simple. The contrast between
 Chinese (\ref{sub}) vs.\ (\ref{nosub}) shows that a sentient nominal,
 that is itself embedded inside another nominal, may antecede
 \textit{ziji} (despite clearly not c-commanding it), just in case the
 embedding nominal is itself \emph{non-}sentient:
  
 \ea\label{sub}\gll Wo de jiaoao hai-le ziji.\\
 I 's pride hurt-{\sc asp} \anaph\\
 \glt `[My$_i$ pride]$_j$ hurt self$_{i/*j}$.'
 \ex\label{nosub}\gll Wo de meimei hai-le ziji.\\
 I 's sister hurt-{\sc asp} \anaph\\
 \glt `[My$_i$ sister]$_j$ hurt self$_{j/*i}$.'  \z

 \noindent To deal with such data, non-sentient nominals that have Agreed with
 \textit{ziji} can no longer be filtered out blindly.  Rather, the
 system must now have a way to look \emph{inside} the nominal, at
 another nominal in a particular structural position, and evaluate the
 sentience of this inner nominal --- a messy state-of-affairs. But if
 such anaphors are defined in terms of something other than \ph-features{}
 --- e.g.\ in terms of a feature that presupposes sentience (like the
 perspectival \dep-feature or an animacy feature itself), the
 account becomes considerably simpler. The antecedent can simply be
 the closest visible nominal in the search domain of the anaphor that
 bears this feature.% \footnote{Of course, we would still need some
   % mechanism like feature percolation to allow the feature (e.g.\
   % animacy) of the possessor nominal to be visible on the possessum,
   % thus enabling Agree with the anaphor.}

 A different sort of problem has to do with the so-called ``subject
 orientation'' of anaphora. Perspectival anaphors typically only take
 subjects, not objects, as antecedents. While this initially looks
 like evidence in favor of a syntactic treatment, there are systematic
 exceptions in both directions. What really matters for antecedence is
 perspective-holding: it just so happens that subjects tend to
 denote perspective-holders more than objects do. Here, again, an
 account in terms of \ph-feature deficiency would find it much harder
 (than one that encodes perspective-sensitivity directly) to deal with
 the problem of how certain nominals can be systematically ``skipped''
 in this manner.



\subsubsection{One language, two anaphors}


In \citet[85, 84a-b]{sundaresan:2012}, I reported that, in
certain Tamil dialects, (local) reflexivity may be expressed either
with a dedicated anaphoric form \taan, as in (\ref{taanloc}), or with
a pro-form \textit{avan}, that is syncretic with a \textsc{3msg}
deictic pronoun, as in (\ref{avanloc}):

\ea\label{avanloc}\gll Raman-\U{}kk\U{}$_i$ avan-æ-yee$_{\{i,j\}}$ piɖikka-læ.\\
Raman-{\sc dat} he-{\sc acc-emph} like-{\sc neg}\\
\glt `Raman$_i$ didn't like (even) himself$_i$/him$_j$.'
\ex\label{taanloc}\gll Raman$_i$ tann-æ-yee$_\subscr$ piɖikka-læ.\\
Raman[{\sc nom}] \anaph-{\sc acc-emph} like-{\sc neg}\\
\glt `Raman$_i$ didn't like (even) himself$_\subscr$.'  \z

\noindent Many languages have dedicated reflexive forms, simplex or
complex. Others, like Frisian, Old English, and Brabant Dutch, use a
reflexive form that is syncretic with the deictic pronominal one (see
\citealt{roorwyn:2011} for discussion). However, for a single language
to allow both types of anaphor in the same position is more
peculiar. Such differences correlate with systematic differences in
interpretation. The use of \taan{} in (\ref{taanloc}) \taan{} favors
an interpretation from the perspective of the antecedent, whereas the
use of the pronoun doesn't.

The challenge for the \ph-deficiency view is this: If \taan{} and
\textit{avan} are purely \ph-deficient elements, why are they
spelled-out differently, and interpreted in distinct ways?  One might
posit that they are both deficient for different \ph-features. But
this then doesn't explain why the interpretive difference between them
has to do with something that putatively has nothing to do with
\ph-features, namely perspective-holding. Note, too, that we cannot
claim, as before, that the two anaphors start out featurally identical
in syntax and are distinguished only later, at LF, since the anaphors
have different morphological forms as well. Under a reference
deficiency view like \citet{Hicks:2009}, we would face essentially the
same problems, since it would be assumed that \taan{} and
\textit{avan} would have identically valued \var{} features at the
point of spell-out.

Such data thus show that we need a distinct featural class for
perspectival anaphors. We could then say that \textit{avan} is \ph- or
reference-deficient while \taan{} is \dep-deficient, this then
accounting for its perspectival nature. There is, indeed, nothing to
prevent a single language from having both types of anaphor in its
lexicon. We will see, however, that the class of perspectival anaphora
runs orthogonal to others: i.e.\ perspectival anaphors may also be
deficient for certain types of \ph-features and vice-versa.

 \subsection{\person-asymmetries in anaphora}

 A different kind of evidence involves data showing that anaphors in
 certain languages are sensitive to 1st/2nd vs.\ 3rd-\person{}
 asymmetries. 

\subsubsection{PCC effects}

The PCC,\footnote{``\textbf{Strong PCC:} In a combination of a weak
  direct object and an indirect object [clitic, agreement marker, weak
  pronoun], the direct object has to be 3$^{rd}$ person.\\\textbf{Weak
    PCC:} In a combination of a weak direct object and an indirect
  object [clitic, agreement marker, weak pronoun], if there is a third
  person it has to be the direct object.''  \citep[][182]{bonet:1991}}
both Strong and Weak, has been shown to apply to a wide
range of languages. For instance, \citet{bonet:1991} discusses this
effect for Arabic, Greek, Basque, Georgian, English, Swiss German and
many Romance languages. Additional languages such as Georgian, Kiowa,
Bantu languages like Chambala, the Malayo Polynesian language Kambera,
Warlpiri, Passamaquoddy and many Slavic languages are reported in
\citet{haspelmath:2004}, \citet{bejarrezac:2003}, \citet{doliana:2013}, among others.

(\ref{str1})-(\ref{str2}) show the Strong PCC at work in
French % (\ref{it}\ref{weak1})-(\ref{it}\ref{weak2})
% instantiate the Weak PCC in Italian, for the analogous sentences
(all French examples below are taken from
\citealt{raynaud:2017}):
\ea\label{fr} \textsc{Strong PCC (French):}
\ea\label{str1} \ding{55} 1/2 ACC > 3 DAT\\
\gll *Ils me lui pr\'esentent.\\
\textsc{3pl.nom} \textsc{1sg.acc} \textsc{3sg.dat} introduce.\textsc{3pl}\\
\glt `They introduce me to him/her.'
\ex\label{str2} \ding{55} 1/2 ACC > 1/2 DAT\\
\gll *Ils me te pr\'esentent.\\
\textsc{3pl.nom} \textsc{1sg.acc} \textsc{2sg.dat} introduce.\textsc{3pl}\\
\glt `They introduce me to you.'  \z \z

\noindent PCC effects are revealing for the purposes of anaphora because, in
certain languages, anaphors pattern just like 1st- and 2nd-person
pronouns with respect to both Strong and Weak PCC effects
\citep{kayne:1975, herschensohn:1979, bonet:1991, anag:2003,
  anag:2005, rivero:2004, nevins:2007, adgerharbour:2007}. Compare
French (\ref{se1}) (originally from \citealt[173]{kayne:1975}), with
French (\ref{str1}), and (\ref{se2}) with (\ref{str2}):

\ea\label{rpcc} \textsc{Strong PCC with reflexives -- French:}
  \ea\label{se1} \ding{55} REFL ACC > 3 DAT\\
\gll *Elle$_i$ se$_i$ lui est donn\'ee enti\`erement\\
She \textsc{refl.acc} \textsc{3msg.dat} is given.\textsc{fsg}
entirely\\
\glt `She$_i$ have herself$_i$ to him entirely.'
\ex\label{se2} \ding{55} REFL ACC > 1/2 DAT\\
\gll *Ils$_i$ se$_i$ me pr\'esentent\\
they \textsc{refl.acc} \textsc{1sg.dat} introduce.\textsc{3pl}\\
\glt `They$_i$ introduce themselves$_i$ to me.'
\z
\z

\noindent Furthermore, just as postulated by the Strong PCC, as long as the
direct object is a weak 3rd-person element, weak indirect objects of
all person may combine with it. Crucially, in such cases, the
reflexive \textit{se} may also licitly combine with it as an indirect
object --- thus showing itself once again to pattern according to the
PCC:

\ea\label{goodpcc} \ding{51} 3 ACC > DAT:
\ea\label{mepcc}\gll Elle me l'a donn\'e.\\
she me.\textsc{dat} {\textsc{3sg.acc=have.3sg}} \textsc{give.msg}\\
\glt `She gave it to me.'
\ex\label{sepcc}\gll Elle$_i$ se$_i$ l'est donn\'e.\\
she herself.\textsc{dat} \textsc{3sg.acc=be.3sg} \textsc{give.msg}\\
\glt `She$_i$ gave it to herself$_i$.'  \z \z

\noindent \citet{rosen:1990} and \citet{baker:2008} also report analogous data for
Southern Tiwa, an Algonquian language. 

% It is worth noting here, a potential correlation between PCC effects
% and animacy in such languages. \citet{richards:2008} reports that in
% Southern Tiwa and Mohawk, the direct object in a double-object
% construction may not be animate. 


\subsubsection{Anaphoric agreement}

The same sensitivity to \person-asymmetries on the part of anaphors is
played out in a different empirical realm, namely that of
agreement. In certain languages --- e.g.\ in Bantu languages like
Swahili \citep{woolford:1999}, Chiche\^wa \citep{baker:2008}, and
Ndebele \citep{bowlot:2002}, and in Warlpiri \citep{legate:2002} ---
the anaphor triggers ``anaphoric agreement'' on the verb.  This is
agreement marking that differs from the normal \ph-paradigm in that
language. Thus, the special \textit{ji} marking on the verb in Swahili
(\ref{swahili-anaph}) (contrast with (\ref{swahili-reg})) does not
\ph-covary, so is a form unique to the anaphor alone:
\ea\label{swahili-reg}\gll Ahmed a-na-m/*ji-penda
Halima\\
Ahmed {\sc 3sbj-prs-3obj}-love Halima.\\
\glt `Ahmed loves Halima.'  \ex\label{swahili-anaph}\gll Ahmed
a-na-ji/*m-penda
mwenyewe.\\
Ahmed {\sc 3sbj-prs-refl/*3obj}-love himself\\
\glt `Ahmed$_i$ loves himself$_i$.' (emphatic) \z

\noindent Furthermore, this \textit{ji-} prefix contrasts with the clearly
\ph-agreeing elements of the paradigm in Swahili \citep[245]
{thompsonschleicher:2001}, Table \ref{swahili}.
\begin{table}[h!]
  \caption{Swahili object agreement paradigm}
\label{swahili}
\begin{tabularx}{\textwidth}{XXX}
 \lsptoprule
 \textsc{\ph} & \textsc{object-marker} & \textsc{verb-form}\\
\midrule
  1sg & -ni- & a-na-\ul{ni}-penda\\
  & & \textit{(s)he loves me}\\
  2sg & -ku- & a-na-\ul{ku}-penda\\
   & & \textit{(s)he loves you}\\
  3\{m/f\}sg (class 1) & -m/mw- & a-na-\ul{m}-penda\\
   & & \textit{(s)he loves him/her}\\
\midrule
  1pl & -tu- & a-na-\ul{tu}-penda\\
  % & & \textit{(s)he loves us}\\
  2pl & -wa- or -ku- & a-na-\ul{wa}-pend\ul{eni}\\
 % & & \textit{(s)he loves you all}\\
  3pl (class 2) & -wa- & a-na-\ul{wa}-penda\\
 % & & \textit{(s)he loves them}\\
  \midrule
  3nsg (class 3) & -u- & a-na-\ul{u}-penda\\
%  & & \textit{(s)he loves it}\\
  \midrule
  3pl (class 4) & -i- & a-na-\ul{i}-penda\\
 %  & & \textit{(s)he loves them}\\
  \midrule
  3nsg (class 5) & -li- & ana-\ul{li}-penda\\
  % & & \textit{(s)he loves it}\\
  \midrule
  3pl (class 6) & -ya- & ana-\ul{ya}-penda\\
 %  & & \textit{(s)he loves them}\\
  \midrule
  3nsg (class 7) & -ki- & ana-\ul{ki}-penda\\
 %  & & \textit{(s)he loves it}\\
  \midrule
  3pl (class 8) & -vi- & ana-\ul{vi}-penda\\
%   & & \textit{(s)he loves them}\\
  \midrule
  3nsg (class 9) & -i- & ana-\ul{i}-penda\\
  % & & \textit{(s)he loves it}\\
  \midrule
  3pl (class 10) & -zi- & ana-\ul{zi}-penda\\
  % & & \textit{(s)he loves them}\\
  \midrule
  3nsg (class 11) & -u- & ana-\ul{u}-penda\\
 %  & & \textit{(s)he loves it}\\
  \midrule
  3nsg (class 14) & -u- & ana-\ul{u}-penda\\
 %  & & \textit{(s)he loves it}\\
  \midrule
  3nsg (class 15) & -ku- & ana-\ul{ku}-penda\\
 %  & & \textit{(s)he loves it}\\
  \midrule
  3nsg (class 16) & -pa- & ana-\ul{pa}-penda\\
 %  & & \textit{(s)he loves it}\\
  \midrule
  3nsg (class 17) & -ku- & ana-\ul{ku}-penda\\
 %  & & \textit{(s)he loves it}\\
  \midrule
  3nsg (class 18) & -mu- & ana-\ul{mu}-penda\\
  % & & \textit{(s)he loves it}\\
   \lspbottomrule
\end{tabularx}
\end{table}
\newpage\noindent
Under a \ph-deficiency approach, such data would be genuinely
difficult to capture because they show that the anaphor must be
featurally distinguishable from all other nominals at the point at
which it triggers verbal agreement. We could imagine, for the sake of
argument, that the anaphor does, indeed, have some or all \ph-features
unvalued when it is merged in the structure. However, we would still
need a mechanism to ensure that it inherits only a \emph{proper
  subset} of features from its binder, in a way that identifies it as
being featurally distinct from its binder even after
feature-valuation. We might avail ourselves of \citet{kratzer:2009}'s
[anaphoric] feature here. But of course, as we have already observed,
once such a choice is made, we have already made the implicit move
away from a purely \ph-deficiency view.

To make matters even more complicated, \citet{baker:2008} shows that
such anaphoric agreement patterns unmistakably like agreement
triggered by 1st- and 2nd-\person{} pronouns and \emph{un}like
3rd-\person{} agreement. 1st- and 2nd-\person{} agreement is
crosslinguistically categorially restricted: e.g.\ adjectives don't
show \person-agreement. Interestingly, adjectival agreement in
languages like Chiche\^{w}a, and other Bantu languages, inflect for
the \num{} and \gender{} of the anaphor, but cannot reflect the
anaphoric agreement that shows up on the verb (\citealt[150-151, 86a-b]{baker:2008}, in Chiche\^{w}a):

\ea\label{chichadj}\gll Ndi-na-i-khal-its-a \textit{pro[CL4]} y-a-i-kali.\\
1s\textsc{s}-\textsc{past-4o}-become-\textsc{caus-fv} {} \textsc{cl4-assoc-cl4}-fierce\\
\glt `I made them (e.g.\ lions) fierce.'
\ex\label{chichrefl}\gll Ndi-na-dzi-khal-its-a \textit{pro[+ana]} w-a-m-kali.\\
1s\textsc{s}-\textsc{past-refl-}become-\textsc{caus-fv} {} \textsc{cl1-assoc-cl1-}fierce\\
\glt `I made myself fierce.'
\z

\noindent This shows that anaphoric agreement is a kind of \person{}
agreement. Interestingly furthermore, Bantu anaphors can be anteceded
by 1st, and 2nd person nominals (in addition to 3rd, as attested by
(\ref{chichrefl})), again suggesting that they have some feature(s) in
common with these. The parallels between 1st- and 2nd-\person{}
agreement and anaphoric agreement don't stop here, as Baker
discusses. Possessive determiners and adpositions --- categories that
can manifest 1st- and 2nd-\person{} agreement --- can also allow
anaphoric agreement in Greenlandic \citep{bittner:1994} and Slave
\citep{rice:1989}, respectively.


The fact that certain anaphors are sensitive to \person-asymmetries
reflected in phenomena like the PCC and anaphoric agreement, shows the
following: (i) such anaphors are themselves not underspecified for
\person{} (at least at the point where the trigger agreement) (ii)
(and potentially relatedly), anaphors of this kind must have something
in common with 1st- and 2nd-person pronouns, which is absent on 3rd,
(iii) the \ph-feature-specification of such an anaphor must be
different from all other nominals at this stage of the derivation (for
the case of anaphoric agreement). 

\subsubsection{A gap in anaphoric antecedence: 1st/2nd vs.\ 3rd}

Many anaphors only take 3rd-\person{} antecedents: e.g.\ German
\textit{sich}, Romance \textit{se/si}, Japanese \textit{zibun}, Korean
\textit{caki}, and Dravidian \taan. A glance at anaphors that take
local (1st/2nd-\person) antecedents initially reveals a somewhat
baffling picture.

There are anaphors that allow 1st, 2nd-\person{} antecedents, but
these crucially also allow 3rd (see \citealt{huangliu:2001}, for a
discussion of Chinese \textit{ziji} in this
regard). It is tempting to conclude from this that anaphors can take
1st/2nd-\person{} antecedents only if they also take 3rd-\person{}
ones.  Yet, a pro-form like \textit{mich} in German \emph{can} take a
1st-person antecedent while not also taking a 3rd (or a 2nd):
\ea\label{ichgood}\gll Ich(/*Du/*Sie) schlug mich.\\
I/*you/*she hit refl.{\sc acc}\\
\glt `I hit myself.'\\
\ding{55} `You hit yourself.'\\
\ding{55} `She hit herself.'  \z

\noindent Interestingly, however, \textit{mich} is ambiguously anaphoric or
pronominal (as indeed is \textit{dich}).  This suggests that there is
no \emph{unambiguous} anaphoric \emph{form} anteceded by 1st/2nd but
not 3rd. Table \ref{lez} for Lezgian (Northeast Caucasian) tells us
that this cannot be accurate either \citep[][184]{haspelmath:1993}.
\begin{table}[h]
	\centering
	\caption{Pro-forms in Lezgian (absolutive, singular)}  
	\label{lez}
	\begin{tabularx}{\textwidth}{XXX} 
		\lsptoprule
		\textsc{Person} & \textsc{Anaphor} & \textsc{Pronoun/Dem.}\\
		\midrule
		\textbf{1st} & \textit{\v{z}uw} & \textit{zun}\\
		\textbf{2nd} & \textit{\v{z}uw} & \textit{wun}\\
		\textbf{3rd} & \textit{wi\v{c}} & \textit{am}\\
		\lspbottomrule
	\end{tabularx}
\end{table}
\newline\noindent In (\ref{lez}), \textit{\v{z}uw} is an unambiguously anaphoric form,
anteceded by 1st \& 2nd, but not 3rd.\footnote{English may be similar,
  but forms like \textit{himself} arguably contain a syncretic
  pro-form (as in the German case) + ``self'' marker.}
But note that Lezgian has, not one, but two dedicated reflexive forms.

What we don't seem to have is a language that is the \emph{inverse} of
one like Italian, German, Tamil or Korean: i.e.\ where the anaphor
that takes a local antecedent has a dedicated reflexive form while the
one that takes a 3rd-\person{} antecedent has a form that is syncretic
with a pronoun. In other words, the correct restriction is that in
(\ref{antfin}), which is also reported in \citet{comrie:1999} as a
typological gap:

\ea\label{antfin} In a language with only one unambiguously anaphoric
  form, this must correspond to an anaphor that takes a 3rd-\person{}
  antecedent.
\z

\noindent It is hard to see how a \ph-based account would be able to capture the
generalization in (\ref{antfin}). An anaphor that is \ph-minimal in
the sense of \citet{kratzer:2009}, for instance, should, by default,
place no \person-restrictions on antecedence: i.e.\ such an anaphor
should behave like Chinese \textit{ziji}. Such data shows that
anaphors need access to a more articulated featural system, one which
can also distinguish inherent asymmetries within the categories of
\person.


\section{Proposal: Unequal anaphors}
\label{secclass}

The discussion above has shown that anaphors in natural language are
not created equal. Some anaphors are contentful for \person{} in a way
that others are not. Yet others are sensitive to properties that are
arguably entirely orthogonal to \ph-features, like perspective, which
also seems to be syntactically instantiated. The data that we have
seen so far thus supports the view that there are many (featural)
routes to anaphora. In other words, two nominals may qualify as being
both anaphoric, despite being featurally quite distinct. This then
naturally raises the question of what an anaphor actually is, and
whether the notion of anaphora is now so diffuse as to be
taxonomically worthless.

The definition in (\ref{anaphdef}) proposes that anaphors are both
syntactically and semantically non-homogenous. At the same time, it is
specific enough to identify anaphors as a meaningful nominal category
in syntax and semantics: 
\pagebreak
  \begin{exe}
    \ex\label{anaphdef} \textsc{Working definition of an
      anaphor:\footnote{I thank Giorgos Spathas (p.c.) for helping me
        finesse aspects of this definition.}}
\begin{description}
\item[In the syntax:] An anaphor defines a nominal that is featurally
  deficient for a (potentially unary) set $\gamma$, which must then be
  checked under Agree with a nominal that is valued for $\gamma$,
  potentially via intervening functional heads.
\item[In the semantics:] An anaphor defines a bound variable or a
  reflexivizing predicate that co-identifies two arguments of a
  predicate. For those semantic anaphors that are also syntactic
  anaphors, feature valuation of $\gamma$ leads either to variable
  binding, with the Goal for $\gamma$ binding the Probe for $\gamma$,
  or arity reduction.\footnote{See \citet{spathas:2010, spathas:2015}
    for arguments that anaphors are semantically non-homogenous, with
    some being bound variables and others arity reducing predicates.}
\item[Output = referential covaluation:] The individuals that the
  binder/bindee denote in the evaluation context covary with respect
  to one another. 
  \end{description}
\end{exe}

\noindent The definition in (\ref{anaphdef}) ensures that the kind of feature that an anaphor
lacks is one that a non-anaphoric nominal is inherently born with ---
since it is a non-anaphoric nominal that must ultimately check the
featural deficiency on the anaphor. This means, the missing feature
cannot be something like case (which would be checked by a functional
head), but must uniquely target the kind of information that is
\emph{inherent} to other nominals, such as a \ph- or reference-feature
(like Hicks' \var) or a perspectival feature (as in the \dep{} feature
from my previous work). The different features all trigger the same
kind of Agree mechanism which then feeds binding at LF, yielding
referential identity as the common output. The definition also leaves
open the possibility that certains nominals, for instance bound
variable pronouns, fake indexicals \citep{kratzer:2009} or certain
types of A-bar elements,\footnote{What precisely the membership of
  this class of elements is, is outside the scope of the current
  paper, and must remain an open question for now.}  count as
anaphoric via the semantic route alone --- i.e.\ without having a
featurally defective nominal counterpart in the syntax.


\subsection{A more articulated feature system}

Against this background, I now propose that a more articulated
\person-categorization than the standard 1st, 2nd, and 3rd is needed
to capture the featural distinctions between the two classes of
anaphor called for here. I base this on a bivalent rather than a
privative feature system. I will avail myself of the binary features
[$\pm$Author] and [$\pm$Addressee] and a private feature
[sentience].

\ea\label{def} \textbf{Featural definitions:}\footnote{The definitions
  for \textsc{[+Author]} and \textsc{[+Addressee]} are adapted from
  \citet{halle:1997, nevins:2007}'s definitions for
  $[\pm Participant]$ and $[\pm Author]$. The [\sentience] feature is
  akin to the [$\pm$ Mental State] feature in \citet{reinhart:2000}.}

\ea \textsc{[+Author]} = the reference set contains the speaker of the
evaluation context (default: utterance-context)

\ex \textsc{[+Addressee]} = the reference set contains the hearer(s)
of the evaluation context (default: utterance context).

\ex \textsc{[\sentience]} = the reference set contains an individual (or
individuals) that is mentally aware and capable of bearing mental
experience in the evaluation context.  \z \z


\noindent Note that while we can think of [sentience] as a kind of \person{}
feature, in the sense that it has a clear relation to [$\pm$Author]
and [$\pm$Addressee], it does not carve up the space of referents like
these features do in terms of the participants of a speech
act. Given the definition of the [sentience] feature in (\ref{def}),
it is clear that all individuals that are contentful for \person{} ---
i.e.\ individuals that are [$\pm$Author] and [$\pm$Addressee]) ---
must automatically also bear the [sentience] feature. At the same
time, we can also have elements that only bear the [sentience]
feature.\footnote{The introduction of the privative [sentience]
  feature thus does not actually constitute a counter-argument to
  proposals like \citet[4]{bobaljik:2008a} which argues that \begin{quotation}``the
  traditional three-value person system over-generates, allowing for
  the expression of universally unattested distinctions. By contrast,
  a two-valued, binary feature system [$\pm$speaker] and [$\pm$hearer]
  (or any equivalent notation) is not only restricted to a four-way
  contrast, it in fact yields exactly the maximally attested contrasts
  and excludes precisely those distinctions that are unattested.''\end{quotation} What we have in our featural toolbox is not a three-value
  \person-system, but a strictly two-value \person{}
  system. Concretely, [sentience] picks out a proper superset of the
  union of the set of referents picked out by [$\pm$Author] and
  [$\pm$Addressee].  It bears a strong similarity to the privative
  [empathy] feature proposed in \citet{adgerharbour:2007}, but
  involves none of the cultural connotations that Adger and Harbour
  attribute to the [empathy] feature. I thank an anonymous reviewer
  for bringing this potential concern to my attention.} % reword

A cross-classification of [$\pm$Author] and [$\pm$Addressee] together
with [sentience] thus yields the set of \person-categories in Table
\ref{fper}.\footnote{Of course, we could also underspecify the
  \person-features themselves to yield a more comprehensive set of
  categories, as in the table below, fleshed out with language help
  from the Surrey Syncretisms Database \citep{surrey:2002}:

      \begin{tabular}{llll}
      \lsptoprule
      & \textbf{Features} & \textbf{Category} & \textbf{Exponents}\\  
      \midrule
   &   {[+Author, sentience]} & \textsc{1} & \textit{I, we}\\
    &   {[+Addressee, sentience]} & \textsc{1incl} $\wedge$ \textsc{2} & \textit{-nto} (Muna, \textsc{2hon.sg=1incl.du})\\   
   1. &  {[-Author, sentience]} & \textsc{$\neg$1} & \textit{ale} (Amele, \textsc{2=3.du})\\
        &   {[-Addressee, sentience]} & \textsc{$\neg$2} & ---\\
        & {[sentience]} & \refl{} & Anaphors in Bantu\\
        &   $\emptyset$ & \nul{} & \textit{ziji} (Chinese),
                                     \textit{man} (German)\\
\midrule
  &     {[-Author, -Addressee, sentience]} & \textsc{3} & \textit{him,
                                                       sie} (German), \textit{si} (Italian)\\
     &  {[+Author, +Addressee, sentience]} & \textsc{1incl.} & \textit{naam}
                                                  (Tamil, \textsc{1incl.pl})\\
   2. &   {[+Author, -Addressee, sentience]} &  \textsc{1excl.} &
                                                   \textit{naaŋgaɭ}
                                                   (Tamil, \textsc{1excl.pl})\\
        &   {[-Author, +Addressee, sentience]} & 2 & \textit{you}\\
          \lspbottomrule
      \end{tabular}
      
      However, I will seek to model syncretism effects for [$\pm$
      Author] and [$\pm$Addressee] via morphological, rather than
      featural, underspecification, where possible, to keep the
      featural toolbox more parsimonious.}
\begin{table}[h]
  \caption{Person Cross-Classification}
\label{fper}
\begin{tabular}{lll}
  \lsptoprule
  \textbf{Features} & \textbf{Category} & \textbf{Exponents}\\  
  \midrule
  {[+Author, +Addressee, sentience]} & \textsc{1incl.} & \textit{naam}
                                                         (Tamil, \textsc{1incl.pl})\\
  {[+Author, -Addressee, sentience]} &  \textsc{1excl.} &
                                                          \textit{naaŋgaɭ}
                                                          (Tamil, \textsc{1excl.pl})\\
  {[-Author, +Addressee, sentience]} & \textsc{2} & \textit{you}\\
  {[-Author, -Addressee, sentience]} & \textsc{3} & \textit{him,
                                                    sie} (German), \textit{si} (Italian)\\
  \midrule
  {[sentience]} & \textsc{Refl} & Anaphors in Bantu\\
  $\emptyset$ & \nul{} & \textit{ziji} (Chinese),
  \\
  \lspbottomrule
    \end{tabular}
  \end{table} \noindent 
  The real innovation of such a system is that it defines three
  distinct types of non-1st and non-2nd \person{} category which our
  classes of anaphor can now invoke. The \nul{} category is based on
  the $\emptyset$ and thus defines an entirely \person-less form.  The
  second category is specified as having \person{} features that are
  \emph{negatively opposed} to those carried by 1st and 2nd-\person,
  this being precisely the kind of distinction that a binary feature
  system allows us to make. The third category, \refl, defines
  nominals that are featurally underspecified: these bear the
  [sentience] feature and nothing else. We will see that such featural
  underspecification characterizes anaphors involved in patterns of
  anaphoric agreement, discussed for some Bantu languages,
  above.\footnote{A [sentience] marked nominal
    might also, in addition, characterize expletives (like German
    \textit{man}) in this class \citep{nevins:2007, ackneeleman:app},
    which have been argued to be \ph-featurally deficient, but
    nevertheless presuppose the sentience of their referent.}


Against the featural classification in Table \ref{fper}, I now
distinguish the following four categories of anaphor:
 \begin{table}[!h]
 	\caption{Four classes of anaphor}
 	\label{anaph-fin}
 	\begin{tabularx}{\textwidth}{llX} 
 		\lsptoprule
 		\textbf{Class} &  \textbf{\person-Features} & \textbf{Exponents}\\  
 		\midrule
 		\textsc{3}rd-anaphor &  {[-Author, -Addressee, sentience]} & \textit{taan} (Tamil),
 		\textit{zich} \textit{(zelf)} (Dutch)\\
 		% \mid
 		\refl{}  & [sentience] & Bantu anaphors\\
 		%\mid
 		\nul-anaphor &  $\emptyset$ & \textit{ziji} (Chinese),
 		\textit{zibun} (Japanese)\\
 		\midrule
 		\textbf{Class} & \textbf{Non-\ph-Feature} & \textbf{Exponents}\\
 		\midrule
 		Perspectival anaphors & [\dep] & \textit{taan},
 		\textit{ziji}, 
 		\textit{sig} (Icelandic)\\
 		
 		
 		
 		
 		\lspbottomrule
 	\end{tabularx}
 \end{table}

\subsection{\nul-\person{} anaphors}

A \nul-\person{} anaphor must have an unvalued \person-feature that is
valued in the course of the syntactic derivation by a nominal or
functional head in the Agree domain. The empirical signature of such
an anaphor is that it can take antecedents of all \person.  

\subsubsection{Deriving phi-matching (\nul-\person)}

We noted again that anaphor-antecedence \ph-matching is typically a
prerequisite crosslinguistically. In the simplest scenario, a
\nul-\person{} anaphor has not just unvalued \person, but also
unvalued \num, and \gender{} features. Such an assumption is
compatible for the Chinese anaphor \textit{ziji}, given that it places
no \ph-restrictions on its antecedent. In such a scenario, all the
\ph-features on the anaphor would simply receive the same values as
those on its antecedent, under Agree, yielding \ph-matching as an
obligatory result. A less straightforward scenario is that the
\nul-\person{} anaphor lacks only the \person{} feature but is born
with inherently valued \num{} and/or \gender{} features (e.g.\
Japanese \textit{zibun}).

What is to prevent such an anaphor from only matching the \person{}
value of its antecedent but differing in values for \num{} and
\gender? It makes sense to think that, in such a case, \ph-mismatch is
ruled out semantically. This follows from the condition that
referential identity typically yields identity of \ph-features. Put
another way, an anaphor (e.g.\ \textit{zibun}) cannot, in the default
case, corefer with a nominal without matching it for \textit{all}
\ph-features. If \ph-matching is not enforced in the syntax, it will
typically be enforced in the semantics, once binding is established,
as we have already discussed. But as already mentioned, the two routes
to referential identity can be teased apart empirically. I discuss a
concrete instance of such a scenario in Section \ref{phiabs}. 



\subsubsection{Deriving morphological underspecification (\nul-\person)}

The morphological underspecification of anaphors could be captured for
a \nul-\person{} anaphor, but it would have to be relegated to the
morphological component. This follows from the assumption that a
\nul-\person{} anaphor start out being \emph{unvalued} for
\person. This means that, once it becomes \ph-valued under Agree, it
will end up with a full set of \ph-features. Any surface lack of
\ph-featural distinctions on such an anaphor will necessarily have to
follow from the underspecification of Vocabulary Items, as again in
(\ref{zijivi1}) and (\ref{tamvi1}): \ea\label{zijivi1} \textsc{[D]}
$\leftrightarrow$ \textit{ziji} \ex\label{tamvi1} \textsc{[3, sg, D]}
$\leftrightarrow$ \taan{} \z

\noindent Thus, the \emph{theory} itself doesn't actually make any predictions
for increased frequency of underspecification on such anaphors,
compared to their deictic pronominal counterparts. Such patterns would
thus have to follow from functional considerations
\citep{roorwyn:2011}, by proposing that anaphors lack, not just the
values but also the attributes, for \ph{} features
\citep{kratzer:2009}, by using featural diacritics to distinguish
valued features from inherent ones \citep{roorwyn:2011} or by
distinguishing anaphors from other pro-forms with respect to their
internal structure \citep{heinat:2008, dechainewiltschko:2012} ---
along the lines discussed in Section \ref{under}.

\subsubsection{Deriving the Anaphor Agreement Effect (\nul-\person)}

The AAE, as we saw, is the restriction that an anaphor cannot directly
trigger covarying \ph-morphology.  AAE effects are straightforwardly
captured with a \nul-\person{} anaphor, as long as we make two, fairly
uncontroversial, assumptions.

First, the timing of Agree operations is crucial. We must ensure that
the anaphor has not itself been valued for \ph-features by the time a
functional head (like T or \lilv) comes around looking to Agree with
it.\footnote{For a non-local anaphor in subject position (e.g.\ Tamil
  \taan \phantom{t}\citealp{sundaresan:2016a, sundaresan:2018}), this falls out
  straightforwardly, because the Agree Probe (e.g.\ T) is merged
  before the nominal binder. In a local reflexive sentence, with an
  object anaphor, we can have subject or object agreement. With object
  agreement, the logic is the same. The Probe is \lilv, which is
  merged earlier than the nominal binder subject. Subject agreement
  typically involves cases of a nominative object under a subject
  which, being oblique, cannot itself trigger agreement, as in Italian
  (\ref{aae-noanaph})-(\ref{aae-noagr}). The Probe is T and is
  actually merged \emph{higher} than the binder. To explain why the
  AAE still holds, we must thus make some additional assumption, e.g.\
  that ``subject agreement'' with an in-situ nominative object
  involves successive cyclic Agree via \lilv. It would then be the
  first Agree cycle that runs into earliness problems as the other
  types of agreement.}

\noindent Second, we must assume that partial agreement with T or \lilv{} is
ruled out. After all, a \nul-\person{} anaphor is only born unvalued
for \person. In other words, X (Probe) cannot Agree with Y (Goal) if Y
has even one unvalued \ph-feature.\footnote{Note that this is distinct
  from another phenomenon sometimes referred to as partial agreement
  which, as a reviewer correctly points out, is well attested. This is
  of the following abstract form. X (Probe) Agrees with Y (Goal),
  which is fully specified for all \ph-features; but X only marks (and
  potentially also only Agrees for) a proper subset of these
  features. For instance, German nouns have fully valued case,
  \person, \num, and \gender{} features. But adjectives modifying such
  nouns show agreement with them only for case, \num, and \gender, and
  plausibly do not even Probe them for \person. Partial agreement in
  this sense is, of course, fully possible in the current system and
  is not what I am talking about here.} Concretely, this means that a
\nul-\person{} anaphor with a valued \num{} and/or valued \gender{}
feature should nevertheless not be able to trigger covaring agreement
for these features on the verb. Agreement must be an ``all or
nothing'' operation.\footnote{On the other hand, if it turns out that
  there \emph{are} languages that allow covarying agreement for
  \gender{} and \num{} in such cases, then the current system has a
  way to make sense of this. The idea would be that, in such
  languages, partial agreement is allowed, perhaps as a parametric
  choice. What is strictly ruled out, however, is a scenario where a
  \nul-\person{} anaphor triggers covarying agreement for \person.}
Finally, anaphoric agreement of the kind noted for Swahili and
Chiche\^{w}a has also been classified as a type of AAE. Such agreement
is not a property of \nul-\person{} anaphors. Given that they have no
valued \person-feature themselves, they are not expected to trigger
agreement (that additionally patterns like 1st and 2nd-\person{}
agreement) on T or \lilv.


\subsection{3rd-\person{} anaphors}

A 3rd-\person{} anaphor has the feature specification
$[-Author, -Addressee]$, and is negatively specified with respect to
1st- and 2nd-\person. The empirical signature of such an anaphor is
that it allows only 3rd-\person{} antecedents. 

3rd-\person{} anaphors must be distinguished from \emph{non}-anaphoric
3rd-\person{} pro-forms, which will also have the same
feature-specification. Assuming that anaphora is defined in terms of
feature-deficiency (which is ``rectified'' via Agree), this means that
3rd-\person{} anaphora must be defective for a non-\person{}
feature. Such anaphors could thus have an unvalued \num{} or \gender{}
feature. Alternatively, or additionally, such anaphors could be
deficient for a perspectival feature like \dep{}
\citep{sundaresan:2012, sundaresan:2018}.

\subsubsection{Deriving phi-matching (3rd-\person)}

Since a 3rd-\person{} anaphor can start out unvalued for \num{} and
\gender, we predict that we would have syntactic feature matching for
these features, because they will be valued by Agree with the
antecedent. But matching for 3rd-\person{} must be via the semantic
route since the anaphor is born with this feature already valued.

\subsubsection{Deriving morphological underspecification (3rd-\person)}

As with \nul-\person{} anaphors, morphological underspecification must
be captured either functionally, structurally, via featural
diacritics, or by positing that the anaphor lacks featural attributes,
not just values.

\subsubsection{Deriving the Anaphor Agreement Effect (3rd-\person)}

Given the discussion above for \nul-\person{} anaphors, we predict
that a 3rd-\person{} anaphor should also be subject to the
AAE. Central to this conclusion is the afore-mentioned premise that
partial agreement with a functional head is ruled out. In other words,
it cannot be the case that a 3rd-\person{} anaphor can satisfy a Probe
by triggering agreement for this feature alone. I assume, as before,
that having unvalued \num{} and \gender{} features will render the
3rd-\person{} anaphor unable to serve as a appropriate Goal for
\ph-agreement. Finally, the timing of Agree is again crucial. The AAE
holds just in case the anaphor has not had its own \ph-features valued
in the course of binding via Agree, by its nominal antecedent, at the
stage when the functional head is trying to Probe it.  


\subsection{The 1/2 vs.\ 3 antecedence gap}

Consider now the 1/2 vs.\ 3 antecedence gap in (\ref{antfin}), repeated
below: 

\ea\label{antfin1} In a language with only one unambiguously anaphoric
form, this must correspond to an anaphor that takes a 3rd-\person{}
antecedent.  \z 

\noindent Both classes of anaphor seen so far are well-behaved with respect to
(\ref{antfin1}).  3rd-\person{} anaphors allow only 3rd-\person{}
antecedents; \nul-\person{} anaphors allow antecedents of all
\person. The only scenario that would allow 1st/2nd-antecedence while
\emph{dis}allowing 3rd, would be if the anaphor were itself specified
as [$+Author$] or [$+Addressee$] (or some combination thereof). But
there don't seem to be dedicated \emph{anaphoric} forms for 1st and
2nd-\person{} alone in any language. For instance, bound-variable uses
of 1st and 2nd-\person{} forms (see discussion of so called
``fake indexicals'' in \citealt{vonstechow:2002, kratzer:2009}, a.o.) as in
(\ref{fake}) always {\em also} involve an indexical use: \vspace{-1ex}

\ea\label{fake} I am the only one who broke my laptop this week.  \z

\noindent But it is admittedly not so clear why this is the case.\footnote{
  Perspectival anaphors are {\em obviative}: i.e.\ cannot {\em cannot}
  refer to the perspective of the utterance-context participant
  \citep{sundaresan:2012, sundaresanpearson:2014,
    sundaresan:2018}. E.g.\ perspectival anaphora in Italian
  \citep{giorgi:2010} and Icelandic (\citealt{Hicks:2009,
    Reuland:2011}, a.o.) are used only across subjunctive clauses --- an
  obviative mood that precludes the utterance-speaker's perspective
  \citep{hellan:1988, sigurdsson:2010}. If this is correct, then we
  can imagine that interpreting the perspectival feature on the
  anaphor together with a feature that is [+Author] or [+Addressee]
  (or both) leads to semantic incompatibility, perhaps even a
  contradiction.}




\subsection{PCC effects and anaphoric agreement: \refl{} anaphors}


We observed earlier that anaphors in French and Southern Tiwa are
sensitive to the PCC, just like 1st and 2nd-\person{} pronouns in
these languages. If the PCC is a person restriction that affects all
(weak) grammatical objects that are (positively or negatively)
specified for \person, then it follows that 3rd-\person{} anaphors
would be subject to the same restriction as 1st- and 2nd. This, in
turn, could be taken to argue that anaphors in such languages belong
to the 3rd-\person{} class. An additional assumption that is needed,
of course, is that, in such languages, a \emph{non}-anaphoric
3rd-person pro-form must lack \person{} altogether.

The fact that anaphoric agreement patterns with 1st- and 2nd-\person{}
agreement could be accounted for by positing that such agreement is
regulated by sensitivity to a positively or negatively specified
\person-feature. But we also saw that anaphoric agreement in a given
language is distinct from all other forms in the \ph-paradigm in that
language (see again (\ref{swahili-reg}) vs.\
(\ref{swahili-anaph}) and the \ph-paradigms in Table
\ref{swahili}). This means that the 3rd-\person{} anaphor must be
featurally distinct from all other nominals at the time of triggering
agreement. Assuming, as before, that partial \ph-agreement is ruled
out, this is harder to implement. After all, once such an anaphor has
been valued for any \num, \gender{} or other (e.g.\ \dep) features,
what is to distinguish it from another nominal (e.g.\ a non-anaphoric
3rd-\person{} pronoun) which bears these features inherently? One
could underspecify the SpellOut rule for agreement, but this seems
clearly the wrong way to go: it doesn't explain why such agreement is
triggered by an anaphor as opposed to any other pro-form with these
features.

\noindent A bigger challenge comes from sentences like (\ref{chichrefl1}),
repeated from (\ref{chichrefl}):
\ea\label{chichrefl1}\gll Ndi-na-dzi-khal-its-a \textit{pro[+ana]} w-a-m-kali.\\
1s\textsc{s}-\textsc{past-refl-}become-\textsc{caus-fv} {} 
\textsc{cl1-assoc-cl1-}fierce\\
\glt `I made myself fierce.'  \z

\noindent Patterns like (\ref{chichrefl1}), reported for other Bantu languages
like Ndebele \citep{bowlot:2002} and Swahili \citep{woolford:1999} ---
show us that the anaphor needs to share some features in common with
1st and 2nd-\person{} as well which, of course, a 3rd-\person{}
anaphor doesn't. 

This is where the privative [sentience] feature comes into play. As
discussed, such a feature underlies all nominals with contentful
\person. An anaphor that takes a 1st and 2nd-\person{} antecedent, as
in (\ref{chichrefl1}), is simply featurally underspecified for all
features except the [sentient] feature. The empirical signature of
such an anaphor (labelled ``\textsc{refl}'') is that it takes only
sentient antecedents. To explain the unique form of anaphoric
agreement in such languages, we must assume that no other nominal in
the language is featurally underspecified such that it denotes
[sentient] and nothing else, at the point in the derivation where the
anaphor triggers agreement on the verb. This means, in turn, that the
anaphor cannot already have Agreed with its antecedent by this point
(assuming that such an Agree operation would render the anaphor and
its antecedent featurally indistinguishable).



        \subsection{Perspectival anaphora}

        
        In the current system, perspectival anaphora comes out as a
        strictly orthogonal category. As such, perspectival anaphors
        can, in theory, be defined for \nul-\person{} and
        3rd-\person{} anaphors, as well as \refl. Dravidian \taan{} is
        a 3rd-\person{} anaphor in the current system, and is
        additionally perspectival. It is thus spelled out by the rule
        in (\ref{taansp}), after having had the [\dep] feature valued
        by its binder: \ea\label{taansp} {[}-Author, -Addressee, sentience,
        Dep: \textit{x}, sg] $\leftrightarrow$ \textit{taan} \z

        \noindent We saw earlier that, in certain Tamil dialects, it is possible
        to have two locally bound reflexive forms --- a \textsc{3msg}
        \textit{avan} (non-perspectival, syncretic) and \taan{}
        (perspectival) (cf.\ (\ref{avanloc}) vs.\ (\ref{taanloc})),
        from \citet{sundaresan:2012}. In the current system, the
        anaphor \textit{avan} would be spelled out by the rule in
        (\ref{avansp}):

        \ea\label{avansp} {[}-Author, -Addressee, sentience, m, sg]
        $\leftrightarrow$ \textit{avan} \z

        \noindent Although the anaphoric and pronominal variants of
        \textit{avan} would differ in terms of which \num{} and
        \gender{} features they were born with --- they would be
        indistinguishable post-valuation. They would thus both be
        subject to the SpellOut rule in (\ref{avansp}), yielding
        syncretic \textit{avan} in this dialect.

        Chinese \textit{ziji} is a \nul-\person{} anaphor but is also
        perspectival, given its sentience and sub-command restrictions
        (cf.\ (\ref{sub}) vs.\ (\ref{nosub})). Note, though, that it
        could also be \refl. Being featurally marked [sentient], its
        sentience restriction would follow automatically. How do we
        decide?  With \textit{ziji}, we see not only animacy
        restrictions but also thematic restrictions on antecedence:
        ultimately, it is subject-oriented like all perspectival
        anaphors are and singles out an antecedent that denotes a
        perspective-holder \citep{huangliu:2001}.  As such, we don't
        need to encode the animacy restriction on \textit{ziji}
        separately with [sentient]; it comes out for free
        with \dep, which is independently needed anyway. So the
        SpellOut rule for \textit{ziji} is just that in
        (\ref{zijisp}):\footnote{This raises the interesting question
          of whether we can ever superficially ``tell'' the difference
          between a \nul-\person{} perspectival anaphor and a \refl{}
          perspectival anaphor. Perhaps not.  The latter is possibly
          just ruled out under conditions of featural economy: i.e.\
          the grammar avoids simultaneously using two features that
          accomplish the same goal, in this case specifying animacy.}
        \ea\label{zijisp} {[}Dep:\textit{x}] $\leftrightarrow$
        \textit{ziji} \z


\section{Empirical predictions}
    \label{secpred}

    The current system makes a range of testable empirical
    predictions. Below, I show that many of these are, indeed,
    confirmed.

\subsection{Phi-matching and its absence}
  \label{phiabs}


  The current model derives anaphor-antecedence \ph-matching in two
  ways. With a \nul-\person{} anaphor, all \ph-matching could happen
  featurally, e.g.\ if such an anaphor is born with \emph{all} its
  \ph-features unvalued. With a 3rd-\person{} anaphor, matching for
  \num{} and \gender{} alone may happen featurally; \person-matching
  is always enforced in the semantics, as a result of referential
  identity between the anaphor and its binder.

  But as mentioned earlier, this distinction can be tested
  empirically. In particular, featural matching should imply strict
  \ph-feature identity since it comes about via goal-probe
  feature-copying under Agree. Semantic matching, on the other hand,
  results in \ph-feature identity \emph{in the default case}, but not
  always. Rather, the requirement is that, applying the interpretation
  of the two sets of \ph-features to a single referent does not yield
  a \emph{contradiction} (e.g.\ a single referent cannot be
  simultaneously 1st and 2nd-\person). 

  But this predicts that we should observe anaphor-antecedent
  \ph-mismatches, just in case applying the interpretation of the two
  sets of \ph-features to a single referent \emph{does}, indeed, yield
  a consistent interpretation. In prior work \citep{sundaresan:2012,
    sundaresan:2018}, I argue that this prediction is confirmed in
  so-called ``monstrous agreement'' sentences in Tamil. {\em Monstrous
    agreement} refers to the phenomenon where the predicate of a
  3rd-person speech report surfaces with 1st-person agreement in the
  scope of a 3rd-\person{} anaphor. I propose that, in such cases, the
  anaphor \taan{} is bound by a shifted 1st-\person{} indexical
  \citep{schlenker:2003, anand:2006} which also triggers the
  1st-\person{} agreement on the verb. We thus have a scenario where
  an anaphor and its local binder have clearly non-identical \person{}
  features, and yet have identical reference.  We can make sense of
  this precisely because it happens under conditions of indexical
  shift.
  
  It is entirely consistent for a single referent to be both the
  speaker of a matrix speech event (thus [+Author] with respect to the
  speech event) and \textit{not} the speaker or addressee with respect
  to the utterance-context (thus, [-Author, -Addressee] with respect
  to the utterance-context). There is no contradiction. Note,
  crucially, that \taan{} is a 3rd-\person{} anaphor; thus,
  referential identity is enforced semantically, not via
  feature-matching.
  

  A different prediction is that a \nul-\person{} anaphor, being
  unvalued for \person, has to match its antecedent for \person, but
  not necessarily for \num{} and \gender. Indeed, such \num{}
  mismatches are possible in Hausa \citep[42, 8]{haspelmath:2008}: crucially, Hausa anaphors can be anteceded by
  all \person{} \citep{Newman:2000}, showing that they belong to the
  class of \nul-\person{} anaphor.

     \subsection{PCC effects}
  
     We predict that \nul-anaphora should not be restricted like 1st-
     and 2nd-\person{} for PCC, since they lack \person. This, too,
     seems to be confirmed. Thus, in Bulgarian, a language that shows
     the Weak PCC, PCC effects do not obtain with the reflexive clitic
     \textit{se} \citep[][500]{rivero:2004} and also
     \citet{nevins:2007}:
     \ea \gll Na Ivan mu se xaresvat tezi momicheta.\\
     to Ivan \textsc{dat} \textsc{refl} like-\textsc{3pl} these
     girls\\
     \glt `Ivan likes these girls.'  \z

 \noindent  Crucially, Bulgarian \textit{se} is underspecified for \person{}
     and can take antecedents for 1st, 2nd, and 3rd-\person.


      \subsection{AAE and the timing of Agree}

      I observed earlier that the timing of Agree plays a central role
      in deriving the AAE. Concretely, the anaphor cannot serve as a
      Goal for Agree for T or \lilv{} because it has unvalued
      \ph-features of its own. This in turn predicts that, in cases
      where an anaphor has already had its \ph-features valued by
      Agree with its antecedent at the stage in the derivation where
      T/\lilv{} Probes it, the AAE should not hold. This prediction
      seems to be met. In recent work \citet{murugesan:2018} presents
      case studies from Gujarati showing that objects in this language
      Agree with T, not \lilv. This means an object anaphor has
      already had its \ph-features valued by its antecedent in [Spec,
      \lilv] by the time T Probes it. It is precisely in such a
      configuration that the AAE seems not to hold. Murugesan argues
      that similar situations arise in Archi, Ingush, and Shona.  
      

      \subsection{Sentience and animacy effects}

      I have argued that an anaphor that triggers anaphoric agreement,
      as in the Bantu languages is of the \textsc{refl} class,
      featurally underspecified as [sentient]. The obvious prediction,
      then, is that anaphors in such languages will not only allow
      antecedents of all \person, which we have already seen to be
      true, but that they will \emph{not} allow non-sentient
      antecedents which (properly) includes inanimate
      antecedents. Such a restriction does, indeed, seem to be
      initially confirmed. \citet{woolford:1999}, \citet{vitale:1981} report
      for Swahili, a language with anaphoric agreement, that object
      agreement may only be trigged by animate entities. 


      
\section*{Acknowledgements}

Thanks to Katharina Hartmann, Johannes Mursell and Peter Smith for
putting this exciting volume together and to two anonymous reviewers
for helpful comments.  Earlier versions of the paper were presented at
the AnaLog Workshop (Harvard), LinG1 (G\"ottingen), WCCFL (UCLA),
GLOW-in-Asia (Singapore), SLE (Naples) and at Leipzig, and benefited
greatly from audience feedback. Thanks to Jonathan Bobaljik and Ad
Neeleman for helpful discussions, and to Gurujegan Murugesan, Louise
Raynaud, and Hedde Zeijlstra for many stimulating discussions on these
topics.  I am extremely grateful, finally, to Tom McFadden for helpful
feedback, particularly on the morphological aspects of this paper.

{\sloppy
	\printbibliography[heading=subbibliography,notkeyword=this]
}

\end{document}
